%===============Kapitola: Měření teploty ===========================================================
\sisetup{
output-decimal-marker = {,},
}

\chapter{Snímače tepelných veličin}
\minitoc
\newpage
  \section{Základní pojmy}
    \href{http://cs.wikipedia.org/wiki/Teplota}{Teplota} je charakteristika tepelného stavu hmoty.
    V obecném významu je to vlastnost předmětů a okolí, kterou je člověk schopen vnímat a přiřadit
    jí pocity studeného, teplého či horkého. V přírodních a technických vědách a jejich aplikacích
    je to \emph{skalární intenzivní veličina}, která je vzhledem ke svému pravděpodobnostnímu
    charakteru vhodná k popisu stavu ustálených makroskopických systémů. Teplota souvisí s
    kinetickou energií částic látky.

    Teplota je základní fyzikální veličinou soustavy \texttt{SI} s jednotkou kelvin (\si{\kelvin})
    a vedlejší jednotkou stupeň Celsia (\si{\degreeCelsius}). Nejnižší možnou teplotou je teplota
    absolutní nuly (\SI{0}{\kelvin}; \SI{-273.15}{\degreeCelsius}), ke které se lze libovolně
    přiblížit, avšak nelze jí dosáhnout.
         
    Do této skupiny patří především rozsáhlá část snímačů teploty. Z hlediska měřených veličin
    můžeme provést následující rozdělení.
    \begin{enumerate}
      \item \textbf{Snímače teploty}
        \begin{enumerate}[label=\emph{\alph*})]
          \item \textit{Snímače pro dotykové měření} 
            \begin{itemize}
              \item elektrické
               \begin{itemize}\addtolength{\itemsep}{-0.5\baselineskip}
                 \item odporové kovové
                 \item odporové polovodičové
                 \item termoelektrické
                 \item polovodičové 
               \end{itemize}   
              \item dilatační
              \item termoelektrické
              \item tlakové
              \item speciální
            \end{itemize}
          \item \textit{Snímače pro bezdotykové měření}
            \begin{itemize}\addtolength{\itemsep}{-0.5\baselineskip}
              \item monochromatické pyrometry
              \item pásmové pyrometry
              \item radiační pyrometry
            \end{itemize}
        \end{enumerate}
      \item \textbf{Snímače tepla}
      \item \textbf{Snímače tepelného toku}
    \end{enumerate}  
       
      \subsection{Elektrické teploměry}
        \subsection{Odporové snímače}
          Odporové snímače využívají princip změny elektrického opdoru vlivem změny teplot.
          Základním požadavkem kladeným na materiál snímače je co největší a stálý teplontí
          součinitel odporu a zároveň co největší měrný odpor. Pro tyto účely se používají kovové a
          polovodičové materiály.
          
          \subsubsection{Kovové odporové snímače} 
            Jsou to především čisté kovy, které se používají pro realizaci vlastního odporového
            článku. Požadavkem je, aby nereagovaly s izolačním nebo ochranným krytem. Jakékoliv
            chemické nebo fyzikální vlivy by mohly způsobit nestálost odporu při stálé teplotě,
            Použitý materiál nemá vykazovat změnu teplotního součinitele odporu s časem (stárnutí)
            a hysterezi. Nejčastěji používanými materiály je \emph{platina, nikl, měď, slitina
            stříbro-zlato} a další \cite[s.~96]{Zehnula1983}.
             
            Platina je výhodná pro velkou chemickou stálost, vysokou teplotou tavení a možností
            dosažení vysoké čistoty. Pro snímače teploty se používá tzv. fyzikálně  čistá platina,
            jejíž čistota se pohybuje kolem 99,93 až 99,99 \% Pt. Měření ukázala, že změny
            základního odporu u sériověš vyráběných přesných teploměru se pohybí kolem
            \num{5e-6}$R_o$ (což odpovídá \SI{0.001}{\kelvin}), u nejlepších teploměrů je tato
            hodnota ještě o řád menší. Proto se používá platina pro etalonový teplměr v oblasti
            teplot \SI{-259.34}{\degreeCelsius} až \SI{630.74}{\degreeCelsius}.
            
            Závislost odporu na teplotě pro rozsah \num{0} až \SI{630}{\degreeCelsius} se vyjadřuje
            rovnicí
            \begin{equation}\label{SAC:kov_Ro1}
              R_\vartheta = R_0(1 + A\vartheta + B\vartheta^2)
            \end{equation}
            \hskip20mm\begin{minipage}{0.8\linewidth}
              kde $R_0$ je odpor při \SI{0}{\degreeCelsius}, $\vartheta\ldots$ teplota ve
              \si{\degreeCelsius}, $A\ldots$ konstanta (\SI{3.9075e-3}{\per\degreeCelsius}),
              $B\ldots$ konstanta (\SI{-0.575e-6}{\per\degreeCelsius}).
            \end{minipage}  
 
             V rozmezí od \SI{0}{\degreeCelsius} do \SI{-190}{\degreeCelsius} se vyjadřuje
             závislost odporu na teplotě rovnicí
             \begin{equation}\label{SAC:kov_Ro2}
               R_\vartheta = R_0[1 + A\vartheta + B\vartheta^2 + C(\vartheta - 100)\vartheta^3)]
             \end{equation}
             \hskip20mm kde $C$ je konstanta (\SI{-4e12}{\degreeCelsius}).            
            
 \printbibliography[heading=subbibliography]   
