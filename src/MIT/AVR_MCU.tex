%==============================Kapitola: Procesory AVR===========================================  
\chapter{Procesory AVR}
\minitoc
\newpage
  \section{AVR Architektura}
    AVR architektura vychází z koncepce rychle přístupného registrového pole, které obsahuje 32 obecně použitelných registrů
    délky 8 bitů. Přístup do registrového pole je proveden v jediném strojovém cyklu. To znamená, že během jednoho strojového
    cyklu lze vykonat jednu aritmeticko-logickou operaci\footnote{oba operandy aritmeticko-logické operace jsou
    načteny z registrového pole, operace je provedena a výsledek směřuje opět do registrového pole v jediném strojovém cyklu}

    Tato technika, umožňuje vyšší výkon ve srovnání s mikrokontroléry řady 8051, které disponují instrukcemi o délce od 12 do 48
    hodinových cyklů, navíc se pro výpočty musí používat akumulátor, který je jen jeden. Registrové pole lze v tomto smyslu
    chápat jako skupinu akumulátorů.

    \subsection{Strojový cyklus}
      Strojový cyklus mikrokontrolérů AVR přímo odpovídá hodinovému cyklu. Nedochází k žádnému dělení hodinových cyklů jako
      například u mikrokontrolérů řady 8051\footnote{jeden strojový cyklus obsahuje 12 hodinových cyklů}
    \subsection{Prefetch a pipelining}
      Mikrokontroléry AVR používají jednoduchý \emph{předvýběr instrukce} (\textbf{prefetch}) umožňující \emph{jednofázové
      zřetězení instrukcí} (\textbf{pipelining})
