% !TeX spellcheck = de_DE
%================ Kapitola 2: Unsere Familie ==============================================
\chapter{Lektion 3: Zu Besuch}\label{NJ:chap_N1_L3}

  \section*{Slovní zásoba}
    \begin{table}[h] % L3_Wortschatz01.jpg
      \begin{tabular}{llll}  
        \hline 
          r Besuch (e)s &  návštěva        & s Stück (e)s        & kus, kousek           \\
          zu Besuch     & na návštěvě, -u  & r Kuchen, s         & koláč                 \\
          r Tag, (e)s   & den              & schmecken           & chutnat               \\
          entschuldigen & omluvit,         & hoffentlich         & snad,                 \\
                        & prominout        &                     & doufejme že           \\
          sprechen      & mluvit           & es schmeckt mir     & chutná mi             \\
          anbieten      & nabídnout,       & wirklich            & skutečný, -ě,         \\
                        & nabízet          &                     & opravdu               \\
          wünschen      & přát (si)        & ausgezeichnet       & výborný, -ě           \\
          r Herr, n     & pán              & lange               & dlouho                \\
          herzlich      & srdečný, -ě      & e Woche             & týden                 \\
          willkommen    & vítán, vítaný    & vor allem           & především             \\
          herein        & dovnitř          & gefallen            & líbit se              \\
          gleich        & hned             & r Wenzelsplatz      & Václavské náměstí     \\
          wieder        & zase, opět       & die Karlsbrücke     & Karlův most           \\
          holen (4.p.)  & dojít (pro)      & hören               & (u)slyšet             \\
          nur           & jen, pouze       & grüßen              & (po)zdravit           \\
          s Brot, (e)s  & chléb            & endlich             & konečně               \\
          r Supermarkt  & supermarket      & da                  & tu, tady              \\
          später        & později          & müssen, ich muss    & muset                 \\
          oder          & nebo             & e Station           & stanice               \\
          dürfen, ich darf   & smět        & fragen (4.p. nach)  & ptát se (koho na)     \\
          warten (auf 4.p.)  & čekat(na)   & r Weg, (e)s         & cesta                 \\
          stören        & rušit            & zeigen              & ukázat,               \\
                        & vyrušovat        &                     & ukazovat              \\
          gar nicht     & vůbec ne         & zuerst              & nejprve               \\
          nehmen        & vzít, brát       & geradeaus           & rovně, přímo          \\
          r Platz, es   & místo, náměstí   & rechts              & vpravo                \\
          (s) Deutsch   & němčina          & nach rechts         & doprava (kam?)        \\
          nichts        & nic              & s Ende, s           & konec                 \\
          e Tasse       & šálek            & am Ende             & na konci              \\
          können, ich kann   & moci; umět  & e Straße            & ulice                 \\
          schaden       & (u)škodit        & links               & vlevo                 \\ 
          bringen       & přinést          & nach links          & doleva                \\
          sofort        & ihned, okamžitě  & sehen               & (u)vidět              \\
        \hline       
      \end{tabular}
      \caption*{ }
    \end{table}
  
    \subsection*{Vazby}
      \begin{table}[ht!]   % L1_Redensart01.jpg
        \begin{tabular}{ll}
          Sprechen Sie Deutsch?             & Mluvíte německy?                           \\
          ein Freund von Martin             & (jeden) Martinův přítel                    \\
          Herzlich willkommen!              & Srdečně (vás/tě) vítám!                    \\
          Kommen Sie herein!                & Pojďte dál!                                \\
          Er holt Brot vom Supermarkt.      & Šel do supermarketu pro chleba.            \\
          Nehmen Sie Platz!                 & Posaďte se!                                \\
          Hoffentlich schmeckt es Ihnen.    & Doufejme, že vám to bude chutnat (chutná). \\
          Vielen Dank! Danke sehr (schön)!  & Děkuji mnohokrát!                          \\
          Wie komme ich zu ...?             & Jak se dostanu na/k ...?
        \end{tabular}
        \caption*{ }
      \end{table}
  
    \subsection*{Grüße - Pozdravy}
      \begin{table}[ht!]   % L3_Grammatik01.jpg
        \begin{tabular}{llll}
          Guten Tag!    & Dobrý den!           & Auf Wiedersehen!   & Nashledanou!           \\
          Guten Morgen! & Dobré ráno!          & Tschüs!            & Ahoj! (při odchodu)    \\
          guten Abend!  & Dobrý večer!         & Bis dann / später! & Na shledanou později   \\
          Hallo!        & Ahoj! (při setkání)! & Bis morgen!        & Na shledanou zítra!    \\
          Grüß dich!    & Ahoj! Nazdar!        & Mach's gut!        & Měj se dobře! Ahoj!    \\
        \end{tabular}
        \caption*{ }
      \end{table}
      
  \section*{Gramatika}
    \subsection*{Všimněte si!} % L3_Grammatik02.jpg
      \hspace*{2em}Sprechen Sie Deutsch?\newline
      Názvy jazyků se používají většinou bez členu.
        \vspace*{-1em}
      \begin{table}[ht!]
        \hspace*{1em}
        \begin{tabular}{ll}  % L3_Grammatik02.jpg
           Nehmen Sie \textbf{Tee} oder \textbf{Kaffee}? & Dáte si čaj nebo kávu?      \\
           Holst du \textbf{Brot}?                       & Dojdeš pro chleba?          \\
           Nehmen Sie ein Stück \textbf{Kuchen}!         & Vezměte si kousek koláče!   \\
           Eine Tasse \textbf{Tee} kann nicht schaden.   & Šálek čaje nemůže uškodit.  \\
        \end{tabular}
        \caption*{ }
      \end{table}

      Člen se vynechává také u podst. jmen látkových označujících neurčité množství, nebo následují-li za 
      podst. jmény označujícími míru nebo množství.
      \newpage
      \begin{table}[ht!]
        \hspace*{1em}
        \begin{tabular}{ll}  % L3_Grammatik02.jpg
           Vor allem gefallen mir das Stadtzentrum  & Především se mi líbí střed města     \\
           \hspace*{3em}und der Hradschin.          & \hspace*{3em}a Hradčany.             \\
           Dort sind Christian und Martin.          & Je tam Christian a Martin.           \\
           Eine Tasse \textbf{Tee} kann nicht schaden.   & Šálek čaje nemůže uškodit.  \\
        \end{tabular}
        \caption*{ }
      \end{table}
      
      Předchází-li sloveso vícenásobnému podmětu, je v němčině sloveso v množném čísle.
    
    \subsection*{Skloňování podstatných jmen v jednotném čísle}
      \begin{table}[ht!]
        \begin{tabular}{lll}  % L3_Grammatik03.jpg
          \hline
               & rod mužský          &                       \\ 
          \hline
           1.p &   der Freund        & ein Freund            \\
           2.p &   des Freund(e)s    & eines Freund(e)s      \\
           3.p &   dem Freund        & einem Freund          \\
           4.p &   den Freund        & einen Freund          \\
        \end{tabular}
        \caption*{ }
      \end{table}
      Koncovka -es v 2. pádě se užívá povinně před sykavkami (s, ß, z, tz, tsch, sch, zt). U podst. 
      jmen zakončených na -er, -el, -en je pouze koncovka -s (des Vaters, des Onkels, des Kuchens). 
      V ostatních případech použití -e- kolísá. Ve slovníku se označuje: des Freund(e)s.
      
      \begin{table}[ht!]
        \begin{tabular}{lll}  % L3_Grammatik04.jpg
          \hline
               & rod mužský            &                        \\ 
          \hline
           1.p &   der Student         & ein Student            \\
           2.p &   des Studenten       & eines Studenten        \\
           3.p &   dem Studenten       & einem Studenten        \\
           4.p &   den Studenten       & einen Studenten        \\
        \end{tabular}
        \caption*{ }
      \end{table}    
      Tento typ skloňování mají pouze některá životná podstatná jména rodu mužského. Patří sem 
      mnoho podstatných jmen cizího původu končících na \emph{-ent} (Dozent, Assistent), 
      \emph{-ant}, \emph{-at}, \emph{-oge}, \emph{-it}, \emph{-ist} a některá jiná: r Kollege, r 
      Herr.	
  
      Některá podstatná jména přibírají pouze koncovku -n: der Herr - des Hern	der Kollege - des 
      Kollegen
  
      \begin{table}[ht!]
        \begin{tabular}{lllll}  % L3_Grammatik05.jpg
          \hline
               & rod ženský      &                & rod střední     &                  \\ 
          \hline
           1.p &   die Schule    & eine Schule    &   das Kind      & ein Kind         \\
           2.p &   der Schule    & einer Schule   &   des Kind(e)s  & eines Kind(e)s   \\
           3.p &   der Schule    & einer Schule   &   dem Kind      & einem Kind       \\
           4.p &   die Schule    & eine Schule    &   das Kind      & ein Kind         \\
        \end{tabular}
        \caption*{}
      \end{table}
      Střední rod má shodný tvar členů vždy v 1. a 4. pádě. Podstatná jména ženského rodu mají vždy 
      všechny pády jednotného čísla bez koncovky. Členy mají shodné tvary \textbf{vždy} v 1. a 4. 
      pádě a v 2. a 3. pádě.
      
    \subsection*{Skloňování tázacích zájmen „wer" (kdo) a „was" (co)}
      \begin{table}[ht!] % L3_Grammatik06.jpg
        \begin{tabular}{ll|lll}  
          \hline
          Wer ist die Frau dort?    & Kdo je ta žena tam?   & 1.p  & wer     & was      \\
          Wessen Freundin ist das?  & Čí je to přítelkyně?  & 2.p  & wessen  &          \\
          Wem sagst du es?          & Komu to řekneš?       & 3.p  & wem     &          \\
          Wen bittest du?           & Koho poprosíš?        & 4.p  & wen     & was      \\
          Was liegt hier?           & Co tu leží?           &      &         &          \\
          Was studiert ihr?         & Co studujete?         &      &         &          \\
          \hline
        \end{tabular}
        \caption*{}
      \end{table}
      
      Pozor: Wessen Kind ist das? Čí je to dítě? Po zájmenu „wessen" následuje bezprostředně podstatné 
      jméno bez členu, ale: Das ist das Kind meiner Schwester. je to dítě nic sestry.
      
    \subsection*{Předložky s 3. pádem}
      \begin{table}[ht!]   % L3_Grammatik07.jpg
        \centering
        \begin{tabular}{l|lllllllll}
          \hline
            1. pád & ich  & du   & er  & sie & es  & wir & ihr  & sie   & Sie   \\
            3. pád & mir  & dir  & ihm & ihr & ihm & uns & euch & ihnen & Ihnen \\
            4. pád & mich & dich & ihn & sei & es  & uns & euch & sie   & Sie   \\
          \hline
        \end{tabular}
        \caption*{ }
      \end{table}
      Skloňování osobních zájmen: 2. pádu osobních zájmen (meiner, deiner, seiner, ihrer atd.) se 
      užívá jen zřídka.
      \begin{itemize}\addtolength{\itemsep}{-0.5\baselineskip}
        \item Ich zeige dir die Schule. Ukážu ti tu školu.
        \item Ich zeige sie dir.Ukážu ti ji.
      \end{itemize}     
      Jsou-li ve větě dva předměty vyjádřené osobními zájmeny, \textbf{předchází} 4. pád 3. pádu.
  
      \begin{table}[ht!]   % L3_Grammatik08.jpg
      \hspace*{1em}
        \begin{tabular}{ll}
          \hline
            Wir kommen \textbf{aus} der Schule.      & Přicházíme \textbf{ze} školy.          \\
            \textbf{Außer} mir wohnt hier noch Eva.  & \textbf{Kromě} mne tu bydlí ještě Eva. \\
            Wir fahren \textbf{mit} der Metro.       & Pojedeme metrem.                       \\
            Er kommt erst \textbf{nach} mir.         & Přijde až \textbf{po} mně.             \\
            Sie fahren \textbf{nach} Dortmund.       & Jedou \textbf{do} Dortmundu.           \\
            Das habe ich \textbf{von} Martin.        & To mám \textbf{od} Martina.            \\
            Wir sprechen \textbf{von} dem Herrn.     & Mluvíme \textbf{o} tom pánovi.         \\
            Er ist \textbf{seit} Mai in Prag.        & Je \textbf{od} května v Praze.         \\
            Er ist \textbf{seit} einer Woche hier.   & Je tady \textbf{už} týden.             \\
            \textbf{Zu} wem gehen Sie?               & \textbf{Ke} komu jdete?                \\
          \hline
        \end{tabular}
        \caption*{ }
      \end{table}

      \begin{table}[ht!]
      \centering
        \begin{tabular}{llll}  % L3_Grammatik09.jpg
          \hline
            aus   & z               & nach  & po, podle, do {\scriptsize (u geograf. jmen)}  \\
            außer & kromě, mimo     & seit  & od (časově)                                    \\
            bei   & u, při          & von   & z, od, o                                       \\
            mit   & s, prostý 7. p. & zu    & k                                              \\
          \hline
        \end{tabular}
        \caption*{}
      \end{table}
      V hovorové řeči je možné sloučit některé předložky s určitým členem: 
      \begin{itemize}
        \item bei dem \textrightarrow beim \hspace*{1em} 
              zu  dem \textrightarrow zum  \hspace*{1em} 
              von dem \textrightarrow vom  \hspace*{1em} 
              zu  der \textrightarrow zur  
      \end{itemize}
    
    \subsection*{Rozkazovací způsob}
      \begin{table}[ht!]   % L3_Grammatik10.jpg
        \hspace*{1em}
        \begin{tabular}{ll}
          \hline
             \textbf{Bitte} ihn!                & Popros ho!                          \\
             \textbf{Besuch(e)} den Onkel!      & Navštiv strýčka!                    \\
             \textbf{Trinken wir} einen Kaffee! & Vypijme si kávu!                    \\
             \textbf{Macht} es doch!            & Udělejte to přece!                  \\
             \textbf{Kommen Sie} bitte auch!    & Přijďte prosím také! (Pojďte ...)    \\
          \hline
        \end{tabular}
        \caption*{ }
      \end{table}

      \begin{table}[ht!]
        \hspace*{1em}
        \begin{tabular}{lll}  % L3_Grammatik10.jpg
          \hline
            2. osoba jedn. č. & Besuch(e)!    & Warte!         \\
            2. osoba množ. č. & Besucht!      & Wartet!        \\
            1. osoba množ. č. & Besuchen wir! & Warten wir!    \\
            3. osoba množ. č. & Besuchen Sie! & Warten Sie!    \\
          \hline
        \end{tabular}
        \caption*{}
      \end{table}
      % L3_Grammatik11.jpg
      Tvary rozkazovacího způsobu v 2. osobě obou čísel nemají osobní zájmeno, jsou tedy bez 
      podmětu. V 1. a 3. os. množného čísla stojí zájmeno za slovesem. Koncovka -e v 2. osobě 
      jednotného čísla je povinná pouze u sloves, která končí v kmeni na -t, -d, -ig, -m, -n, -eln, 
      -ern.
      
      Sloveso \textbf{\uv{sein}} má v rozkazovacím způsobu nepravidelné tvary:
      \begin{table}[ht!]
        \hspace*{1em}
        \begin{tabular}{ll}  % L3_Grammatik12.jpg
          \hline
          \textbf{Sei} so gut!              &  Buď tak hodný!           \\
          \textbf{Seien Sie} bitte so nett! &  Buďte prosím tak laskav! \\
          \textbf{Seien wir} nett zu ihr!   &  Buďme k ní milí!         \\
          \textbf{Seid auch} nett zu ihr!   &  Buďte k ní také milí!    \\
          \hline
        \end{tabular}
        \caption*{}
      \end{table}

    \subsection*{Způsobová slovesa \uv{müssen}, \uv{können}, \uv{dürfen}}
      \begin{table}[ht!]   % L3_Grammatik13.jpg
        \hspace*{1em}
        \begin{tabular}{ll}
          \hline
             Sie müssen fleißig sein.         & Musíte být pilný.             \\
             Können Sie es ihm sagen?         & Můžete mu to říci?            \\
             Wir können noch nicht Deutsch.   & Neumíme ještě německy.        \\
             Darf ich hier warten?            & Smím (mohu) zde počkat?       \\
          \hline
        \end{tabular}
        \caption*{ }
      \end{table}
      
      \begin{table}[ht!]   % L3_Grammatik13.jpg
        \hspace*{1em}
        \begin{tabular}{llllll}
          \hline
          \multicolumn{2}{c}{müssen (muset)}  & \multicolumn{2}{c}{können (moci, umět)}
                                              & \multicolumn{2}{c}{dürfen (smět)}       \\ 
          \hline
          ich muss  & wir müssen & ich kann   & wir können  & ich darf   & wir dürfen   \\
          du  musst & ihr müsst  & du  kannst & ihr könnt   & du  darfst & ihr dürft    \\
          er  muss  & sie müssen & er  kann   & sie können  & er  darf   & sie dürfen   \\
          \hline
        \end{tabular}
        \caption*{ }
      \end{table}
      \begin{table}[ht!]   % L3_Grammatik13.jpg
        \hspace*{1em}
        \begin{tabular}{lllll}
          \hline
             Sie   &  muss & es  & auch noch           & machen.          \\
                   & Darf  & ich & Ihnen meine Familie & vorstellen?      \\
          \hline
        \end{tabular}
        \caption*{ }
      \end{table}
      Způsobová slovesa bývají zpravidla doplněna infinitivem významového slovesa. Infinitiv stojí 
      vždy na konci věty a tvoří s určitým tvarem způsobového slovesa tzv. \textbf{větný rámec}.
      \begin{table}[ht!]   % L3_Grammatik13.jpg
        \hspace*{1em}
        \begin{tabular}{ll}
          \hline
          Darf ich stören?                &  Smím (mohu) vyrušit?           \\
          Darf ich Ihnen Kaffee anbieten? &  Mohu vám nabídnout kávu?       \\
          \hline
        \end{tabular}
        \caption*{ }
      \end{table}
      
      V němčině se na rozdíl od češtiny používá v případě, kdy žádáme o dovolení, většinou sloveso 
      „důrfen".