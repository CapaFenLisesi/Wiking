%===================================================================================================
% Physics
% Fyz1.tex
%===================================================================================================
% notes:
%~~~~~~~~~
% \label{fyz:eq359}
% \label{fyz:fig200}
% \label{fyz:exam010}
% \label{fyz:tab006}
%---------------------------------------------------------------------------------------------------
% Setting path to image 
  \graphicspath{{../src/FYZ/img/}}
  \makeatletter
    \@addtoreset{chapter}{part}
  \makeatother
%---------------------------------------------------------------------------------------------------
\lstset{ %
  language=Matlab,                       % choose the language of the code
  basicstyle=\footnotesize,              % the size of the fonts that are used for the code
%  backgroundcolor=\color{White},        % choose the background color.
  commentstyle=\color{help}\textit,
  keywordstyle=\color{keyword}\textbf,
  breaklines=true,                       % sets automatic line breaking
  breakatwhitespace=true,                % sets if automatic breaks should only happen at whitespace
  showspaces=false,                      % show spaces adding particular underscores
  showstringspaces=true,                 % underline spaces within strings
  showtabs=true,                         % show tabs within strings adding particular underscores
  frame=none,                            % adds a frame around the code - none, single
  tabsize=8,                             % sets default tabsize to 8 spaces
  captionpos=b,                          % sets the caption-position to bottom
  numbers=left,                          % where to put the line-numbers -none, left, right
  numberstyle=\footnotesize,             % the size of the fonts that are used for the line-numbers
  stepnumber=1,                          % the step between two line-numbers. If it's 1 each line
                                         % will be numbered
  xleftmargin=3em,                       % adjust left margin
}

\part{Fyzika I}\label{part:FYZI}
\parttoc
  
\iftoggle{DEBUG}{
%  DEBUG was on
% ~~~~~~~~~~~~~~~~~~~~~~~~~~~~~~~~~~~~~~~~~~~~~~~~
%  \input{../src/FYZ/chap/fey1ch01_02_03.tex}
%  \input{../src/FYZ/chap/fey1ch04.tex}
%  \input{../src/FYZ/chap/fey1ch05.tex}
%  \input{../src/FYZ/chap/fey1ch06.tex}
%  \input{../src/FYZ/chap/fey1ch07.tex}
%  \input{../src/FYZ/chap/fey1ch08.tex}
%  \input{../src/FYZ/chap/fey1ch09.tex}
%  \input{../src/FYZ/chap/fey1ch10.tex}
%  \input{../src/FYZ/chap/fey1ch11.tex}
%  \input{../src/FYZ/chap/fey1ch12.tex}
%  \input{../src/FYZ/chap/fey1ch13_14.tex}
%  \input{../src/FYZ/chap/fey1ch15.tex}
%  \input{../src/FYZ/chap/fey1ch16.tex}
%  \input{../src/FYZ/chap/fey1ch17.tex}
  \input{../src/FYZ/chap/fey1ch21.tex} 
  \input{../src/FYZ/chap/fey1ch24.tex} 
  \input{../src/FYZ/chap/fey1ch25.tex} 
  \input{../src/FYZ/chap/fey1ch26.tex} 
  % !TeX spellcheck = cs_CZ
{\tikzset{external/prefix={tikz/FYZI/}}
 \tikzset{external/figure name/.add={ch24_}{}}
%---------------------------------------------------------------------------------------------------
% file fey1ch27.tex
%---------------------------------------------------------------------------------------------------
%========================= Geometrická optika ======================================================
\chapter{Geometrická optika}\label{fyz:IchapXXVII}
\minitoc

  \section{Úvod}\label{fyz:IchapXXVIIsecI}
    Na několika přístrojích předvedeme aproximaci nazvanou \emph{geometrická optika}. Je to
    nejužitečnější aproximace pro praktickou konstrukci mnoha optických systémů a přístrojů.
    Geometrická optika je buď velmi jednoduchá nebo velmi komplikovaná.
    
    Abychom mohli pokračovat potřebujeme jeden geometrický vztah a to: máme-li trojúhelník s malou
    výškou $h$ a velkou základnou $d$, pak přepona $s$ je delší než základna (viz obr.
    \ref{fyz:fig156}).  
    
    Tedy 
    \begin{equation}\label{FYZ:eq_triangle}
     \Delta \approx \frac{h^2}{2s}.
    \end{equation}
    To je celá geometrie, kterou je třeba znát, aby bylo možné diskutovat vznik obrazů pomocí
    zakřivených ploch.
    
    \begin{figure}[ht!]
      \centering
      \includegraphics[width=0.6\linewidth]{fyz_fig156.pdf}
      \captionof{figure}{Trojúhelník s malou výškou a velkou základnou
                 \cite[s.~358]{Feynman01}}
      \label{fyz:fig156}  
    \end{figure}

  \section{Ohnisková vzdálenost kulové čočky}\label{fyz:IchapXXVIIsecII}
  \section{Ohnisková vzdálenost čočky}\label{fyz:IchapXXVIIsecIII}
  \section{Zvětšení}\label{fyz:IchapXXVIIsecIV}
  \section{Složené čočky}\label{fyz:IchapXXVIIsecV}
  \section{Aberace}\label{fyz:IchapXXVIIsecVI}
  \section{Rozlišovací schopnost}\label{fyz:IchapXXVIIsecVII}
  \section{Příklady a cvičení}\label{fyz:IchapXXVIIsecVIII}
  
} %tikzset
%~~~~~~~~~~~~~~~~~~~~~~~~~~~~~~~~~~~~~~~~~~~~~~~~~~~~~~~~~~~~~~~~~~~~~~~~~~~~~~~~~~~~~~~~~~~~~~~~~~
\printbibliography[title={Seznam literatury}, heading=subbibliography]
\addcontentsline{toc}{section}{Seznam literatury}
%  \input{../src/FYZ/chap/fey1ch28.tex}
%  \input{../src/FYZ/chap/fey1ch29.tex}
%  \input{../src/FYZ/chap/fey1ch30.tex} 
%  \input{../src/FYZ/chap/fey1ch31.tex} 
%  \input{../src/FYZ/chap/fey1ch32.tex} 
%  \input{../src/FYZ/chap/fey1ch33.tex} 
%  \input{../src/FYZ/chap/fey1ch35.tex}
%  % !TeX spellcheck = cs_CZ
{\tikzset{external/prefix={tikz/FYZI/}}
 \tikzset{external/figure name/.add={ch33_}{}}
%---------------------------------------------------------------------------------------------------
% file fey1ch36.tex
%---------------------------------------------------------------------------------------------------
%=========================== Kapitola: Mechanizmus vidění =========================================
\chapter{Mechanizmus vidění}\label{fyz:IchapXXXVI}
\minitoc
  \section{Barevný vjem}\label{fyz:IchapXXXVIsecI}
  \section{Fyziologie oka}\label{fyz:IchapXXXVIsecII}
  \section{Tyčinky}\label{fyz:IchapXXXVIsecIII}
  \section{Složené oko hmyzu}\label{fyz:IchapXXXVIsecIV}
  \section{Jiné oči}\label{fyz:IchapXXXVIsecV}

} %tikzset
%---------------------------------------------------------------------------------------------------
%\printbibliography[title={Seznam literatury},heading=subbibliography]
\addcontentsline{toc}{section}{Seznam literatury}
%  % !TeX spellcheck = cs_CZ
{\tikzset{external/prefix={tikz/FYZI/}}
 \tikzset{external/figure name/.add={ch39_}{}}
%=========================== Kapitola: Kinetická teorie plynů =====================================
\chapter{Kinetická teorie plynů}\label{fyz:IchapXXXIX}
\minitoc
  \section{Vlastnosti látek}\label{fyz:IchapXXXIXsecI}
  \section{Tlak plynu}\label{fyz:IchapXXXIXsecII}
  \section{Stlačitelnost záření}\label{fyz:IchapXXXIXsecIII}
  \section{Teplota a kinetická energie}\label{fyz:IchapXXXIXsecIV}
  \section{Zákon ideálního plynu}\label{fyz:IchapXXXIXsecV}
  \section{Příklady a cvičení}\label{fyz:IchapXXXIXsecVI}


    \begin{figure}[ht!] %\ref{fyz_fig242}
      \centering
      \includegraphics[width=0.5\linewidth]{fyz_fig242.pdf}
      \caption{
               (\cite[s.~525]{Feynman01})}
      \label{fyz_fig242}
    \end{figure}

    \begin{figure}[ht!] %\ref{fyz_fig243}
      \centering
      \includegraphics[width=0.5\linewidth]{fyz_fig243.pdf}
      \caption{
               (\cite[s.~530]{Feynman01})}
      \label{fyz_fig243}
    \end{figure}

    \begin{figure}[ht!] %\ref{fyz_fig244}
      \centering
      \includegraphics[width=0.5\linewidth]{fyz_fig244.pdf}
      \caption{
               (\cite[s.~531]{Feynman01})}
      \label{fyz_fig244}
    \end{figure}

    \begin{figure}[ht!] %\ref{fyz_fig245}
      \centering
      \includegraphics[width=0.5\linewidth]{fyz_fig245.pdf}
      \caption{
               (\cite[s.~533]{Feynman01})}
      \label{fyz_fig245}
    \end{figure}

} %tikzset
%---------------------------------------------------------------------------------------------------
\printbibliography[title={Seznam literatury}, heading=subbibliography]
\addcontentsline{toc}{section}{Seznam literatury} 
%  % Kinematika částice
%---------------------------------------------------------------------------------------------------
% file kinematika.tex
%---------------------------------------------------------------------------------------------------
\chapter{Kinematika částice}
\minitoc
\newpage
  Nejjednodušší fyzikální soustava je jeden hmotný bod, který se pohybuje v prostoru a čase. Pojem
  hmotný bod je ovšem abstrakce, model, kterým nahrazujeme reálnou částici. Vyjadřujeme jím, že
  odhlížíme od tvaru a rozměru částice, považujeme ji za bodovou, a kromě její geometrické polohy v
  daném okamžiku jí připisujeme pouze jedinou fyzikální vlastnost, hmotnost. V tomto smyslu budeme
  v mechanice často místo hmotného bodu hovořit prostě o částici.
  %----------------- Kinematický popis pohybu částice ---------------------------------------------- 
  \section{Kinematický popis pohybu částice}
    V kinematice se zajímáme pouze o průběh pohybu částice v prostoru a čase a nepátráme po
    příčinách tohoto pohybu a jeho změn. Předpokládáme, že částice se pohybuje po spojité křivce,
    trajektorii, a snažíme se určit jednak tvar této trajektorie a zákon pohybu po ní, tj. polohu
    částice na trajektorii v závislosti na čase\footnote{Představa o pohybu částice po trajektorii
    jako po spojité křivce vyplývá z naší smyslové zkušenosti. Ukazuje se, že v mikrosvětě tato
    představa neodpovídá skutečnosti a pojem trajektorie tam ztrácí smysl. Částice se v mikrosvětě
    pohybuje podle zákonu kvantové mechaniky a v daném okamžiku není možné současné přesně stanovit
    její polohu a rychlost}. Spojitá křivka má v každém bodě tečnu a můžeme zavést pojem okamžité
    rychlosti částice mířící ve směru této tečny.
  
    \begin{figure}[ht!]
      \centering
      \includegraphics[width=\linewidth]{trajectory.pdf}
      \caption{Příklad trajektorie částice a zavedení kartézské soustavy souřadnic}
      \label{mech:fig_trjctr}
    \end{figure}
  
    Předpokládejme nejprve, že trajektorie částice je zadána. Pak můžeme od zvoleného bodu na
    trajektorii a zvoleného okamžiku měřit dráhu částice $s(t)$, tedy délku křivky, kterou částice
    za určitou dobu prošla (obr. \ref{mech:fig_trjctr}). V okamžiku $t$ je částice v bodě daném
    prošlou dráhou $s$, v okamžiku $t + \Delta t$ v bodě $s + \Delta s$. Dráha $s$ tu vlastně
    představuje parametr udávající polohu bodu na křivce; tímto způsobem popisujeme například pohyb
    automobilu na dálnici a udáváme na kterém je právě kilometru.
  
    Přitom můžeme zavést \textbf{střední rychlost částice} v intervalu $\Delta t$
    \begin{equation}\label{mech:eq_stredni_rychlost}
      \langle v\rangle=\frac{\Delta s}{\Delta t},
    \end{equation}
    \textbf{okamžitou rychlost částice} v okamžiku $t$
    \begin{equation}\label{mech:eq_okamzita_rychlost}
      v(t)=\lim_{\Delta t\rightarrow0}\frac{\Delta s}{\Delta t}=\frac{ds}{dt}=\dot{s}
    \end{equation}
    a \textbf{okamžité zrychlení}
    \begin{equation}\label{mech:eq_okamzite_zrychleni}
      a(t)=\lim_{\Delta t\rightarrow0}\frac{\Delta v}{\Delta t}
          =\frac{dv}{dt}=\lim_{\Delta t\rightarrow0}\frac{d^2s}{dt^2}=\dot{v}=\ddot{s}
    \end{equation}
    Takto zavedené rychlost a zrychlení jsou skalární funkce času a udávají pouze jak se mění dráha
    a rychlost při pohybu po zadané trajektorii, ve směru tečny k této trajektorii.
  
    Obecně však musíme udat polohu částice v prostoru vzhledem k nějaké vztažné soustavě. Tato
    soustava, například kartézská, je spojena s nějakým tuhým tělesem a doplněna hodinami
    umístěnými   například v počátku. V místnosti mohou jako kartézské osy sloužit průsečnice stěn
    a podlahy. Potom udáváme tři kartézské souřadnice částice jako funkce času:
    \begin{equation}\label{mech:eq_xyz}
      x=x(t),\quad y=y(t),\quad z=z(t)
    \end{equation}
    Soustava tří rovnic (rov. \ref{mech:eq_xyz}) představuje parametrické vyjádření tvaru
    trajektorie. Rovnici trajektorie v kartézských souřadnicích dostaneme, vyloučíme-li z rov.
    \ref{mech:eq_xyz} čas. Parametrem pohybu může být ovšem i dráha:
    \begin{equation}\label{mech:eq_draha}
      x = x(s),\quad y = y(s),\quad z = z(s).
    \end{equation}
    Přitom $s = s[x(t), y(t), z(t)]$ vystupuje jako složená funkce času. Výše zavedená skalární
    rychlost bude
  
    %-------------------------- Základní pohyby a jejich skládání---------------------------------------------
    \subsection{Základní pohyby a jejich skládání}
      Uvedeme nyní některé základní typy pohybu částice.
      \subsubsection{Pohyb přímočarý}
          Nechť přímočarý pohyb probíhá podél osy x s počátečními podmínkami $x = x_0,v_x =
          \dot{x}=v_{0_x}$ při $t = t_0$. Pak rozlišujeme
          \begin{itemize}
            \item \emph{Pohyb rovnoměrný} s konstantní rychlostí $v_{0_x}$ a nulovým zrychlením
                  $a_x=0$. Integrací a použitím počátečních podmínek dostáváme zákon pohybu:
                  \begin{equation}\label{mech:eq_primocar_rovnomer}
                    x=x_0+v_{0_x}(t-t_0)
                  \end{equation}
            \item \emph{Pohyb rovnoměrně zrychlený} s konstantním zrychlením $a_{0_x}$ kladným nebo
                  záporným. Integrací a použitím počátečních podmínek dostáváme zákon ry\-chlo\-sti a
                  zákon pohybu:
                  \begin{align}
                    v &= v_0x+a_0x(t-t_0), \\
                    x &= x_0+v_{0_x}(t-t_0)+\frac{1}{2}a_{0_x}(t-t_0)^2 \label{mech:eq_const_acc}.
                  \end{align}
                  Je-li při $t = 0 x = 0, v = 0$ dostaneme známé vztahy $$v=a_{0_x}t,\quad
                  x=\frac{1}{2}a_{0_x}t$$
            \item \emph{Pohyb nerovnoměrný} se zrychlením obecně závislým na čase $a(t)$. Pak
                  do\-sta\-ne\-me zákon rychlosti a zákon pohybu integrováním
                  \begin{align}
                    v &= v_{0_x}+\int_{t_0}^{t}{a(t)dt} \\
                    x &= x_0+v_{0_x}(t-t_0)+\int_{t_0}^{t}{v(t)dt}
                  \end{align}
          \end{itemize}
      \subsubsection{Pohyb kruhový}
      \subsubsection{Pohyb harmonický}
        Pohyb harmonický dostaneme jako projekci rovnoměrného kruhového pohybu kolem počátku do
        jedné z kartézských os. Například v ose $y$ pak máme
        \begin{equation}\label{mech:eq_p_harmon}
          y(t)=A\sin(\omega t+\varphi_0)
        \end{equation}
        kde 
        \begin{labeling}{$\omega t+\varphi_0$}
          \setlength{\itemindent}{2cm}
          \item[\(y\)]                     \(\ldots\)\emph{výchylka (elongace)}, 
          \item[\(A\)]                     \(\ldots\)\emph{amplituda}, 
          \item[\(\omega\)]                \(\ldots\)\emph{úhlová rychlost} $[rad\cdot s^{-1}]$,
          \item[\(T=\frac{2\pi}{\omega}\)] \(\ldots\)\emph{perioda} $[s]$, 
          \item[\(f=\frac{1}{T}\)]         \(\ldots\)\emph{frekvence} $[Hz]$, 
          \item[\(\omega t+\varphi_0\)]    \(\ldots\)\emph{fáze}, 
          \item[\(\varphi_0\)]             \(\ldots\)\emph{počáteční fáze při} $t=0$ neboli
                                                     \emph{fázová konstanta}.
        \end{labeling}
  
        Souřadnice vektorů rychlosti a zrychlení při harmonickém pohybu jsou
        \begin{subequations}
          \label{mech:eq_harm} 
          \begin{align}
            v_y = \dot{y} 
              & = \omega A\cos(\omega t+\varphi_0 )=
                  \omega A\sin(\omega t+\varphi_0+\frac{\pi}{2}), \label{mech:eq_harm_vy}         \\
             a_y = \ddot{y} 
              &= -\omega^2A\sin(\omega t+\varphi_0 )=
                  \omega^2A\sin(\omega t+\varphi_0+\pi).          \label{mech:eq_harm_ay}
          \end{align}
        \end{subequations}  
        Z těchto vztahů je vidět, že při harmonickém pohybu rychlost předbíhá výchylku o
        $\frac{\pi}{2}$ a zrychlení o $\pi$ (je v protifázi).

    %----------------- Skládání pohybů -------------------------------------------------------------
    \subsection{Skládání pohybů}
      Ačkoliv částice může konat současně několik pohybů, lze je vektorově skládat. Tento netriviální 
      poznatek usnadňuje studium mechanických pohybů. Ukážeme nyní některé zajímavé případy skládání pohybu.
  
    \subsubsection{Skládání kolmých přímočarých pohybů}
      Se skládáním kolmých přímočarých pohybů se setkáváme při \emph{vrhu těles v homogenním tíhovém poli ve 
      vakuu}. Uvažujme rovinný pohyb v rovině $x, z,$ při čemž v záporném směru osy $z$ má pohyb zrychlení 
      velikosti $g$.

      \begin{example}Výstřel z děla (ve vakuu).
        Dělová koule opouští hlaveň zadanou rychlostí. Určete:
        \begin{itemize}\addtolength{\itemsep}{-0.5\baselineskip}
          \item maximální dostřel pro zadanou úsťovou rychlost,
          \item hranice oblasti, ve kterém lze zasáhnout cíl,
          \item stanovte velikost potřebného náměru děla pro zasažení libovolného cíle uvnitř
                ochranné paraboly.
        \end{itemize}
        \begin{figure}[ht!]
          \centering
          \begin{tabular}{c}
            \subfloat[ ]{\label{mech:fig_delo1}
               \includegraphics[width=0.9\linewidth]{kinematika_delo_vakuum.pdf}}      \\
            \subfloat[ ]{\label{mech:fig_tan_alpha}   
               \includegraphics[width=0.3\linewidth]{kinematika_delo_tan_alpha.pdf}}
          \end{tabular}   
          \caption{K příkladu výpočtu trajektorie projektilu. Goniometrický vzorec
                   $|\cos\alpha|=\frac{1}{\sqrt{1+\tan\alpha^2}}$ lze snadno odvodit z náčrtu
                   pomocí Pythagorovy věty (Přepona pravoúhlého trojúhelníka je
                   $\sqrt{1+\tan\alpha^2}$)}            
        \end{figure}
        \textbf{Řešení:}
          Uvažujte rovinný pohyb v rovině $xz$, při čemž v záporném směru osy $z$ má pohyb zrychlení 
          velikosti $g$. Ve směru osy $z$ tedy probíhá rovnoměrně zrychlený pohyb podle rov. 
          \ref{mech:eq_const_acc}. Vztáhneme-li počáteční podmínky k okamžiku \(t = 0\), máme
          \begin{equation}\label{mech_eq_delo_vakuum_osa_z}
            z(t)=z_0+v_{0z}t-\frac{1}{2}gt^2, \qquad v_z(t)=v_{0z}-gt
          \end{equation}
          Ve směru osy $x$ je pohyb rovnoměrný:
          \begin{equation}\label{mech_eq_delo_vakuum_osa_x}
            x(t)=x_0+v_{0x} t,\qquad v_x(t)=v_{0x}=\mathrm{konst}
          \end{equation}

          Dělová koule opouští hlaveň pod elevačním úhlem $\alpha$ za podmínek dle obr.
          \ref{mech:fig_delo1} platí  $x_0=0, z_0=0, v_{0x}=v_0\cos\alpha>0,
          v_{0z}=v_0\sin\alpha>0$. Jde tedy o skládání \emph{rovnoměrného přímočarého pohybu s
          rychlostí} $v_0\cos\alpha$ ve směru osy $x$ a svislého pohybu vzhůru. Získané rovnice
          \begin{equation}\label{mech:eq_delo_rce_pohybu}
            z(t)=v_{0z}t-\frac{1}{2}gt^2, \qquad x(t)=v_{0x}t
          \end{equation}
          představují \emph{parametrické rovnice trajektorie}. Vyloučíme-li z nich čas $t$,
          dostaneme rovnici křivky v kartézských souřadnicích
          \begin{equation}\label{mech:eq_delo_vakuum_param_rce}
            z(x)=  \frac{v_{0z}}{v_{0x}}x-\frac{1}{2}\frac{g}{v_{0x}^2}x^2
                = x\tan\alpha-\frac{1}{2}\frac{g}{v_0^2\cos^2\alpha}x^2
          \end{equation}
          Nyní aplikujeme goniometrický vzorec
          \begin{equation*}
            |cos\alpha|=   \frac{1}{\sqrt{1+\tan^2\alpha}}\Rightarrow \frac{1}{\cos^2\alpha} 
                       = 1+\tan^2\alpha
          \end{equation*}
          odvozený dle náčrtku na obrázku \ref{mech:fig_tan_alpha} a dostáváme rovnici
          \begin{equation}\label{mech_eq_example_vysledna_rce}
            z(x)=x\tan\alpha-\frac{1}{2}\frac{g}{v_0^2}(1+\tan^2\alpha)x^2
          \end{equation}
          Pohyb projektilu (dělové koule) probíhá po stejné trajektorii, jako šikmý vrh v
          homogenním tíhovém poli ve vakuu, tedy po parabole. Snadno dostaneme \emph{souřadnice
          vrcholu dráhy, dálku doletu a celkovou dobu letu}.

          \begin{itemize}
            \item Maximální dolet pro daný elevační úhel:
              \begin{equation}\label{mech:eq_elevacni_uhel_1}
                0=x\tan\alpha-\frac{1}{2}\frac{g}{v_0^2}(1+\tan^2\alpha)x^2
              \end{equation}
              \emph{Netriviální kořen této kvadratické rovnice je námi hledaný dolet dělové koule}
              \begin{equation}\label{mech:eq_elevacni_uhel_2}
                x_d=\frac{2v_0^2\tan\alpha}{g(1+\tan^2\alpha)}(1+\tan^2\alpha)
                   =\frac{2v_0^2\sin\alpha\cos\alpha}{g}=\frac{v_0^2\sin{2\alpha}}{g}
              \end{equation}

            \item Celková doba letu:
              \begin{equation}\label{mech:eq_doba_letu}
                t_d=\frac{x_d}{v_{0x}} =\frac{2v_0^2\sin\alpha\cos\alpha}{gv_0\cos\alpha}
                   =\frac{2v_0\sin\alpha}{g}
              \end{equation}

            \item Souřadnice vrcholu dráhy: \emph{získáme derivováním rov.
                  \ref{mech_eq_example_vysledna_rce}}
                  \begin{align}
                    0   &= tan\alpha-\frac{g}{v_0^2(1+\tan^2\alpha)}x_v                         \\
                    x_v &= \frac{v_0^2}{g}\frac{\tan\alpha}{1+\tan^2\alpha}=
                           \frac{v_0^2}{g}\frac{\sin\alpha}{\cos\alpha}
                           \cdot\cos^2\alpha\cdot\frac{2}{2}                                   \\
                    x_v &= \frac{v_0^2\sin{2\alpha}}{2g}
                   \end{align}
                   \emph{Souřadnici $z_v$ dostaneme dosazením $x_v$  do rov.
                   \ref{mech_eq_example_vysledna_rce}}
                   \begin{align}
                     z_v &= \frac{v_0^2}{g}\frac{\tan^2\alpha}{1+\tan^2\alpha}-
                            \frac{1}{2}\frac{g}{v_0^2}(1+\tan^2\alpha)\frac{v_0^4}{g^2}
                            \frac{\tan^2\alpha}{(1+\tan^2\alpha)^2}                            \\
                     z_v &= \frac{v_0^2}{g}\frac{\tan^2\alpha}{1+\tan^2\alpha}-
                            \frac{1}{2}\frac{v_0^2}{g}\frac{\tan^2\alpha}{1+\tan^2\alpha}      \\
                     z_v &= \frac{v_0^2\sin^2\alpha}{2g}
              \end{align}
              \emph{Odtud je zřejmé, že maximální délka doletu odpovídá úhlu $\frac{\pi}{4}$ a že
              obecně daného bodu doletu lze dosáhnout pod dvěma různými úhly
              $\frac{\pi}{4}\pm\Delta\alpha$.}

            \item Stanovení elevačního úhlu pro zasažení zadaných souřadnic $[X_c, Z_c]$ cíle:
              \emph{Opět vycházíme z rov. \ref{mech_eq_example_vysledna_rce}, ovšem tentokrát
              nejsou neznáme $x$ a $z$, ale $\alpha$: Použijeme substituci $\tan\alpha=p$ a
              vypočítáme kořeny této kvadratické rovnice:}
              \begin{align}
                0       &= gx^2p^2-2v_0^2xp+(gx^2+2zv_0^2) \\
                p_{1,2} &= \frac{v_0^2\pm\sqrt{v_0^4-g(gx^2+2zv_0^2)}}{gx} \\
                \alpha  &= \tan^{-1}\left(\frac{v_0^2\pm\sqrt{v_0^4-g(gx^2+2zv_0^2)}}{gx}\right)
              \end{align}
              \emph{Je-li cíl zadán v polárních souřadnicích $[r,\varphi]$, lze potřebný náměr
              stanovit takto:}
              \begin{equation}\label{mech:eq_namer}
                \alpha=\tan^{-1}\left(\frac{v_0^2\pm
                       \sqrt{v_0^4-g(gr^2\cos^2\varphi+2r\sin\varphi
                             v_0^2)}}{gr\cos\varphi}\right)
              \end{equation}
              \emph{Pokud ovšem bude diskriminant menší než 0, leží cíl mimo dosah děla. Tj.
               neexistuje takový náměr děla, kterým by bylo možné cíl zasáhnout. Je-li
               diskriminant roven nule, jedná se o hranici, za kterou již při dané úsťové
               rychlosti nelze dostřelit. Body ležící na této obálce tzv. ochranná parabola mohou
               být zasaženy pouze při jedné hodnotě elevačního úhlu.}

            \item Stanovení rovnice ochranné paraboly:
              \emph{To provedeme tak, že položíme diskriminant rovnice pro $\tan\alpha$ roven
               nule a dostaneme rovnici obálky}
              \begin{equation}\label{mech:eq_ochr_parabola}
                v_0^4-g(gx^2+2zv_0^2)\Rightarrow z=-\frac{v_0^2}{2g^2}x^2+\frac{v_0^2}{2g}
              \end{equation}

          \end{itemize}

          %------------------------------Dělo---------------------------------
          \lstinputlisting{../src/MECH/img/kinematika_delo_ve_vakuu.m}
          \begin{lstlisting}[caption=\texttt{kinematika\_delo\_ve\_vakuu.m} pro ověření výpočtu balistické 
          dráhy projektilu.]
          \end{lstlisting}
          %-------------------------------------------------------------------
          \begin{figure}[ht!]
            \centering
            \includegraphics[width=\linewidth]{kinematika_delo_vakuum_matlab.pdf}
            \caption[Výpočet trajektorie projektilu]{Výpočet trajektorie projektilu ve vakuu při
                     úsťové rychlosti $210 m/s$ pomocí sw
                     MATLAB\textsuperscript{\textregistered}.}
            \label{mech:fig_delo_matlab}
          \end{figure}
        \end{example}
        \newpage
        \begin{example}
          Kolo vagónu se valí po vodorovné kolejnici. Uvažujte bod, který je v počátečním okamžiku
          pod středem kola ve vzdálenosti, která může být menší, rovna nebo větší než vzdálenost
          středu kola od kolejnice.
          \begin{figure}[ht!]
            \centering
            \includegraphics[scale=1]{trajectory_wheel_carriage.pdf}
            \caption{Kolo vagónu a tři možné polohy bodu}
            \label{mech:fig_wheel_1}
          \end{figure}
          \newline
          Určete  parametrické rovnice dráhy zvoleného bodu, složky rychlosti a její velikost,
          složky zrychlení a jeho velikost, tečné a normálové zrychlení a poloměr křivosti dráhy.
          \cite[p.~11]{Slavik}
          \newline
          \textbf{Řešení}: Obvodová rychlost v místě dotyku s kolejnicí je $v=\omega R$, což
          vzhledem k předpokladu o valení představuje posuvnou rychlost kola. Parametrické
          rovnice pro střed kola jsou pak
          \begin{align}\label{mech:eq_wheel_center}
            x_S &= \omega R t \\
            y_S &= R
          \end{align}
          Uvažovaný bod $B_3$ na obr. \ref{mech:fig_wheel_xy} je ve své nové pozici v čase $t_1$
          posunut vůči středu o vzdálenost $r\cdot\sin\omega t$ ve směru osy $x$ a o vzdálenost
          $r\cdot\cos\omega t$ ve směru osy $y$. Z obrázku \ref{mech:fig_wheel_xy} lze odvodit
          následující rovnice pro souřadnice libovolného bodu \emph{B} na kole vagónu.
          \begin{figure}[ht!]
            \centering
            \includegraphics[scale=1.2]{trajectory_wheel_carriage_solve.pdf}
            \caption[Náčrt pro odvození rovnic pohybu bodu na kole vagónu]{Náčrt pro odvození
                     parametrických rovnic pohybu libovolně zvoleného bodu na kole vagónu}
            \label{mech:fig_wheel_xy}
          \end{figure}
                   
          \begin{minipage}[t]{0.5\textwidth}% first column
            \begin{itemize}
              \item ve směru osy x:                                        
              \begin{align*}
                x &= x_S + x'                        \\
                x &= x_S + r\cos(\psi-\frac{\pi}{2}) \\
                x &= x_S + r\sin\psi                 \\
                x &= \omega R t + r\sin\omega t
              \end{align*}
              \end{itemize}  
          \end{minipage}%second column           
          \begin{minipage}[t]{0.5\textwidth}
           \begin{itemize}
              \item ve směru osy y: 
              \begin{align*}
                y &= y_S - y'                        \\
                y &= y_S - r\sin(\psi-\frac{\pi}{2}) \\
                y &= y_S + r\cos\psi                 \\
                y &= R + r\cos\omega t
              \end{align*}
            \end{itemize}                 
          \end{minipage}
          % VERY important here is that there is no space between the /end{minipage} and the next
          %\begin{minipage} (obviously not counting comments). Otherwise LaTeX will not render the
          % columns side by side. 
          takže, parametrické rovnice dráhy mají tvar \textbf{cykloidy} viz
          \ref{mech_eq_cykloida}.
          \begin{align}\label{mech_eq_cykloida}
            x &= \omega R t + r\sin\omega t\\
            y &= R + r\cos\omega t
          \end{align}
          \begin{itemize}
            \item Složky rychlosti:
              \begin{align}\label{mech_eq_cykloida_v}
               v_x &= \frac{dx}{dt} = \omega R + r\omega\cos\omega t \nonumber \\
               v_y &= \frac{dy}{dt} = -r\omega\sin\omega t           \nonumber \\
               v   &= \sqrt{v_x^2 + v_y^2}= \omega\sqrt{R^2 + 2Rr\cos\omega t + r^2}
              \end{align}
            \item Složky zrychlení:
              \begin{align}\label{mech_eq_cykloida_a}
                a_x &= \frac{dv_x}{dt} = -r\omega^2\sin\omega t \\
                a_y &= \frac{dv_y}{dt} = -r\omega^2\cos\omega t \\
                a   &= \sqrt{a_x^2 + a_y^2}= r\omega^2\sqrt{\sin^2\omega t + 
                       \cos^2\omega t} = r\omega^2
              \end{align}
              Tento výsledek je superpozicí rovnoměrného kruhového a rovnoměrného přímočarého
              pohybu.
            \item Tečné zrychlení dostaneme derivací velikosti rychlosti
              \begin{align}\label{mech_eq_cykloida_at}
                a_t &= \frac{dv}{dt} = \omega\cdot\frac{1}{2\sqrt{R^2 + 2Rr\cos\omega t + r^2}}
                       \cdot(-1)\cdot2Rr\omega\sin\omega t  \\
                a_t &= \frac{r\omega^2\cdot|R\cos\omega t-r|}{\sqrt{R^2 + 2Rr\cos\omega t + r^2}}
              \end{align}
            \item Normálové zrychlení získáme užitím Pythagorovy věty
              \begin{align*}
                a_n &= \sqrt{a^2 - a_t^2} \\
                a_n &= \sqrt{(r\omega^2)^2-\left(
                       \frac{Rr\omega^2\sin\omega t}
                            {\sqrt{R^2 + 2Rr\cos\omega t + r^2}}\right)^2}                      \\
                a_n &= \frac{r\omega^2|R\cos\omega t - r|}{\sqrt{R^2 + 2Rr\cos\omega t + r^2}}
              \end{align*}
            \item Poloměr křivosti $R_0$ dostaneme ze vztahu $a_n=\frac{v^2}{R_0}$:
              \begin{align*}\label{mech:eq_cykloida_R0}
                  R_0 &=  \cfrac{\omega^2(R^2 + 2Rr\cos\omega t + r^2)}
                         {\cfrac{r\omega^2|R\cos\omega t - r|}
                         {\sqrt{R^2 + 2Rr\cos\omega t + r^2}}}                                  \\
                  R_0 &=  \frac{(R^2 + 2Rr\cos\omega t + r^2)^
                         {\frac{3}{2}}}{|Rr\cos\omega t - r^2|}
              \end{align*}
              Poloměr křivosti není roven vzdálenosti od středu kola r: drahou bodu není
              kružnice, nýbrž cykloida (viz obr. \ref{mech:fig_wheel_cycloid}).
          \end{itemize}
          \begin{align*}
            (x - \omega R t)^2            &= r^2\sin^2\omega t                          \\
            (y-R)^2                       &= r^2\cos\omega t                            \\
            (x - \omega R t)^2 + (y-R)^2  &= r^2\sin^2\omega t + r^2\cos\omega t        \\
            (x - \omega R t)^2 + (y-R)^2  &= r^2 \quad \text{kde}\; t = 
                                             \frac{1}{\omega}\arccos\frac{y-R}{r}       \\
            \left(x - R\arccos\frac{y-R}{r}\right)^2 + (y-R)^2  &= r^2
          \end{align*}
          \begin{figure}[ht!]
            \centering
            \includegraphics[width=0.9\linewidth]{trajectory_wheel_cycloid.pdf}
            \caption[Cykloida]{Cykloida: pro $B_2\ldots r=R$ je cykloida prostá; $B_3\ldots r>R$
                               cykloida prodloužená; $B_1\ldots r<R$ cykloida zkrácená;
                               \texttt{[cykloida.m]} }
            \label{mech:fig_wheel_cycloid}
          \end{figure}
        \end{example}
          
    \subsubsection{Skládání harmonických pohybů v kolmých směrech}
      Zmíníme se ještě o skládání \textbf{harmonických pohybů v kolmých smě\-rech}. Sklá\-dá\-me-li dva 
      takové pohyby o stejné úhlové frekvenci, bude výsledný pohyb probíhat po trajektorii dané parametrický 
      jako
      \begin{equation}\label{mech:eq_lissaujous1}
          x=A\sin(\omega t+\varphi_{01}),\qquad y=B\sin(\omega t +\varphi_{02})
      \end{equation}
      Výsledný pohyb vytváří zajímavé geometrické tvary známé pod názvem Lissajousovy obrazce. Jejich vzhled 
      závisí na poměru frekvencí a na fázovém úhlu \cite{Okrouhlik}.

      Označíme fázi kmitů ve směru $x$ jako $\omega t+\varphi_{01} = \varphi$, rozdíl fází obou kmitů jako 
      $\varphi_{02}-\varphi_{01} =\delta$. Dále vyloučíme z parametrických rovnic čas. K tomu cíli vyjádříme 
      $sin\varphi$ a $cos\varphi$ pomocí veličin na čase nezávisejících a použijeme známý vztah $sin^2\varphi 
      + cos^2\varphi = 1$. Máme
      \begin{equation}\label{mech:eq_lissajous2}
          \sin\varphi=\frac{x}{A}, \qquad 
          \sin(\varphi+\delta)=\sin\varphi\cos\delta+\cos\varphi\sin\delta=\frac{y}{B}
      \end{equation}
      odkud
      \begin{equation}\label{mech:eq_lissajous3}
          \cos\varphi=\frac{1}{\sin\delta}\left(\frac{y}{B}-\frac{x}{A}\cos\delta\right)
      \end{equation}
      Sečteme-li nyní $sin^2\varphi$ a $cos^2\varphi$, dostaneme rovnici trajektorie
      \begin{equation}\label{mech:eq_lissajous4}
          \frac{x^2}{A^2}+\frac{y^2}{B^2}-\frac{2xy}{AB}\cos\delta=\sin^2\delta
      \end{equation}
      V závislosti na $\delta$ může tato rovnice odpovídat rovnici \emph{úsečky}, nebo \emph{elipsy}. Je-li 
      $\delta = n\pi$, probíhají kmity po úsečce, jejíž přímka má směrnici $k = \pm\frac{B}{A}$, je-li 
      $\delta = \left(n + \frac{1}{2}\right)\pi$, je trajektorií
      elipsa
      \begin{equation}\label{mech:eq_lissajous5}
          \frac{x^2}{A^2}+\frac{y^2}{B^2}=1
      \end{equation}

      Jsou-li amplitudy obou pohybů stejné, přejde pro $\delta = \left(n+\frac{1}{2}\right)\pi$ elipsa v 
      kružnici. S uvedeným skládáním dvou kolmých pohybů o stejných frekvencích se setkáváme nejen v 
      mechanice, ale například i v elektromagnetismu a optice při studiu polarizace světla. Výsledné 
      trajektorie získané pomocí počítače jsou na obr. \ref{mech:fig_lissajous_1} a obr. 
      \ref{mech:fig_lissajous_2} \cite{Stoll}.

      \begin{figure*}
        \centering
        \subfloat[$A=B$ a $\omega_1=\omega_2$.]{\label{mech:fig_lissajous_1}
           \includegraphics[width=0.4\textwidth]{lissajous_1.pdf}}
        \subfloat[$A=B$ a $\frac{\omega_1}{\omega_2}=\frac{2}{3}$.]{\label{mech:fig_lissajous_2}
           \includegraphics[width=0.6\textwidth]{lissajous_2.pdf}}
        \caption[Skládání harm. pohybů v kolmých směrech]{Trajektorie harmonických pohybů
                 $x=A\sin(\omega_1 t)$ a $y=B\sin(\omega_2 t+\varphi)$ v kolmých směrech}
        \label{mech:fig_lissajous_obrazce}
      \end{figure*}

      Jsou-li úhlové frekvence kolmých pohybů různé, vznikají složité tzv. \textbf{Lissajousovy obrazce} viz 
      \ref{mech:fig_lissajous_2}. Program ukazuje, jak se projevuje změna fázového úhlu při daném poměru 
      frekvencí obou pohybů.

      %---------------------------------------------------------------
        \lstinputlisting{../src/MECH/img/Lissajous.m}
        \begin{lstlisting}[caption=\texttt{Lissajous.m}vykreslí skládání harmonických pohybů v kolmých 
        směrech.]
        \end{lstlisting}
      %---------------------------------------------------------------

%  %========================================= Kapitola: Astrofyzika =============================================
\chapter{Úvod}
\minitoc
\newpage
  \wikiAstrofyzika  je vědní obor ležící na rozhraní \emph{fyziky} a \emph{astronomie}. Zabývá se fyzikou 
  vesmíru, včetně fyzikálních vlastností (svítivost, hustota, teplota, chemické složení) astronomických 
  objektů jako jsou hvězdy, galaxie a mezihvězdná hmota, jakož i jejich vzájemné působení.
  
  Podle metod výzkumu těchto objektů se dělí na \emph{fotometrii}, \emph{spektroskopii},
  \emph{radioastronomii}, \emph{astrofyziku rentgenovou}, \emph{infračervenou}, \emph{ultrafialovou} a 
  \emph{neutrinovou}. Každý z těchto podoborů se dále dělí na praktickou a teoretickou část. Praktická 
  získává potřebná data. Teoretická s pomocí fyzikálních zákonů vysvětluje pozorované cho\-vá\-ní vesmírných 
  těles.
  
  \section{Historie astrofyziky}
  
  \section{Základní vztahy}
    \begin{itemize}
      \item \wikiAU - \emph{astronomická jednotka}: průměrná vzdálenost Země od Slunce, $150\cdot10^6\ km$. 
      Vzájemné vzdálenosti planet či jiných objektů sluneční soustavy vy\-já\-dře\-né v AU poskytují 
      relativně názorné měřítko vzdáleností těchto objektů od sebe. Přesná hodnota je $$1 AU = 149\ 597\ 870\ 
      691 \pm 6\ m$$ Kvůli vyšší přesnosti \emph{Mezinárodní astronomická unie} (International Astronomical 
      Union, IAU) přijala novou de\-fi\-ni\-ci, podle které je AU délka poloměru nerušené oběžné kruhové 
      dráhy tělesa se zanedbatelnou hmotností, pohybujícího se okolo Slunce rychlostí $0,017\ 202\ 098\ 950$ 
      radiánů za den ($86\ 400\ s$). 
        \begin{itemize}
          \item Vzdálenost Země od Slunce je $1,00 ± 0,02\ AU$.
          \item Měsíc obíhá kolem Země ve vzdálenosti $0,0026 \pm 0,0001\ AU$.
          \item Mars je od Slunce vzdálen $1,52 \pm 0,14 AU$.
          \item Jupiter je od Slunce vzdálen $5,20 \pm 0,05\ AU$.
          \item Nejvzdálenější člověkem vyrobené těleso, sonda Voyager 1, bylo 31. prosince 2007 ve    
                vzdálenosti $104,93\ AU$ od Slunce.
          \item Průměr sluneční soustavy bez \emph{Oortova oblaku} je přibližně $105\ AU$.
          \item Průměr sluneční soustavy s Oortovým oblakem se odhaduje na $50\ 000$ až $100\ 000\ AU$.
          \item Nejbližší hvězda (po Slunci), Proxima Centauri, se nachází přibližně ve vzdálenosti 
                $268\ 000\ AU$.
          \item Průměr hvězdy Betelgeuze je $2,57\ AU$.
          \item Vzdálenost Slunce od středu Galaxie je přibližně $1,7\cdot10^9\ AU$.
          \item Velikost viditelného vesmíru je asi $8,66\cdot10^{14}\ AU$.
        \end{itemize}
      \item \textbf{l.y.} - \emph{světelný rok}: vzdálenost, kterou světlo ulétna za jeden rok, 
            $9,46\cdot10^{12}\ km$,
      \item \textbf{pc} - \emph{parsek, paralaktická sekunda}: vzdálenost, ze které by poloměr oběžné dráhy   
            Země byl kolmo k zornému paprsku vidět pod úhlem $1''$, $30,9\cdot10^{12}$ km. 
    \end{itemize}
    
    \begin{example} 
      Spočtěte, jakou vzdálenost v metrech vyjadřuje jeden parsek \protect\cite[s.~3]{Kulhanek2009}.
      
      \textbf{řešení}: $1\ pc$ (paralaktická sekunda) je vzdálenost, ze které vidíme velkou poloosu oběžné 
      dráhy Země kolem Slunce pod uhlem $\varphi = 1''$. Úhel $1''$ je tak malý, že strany $VS$ a $VZ$ na 
      obrázku prakticky splývají a místo pravého trojúhelníka $VSZ$ můžeme použít definiční vztah úhlu v 
      obloukové míře (\emph{velkost úhlu je možné určit jako poměr délky oblouku vymezeného rameny na 
      kružnici opsané kolem vrcholu k poloměru této kružnice}). Proto $$\varphi = \frac{R_{SZ}}{l} 
      \rightarrow l = \frac{R_{SZ}}{\varphi},$$ 

      \begin{figure}[ht!]
          \centering
          \includegraphics[width=0.8\linewidth]{ex_parsek.pdf}
          \caption[Parsek]{Parsek}
          \label{afyz:fig_ex_parsek}
      \end{figure}
      
      kde $l$ je vzdálenost $1\ pc$ v metrech, $R_{SZ}$ je vzdálenost země od Slunkce a $\varphi$ je úhel jedné vteřiny vyjádřený v radiánech. 
      $$l = \frac{1,5\cdot10^{11}\ m}{\frac{1}{60\cdot60}\cdot\frac{2\pi}{360}}\cong 3\cdot10^{16}\ m.$$  
      
    \end{example}
    
   Další jednotkou, kterou se v astrofyzice měří vzdálenost dvou vesmírných těles, je \emph{paralaxa}. 
   Pozorovací místa musí být od sebe výrazně vzdálena, aby například při měření vzdálenosti naší nejbližší 
   hvězdy - \emph{Proxima Centauri} byla paralaxa vůbec měřitelná. Vzdálenost této hvězdy je 4,2 světelných 
   let (nebo 270 000 AU) od Země.
   
    \begin{example}
      Najděte paralaxu Proximy Centauri, která je od nás vzdálená asi 4,2 světelného roku \protect\cite[s.~4]{Kulhanek2009}.
      
      \begin{figure}[ht!]
          \centering
          \includegraphics[width=\linewidth]{ex_Proxima_Centauri_paralaxa.pdf}
          \caption[Paralaxa naší nejbližší hvězdy]{Paralaxa naší nejbližší hvězdy}
          \label{afyz:fig_ex_paralaxa}
      \end{figure}
            
      \textbf{Řešení}: Díky pohybu Země kolem Slunce se zdá, že blízké hvězdy opisují oproti vzdáleným 
      elipsu. Úhlový poloměr této elipsy se nazývá paralaxa hvězdy. Lze ji změřit jen pro nejbližší hvězdy. Z 
      definice úhlu (jako v předchozím příkladě) tedy vyplývá, že
      $$\pi = \frac{R_{ZS}}{l} = \frac{1,5\cdot10^{11}\ m}{4,2 l.y} = \frac{1,5\cdot10^{11}\ 
      m}{4,2\cdot9,5\cdot10^{15}\ m}	\cong 3,7\cdot10^{-6}\ rad,$$ což je přibližně $0.76''$. Vidíme, že i u 
      druhé nejbližší hvězdy po Slunci není paralaxa ani celá $1''$.
    \end{example}

\printbibliography[heading=subbibliography]
}
{
% DEBUG was off
%=========================== Kapitola 01: Hlavní etapy vývoje fyziky ==============================
  \input{../src/FYZ/chap/fey1ch01_02_03.tex}
%=========================== Kapitola 02: Zachování energie =======================================
  \input{../src/FYZ/chap/fey1ch04.tex} 
%=========================== Kapitola 03: Čas a vzdálenost ========================================
  \input{../src/FYZ/chap/fey1ch05.tex} 
%=========================== Kapitola 04: Pravděpodobnost =========================================
  \input{../src/FYZ/chap/fey1ch06.tex} 
%=========================== Kapitola 05: Teorie gravitace ========================================
  \input{../src/FYZ/chap/fey1ch07.tex}
%=========================== Kapitola 06: Pohyb ===================================================
  \input{../src/FYZ/chap/fey1ch08.tex} 
%=========================== Kapitola 07: Newtonovy zákony dynamiky ===============================
  \input{../src/FYZ/chap/fey1ch09.tex} 
%=========================== Kapitola 08: Zachování hybnosti ======================================
  \input{../src/FYZ/chap/fey1ch10.tex} 
%=========================== Kapitola 09: Vektory =================================================
  \input{../src/FYZ/chap/fey1ch11.tex}
%=========================== Kapitola 10: Charakteristiky síly ====================================
  \input{../src/FYZ/chap/fey1ch12.tex} 
%=========================== Kapitola 11: Práce a potenciální energie =============================
  \input{../src/FYZ/chap/fey1ch13_14.tex} 
%=========================== Kapitola 12: Speciální teorie relativity =============================
  \input{../src/FYZ/chap/fey1ch15.tex}
%=========================== Kapitola 13: Relativistická energie a hybnost ========================
  \input{../src/FYZ/chap/fey1ch16.tex} 
%=========================== Kapitola 14: Prostoročas =============================================
  \input{../src/FYZ/chap/fey1ch17.tex}
%=========================== Kapitola 15: Dvojrozměrná rotace =====================================
%  % !TeX spellcheck = cs_CZ
%=========================== Kapitola: Dvojrozměrná rotace ========================================
\chapter{Dvojrozměrná rotace}\label{fyz:IchapXVIII}
\minitoc
  \section{Hmotný střed}\label{fyz:IchapXVIIIsecI}
  \section{Rotace tuhého tělese}\label{fyz:IchapXVIIIsecII}
  \section{Moment hybnosti}\label{fyz:IchapXVIIIsecIII}
  \section{Zachování momentu hybnosti}\label{fyz:IchapXVIIIsecIV}
  \section{Příklady a cvičení}\label{fyz:IchapXVIIIsecVI} 
%=========================== Kapitola 16: Hmotný střed; Moment setrvačnosti =======================
%  % !TeX spellcheck = cs_CZ
%=========================== Kapitola: Hmotný střed; Moment setrvačnosti ==========================
\chapter{Hmotný střed; Moment setrvačnosti}\label{fyz:IchapXIX}
\minitoc
  \section{Vlastnosti hmotného středu}\label{fyz:IchapXIXsecI}
  \section{Poloha hmotného bodu}\label{fyz:IchapXIXsecII}
  \section{Určení momentu setrvačnosti}\label{fyz:IchapXIXsecIII}
  \section{Kinetická energie rotace}\label{fyz:IchapXIXsecIV}
  \section{Příklady a cvičení}\label{fyz:IchapXIXsecV}
%=========================== Kapitola 17: Rotace v prostoru =======================================
%  % !TeX spellcheck = cs_CZ
%=========================== Kapitola: Rotace v prostoru ==========================================
\chapter{Rotace v prostoru}\label{fyz:IchapXX}
\minitoc
  \section{Momenty sil v prostoru}\label{fyz:IchapXXsecI}
  \section{Rovnice rotace a vektorový součin}\label{fyz:IchapXXsecII}
  \section{Setrvačník}\label{fyz:IchapXXsecIII}
  \section{Moment hybnosti tuhého tělesa}\label{fyz:IchapXXsecIV}
  \section{Příklady a cvičení}\label{fyz:IchapXXsecV}   
%=========================== Kapitola 18: Harmonický oscilátor ====================================
  \input{../src/FYZ/chap/fey1ch21.tex} 
%=========================== Kapitola 19: Algebra =================================================
%  % !TeX spellcheck = cs_CZ
%=========================== Kapitola: Algebra ====================================================
\chapter{Algebra}\label{fyz:IchapXXII}
\minitoc
  \section{Sčítání a násobení}\label{fyz:IchapXXIIsecI}
  \section{Inverzní operace}\label{fyz:IchapXXIIsecII}
  \section{Abstrakce a zobecnění}\label{fyz:IchapXXIIsecIII}
  \section{Aproximace iracionálních čísel}\label{fyz:IchapXXIIsecIV}
  \section{Komplexní čísla}\label{fyz:IchapXXIIsecV}
  \section{Imaginární exponenty}\label{fyz:IchapXXIIsecVI}
  \section{Příklady a cvičení}\label{fyz:IchapXXIIsecVIII} 
%=========================== Kapitola 20: Rezonance ===============================================
%  % !TeX spellcheck = cs_CZ
%=========================== Kapitola: Rezonance ==================================================
\chapter{Rezonance}\label{fyz:IchapXXIII}
\minitoc
  \section{Komplexní čísla a harmonický pohyb}\label{fyz:IchapXXIIIsecI}
  \section{Tlumené nucené kmity}\label{fyz:IchapXXIIIsecII}
  \section{Rezonance v elektrických obvodech}\label{fyz:IchapXXIIIsecIII}
  \section{Rezonance v přírodě}\label{fyz:IchapXXIIIsecIV}
  \section{Příklady a cvičení}\label{fyz:IchapXXIIIsecV} 
%=========================== Kapitola 21: Přechodové jevy =========================================
  \input{../src/FYZ/chap/fey1ch24.tex} 
%=========================== Kapitola 22: Lineární systémy. Přehled ===============================
  \input{../src/FYZ/chap/fey1ch25.tex} 
%=========================== Kapitola 23: Optika: Princip nejkratšího času ========================
  \input{../src/FYZ/chap/fey1ch26.tex} 
%=========================== Kapitola 24: Geometrická optika ======================================
  % !TeX spellcheck = cs_CZ
{\tikzset{external/prefix={tikz/FYZI/}}
 \tikzset{external/figure name/.add={ch24_}{}}
%---------------------------------------------------------------------------------------------------
% file fey1ch27.tex
%---------------------------------------------------------------------------------------------------
%========================= Geometrická optika ======================================================
\chapter{Geometrická optika}\label{fyz:IchapXXVII}
\minitoc

  \section{Úvod}\label{fyz:IchapXXVIIsecI}
    Na několika přístrojích předvedeme aproximaci nazvanou \emph{geometrická optika}. Je to
    nejužitečnější aproximace pro praktickou konstrukci mnoha optických systémů a přístrojů.
    Geometrická optika je buď velmi jednoduchá nebo velmi komplikovaná.
    
    Abychom mohli pokračovat potřebujeme jeden geometrický vztah a to: máme-li trojúhelník s malou
    výškou $h$ a velkou základnou $d$, pak přepona $s$ je delší než základna (viz obr.
    \ref{fyz:fig156}).  
    
    Tedy 
    \begin{equation}\label{FYZ:eq_triangle}
     \Delta \approx \frac{h^2}{2s}.
    \end{equation}
    To je celá geometrie, kterou je třeba znát, aby bylo možné diskutovat vznik obrazů pomocí
    zakřivených ploch.
    
    \begin{figure}[ht!]
      \centering
      \includegraphics[width=0.6\linewidth]{fyz_fig156.pdf}
      \captionof{figure}{Trojúhelník s malou výškou a velkou základnou
                 \cite[s.~358]{Feynman01}}
      \label{fyz:fig156}  
    \end{figure}

  \section{Ohnisková vzdálenost kulové čočky}\label{fyz:IchapXXVIIsecII}
  \section{Ohnisková vzdálenost čočky}\label{fyz:IchapXXVIIsecIII}
  \section{Zvětšení}\label{fyz:IchapXXVIIsecIV}
  \section{Složené čočky}\label{fyz:IchapXXVIIsecV}
  \section{Aberace}\label{fyz:IchapXXVIIsecVI}
  \section{Rozlišovací schopnost}\label{fyz:IchapXXVIIsecVII}
  \section{Příklady a cvičení}\label{fyz:IchapXXVIIsecVIII}
  
} %tikzset
%~~~~~~~~~~~~~~~~~~~~~~~~~~~~~~~~~~~~~~~~~~~~~~~~~~~~~~~~~~~~~~~~~~~~~~~~~~~~~~~~~~~~~~~~~~~~~~~~~~
\printbibliography[title={Seznam literatury}, heading=subbibliography]
\addcontentsline{toc}{section}{Seznam literatury} 
%=========================== Kapitola 25: Elektromagnetické záření ================================
  \input{../src/FYZ/chap/fey1ch28.tex}
%=========================== Kapitola 26: Interference ============================================
  \input{../src/FYZ/chap/fey1ch29.tex} 
%=========================== Kapitola 27: Difrakce ================================================
  \input{../src/FYZ/chap/fey1ch30.tex} 
%=========================== Kapitola 28: Původ indexu lomu =======================================
  \input{../src/FYZ/chap/fey1ch31.tex} 
%=========================== Kapitola 29: Radiační útlum. Rozptyl světla ==========================
  \input{../src/FYZ/chap/fey1ch32.tex} 
%=========================== Kapitola 30: Polarizace ==============================================
  \input{../src/FYZ/chap/fey1ch33.tex} 
%=========================== Kapitola 31: Relativistické jevy a záření ============================
%  % !TeX spellcheck = cs_CZ
%=========================== Kapitola: Relativistické jevy a záření ===============================
\chapter{Relativistické jevy a záření}\label{fyz:IchapXXXIV}
\minitoc
  \section{Pohybující se zdroje}\label{fyz:IchappXXXIVsecI}
  \section{\uv{Zdánlivý} pohyb}\label{fyz:IchappXXXIVsecII}
  \section{Synchrotronové záření}\label{fyz:IchappXXXIVsecIII}
  \section{Kosmické synchrotronové záření}\label{fyz:IchappXXXIVsecIV}
  \section{Brzdné záření}\label{fyz:IchappXXXIVsecV}
  \section{Dopplerův jev}\label{fyz:IchappXXXIVsecVI}
  \section{Vlnový čtyřvektor}\label{fyz:IchappXXXIVsecVII}
  \section{Aberace}\label{fyz:IchappXXXIVsecVIII}
  \section{Hybnost světla}\label{fyz:IchappXXXIVsecIX}
  \section{Příklady a cvičení}\label{fyz:IchappXXXIVsecX} 
%=========================== Kapitola 32: Barevné vidění ==========================================
  \input{../src/FYZ/chap/fey1ch35.tex} 
%=========================== Kapitola 33: Mechanizmus vidění ======================================
  % !TeX spellcheck = cs_CZ
{\tikzset{external/prefix={tikz/FYZI/}}
 \tikzset{external/figure name/.add={ch33_}{}}
%---------------------------------------------------------------------------------------------------
% file fey1ch36.tex
%---------------------------------------------------------------------------------------------------
%=========================== Kapitola: Mechanizmus vidění =========================================
\chapter{Mechanizmus vidění}\label{fyz:IchapXXXVI}
\minitoc
  \section{Barevný vjem}\label{fyz:IchapXXXVIsecI}
  \section{Fyziologie oka}\label{fyz:IchapXXXVIsecII}
  \section{Tyčinky}\label{fyz:IchapXXXVIsecIII}
  \section{Složené oko hmyzu}\label{fyz:IchapXXXVIsecIV}
  \section{Jiné oči}\label{fyz:IchapXXXVIsecV}

} %tikzset
%---------------------------------------------------------------------------------------------------
%\printbibliography[title={Seznam literatury},heading=subbibliography]
\addcontentsline{toc}{section}{Seznam literatury}
%=========================== Kapitola 34: Kvantové chování ========================================
%  % !TeX spellcheck = cs_CZ
%=========================== Kapitola: Kvantové chování ===========================================
\chapter{Kvantové chování}\label{fyz:IchapXXXVII}
\minitoc
  \section{Mechanika atomů}\label{fyz:IchapXXXVIIsecI}
  \section{Experiment s kulkami}\label{fyz:IchapXXXVIIsecII}
  \section{Experiment s vlnami}\label{fyz:IchapXXXVIIsecIII}
  \section{Experiment s elektrony}\label{fyz:IchapXXXVIIsecIV}
  \section{Interference elektronových vln}\label{fyz:IchapXXXVIIsecV}
  \section{Sledování elektronů}\label{fyz:IchapXXXVIIsecVI}
  \section{Základní principy kvantové mechaniky}\label{fyz:IchapXXXVIIsecVII}
  \section{Princip neurčitosti}\label{fyz:IchapXXXVIIsecVIII}
  \section{Příklady a cvičení}\label{fyz:IchapXXXVIIsecIX} 
%=========================== Kapitola 35: Souvislost mezi vlnovým a korpuskulárním hlediskem ======
%  % !TeX spellcheck = cs_CZ
%=========================== Kapitola: Souvislost mezi vlnovým a korpuskulárním hlediskem =========
\chapter{Souvislost mezi vlnovým a korpuskulárním hlediskem}\label{fyz:IchapXXXVIII}
\minitoc
  \section{Amplitudy vln pravděpodobnosti}\label{fyz:IchapXXXVIIIsecI}
  \section{Měření polohy a hybnosti}\label{fyz:IchapXXXVIIIsecII}
  \section{Difrakce na krystalech}\label{fyz:IchapXXXVIIIsecIII}
  \section{Velikost atomu}\label{fyz:IchapXXXVIIIsecIV}
  \section{Energetické hladiny}\label{fyz:IchapXXXVIIIsecV}
  \section{Filozofické důsledky}\label{fyz:IchapXXXVIIIsecVI}
  \section{Příklady a cvičení}\label{fyz:IchapXXXVIIIsecVII} 
%=========================== Kapitola 36: Kinetická teorie plynů ==================================
  % !TeX spellcheck = cs_CZ
{\tikzset{external/prefix={tikz/FYZI/}}
 \tikzset{external/figure name/.add={ch39_}{}}
%=========================== Kapitola: Kinetická teorie plynů =====================================
\chapter{Kinetická teorie plynů}\label{fyz:IchapXXXIX}
\minitoc
  \section{Vlastnosti látek}\label{fyz:IchapXXXIXsecI}
  \section{Tlak plynu}\label{fyz:IchapXXXIXsecII}
  \section{Stlačitelnost záření}\label{fyz:IchapXXXIXsecIII}
  \section{Teplota a kinetická energie}\label{fyz:IchapXXXIXsecIV}
  \section{Zákon ideálního plynu}\label{fyz:IchapXXXIXsecV}
  \section{Příklady a cvičení}\label{fyz:IchapXXXIXsecVI}


    \begin{figure}[ht!] %\ref{fyz_fig242}
      \centering
      \includegraphics[width=0.5\linewidth]{fyz_fig242.pdf}
      \caption{
               (\cite[s.~525]{Feynman01})}
      \label{fyz_fig242}
    \end{figure}

    \begin{figure}[ht!] %\ref{fyz_fig243}
      \centering
      \includegraphics[width=0.5\linewidth]{fyz_fig243.pdf}
      \caption{
               (\cite[s.~530]{Feynman01})}
      \label{fyz_fig243}
    \end{figure}

    \begin{figure}[ht!] %\ref{fyz_fig244}
      \centering
      \includegraphics[width=0.5\linewidth]{fyz_fig244.pdf}
      \caption{
               (\cite[s.~531]{Feynman01})}
      \label{fyz_fig244}
    \end{figure}

    \begin{figure}[ht!] %\ref{fyz_fig245}
      \centering
      \includegraphics[width=0.5\linewidth]{fyz_fig245.pdf}
      \caption{
               (\cite[s.~533]{Feynman01})}
      \label{fyz_fig245}
    \end{figure}

} %tikzset
%---------------------------------------------------------------------------------------------------
\printbibliography[title={Seznam literatury}, heading=subbibliography]
\addcontentsline{toc}{section}{Seznam literatury} 
%=========================== Kapitola 37: Principy statistické mechaniky ==========================
%  % !TeX spellcheck = cs_CZ
%=========================== Kapitola: Principy statistické mechaniky =============================
\chapter{Principy statistické mechaniky}\label{fyz:IchapXL}
\minitoc
  \section{Exponenciální atmosféra}\label{fyz:IchapXLsecI}
  \section{Boltzmannův zákon}\label{fyz:IchapXLsecII}
  \section{Vypařování kapaliny}\label{fyz:IchapXLsecIII}
  \section{Rozdělení molekul podle rychlosti}\label{fyz:IchapXLsecIV}
  \section{Měrná tepelná kapacita plynů}\label{fyz:IchapXLsecV}
  \section{Selhání klasické fyziky}\label{fyz:IchapXLsecVI}
  \section{Příklady a cvičení}\label{fyz:IchapXLsecVII} 
%=========================== Kapitola 38: Brownův pohyb ===========================================
%  % !TeX spellcheck = cs_CZ
%=========================== Kapitola: Brownův pohyb ==============================================
\chapter{Brownův pohyb}\label{fyz:IchapXLI}
\minitoc
  \section{Ekvipartičnost energie}\label{fyz:IchapXLIsecI}
  \section{Tepelná rovnováha záření}\label{fyz:IchapXLIsecII}
  \section{Ekvipartičnost a kvantový oscilátor}\label{fyz:IchapXLIsecIII}
  \section{Náhodná procházka}\label{fyz:IchapXLIsecIV}
  \section{Příklady a cvičení}\label{fyz:IchapXLIsecV} 
%=========================== Kapitola 39: Aplikace kinetické teorie ===============================
%  % !TeX spellcheck = cs_CZ
%=========================== Kapitola: Aplikace kinetické teorie ==================================
\chapter{Aplikace kinetické teorie}\label{fyz:IchapXLII}
\minitoc
  \section{Vypařování}\label{fyz:IchapXLIIsecI}
  \section{Termoemise}\label{fyz:IchapXLIIsecII}
  \section{Termoionizace}\label{fyz:IchapXLIIsecIII}
  \section{Chemická kinetika}\label{fyz:IchapXLIIsecIV}
  \section{Einsteinovy zákony záření}\label{fyz:IchapXLIIsecV}
  \section{Příklady a cvičení}\label{fyz:IchapXVLIIsecVI} 
%=========================== Kapitola 40: Difuze ==================================================
%  % !TeX spellcheck = cs_CZ
%=========================== Kapitola: Difuze =====================================================
\chapter{Difuze}\label{fyz:IchapXLIII}
\minitoc
  \section{Srážky molekul}\label{fyz:IchapXLIIIsecI}
  \section{Střední volná dráha}\label{fyz:IchapXLIIIsecII}
  \section{Driftová rychlost}\label{fyz:IchapXLIIIsecIII}
  \section{Iontová vodivost}\label{fyz:IchapXLIIIsecIV}
  \section{Molekulová difuze}\label{fyz:IchapXLIIIsecV}
  \section{Tepelná vodivost}\label{fyz:IchapXLIIIsecVI}
  \section{Příklady a cvičení}\label{fyz:IchapXLIIIsecVII} 
%=========================== Kapitola 41: Zákony termodynamiky ====================================
%  % !TeX spellcheck = cs_CZ
%=========================== Kapitola: Zákony termodynamiky =======================================
\chapter{Zákony termodynamiky}\label{fyz:IchapXLIV}
\minitoc
  \section{Tepelné stroje; první zákon}\label{fyz:IchapXLIVsecI}
  \section{Druhý zákon}\label{fyz:IchapXLIVsecII}
  \section{Vratné stroje}\label{fyz:IchapXLIVsecIII}
  \section{Účinnost ideálního stroje}\label{fyz:IchapXLIVsecIV}
  \section{Termodynamická teplota}\label{fyz:IchapXLIVsecV}
  \section{Entropie}\label{fyz:IchapXLIVsecVI}
  \section{Příklady a cvičení}\label{fyz:IchapXLIVsecVII} 
%=========================== Kapitola 42: Ilustrace termodynamiky =================================
%  % !TeX spellcheck = cs_CZ
%=========================== Kapitola: Ilustrace termodynamiky ====================================
\chapter{Ilustrace termodynamiky}\label{fyz:IchapXLV}
\minitoc
  \section{Vnitřní energie}\label{fyz:IchapXLVsecI}
  \section{Aplikace}\label{fyz:IchapXLVsecII}
  \section{Clasiova-Clapeyronova rovnice}\label{fyz:IchapXLVsecIII}
  \section{Příklady a cvičení}\label{fyz:IchapXLVsecIV} 
%=========================== Kapitola 43: Rohatka se západkou =====================================
%  % !TeX spellcheck = cs_CZ
%=========================== Kapitola: Rohatka se západkou ========================================
\chapter{Rohatka se západkou}\label{fyz:IchapXLVI}
\minitoc
  \section{Jak pracuje rohatka}\label{fyz:IchapXLVIsecI}
  \section{Rohatka jako stroj}\label{fyz:IchapXLVIsecII}
  \section{Vratnost v mechanice}\label{fyz:IchapXLVIsecIII}
  \section{Nevratnost}\label{fyz:IchapXLVIsecIV}
  \section{Uspořádání a entropie}\label{fyz:IchapXLVIsecV}
  \section{Příklady a cvičení}\label{fyz:IchapXLVIsecVI} 
%=========================== Kapitola 44: Zvuk. Vlnová rovnice ====================================
%  % !TeX spellcheck = cs_CZ
%=========================== Kapitola: Zvuk. Vlnová rovnice =======================================
\chapter{Zvuk. Vlnová rovnice}\label{fyz:IchapXLVII}
\minitoc
  \section{Vlny}\label{fyz:IchapXLVIIsecI}
  \section{Šíření zvuku}\label{fyz:IchapXLVIIsecII}
  \section{Vlnová rovnice}\label{fyz:IchapXLVIIsecIII}
  \section{Řešení vlnové rovnice}\label{fyz:IchapXLVIIsecIV}
  \section{Rychlost zvuku}\label{fyz:IchapXLVIIsecV}
  \section{Příklady a cvičení}\label{fyz:IchapXLVIIsecVI} 
%=========================== Kapitola 45: Rázy ====================================================
%  % !TeX spellcheck = cs_CZ
%=========================== Kapitola: Rázy =======================================================
\chapter{Rázy}\label{fyz:IchapXLVIII}
\minitoc
  \section{Skládání dvou vln}\label{fyz:IchapXLVIIIsecI}
  \section{Záznějové tóny a modulace}\label{fyz:IchapXLVIIIsecII}
  \section{Postranní pásy}\label{fyz:IchapXLVIIIsecIII}
  \section{Lokalizované vlnové balíky}\label{fyz:IchapXLVIIIsecIV}
  \section{Amplitudy pravděpodobnosti pro částice}\label{fyz:IchapXLVIIIsecV}
  \section{Vlny v trojrozměrném prostoru}\label{fyz:IchapXLVIIIsecVI}
  \section{Normální kmity}\label{fyz:IchapXLVIIIsecVII}
  \section{Příklady a cvičení}\label{fyz:IchapXLVIIIsecVIII} 
%=========================== Kapitola 46: Mody ====================================================
%  % !TeX spellcheck = cs_CZ
%=========================== Kapitola: Mody =======================================================
\chapter{Mody}\label{fyz:IchapIL}
\minitoc
  \section{Odraz vln}\label{fyz:IchapILsecI}
  \section{Vlny v ohraničené oblasti, vlastní frekvence}\label{fyz:IchapILsecII}
  \section{Dvojrozměrné mody}\label{fyz:IchapILsecIII}
  \section{Vázaná kyvadla}\label{fyz:IchapILsecIV}
  \section{Lineární soustavy}\label{fyz:IchapILsecV}
  \section{Příklady a cvičení}\label{fyz:IchapILsecVI} 
%=========================== Kapitola 47: Harmonické kmity ========================================
  % !TeX spellcheck = cs_CZ
%=========================== Kapitola: Harmonické kmity ===========================================
\chapter{Harmonické kmity}\label{fyz:IchapL}
\minitoc
  \section{Hudební tóny}\label{fyz:IchapLsecI}
  \section{Fourierovy řady}\label{fyz:IchapLsecII}
  \section{Kvalita a libozvučnost}\label{fyz:IchapLsecIII}
  \section{Fourierorvy koeficienty}\label{fyz:IchapLsecIV}
  \section{Věta o energii}\label{fyz:IchapLsecV}
  \section{Nelineární odezvy}\label{fyz:IchapLsecVI}
  \section{Příklady a cvičení}\label{fyz:IchapLsecVII} 
%=========================== Kapitola 48: Vlny ====================================================
%  % !TeX spellcheck = cs_CZ
%=========================== Kapitola: Vlny =======================================================
\chapter{Vlny}\label{fyz:IchapLI}
\minitoc
  \section{Kuželové vlny}\label{fyz:IchapLIsecI}
  \section{Rázové vlny}\label{fyz:IchapLIsecII}
  \section{Vlny v pevných látkách}\label{fyz:IchapLIsecIII}
  \section{Povrchové vlny}\label{fyz:IchapLIsecIV}
  \section{Příklady a cvičení}\label{fyz:IchapLIsecVI} 
%=========================== Kapitola 49: Symetrie fyzikálních zákonů =============================
%  % !TeX spellcheck = cs_CZ
%=========================== Kapitola: Symetrie fyzikálních zákonů ================================
\chapter{Symetrie fyzikálních zákonů}\label{fyz:IchapLII}
\minitoc
  \section{Symetrické operace}\label{fyz:IchapLIIsecI}
  \section{Symetrie v prostoru a čse}\label{fyz:IchapLIIsecII}
  \section{Symetrie a zákony zachování}\label{fyz:IchapLIIsecIII}
  \section{Zrcadlový obraz}\label{fyz:IchapLIIsecIV}
  \section{Polární a axiální vektory}\label{fyz:IchapLIIsecV}
  \section{Parita se nezachovává}\label{fyz:IchapLIIsecVI}
  \section{Antihmota}\label{fyz:IchapLIIsecVII}
  \section{Porušení symetrie}\label{fyz:IchapLIIsecVIII}
  \section{Příklady a cvičení}\label{fyz:IchapLIIsecIX} 
%=========================== Kapitola 50: Kinematika částice ======================================
 % Kinematika částice
%---------------------------------------------------------------------------------------------------
% file kinematika.tex
%---------------------------------------------------------------------------------------------------
\chapter{Kinematika částice}
\minitoc
\newpage
  Nejjednodušší fyzikální soustava je jeden hmotný bod, který se pohybuje v prostoru a čase. Pojem
  hmotný bod je ovšem abstrakce, model, kterým nahrazujeme reálnou částici. Vyjadřujeme jím, že
  odhlížíme od tvaru a rozměru částice, považujeme ji za bodovou, a kromě její geometrické polohy v
  daném okamžiku jí připisujeme pouze jedinou fyzikální vlastnost, hmotnost. V tomto smyslu budeme
  v mechanice často místo hmotného bodu hovořit prostě o částici.
  %----------------- Kinematický popis pohybu částice ---------------------------------------------- 
  \section{Kinematický popis pohybu částice}
    V kinematice se zajímáme pouze o průběh pohybu částice v prostoru a čase a nepátráme po
    příčinách tohoto pohybu a jeho změn. Předpokládáme, že částice se pohybuje po spojité křivce,
    trajektorii, a snažíme se určit jednak tvar této trajektorie a zákon pohybu po ní, tj. polohu
    částice na trajektorii v závislosti na čase\footnote{Představa o pohybu částice po trajektorii
    jako po spojité křivce vyplývá z naší smyslové zkušenosti. Ukazuje se, že v mikrosvětě tato
    představa neodpovídá skutečnosti a pojem trajektorie tam ztrácí smysl. Částice se v mikrosvětě
    pohybuje podle zákonu kvantové mechaniky a v daném okamžiku není možné současné přesně stanovit
    její polohu a rychlost}. Spojitá křivka má v každém bodě tečnu a můžeme zavést pojem okamžité
    rychlosti částice mířící ve směru této tečny.
  
    \begin{figure}[ht!]
      \centering
      \includegraphics[width=\linewidth]{trajectory.pdf}
      \caption{Příklad trajektorie částice a zavedení kartézské soustavy souřadnic}
      \label{mech:fig_trjctr}
    \end{figure}
  
    Předpokládejme nejprve, že trajektorie částice je zadána. Pak můžeme od zvoleného bodu na
    trajektorii a zvoleného okamžiku měřit dráhu částice $s(t)$, tedy délku křivky, kterou částice
    za určitou dobu prošla (obr. \ref{mech:fig_trjctr}). V okamžiku $t$ je částice v bodě daném
    prošlou dráhou $s$, v okamžiku $t + \Delta t$ v bodě $s + \Delta s$. Dráha $s$ tu vlastně
    představuje parametr udávající polohu bodu na křivce; tímto způsobem popisujeme například pohyb
    automobilu na dálnici a udáváme na kterém je právě kilometru.
  
    Přitom můžeme zavést \textbf{střední rychlost částice} v intervalu $\Delta t$
    \begin{equation}\label{mech:eq_stredni_rychlost}
      \langle v\rangle=\frac{\Delta s}{\Delta t},
    \end{equation}
    \textbf{okamžitou rychlost částice} v okamžiku $t$
    \begin{equation}\label{mech:eq_okamzita_rychlost}
      v(t)=\lim_{\Delta t\rightarrow0}\frac{\Delta s}{\Delta t}=\frac{ds}{dt}=\dot{s}
    \end{equation}
    a \textbf{okamžité zrychlení}
    \begin{equation}\label{mech:eq_okamzite_zrychleni}
      a(t)=\lim_{\Delta t\rightarrow0}\frac{\Delta v}{\Delta t}
          =\frac{dv}{dt}=\lim_{\Delta t\rightarrow0}\frac{d^2s}{dt^2}=\dot{v}=\ddot{s}
    \end{equation}
    Takto zavedené rychlost a zrychlení jsou skalární funkce času a udávají pouze jak se mění dráha
    a rychlost při pohybu po zadané trajektorii, ve směru tečny k této trajektorii.
  
    Obecně však musíme udat polohu částice v prostoru vzhledem k nějaké vztažné soustavě. Tato
    soustava, například kartézská, je spojena s nějakým tuhým tělesem a doplněna hodinami
    umístěnými   například v počátku. V místnosti mohou jako kartézské osy sloužit průsečnice stěn
    a podlahy. Potom udáváme tři kartézské souřadnice částice jako funkce času:
    \begin{equation}\label{mech:eq_xyz}
      x=x(t),\quad y=y(t),\quad z=z(t)
    \end{equation}
    Soustava tří rovnic (rov. \ref{mech:eq_xyz}) představuje parametrické vyjádření tvaru
    trajektorie. Rovnici trajektorie v kartézských souřadnicích dostaneme, vyloučíme-li z rov.
    \ref{mech:eq_xyz} čas. Parametrem pohybu může být ovšem i dráha:
    \begin{equation}\label{mech:eq_draha}
      x = x(s),\quad y = y(s),\quad z = z(s).
    \end{equation}
    Přitom $s = s[x(t), y(t), z(t)]$ vystupuje jako složená funkce času. Výše zavedená skalární
    rychlost bude
  
    %-------------------------- Základní pohyby a jejich skládání---------------------------------------------
    \subsection{Základní pohyby a jejich skládání}
      Uvedeme nyní některé základní typy pohybu částice.
      \subsubsection{Pohyb přímočarý}
          Nechť přímočarý pohyb probíhá podél osy x s počátečními podmínkami $x = x_0,v_x =
          \dot{x}=v_{0_x}$ při $t = t_0$. Pak rozlišujeme
          \begin{itemize}
            \item \emph{Pohyb rovnoměrný} s konstantní rychlostí $v_{0_x}$ a nulovým zrychlením
                  $a_x=0$. Integrací a použitím počátečních podmínek dostáváme zákon pohybu:
                  \begin{equation}\label{mech:eq_primocar_rovnomer}
                    x=x_0+v_{0_x}(t-t_0)
                  \end{equation}
            \item \emph{Pohyb rovnoměrně zrychlený} s konstantním zrychlením $a_{0_x}$ kladným nebo
                  záporným. Integrací a použitím počátečních podmínek dostáváme zákon ry\-chlo\-sti a
                  zákon pohybu:
                  \begin{align}
                    v &= v_0x+a_0x(t-t_0), \\
                    x &= x_0+v_{0_x}(t-t_0)+\frac{1}{2}a_{0_x}(t-t_0)^2 \label{mech:eq_const_acc}.
                  \end{align}
                  Je-li při $t = 0 x = 0, v = 0$ dostaneme známé vztahy $$v=a_{0_x}t,\quad
                  x=\frac{1}{2}a_{0_x}t$$
            \item \emph{Pohyb nerovnoměrný} se zrychlením obecně závislým na čase $a(t)$. Pak
                  do\-sta\-ne\-me zákon rychlosti a zákon pohybu integrováním
                  \begin{align}
                    v &= v_{0_x}+\int_{t_0}^{t}{a(t)dt} \\
                    x &= x_0+v_{0_x}(t-t_0)+\int_{t_0}^{t}{v(t)dt}
                  \end{align}
          \end{itemize}
      \subsubsection{Pohyb kruhový}
      \subsubsection{Pohyb harmonický}
        Pohyb harmonický dostaneme jako projekci rovnoměrného kruhového pohybu kolem počátku do
        jedné z kartézských os. Například v ose $y$ pak máme
        \begin{equation}\label{mech:eq_p_harmon}
          y(t)=A\sin(\omega t+\varphi_0)
        \end{equation}
        kde 
        \begin{labeling}{$\omega t+\varphi_0$}
          \setlength{\itemindent}{2cm}
          \item[\(y\)]                     \(\ldots\)\emph{výchylka (elongace)}, 
          \item[\(A\)]                     \(\ldots\)\emph{amplituda}, 
          \item[\(\omega\)]                \(\ldots\)\emph{úhlová rychlost} $[rad\cdot s^{-1}]$,
          \item[\(T=\frac{2\pi}{\omega}\)] \(\ldots\)\emph{perioda} $[s]$, 
          \item[\(f=\frac{1}{T}\)]         \(\ldots\)\emph{frekvence} $[Hz]$, 
          \item[\(\omega t+\varphi_0\)]    \(\ldots\)\emph{fáze}, 
          \item[\(\varphi_0\)]             \(\ldots\)\emph{počáteční fáze při} $t=0$ neboli
                                                     \emph{fázová konstanta}.
        \end{labeling}
  
        Souřadnice vektorů rychlosti a zrychlení při harmonickém pohybu jsou
        \begin{subequations}
          \label{mech:eq_harm} 
          \begin{align}
            v_y = \dot{y} 
              & = \omega A\cos(\omega t+\varphi_0 )=
                  \omega A\sin(\omega t+\varphi_0+\frac{\pi}{2}), \label{mech:eq_harm_vy}         \\
             a_y = \ddot{y} 
              &= -\omega^2A\sin(\omega t+\varphi_0 )=
                  \omega^2A\sin(\omega t+\varphi_0+\pi).          \label{mech:eq_harm_ay}
          \end{align}
        \end{subequations}  
        Z těchto vztahů je vidět, že při harmonickém pohybu rychlost předbíhá výchylku o
        $\frac{\pi}{2}$ a zrychlení o $\pi$ (je v protifázi).

    %----------------- Skládání pohybů -------------------------------------------------------------
    \subsection{Skládání pohybů}
      Ačkoliv částice může konat současně několik pohybů, lze je vektorově skládat. Tento netriviální 
      poznatek usnadňuje studium mechanických pohybů. Ukážeme nyní některé zajímavé případy skládání pohybu.
  
    \subsubsection{Skládání kolmých přímočarých pohybů}
      Se skládáním kolmých přímočarých pohybů se setkáváme při \emph{vrhu těles v homogenním tíhovém poli ve 
      vakuu}. Uvažujme rovinný pohyb v rovině $x, z,$ při čemž v záporném směru osy $z$ má pohyb zrychlení 
      velikosti $g$.

      \begin{example}Výstřel z děla (ve vakuu).
        Dělová koule opouští hlaveň zadanou rychlostí. Určete:
        \begin{itemize}\addtolength{\itemsep}{-0.5\baselineskip}
          \item maximální dostřel pro zadanou úsťovou rychlost,
          \item hranice oblasti, ve kterém lze zasáhnout cíl,
          \item stanovte velikost potřebného náměru děla pro zasažení libovolného cíle uvnitř
                ochranné paraboly.
        \end{itemize}
        \begin{figure}[ht!]
          \centering
          \begin{tabular}{c}
            \subfloat[ ]{\label{mech:fig_delo1}
               \includegraphics[width=0.9\linewidth]{kinematika_delo_vakuum.pdf}}      \\
            \subfloat[ ]{\label{mech:fig_tan_alpha}   
               \includegraphics[width=0.3\linewidth]{kinematika_delo_tan_alpha.pdf}}
          \end{tabular}   
          \caption{K příkladu výpočtu trajektorie projektilu. Goniometrický vzorec
                   $|\cos\alpha|=\frac{1}{\sqrt{1+\tan\alpha^2}}$ lze snadno odvodit z náčrtu
                   pomocí Pythagorovy věty (Přepona pravoúhlého trojúhelníka je
                   $\sqrt{1+\tan\alpha^2}$)}            
        \end{figure}
        \textbf{Řešení:}
          Uvažujte rovinný pohyb v rovině $xz$, při čemž v záporném směru osy $z$ má pohyb zrychlení 
          velikosti $g$. Ve směru osy $z$ tedy probíhá rovnoměrně zrychlený pohyb podle rov. 
          \ref{mech:eq_const_acc}. Vztáhneme-li počáteční podmínky k okamžiku \(t = 0\), máme
          \begin{equation}\label{mech_eq_delo_vakuum_osa_z}
            z(t)=z_0+v_{0z}t-\frac{1}{2}gt^2, \qquad v_z(t)=v_{0z}-gt
          \end{equation}
          Ve směru osy $x$ je pohyb rovnoměrný:
          \begin{equation}\label{mech_eq_delo_vakuum_osa_x}
            x(t)=x_0+v_{0x} t,\qquad v_x(t)=v_{0x}=\mathrm{konst}
          \end{equation}

          Dělová koule opouští hlaveň pod elevačním úhlem $\alpha$ za podmínek dle obr.
          \ref{mech:fig_delo1} platí  $x_0=0, z_0=0, v_{0x}=v_0\cos\alpha>0,
          v_{0z}=v_0\sin\alpha>0$. Jde tedy o skládání \emph{rovnoměrného přímočarého pohybu s
          rychlostí} $v_0\cos\alpha$ ve směru osy $x$ a svislého pohybu vzhůru. Získané rovnice
          \begin{equation}\label{mech:eq_delo_rce_pohybu}
            z(t)=v_{0z}t-\frac{1}{2}gt^2, \qquad x(t)=v_{0x}t
          \end{equation}
          představují \emph{parametrické rovnice trajektorie}. Vyloučíme-li z nich čas $t$,
          dostaneme rovnici křivky v kartézských souřadnicích
          \begin{equation}\label{mech:eq_delo_vakuum_param_rce}
            z(x)=  \frac{v_{0z}}{v_{0x}}x-\frac{1}{2}\frac{g}{v_{0x}^2}x^2
                = x\tan\alpha-\frac{1}{2}\frac{g}{v_0^2\cos^2\alpha}x^2
          \end{equation}
          Nyní aplikujeme goniometrický vzorec
          \begin{equation*}
            |cos\alpha|=   \frac{1}{\sqrt{1+\tan^2\alpha}}\Rightarrow \frac{1}{\cos^2\alpha} 
                       = 1+\tan^2\alpha
          \end{equation*}
          odvozený dle náčrtku na obrázku \ref{mech:fig_tan_alpha} a dostáváme rovnici
          \begin{equation}\label{mech_eq_example_vysledna_rce}
            z(x)=x\tan\alpha-\frac{1}{2}\frac{g}{v_0^2}(1+\tan^2\alpha)x^2
          \end{equation}
          Pohyb projektilu (dělové koule) probíhá po stejné trajektorii, jako šikmý vrh v
          homogenním tíhovém poli ve vakuu, tedy po parabole. Snadno dostaneme \emph{souřadnice
          vrcholu dráhy, dálku doletu a celkovou dobu letu}.

          \begin{itemize}
            \item Maximální dolet pro daný elevační úhel:
              \begin{equation}\label{mech:eq_elevacni_uhel_1}
                0=x\tan\alpha-\frac{1}{2}\frac{g}{v_0^2}(1+\tan^2\alpha)x^2
              \end{equation}
              \emph{Netriviální kořen této kvadratické rovnice je námi hledaný dolet dělové koule}
              \begin{equation}\label{mech:eq_elevacni_uhel_2}
                x_d=\frac{2v_0^2\tan\alpha}{g(1+\tan^2\alpha)}(1+\tan^2\alpha)
                   =\frac{2v_0^2\sin\alpha\cos\alpha}{g}=\frac{v_0^2\sin{2\alpha}}{g}
              \end{equation}

            \item Celková doba letu:
              \begin{equation}\label{mech:eq_doba_letu}
                t_d=\frac{x_d}{v_{0x}} =\frac{2v_0^2\sin\alpha\cos\alpha}{gv_0\cos\alpha}
                   =\frac{2v_0\sin\alpha}{g}
              \end{equation}

            \item Souřadnice vrcholu dráhy: \emph{získáme derivováním rov.
                  \ref{mech_eq_example_vysledna_rce}}
                  \begin{align}
                    0   &= tan\alpha-\frac{g}{v_0^2(1+\tan^2\alpha)}x_v                         \\
                    x_v &= \frac{v_0^2}{g}\frac{\tan\alpha}{1+\tan^2\alpha}=
                           \frac{v_0^2}{g}\frac{\sin\alpha}{\cos\alpha}
                           \cdot\cos^2\alpha\cdot\frac{2}{2}                                   \\
                    x_v &= \frac{v_0^2\sin{2\alpha}}{2g}
                   \end{align}
                   \emph{Souřadnici $z_v$ dostaneme dosazením $x_v$  do rov.
                   \ref{mech_eq_example_vysledna_rce}}
                   \begin{align}
                     z_v &= \frac{v_0^2}{g}\frac{\tan^2\alpha}{1+\tan^2\alpha}-
                            \frac{1}{2}\frac{g}{v_0^2}(1+\tan^2\alpha)\frac{v_0^4}{g^2}
                            \frac{\tan^2\alpha}{(1+\tan^2\alpha)^2}                            \\
                     z_v &= \frac{v_0^2}{g}\frac{\tan^2\alpha}{1+\tan^2\alpha}-
                            \frac{1}{2}\frac{v_0^2}{g}\frac{\tan^2\alpha}{1+\tan^2\alpha}      \\
                     z_v &= \frac{v_0^2\sin^2\alpha}{2g}
              \end{align}
              \emph{Odtud je zřejmé, že maximální délka doletu odpovídá úhlu $\frac{\pi}{4}$ a že
              obecně daného bodu doletu lze dosáhnout pod dvěma různými úhly
              $\frac{\pi}{4}\pm\Delta\alpha$.}

            \item Stanovení elevačního úhlu pro zasažení zadaných souřadnic $[X_c, Z_c]$ cíle:
              \emph{Opět vycházíme z rov. \ref{mech_eq_example_vysledna_rce}, ovšem tentokrát
              nejsou neznáme $x$ a $z$, ale $\alpha$: Použijeme substituci $\tan\alpha=p$ a
              vypočítáme kořeny této kvadratické rovnice:}
              \begin{align}
                0       &= gx^2p^2-2v_0^2xp+(gx^2+2zv_0^2) \\
                p_{1,2} &= \frac{v_0^2\pm\sqrt{v_0^4-g(gx^2+2zv_0^2)}}{gx} \\
                \alpha  &= \tan^{-1}\left(\frac{v_0^2\pm\sqrt{v_0^4-g(gx^2+2zv_0^2)}}{gx}\right)
              \end{align}
              \emph{Je-li cíl zadán v polárních souřadnicích $[r,\varphi]$, lze potřebný náměr
              stanovit takto:}
              \begin{equation}\label{mech:eq_namer}
                \alpha=\tan^{-1}\left(\frac{v_0^2\pm
                       \sqrt{v_0^4-g(gr^2\cos^2\varphi+2r\sin\varphi
                             v_0^2)}}{gr\cos\varphi}\right)
              \end{equation}
              \emph{Pokud ovšem bude diskriminant menší než 0, leží cíl mimo dosah děla. Tj.
               neexistuje takový náměr děla, kterým by bylo možné cíl zasáhnout. Je-li
               diskriminant roven nule, jedná se o hranici, za kterou již při dané úsťové
               rychlosti nelze dostřelit. Body ležící na této obálce tzv. ochranná parabola mohou
               být zasaženy pouze při jedné hodnotě elevačního úhlu.}

            \item Stanovení rovnice ochranné paraboly:
              \emph{To provedeme tak, že položíme diskriminant rovnice pro $\tan\alpha$ roven
               nule a dostaneme rovnici obálky}
              \begin{equation}\label{mech:eq_ochr_parabola}
                v_0^4-g(gx^2+2zv_0^2)\Rightarrow z=-\frac{v_0^2}{2g^2}x^2+\frac{v_0^2}{2g}
              \end{equation}

          \end{itemize}

          %------------------------------Dělo---------------------------------
          \lstinputlisting{../src/MECH/img/kinematika_delo_ve_vakuu.m}
          \begin{lstlisting}[caption=\texttt{kinematika\_delo\_ve\_vakuu.m} pro ověření výpočtu balistické 
          dráhy projektilu.]
          \end{lstlisting}
          %-------------------------------------------------------------------
          \begin{figure}[ht!]
            \centering
            \includegraphics[width=\linewidth]{kinematika_delo_vakuum_matlab.pdf}
            \caption[Výpočet trajektorie projektilu]{Výpočet trajektorie projektilu ve vakuu při
                     úsťové rychlosti $210 m/s$ pomocí sw
                     MATLAB\textsuperscript{\textregistered}.}
            \label{mech:fig_delo_matlab}
          \end{figure}
        \end{example}
        \newpage
        \begin{example}
          Kolo vagónu se valí po vodorovné kolejnici. Uvažujte bod, který je v počátečním okamžiku
          pod středem kola ve vzdálenosti, která může být menší, rovna nebo větší než vzdálenost
          středu kola od kolejnice.
          \begin{figure}[ht!]
            \centering
            \includegraphics[scale=1]{trajectory_wheel_carriage.pdf}
            \caption{Kolo vagónu a tři možné polohy bodu}
            \label{mech:fig_wheel_1}
          \end{figure}
          \newline
          Určete  parametrické rovnice dráhy zvoleného bodu, složky rychlosti a její velikost,
          složky zrychlení a jeho velikost, tečné a normálové zrychlení a poloměr křivosti dráhy.
          \cite[p.~11]{Slavik}
          \newline
          \textbf{Řešení}: Obvodová rychlost v místě dotyku s kolejnicí je $v=\omega R$, což
          vzhledem k předpokladu o valení představuje posuvnou rychlost kola. Parametrické
          rovnice pro střed kola jsou pak
          \begin{align}\label{mech:eq_wheel_center}
            x_S &= \omega R t \\
            y_S &= R
          \end{align}
          Uvažovaný bod $B_3$ na obr. \ref{mech:fig_wheel_xy} je ve své nové pozici v čase $t_1$
          posunut vůči středu o vzdálenost $r\cdot\sin\omega t$ ve směru osy $x$ a o vzdálenost
          $r\cdot\cos\omega t$ ve směru osy $y$. Z obrázku \ref{mech:fig_wheel_xy} lze odvodit
          následující rovnice pro souřadnice libovolného bodu \emph{B} na kole vagónu.
          \begin{figure}[ht!]
            \centering
            \includegraphics[scale=1.2]{trajectory_wheel_carriage_solve.pdf}
            \caption[Náčrt pro odvození rovnic pohybu bodu na kole vagónu]{Náčrt pro odvození
                     parametrických rovnic pohybu libovolně zvoleného bodu na kole vagónu}
            \label{mech:fig_wheel_xy}
          \end{figure}
                   
          \begin{minipage}[t]{0.5\textwidth}% first column
            \begin{itemize}
              \item ve směru osy x:                                        
              \begin{align*}
                x &= x_S + x'                        \\
                x &= x_S + r\cos(\psi-\frac{\pi}{2}) \\
                x &= x_S + r\sin\psi                 \\
                x &= \omega R t + r\sin\omega t
              \end{align*}
              \end{itemize}  
          \end{minipage}%second column           
          \begin{minipage}[t]{0.5\textwidth}
           \begin{itemize}
              \item ve směru osy y: 
              \begin{align*}
                y &= y_S - y'                        \\
                y &= y_S - r\sin(\psi-\frac{\pi}{2}) \\
                y &= y_S + r\cos\psi                 \\
                y &= R + r\cos\omega t
              \end{align*}
            \end{itemize}                 
          \end{minipage}
          % VERY important here is that there is no space between the /end{minipage} and the next
          %\begin{minipage} (obviously not counting comments). Otherwise LaTeX will not render the
          % columns side by side. 
          takže, parametrické rovnice dráhy mají tvar \textbf{cykloidy} viz
          \ref{mech_eq_cykloida}.
          \begin{align}\label{mech_eq_cykloida}
            x &= \omega R t + r\sin\omega t\\
            y &= R + r\cos\omega t
          \end{align}
          \begin{itemize}
            \item Složky rychlosti:
              \begin{align}\label{mech_eq_cykloida_v}
               v_x &= \frac{dx}{dt} = \omega R + r\omega\cos\omega t \nonumber \\
               v_y &= \frac{dy}{dt} = -r\omega\sin\omega t           \nonumber \\
               v   &= \sqrt{v_x^2 + v_y^2}= \omega\sqrt{R^2 + 2Rr\cos\omega t + r^2}
              \end{align}
            \item Složky zrychlení:
              \begin{align}\label{mech_eq_cykloida_a}
                a_x &= \frac{dv_x}{dt} = -r\omega^2\sin\omega t \\
                a_y &= \frac{dv_y}{dt} = -r\omega^2\cos\omega t \\
                a   &= \sqrt{a_x^2 + a_y^2}= r\omega^2\sqrt{\sin^2\omega t + 
                       \cos^2\omega t} = r\omega^2
              \end{align}
              Tento výsledek je superpozicí rovnoměrného kruhového a rovnoměrného přímočarého
              pohybu.
            \item Tečné zrychlení dostaneme derivací velikosti rychlosti
              \begin{align}\label{mech_eq_cykloida_at}
                a_t &= \frac{dv}{dt} = \omega\cdot\frac{1}{2\sqrt{R^2 + 2Rr\cos\omega t + r^2}}
                       \cdot(-1)\cdot2Rr\omega\sin\omega t  \\
                a_t &= \frac{r\omega^2\cdot|R\cos\omega t-r|}{\sqrt{R^2 + 2Rr\cos\omega t + r^2}}
              \end{align}
            \item Normálové zrychlení získáme užitím Pythagorovy věty
              \begin{align*}
                a_n &= \sqrt{a^2 - a_t^2} \\
                a_n &= \sqrt{(r\omega^2)^2-\left(
                       \frac{Rr\omega^2\sin\omega t}
                            {\sqrt{R^2 + 2Rr\cos\omega t + r^2}}\right)^2}                      \\
                a_n &= \frac{r\omega^2|R\cos\omega t - r|}{\sqrt{R^2 + 2Rr\cos\omega t + r^2}}
              \end{align*}
            \item Poloměr křivosti $R_0$ dostaneme ze vztahu $a_n=\frac{v^2}{R_0}$:
              \begin{align*}\label{mech:eq_cykloida_R0}
                  R_0 &=  \cfrac{\omega^2(R^2 + 2Rr\cos\omega t + r^2)}
                         {\cfrac{r\omega^2|R\cos\omega t - r|}
                         {\sqrt{R^2 + 2Rr\cos\omega t + r^2}}}                                  \\
                  R_0 &=  \frac{(R^2 + 2Rr\cos\omega t + r^2)^
                         {\frac{3}{2}}}{|Rr\cos\omega t - r^2|}
              \end{align*}
              Poloměr křivosti není roven vzdálenosti od středu kola r: drahou bodu není
              kružnice, nýbrž cykloida (viz obr. \ref{mech:fig_wheel_cycloid}).
          \end{itemize}
          \begin{align*}
            (x - \omega R t)^2            &= r^2\sin^2\omega t                          \\
            (y-R)^2                       &= r^2\cos\omega t                            \\
            (x - \omega R t)^2 + (y-R)^2  &= r^2\sin^2\omega t + r^2\cos\omega t        \\
            (x - \omega R t)^2 + (y-R)^2  &= r^2 \quad \text{kde}\; t = 
                                             \frac{1}{\omega}\arccos\frac{y-R}{r}       \\
            \left(x - R\arccos\frac{y-R}{r}\right)^2 + (y-R)^2  &= r^2
          \end{align*}
          \begin{figure}[ht!]
            \centering
            \includegraphics[width=0.9\linewidth]{trajectory_wheel_cycloid.pdf}
            \caption[Cykloida]{Cykloida: pro $B_2\ldots r=R$ je cykloida prostá; $B_3\ldots r>R$
                               cykloida prodloužená; $B_1\ldots r<R$ cykloida zkrácená;
                               \texttt{[cykloida.m]} }
            \label{mech:fig_wheel_cycloid}
          \end{figure}
        \end{example}
          
    \subsubsection{Skládání harmonických pohybů v kolmých směrech}
      Zmíníme se ještě o skládání \textbf{harmonických pohybů v kolmých smě\-rech}. Sklá\-dá\-me-li dva 
      takové pohyby o stejné úhlové frekvenci, bude výsledný pohyb probíhat po trajektorii dané parametrický 
      jako
      \begin{equation}\label{mech:eq_lissaujous1}
          x=A\sin(\omega t+\varphi_{01}),\qquad y=B\sin(\omega t +\varphi_{02})
      \end{equation}
      Výsledný pohyb vytváří zajímavé geometrické tvary známé pod názvem Lissajousovy obrazce. Jejich vzhled 
      závisí na poměru frekvencí a na fázovém úhlu \cite{Okrouhlik}.

      Označíme fázi kmitů ve směru $x$ jako $\omega t+\varphi_{01} = \varphi$, rozdíl fází obou kmitů jako 
      $\varphi_{02}-\varphi_{01} =\delta$. Dále vyloučíme z parametrických rovnic čas. K tomu cíli vyjádříme 
      $sin\varphi$ a $cos\varphi$ pomocí veličin na čase nezávisejících a použijeme známý vztah $sin^2\varphi 
      + cos^2\varphi = 1$. Máme
      \begin{equation}\label{mech:eq_lissajous2}
          \sin\varphi=\frac{x}{A}, \qquad 
          \sin(\varphi+\delta)=\sin\varphi\cos\delta+\cos\varphi\sin\delta=\frac{y}{B}
      \end{equation}
      odkud
      \begin{equation}\label{mech:eq_lissajous3}
          \cos\varphi=\frac{1}{\sin\delta}\left(\frac{y}{B}-\frac{x}{A}\cos\delta\right)
      \end{equation}
      Sečteme-li nyní $sin^2\varphi$ a $cos^2\varphi$, dostaneme rovnici trajektorie
      \begin{equation}\label{mech:eq_lissajous4}
          \frac{x^2}{A^2}+\frac{y^2}{B^2}-\frac{2xy}{AB}\cos\delta=\sin^2\delta
      \end{equation}
      V závislosti na $\delta$ může tato rovnice odpovídat rovnici \emph{úsečky}, nebo \emph{elipsy}. Je-li 
      $\delta = n\pi$, probíhají kmity po úsečce, jejíž přímka má směrnici $k = \pm\frac{B}{A}$, je-li 
      $\delta = \left(n + \frac{1}{2}\right)\pi$, je trajektorií
      elipsa
      \begin{equation}\label{mech:eq_lissajous5}
          \frac{x^2}{A^2}+\frac{y^2}{B^2}=1
      \end{equation}

      Jsou-li amplitudy obou pohybů stejné, přejde pro $\delta = \left(n+\frac{1}{2}\right)\pi$ elipsa v 
      kružnici. S uvedeným skládáním dvou kolmých pohybů o stejných frekvencích se setkáváme nejen v 
      mechanice, ale například i v elektromagnetismu a optice při studiu polarizace světla. Výsledné 
      trajektorie získané pomocí počítače jsou na obr. \ref{mech:fig_lissajous_1} a obr. 
      \ref{mech:fig_lissajous_2} \cite{Stoll}.

      \begin{figure*}
        \centering
        \subfloat[$A=B$ a $\omega_1=\omega_2$.]{\label{mech:fig_lissajous_1}
           \includegraphics[width=0.4\textwidth]{lissajous_1.pdf}}
        \subfloat[$A=B$ a $\frac{\omega_1}{\omega_2}=\frac{2}{3}$.]{\label{mech:fig_lissajous_2}
           \includegraphics[width=0.6\textwidth]{lissajous_2.pdf}}
        \caption[Skládání harm. pohybů v kolmých směrech]{Trajektorie harmonických pohybů
                 $x=A\sin(\omega_1 t)$ a $y=B\sin(\omega_2 t+\varphi)$ v kolmých směrech}
        \label{mech:fig_lissajous_obrazce}
      \end{figure*}

      Jsou-li úhlové frekvence kolmých pohybů různé, vznikají složité tzv. \textbf{Lissajousovy obrazce} viz 
      \ref{mech:fig_lissajous_2}. Program ukazuje, jak se projevuje změna fázového úhlu při daném poměru 
      frekvencí obou pohybů.

      %---------------------------------------------------------------
        \lstinputlisting{../src/MECH/img/Lissajous.m}
        \begin{lstlisting}[caption=\texttt{Lissajous.m}vykreslí skládání harmonických pohybů v kolmých 
        směrech.]
        \end{lstlisting}
      %---------------------------------------------------------------

%=========================== Kapitola 51: Dynamika částice ========================================
%  \input{../src/FYZ//chap/dynamika.tex} 
%=========================== Kapitola 52: Historie Astrofyziky ====================================
  %========================================= Kapitola: Astrofyzika =============================================
\chapter{Úvod}
\minitoc
\newpage
  \wikiAstrofyzika  je vědní obor ležící na rozhraní \emph{fyziky} a \emph{astronomie}. Zabývá se fyzikou 
  vesmíru, včetně fyzikálních vlastností (svítivost, hustota, teplota, chemické složení) astronomických 
  objektů jako jsou hvězdy, galaxie a mezihvězdná hmota, jakož i jejich vzájemné působení.
  
  Podle metod výzkumu těchto objektů se dělí na \emph{fotometrii}, \emph{spektroskopii},
  \emph{radioastronomii}, \emph{astrofyziku rentgenovou}, \emph{infračervenou}, \emph{ultrafialovou} a 
  \emph{neutrinovou}. Každý z těchto podoborů se dále dělí na praktickou a teoretickou část. Praktická 
  získává potřebná data. Teoretická s pomocí fyzikálních zákonů vysvětluje pozorované cho\-vá\-ní vesmírných 
  těles.
  
  \section{Historie astrofyziky}
  
  \section{Základní vztahy}
    \begin{itemize}
      \item \wikiAU - \emph{astronomická jednotka}: průměrná vzdálenost Země od Slunce, $150\cdot10^6\ km$. 
      Vzájemné vzdálenosti planet či jiných objektů sluneční soustavy vy\-já\-dře\-né v AU poskytují 
      relativně názorné měřítko vzdáleností těchto objektů od sebe. Přesná hodnota je $$1 AU = 149\ 597\ 870\ 
      691 \pm 6\ m$$ Kvůli vyšší přesnosti \emph{Mezinárodní astronomická unie} (International Astronomical 
      Union, IAU) přijala novou de\-fi\-ni\-ci, podle které je AU délka poloměru nerušené oběžné kruhové 
      dráhy tělesa se zanedbatelnou hmotností, pohybujícího se okolo Slunce rychlostí $0,017\ 202\ 098\ 950$ 
      radiánů za den ($86\ 400\ s$). 
        \begin{itemize}
          \item Vzdálenost Země od Slunce je $1,00 ± 0,02\ AU$.
          \item Měsíc obíhá kolem Země ve vzdálenosti $0,0026 \pm 0,0001\ AU$.
          \item Mars je od Slunce vzdálen $1,52 \pm 0,14 AU$.
          \item Jupiter je od Slunce vzdálen $5,20 \pm 0,05\ AU$.
          \item Nejvzdálenější člověkem vyrobené těleso, sonda Voyager 1, bylo 31. prosince 2007 ve    
                vzdálenosti $104,93\ AU$ od Slunce.
          \item Průměr sluneční soustavy bez \emph{Oortova oblaku} je přibližně $105\ AU$.
          \item Průměr sluneční soustavy s Oortovým oblakem se odhaduje na $50\ 000$ až $100\ 000\ AU$.
          \item Nejbližší hvězda (po Slunci), Proxima Centauri, se nachází přibližně ve vzdálenosti 
                $268\ 000\ AU$.
          \item Průměr hvězdy Betelgeuze je $2,57\ AU$.
          \item Vzdálenost Slunce od středu Galaxie je přibližně $1,7\cdot10^9\ AU$.
          \item Velikost viditelného vesmíru je asi $8,66\cdot10^{14}\ AU$.
        \end{itemize}
      \item \textbf{l.y.} - \emph{světelný rok}: vzdálenost, kterou světlo ulétna za jeden rok, 
            $9,46\cdot10^{12}\ km$,
      \item \textbf{pc} - \emph{parsek, paralaktická sekunda}: vzdálenost, ze které by poloměr oběžné dráhy   
            Země byl kolmo k zornému paprsku vidět pod úhlem $1''$, $30,9\cdot10^{12}$ km. 
    \end{itemize}
    
    \begin{example} 
      Spočtěte, jakou vzdálenost v metrech vyjadřuje jeden parsek \protect\cite[s.~3]{Kulhanek2009}.
      
      \textbf{řešení}: $1\ pc$ (paralaktická sekunda) je vzdálenost, ze které vidíme velkou poloosu oběžné 
      dráhy Země kolem Slunce pod uhlem $\varphi = 1''$. Úhel $1''$ je tak malý, že strany $VS$ a $VZ$ na 
      obrázku prakticky splývají a místo pravého trojúhelníka $VSZ$ můžeme použít definiční vztah úhlu v 
      obloukové míře (\emph{velkost úhlu je možné určit jako poměr délky oblouku vymezeného rameny na 
      kružnici opsané kolem vrcholu k poloměru této kružnice}). Proto $$\varphi = \frac{R_{SZ}}{l} 
      \rightarrow l = \frac{R_{SZ}}{\varphi},$$ 

      \begin{figure}[ht!]
          \centering
          \includegraphics[width=0.8\linewidth]{ex_parsek.pdf}
          \caption[Parsek]{Parsek}
          \label{afyz:fig_ex_parsek}
      \end{figure}
      
      kde $l$ je vzdálenost $1\ pc$ v metrech, $R_{SZ}$ je vzdálenost země od Slunkce a $\varphi$ je úhel jedné vteřiny vyjádřený v radiánech. 
      $$l = \frac{1,5\cdot10^{11}\ m}{\frac{1}{60\cdot60}\cdot\frac{2\pi}{360}}\cong 3\cdot10^{16}\ m.$$  
      
    \end{example}
    
   Další jednotkou, kterou se v astrofyzice měří vzdálenost dvou vesmírných těles, je \emph{paralaxa}. 
   Pozorovací místa musí být od sebe výrazně vzdálena, aby například při měření vzdálenosti naší nejbližší 
   hvězdy - \emph{Proxima Centauri} byla paralaxa vůbec měřitelná. Vzdálenost této hvězdy je 4,2 světelných 
   let (nebo 270 000 AU) od Země.
   
    \begin{example}
      Najděte paralaxu Proximy Centauri, která je od nás vzdálená asi 4,2 světelného roku \protect\cite[s.~4]{Kulhanek2009}.
      
      \begin{figure}[ht!]
          \centering
          \includegraphics[width=\linewidth]{ex_Proxima_Centauri_paralaxa.pdf}
          \caption[Paralaxa naší nejbližší hvězdy]{Paralaxa naší nejbližší hvězdy}
          \label{afyz:fig_ex_paralaxa}
      \end{figure}
            
      \textbf{Řešení}: Díky pohybu Země kolem Slunce se zdá, že blízké hvězdy opisují oproti vzdáleným 
      elipsu. Úhlový poloměr této elipsy se nazývá paralaxa hvězdy. Lze ji změřit jen pro nejbližší hvězdy. Z 
      definice úhlu (jako v předchozím příkladě) tedy vyplývá, že
      $$\pi = \frac{R_{ZS}}{l} = \frac{1,5\cdot10^{11}\ m}{4,2 l.y} = \frac{1,5\cdot10^{11}\ 
      m}{4,2\cdot9,5\cdot10^{15}\ m}	\cong 3,7\cdot10^{-6}\ rad,$$ což je přibližně $0.76''$. Vidíme, že i u 
      druhé nejbližší hvězdy po Slunci není paralaxa ani celá $1''$.
    \end{example}

\printbibliography[heading=subbibliography]
}    % DEBUG was off


  \part{Fyzika II}\label{part:FYZII}
  \parttoc

\iftoggle{DEBUG}{
%  DEBUG was on
%  \input{../src/FYZ/chap/fey2ch01.tex}
%  \input{../src/FYZ/chap/fey2ch02.tex}
  \input{../src/FYZ/chap/fey2ch03.tex} 
%  \input{../src/FYZ/chap/fey2ch04.tex}
%  \input{../src/FYZ/chap/fey2ch05_06_07.tex}
%  \input{../src/FYZ/chap/fey2ch08.tex}
%  \input{../src/FYZ/chap/fey2ch13.tex}
}
{
% DEBUG was off
%=========================== Kapitola: Elektromagnetizmus==========================================
  \input{../src/FYZ/chap/fey2ch01.tex} 
%=========================== Kapitola: Diferenciální počet vektorových polí =======================
  \input{../src/FYZ/chap/fey2ch02.tex} 
%=========================== Kapitola: Integrální počet vektorových polí ==========================
  \input{../src/FYZ/chap/fey2ch03.tex} 
%=========================== Kapitola: Elektrostatika =============================================
  \input{../src/FYZ/chap/fey2ch04.tex}
%=========================== Kapitola: Elektrostatika v různých případech =========================
  \input{../src/FYZ/chap/fey2ch05_06_07.tex}
%=========================== Kapitola: Elektrostatická energie ====================================
  \input{../src/FYZ/chap/fey2ch08.tex}
%=========================== Kapitola: Magnetostatika =============================================
  \input{../src/FYZ/chap/fey2ch13.tex}
}