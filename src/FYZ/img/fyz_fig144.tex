% !TeX spellcheck = cs_CZ

% note:
% ~~~~~~
% This figure uses expl3 package for coordinate calculating !!!
%\usepackage{expl3}                     % fpeval
%\ExplSyntaxOn
%\cs_set_eq:NN \fpeval \fp_eval:n
%\ExplSyntaxOff
%-------------------------------------------------------------------------------------------------
\def\xmin{0}
\def\xmax{8.5}
\def\ymin{-0.75}
\def\ymax{+2}
\def\zero{0}
\def\R{0.5}
\def\r{0.5}
\def\omega{4/3.14}
\def\angle{360}

{\centering
 \captionsetup{type=figure}
\begin{animateinline}
 [autoplay,
  palindrome,
  begin={
    \begin{tikzpicture}
      \useasboundingbox (\xmin,\ymin) rectangle (\xmax,\ymax);

      \draw [step=1cm, black!40!white, very thin] (\xmin cm,\ymin cm) grid (\xmax cm,\ymax cm);
      \draw [red, thick,  domain=0:6*pi, samples=100] 
        plot ({\omega*\R*\x + \r*sin(deg(\omega*\x))}, {\R+\r*cos(deg(\omega*\x))} );
        
      \draw [Mahogany, thick,  domain=0:6*pi, samples=100] 
        plot ({\omega*\R*\x + 1.5*\r*sin(deg(\omega*\x))}, {\R+1.5*\r*cos(deg(\omega*\x))} );
        
      \draw [orange, thick,  domain=0:6*pi, samples=100] 
        plot ({\omega*\R*\x + 0.4*\r*sin(deg(\omega*\x))}, {\R+0.4*\r*cos(deg(\omega*\x))} );
        
      \draw[black, thick] (\xmin cm,\ymin cm) rectangle (\xmax cm,\ymax cm);
    
      \draw [Sepia,line width=0.5mm] (\xmin,0) -- (\xmax,0);
        \foreach \x/\xtext in {1/1,2/2,3/3,4/4,5/5,6/6,7/7,8/8}
          \draw (\x cm,-3pt) -- (\x cm,2pt) node[below, yshift=-1mm, fill=white] {\xtext};
        },
  end={\end{tikzpicture}}
]{5}
  \multiframe{40}{nstep=0.1+20}{% 
    % wheel
    \filldraw[fill=black!40!white, draw=black, opacity=0.2] 
      (\nstep/57.295/3.14*2, \R cm) circle (\R cm);
    %point 1
    \filldraw[fill=red!40!white, draw=black] (\fpeval{\omega*\R*\nstep/57.295 + 
       1.0*\r*sin(deg(\omega*\nstep))},\fpeval{\R+1.0*\r*cos(deg(\omega*\nstep))}) circle (0.5 mm);
    %point 2
    \filldraw[fill=red!40!white, draw=black] (\fpeval{\omega*\R*\nstep/57.295 + 
       1.5*\r*sin(deg(\omega*\nstep))},\fpeval{\R+1.5*\r*cos(deg(\omega*\nstep))}) circle (0.5 mm);
    %point 3
    \filldraw[fill=red!40!white, draw=black] (\fpeval{\omega*\R*\nstep/57.295 + 
       0.4*\r*sin(deg(\omega*\nstep))},\fpeval{\R+0.4*\r*cos(deg(\omega*\nstep))}) circle (0.5 mm);
}
\end{animateinline}
\captionof{figure}{Animace parametrických rovnic pohybu libovolně zvoleného bodu na kole}
\label{fyz:fig144}
\par}