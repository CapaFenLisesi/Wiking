\chapter{Základy fyziky}
\minitoc
\newpage
  \section{Úvod}
    V této kapitole budeme zkoumat nejzákladnější myšlenky, s nimiž se ve fyzice setkáváme - budeme hovořit o 
    tom, jak si v současnosti představujeme povahu věcí. Nebudeme však hovořit o tom, jak se poznala 
    správnost těchto představ - o těchto detailech se dozvíte v pravý čas.
    
    Věci, o něž se v naší vědě zajímáme, se nám ukazují množstvím projevů a atributů. Stojíme-li například na 
    břehu a hledíme na moře, vidíme vodu, na vodě pěnu, nad mořem oblaka, slunce, modrou oblohu a vůbec 
    světlo, slyšíme zvuk, nárazy vln, svištění větru, cítíme vzduch. Na břehu je písek a skály, a každá má 
    jinou tvrdost a pevnost, barvu a složení. Jsou tam zvířata a vodní tráva, je tam hlad i nemoc a na břehu 
    je pozorovatel se svými myšlenkami a snad i štěstím. Každé jiné místo v přírodě se vyznačuje podobnou 
    pestrostí věcí a vlivů, podobnou složitostí. Naše zvědavost nás nutí klást otázky, hledat souvislosti a 
    chápat mnohotvárnost věcí jako následek snad relativně malého počtu nejjednodušších věcí a sil působících 
    nekonečně rozmanitě.
    
    Klademe si otázku: Je písek jiný než skály? Není snad písek nic jiného, než velký počet velmi malých 
    kamínků? Je Měsíc velká skála? Kdybychom porozuměli tomu, co jsou skály, znamená to, že bychom pochopili 
    i podstatu písku a Měsíce? Co je to vítr? Jsou to nárazy vzduchu podobné nárazům vody na břeh? Jaké 
    společné rysy mají rozličné druhy pohybu? Co mají společného různé druhy zvuku? Kolik různých barev 
    existuje? A tak dále. Takovým způsobem se snažíme postupně analyzovat všechny věci. Dáváme do souvislostí 
    věci, které na první pohled vzájemně nesouvisí. Děláme to s nadějí, že se nám podaří redukovat počet 
    rozličných věcí a tak je lépe poznat.
    
    Před několika sty lety vznikla metoda hledání částečných odpovědí na uvedené otázky. \emph{Pozorování, 
    usuzování a experiment} vytvářejí to, co nazýváme \emph{vědeckou metodou}. Budeme se muset omezit jen na 
    holý popis našich představ o tom, co se nazývá z\emph{základní fyzikou} nebo základními myšlenkami, které 
    vznikly aplikováním vědecké metody.
    
    Co to znamená něco „pochopit“? Můžeme si představit, že to složité nahromadění pohybujících se věcí, 
    které vytvářejí „svět“, je šachová hra bohů a my vystupujeme jako diváci, kteří neznají pravidla hry, ale 
    je jim dovoleno hru \emph{pozorovat}. Samozřejmě, pozorujeme-li dostatečně dlouho, můžeme nakonec 
    pochytit několik pravidel. \emph{Pravidla hry} představují to, co chápeme jako \emph{základní fyziku}. I 
    kdybychom znali všechna pravidla, nemuseli bychom ještě rozumět každému kroku hry, protože je příliš 
    složitá a možnosti našeho rozumu omezené. Hrajete-li šachy, jistě víte, že je jednoduché naučit se 
    všechna pravidla, ale i tak je velmi těžké zvolit ten správný tah nebo pochopit záměry protihráče. Stejné 
    je to i s přírodou, jen mnohem těžší. Máme však možnost najít alespoň všechna pravidla. Zatím je všechna 
    neznáme. (Každou chvíli se objevuje něco takového jako rošáda, kterou ještě neznáme.) Nejen, že neznáme 
    všechna pravidla, ale pomocí těch, která známe, umíme jen velmi málo vysvětlit. Je tomu tak proto, že 
    téměř všechny situace jsou ohromně složité a známá pravidla nám neumožní sledovat všechny obraty hry, 
    nemluvě o předvídání dalších kroků. Musíme se proto omezit na základnější otázku pravidel hry. Naučíme-li 
    se pravidla, budeme to považovat za „pochopení“ světa.
    
    Jak můžeme rozhodnout, zda pravidla, která vlastně jen „odhadujeme“, jsou skutečně správná, když 
    nemůžeme dokonale analyzovat hru? Existují zhruba tři způsoby. Především nám příroda může poskytnout 
    (nebo my si od přírody vynutíme) jednoduché situace skládající se z malého počtu částí, umožňující 
    přesnou předpověď budoucího dění, a tím i zkoušku pravidel. (V rohu šachovnice zůstalo jen málo figurek, 
    jejichž tahy již umíme přesně určit)
    
    Druhý způsob zkoušky pravidel spočívá v jejich použití k odvození obecnějších pravidel. Například, 
    střelec se na šachovnici pohybuje úhlopříčně. Odtud je možné usuzovat na skutečnost, že určitý střelec 
    bude vždy na bílém poli. Odhlédneme-li od podrobností, můžeme prověřovat naše pravidlo o pohybu 
    uvedeného střelce tak, že sledujeme, jestli se vždy nachází na bílém poli. Po dlouhém čase se samozřejmě 
    může stát, že se náhle objeví na černém poli (v průběhu hry byl vzat, ale jeden pěšec došel na konec 
    šachovnice a proměnil se na střelce na černém poli). Tak to bývá i ve fyzice. Dlouho používáme pravidlo, 
    které ve všech směrech dobře vyhovuje, ačkoliv neznáme detaily, a potom najednou objevíme \emph{nové 
    pravidlo}. Z hlediska základů fyziky probíhají nejzajímavější jevy na nových místech, na místech, kde 
    pravidla neplatí a ne tam, kde pravidla \emph{platí}. To je způsob, jakým objevujeme nová pravidla.
    
    Třetí ze způsobů, kterými se můžeme přesvědčit o správnosti našich myšlenek, je poměrně hrubý, ale snad 
    nejúčinnější. Je to způsob přibližného odhadu. Ačkoliv nejsme schopni říci, proč Aljechin \emph{táhl 
    právě tou figurkou}, můžeme v \emph{hrubých rysech} chápat, že seskupuje figurky okolo krále, aby ho 
    chránil, protože za daných okolností je to nejrozumnější. Podobně je to i s naším chápáním přírody. Často 
    ji více či méně chápeme, aniž bychom byli schopni znát význam tahu \emph{každé jednotlivé figurky}.
    
    Zpočátku se přírodní jevy hrubě rozdělovaly do tříd jako teplo, elektřina, mechanika, magnetizmus, 
    vlastnosti látek, chemické děje, světlo nebo optika, rentgenové paprsky, jaderná fyzika, gravitace, 
    mezonové jevy atd. Cílem je však pochopení \emph{celé přírody} jako různých aspektů \emph{jednoho 
    souboru} jevů. Úkolem základní teoretické fyziky dneška je \emph{nalezení zákonů stojících za 
    experimentem a sjednocení uvedených tříd}. Historicky se nám vždy podařilo sloučit je, ale postupem času 
    se objevovaly nové věci. Když jsme si již vytvořili ucelenou představu, objevily se najednou rentgenové 
    paprsky. Když se i tento jev dostal do jednotného schématu, objevily se mezony. Proto v každém stádiu hry 
    vypadá situace dost chaoticky. Mnohé se objasnilo z jednotného hlediska, ale ještě stále je mnoho volných 
    konců nitek, o nichž nevíme, kam patří. Takový je dnes stav věcí a my se ho pokusíme popsat.
    
    Všimněme si v historii několika příkladů uvedeného sjednocování. Uvažujme nejdříve \emph{teplo a 
    mechaniku}. Jsou-li atomy v pohybu, obsahuje systém tím více tepla, čím více pohybu v něm je, takže 
    \emph{teplo a všechny tepelné efekty je možné vyjádřit pomocí zákonů mechaniky}. Dalším úžasným 
    sjednocením bylo objevení souvislosti mezi \emph{elektřinou, magnetizmem} a světlem, o nichž se zjistilo, 
    že jsou různými aspekty stejné věci, kterou dnes nazýváme \emph{elektromagnetické pole}. Dále chemické 
    děje, rozmanité vlastnosti různých látek a chování atomových částic byly sjednoceny do \emph{kvantové 
    chemie}.
    
    Zůstává zde však otázka, zda bude možné vše sjednotit tak, abychom mohli prohlásit, že svět představuje 
    rozmanité aspekty jediné věci? To nikdo neví. Víme pouze, že na naší cestě vpřed se nám daří spojovat 
    fragmenty, přičemž vždy nalézáme cosi, co nezapadá do obecného obrazu, a proto se opět pokoušíme doplnit 
    skládačku. Nevíme, zda tato skládačka má konečný počet částí a zda má tato hra vůbec hranice. Dozvíme se 
    to až tehdy, když složíme výsledný obraz, jestli ho vůbec kdy složíme. Chtěli bychom však ukázat, kam až 
    tento proces sjednocování pokročil a jaká je dnešní situace při objasňování základních jevů pomocí co 
    nejmenšího počtu principů. Jednodušeji řečeno: \textbf{z čeho jsou složeny věci a kolik je těch 
    stavebních prvků?}
    
  \section{Hlavní etapy vývoje}
    Fyzika prošla dlouhým historickým vývojem a znalost tohoto vývoje pomáhá lépe pochopit logiku soustavy
    fyzikálních poznatků a dokonce do\-cházet k poznatkům novým. V krátkosti dějiny fyziky můžeme 
    rozdělit na 
    tři hlavní etapy:
    \begin{itemize}
     	\item Stará fyzika - od starověku do počátku 17. století (orientačně do roku 1600).
     \item Klasická fyzika - 1600 – 1900.
     \item Moderní fyzika - 1900 – dosud.
    \end{itemize}
    Starou fyziku nemůžeme považovat za vědu ve vlastním smyslu, i když se dobrala celé řady významných 
    vědeckých poznatku. První z nich znali již staří Sumerové, Babyloňané, Egypťané a Číňané. Šlo zejména 
    o  poznatky astronomické a geometrické (Pythagorova veta) a také o metody měření některých 
    fyzikálních veličin (délka, hmotnost, čas). Fyzika ve starém Řecku byla jako součást filosofie 
    převážně spekulativní a tento charakter si pod vlivem aristotelismu udržela, až do počátku novověku. 
    Skutečný fyzikální výzkum prováděli až helenističtí Řekové, kdy se centrem vědy a kultury antického 
    světa stala Alexandrie. V Alexandrii studoval největší fyzik starověku Archimédes, který dospěl k 
    důležitým poznatkům o statické rovnováze těles a plování těles a v matematice se těsně přiblížil 
    objevu diferenciálního a integrálního počtu. Alexandrijští Řekové znali také zákon odrazu světla 
    (nikoli lomu) a prováděli první měření teploty. Poznatky antiky byly středověké Evropě 
    zprostředkovány Araby, kteří se též intenzivně zabývali optikou (Alhazen) a určováním měrné hmotnosti 
    látek. Zatímco ve středověku byly hlavní přírodovědné poznatky čerpány z Euklidových ”Základu” 
    (geometrie), ”Almagestu” Klaudia Ptolemaia (geocentrický výklad astronomie sluneční soustavy) a spisu 
    Aristotelových (mj. ”Fysika”), vešly práce Archimédovy v Evropě ve známost až teprve začátkem 
    novověku. Ve starověku a středověku však fyzika neprováděla systematické experimenty, nevyužívala 
    matematický aparát k popisu přírodních jevu a neměla ani přesně definovány základní pojmy (rychlost, 
    zrychlení, síla apod.) Zrod fyziky jako vědy se datuje začátkem 17. století. Na základě  
    astronomických výzkumu Keplerových (1571-1630) a pozemských mechanických experimentů Galileových 
    (1564-1642) mohl Isaac Newton (1643-1727) vytvořit první fyzikální teorii, klasickou mechaniku,   
    využívající matematický aparát diferenciálního a integrálního poctu. Newton přišel s koncepcí   
    všeobecné gravitace a ukázal, že není přehrady mezi nebeskou a pozemskou fyzikou, že síla, která    
    udržuje planety na jejich drahách kolem Slunce je táž jako síla, která nutí jablko padat k zemi. 
    Základní Newtonovo dílo z r. l687 nese název ”Matematické základy přírodní filosofie” (”Philosophiae 
    naturalis principia mathematica”) a představuje pravděpodobně nejvýznamnější vědeckou knihu, která 
    byla kdy napsána. Newton se zabýval též optikou a rozpracoval teorii rozkladu bílého světla do 
    spektra. V té době byl již zásluhou Snellovou a Descartovou znám i zákon lomu světla. Z roku 1600 
    pochází první vědecký spis o elektřině a magnetismu od anglického lékaře a fyzika Gilberta. Výzkumem 
    těchto jevu se v následujících stoletích zabývala celá řada fyziků (Coulomb, Volta, Oersted, 
    Amp\`{e}re a další). Tento výzkum pak završil Faraday (1791-1867) svým objevem zákona 
    elektromagnetické indukce a svou koncepcí siločar elektromagnetického pole. Úlohu Newtona 
    elektromagnetismu pak sehrál James Clerk Maxwell (1831-1879), který ve svém ”Traktátě o elektřině a 
    magnetismu” z r. 1873 sestavil slavné Maxwellovy rovnice popisující vlastnosti elektromagnetického 
    pole. Maxwell zároveň teoreticky zdůvodnil elektromagnetickou povahu světla a ukázal, že jevy spojené 
    s vlastnostmi elektrického náboje (”elektřina”), elektrického proudu (”galvanismus”), magnetického 
    pole a světla (optika), jsou jedné a téže elektromagnetické povahy. V devatenáctém století byl tak 
    dovršen výzkum mechanických jevů a elektromagnetismu a klasická fyzika tím za\-vršena. V přírodě tedy 
    existovaly pouze dvě síly, dva způsoby vzájemné interakce mezi částicemi: gravitační a 
    elektromagnetická. Mezi nimi se však projevoval určitý rozpor. Jak Newtonovy tak Maxwellovy rovnice 
    platí v libovolné inerciální vztažné soustavě. Při přechodu od jedné inerciální soustavy k druhé se 
    však Newtonovy rovnice transformují pomocí tzv. Galileiho transformací a Maxwellovy rovnice pomocí 
    Lorentzových transformací. Fyzika se tak rozdvojila, mechanické a elektromagnetické děje se zdály být 
    neslučitelné. Kromě toho existovaly některé experimenty, jejichž výsledek nedokázala klasická fyzika 
    vysvětlit: průběh spektra rovnovážného elektromagnetického záření (tzv. záření absolutně černého 
    tělesa) a pokus Michelsonův, který svědčil o neexistenci světelného éteru. Tyto zdánlivě nepodstatné 
    rozpory vyústily ve 20. století ve vznik moderní fyziky, tj. fyziky kvantové a relativistické. Právě 
    koncem roku 1900 vyslovil Planck tzv. kvantovou hypotézu, jíž vysvětlil záření absolutně černého 
    tělesa, a v r. 1905 publikoval Einstein práci o speciální teorii relativity. V ní překlenul rozpor 
    mezi Newtonovou a Maxwellovou fyzikou a fyziku opět sjednotil. Předpoklad o existenci světelného 
    éteru se teorií relativity stal zbytečným. V roce 1916 vytvořil Einstein i obecnou teorii
    relativity jako moderní teorii gravitace. Gravitační síly podle této teorie souvisejí se zakřivením 
    prostoročasu. Jak speciální, tak obecná teorie relativity přecházejí při rychlostech objektu 
    podstatně menších než je rychlost světla ve vakuu a při slabých gravitačních polích v teorii 
    Newtonovu. Přelom 19. a 20. století je též poznamenán objevem radioaktivity a vznikem jaderné fyziky, 
    která tak významným způsobem zasáhla do života celého lidstva. V jaderné fyzice se uplatní další dvě 
    přírodní síly - tzv. silná, která udržuje nukleony v atomových jádrech a slabá, která se projevuje 
    při radioaktivní přeměně beta za vzniku neutrin. Moderní fyzika odhalila v kosmickém záření a pomocí 
    urychlovačů obrovské množství částic, jejichž vlastnosti studuje a snaží se je utřídit a vysvětlit. 
    Mezi všemi těmito částicemi působí čtyři základní síly přírody: gravitační, elektromagnetická, silná 
    a slabá. V nedávné době se podařilo prokázat, že i elektromagnetická a slabá interakce jsou téže 
    podstaty a tvoří jedinou sílu elektroslabou. V průběhu historie fyziky od Newtona a Maxwella k dnešku 
    tak probíhá úsilí o sjednocování interakcí, které pokračuje i dnes. Fyzika se pokouší prokázat, že i 
    silná a elektroslabá interakce jsou téže povahy, a že k nim konečně přistupuje i síla gravitační. Tím 
    by vznikla idea jediné přírodní síly sjednocující všechny přírodní jevy a děje. Fyzika ovšem nemůže k 
    takovému závěru dojít pouhým uvažováním, musí matematicky vypracovat a zdůvodnit příslušnou teorii a 
    její závěry experimentálně ověřit. To vede ke snaze budovat stále větší a větší urychlovače a také k 
    intenzivnímu výzkumu jevů v kosmu. Sjednocování interakcí má totiž těsnou návaznost na vývoj vesmíru 
    podle hypotézy o tzv. ”velkém tresku”. Právě v počátcích vývoje vesmíru by se měly všechny čtyři 
    (resp. tři) interakce uplatňovat rovnocenným způsobem a teprve v průběhu dalšího vývoje a rozpínání 
    vesmíru se postupně oddělovat. Tak jako počátky vzniku vědecké fyziky v 17. století jsou spjaty s 
    astronomickými pozorováními sluneční soustavy, je i dnes fyzika stále více propojena s astrofyzikou. 
    Vesmír zůstává největší fyzikální laboratoří.
\printbibliography[heading=subbibliography]
