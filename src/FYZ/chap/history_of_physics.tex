\chapter{Historie fyziky}
\minitoc
\newpage
  \section{Hlavní etapy vývoje}
    Fyzika prošla dlouhým historickým vývojem a znalost tohoto vývoje pomáhá lépe pochopit logiku soustavy
    fyzikálních poznatků a dokonce do\-cházet k poznatkům novým. V krátkosti dějiny fyziky můžeme 
    rozdělit na 
    tři hlavní etapy:
    \begin{itemize}
     	\item Stará fyzika - od starověku do počátku 17. století (orientačně do roku 1600).
     \item Klasická fyzika - 1600 – 1900.
     \item Moderní fyzika - 1900 – dosud.
    \end{itemize}
    Starou fyziku nemůžeme považovat za vědu ve vlastním smyslu, i když se dobrala celé řady významných 
    vědeckých poznatku. První z nich znali již staří Sumerové, Babyloňané, Egypťané a Číňané. Šlo zejména 
    o  poznatky astronomické a geometrické (Pythagorova veta) a také o metody měření některých 
    fyzikálních veličin (délka, hmotnost, čas). Fyzika ve starém Řecku byla jako součást filosofie 
    převážně spekulativní a tento charakter si pod vlivem aristotelismu udržela, až do počátku novověku. 
    Skutečný fyzikální výzkum prováděli až helenističtí Řekové, kdy se centrem vědy a kultury antického 
    světa stala Alexandrie. V Alexandrii studoval největší fyzik starověku Archimédes, který dospěl k 
    důležitým poznatkům o statické rovnováze těles a plování těles a v matematice se těsně přiblížil 
    objevu diferenciálního a integrálního počtu. Alexandrijští Řekové znali také zákon odrazu světla 
    (nikoli lomu) a prováděli první měření teploty. Poznatky antiky byly středověké Evropě 
    zprostředkovány Araby, kteří se též intenzivně zabývali optikou (Alhazen) a určováním měrné hmotnosti 
    látek. Zatímco ve středověku byly hlavní přírodovědné poznatky čerpány z Euklidových ”Základu” 
    (geometrie), ”Almagestu” Klaudia Ptolemaia (geocentrický výklad astronomie sluneční soustavy) a spisu 
    Aristotelových (mj. ”Fysika”), vešly práce Archimédovy v Evropě ve známost až teprve začátkem 
    novověku. Ve starověku a středověku však fyzika neprováděla systematické experimenty, nevyužívala 
    matematický aparát k popisu přírodních jevu a neměla ani přesně definovány základní pojmy (rychlost, 
    zrychlení, síla apod.) Zrod fyziky jako vědy se datuje začátkem 17. století. Na základě  
    astronomických výzkumu Keplerových (1571-1630) a pozemských mechanických experimentů Galileových 
    (1564-1642) mohl Isaac Newton (1643-1727) vytvořit první fyzikální teorii, klasickou mechaniku,   
    využívající matematický aparát diferenciálního a integrálního poctu. Newton přišel s koncepcí   
    všeobecné gravitace a ukázal, že není přehrady mezi nebeskou a pozemskou fyzikou, že síla, která    
    udržuje planety na jejich drahách kolem Slunce je táž jako síla, která nutí jablko padat k zemi. 
    Základní Newtonovo dílo z r. l687 nese název ”Matematické základy přírodní filosofie” (”Philosophiae 
    naturalis principia mathematica”) a představuje pravděpodobně nejvýznamnější vědeckou knihu, která 
    byla kdy napsána. Newton se zabýval též optikou a rozpracoval teorii rozkladu bílého světla do 
    spektra. V té době byl již zásluhou Snellovou a Descartovou znám i zákon lomu světla. Z roku 1600 
    pochází první vědecký spis o elektřině a magnetismu od anglického lékaře a fyzika Gilberta. Výzkumem 
    těchto jevu se v následujících stoletích zabývala celá řada fyziků (Coulomb, Volta, Oersted, 
    Amp\`{e}re a další). Tento výzkum pak završil Faraday (1791-1867) svým objevem zákona 
    elektromagnetické indukce a svou koncepcí siločar elektromagnetického pole. Úlohu Newtona 
    elektromagnetismu pak sehrál James Clerk Maxwell (1831-1879), který ve svém ”Traktátě o elektřině a 
    magnetismu” z r. 1873 sestavil slavné Maxwellovy rovnice popisující vlastnosti elektromagnetického 
    pole. Maxwell zároveň teoreticky zdůvodnil elektromagnetickou povahu světla a ukázal, že jevy spojené 
    s vlastnostmi elektrického náboje (”elektřina”), elektrického proudu (”galvanismus”), magnetického 
    pole a světla (optika), jsou jedné a téže elektromagnetické povahy. V devatenáctém století byl tak 
    dovršen výzkum mechanických jevů a elektromagnetismu a klasická fyzika tím za\-vršena. V přírodě tedy 
    existovaly pouze dvě síly, dva způsoby vzájemné interakce mezi částicemi: gravitační a 
    elektromagnetická. Mezi nimi se však projevoval určitý rozpor. Jak Newtonovy tak Maxwellovy rovnice 
    platí v libovolné inerciální vztažné soustavě. Při přechodu od jedné inerciální soustavy k druhé se 
    však Newtonovy rovnice transformují pomocí tzv. Galileiho transformací a Maxwellovy rovnice pomocí 
    Lorentzových transformací. Fyzika se tak rozdvojila, mechanické a elektromagnetické děje se zdály být 
    neslučitelné. Kromě toho existovaly některé experimenty, jejichž výsledek nedokázala klasická fyzika 
    vysvětlit: průběh spektra rovnovážného elektromagnetického záření (tzv. záření absolutně černého 
    tělesa) a pokus Michelsonův, který svědčil o neexistenci světelného éteru. Tyto zdánlivě nepodstatné 
    rozpory vyústily ve 20. století ve vznik moderní fyziky, tj. fyziky kvantové a relativistické. Právě 
    koncem roku 1900 vyslovil Planck tzv. kvantovou hypotézu, jíž vysvětlil záření absolutně černého 
    tělesa, a v r. 1905 publikoval Einstein práci o speciální teorii relativity. V ní překlenul rozpor 
    mezi Newtonovou a Maxwellovou fyzikou a fyziku opět sjednotil. Předpoklad o existenci světelného 
    éteru se teorií relativity stal zbytečným. V roce 1916 vytvořil Einstein i obecnou teorii
    relativity jako moderní teorii gravitace. Gravitační síly podle této teorie souvisejí se zakřivením 
    prostoročasu. Jak speciální, tak obecná teorie relativity přecházejí při rychlostech objektu 
    podstatně menších než je rychlost světla ve vakuu a při slabých gravitačních polích v teorii 
    Newtonovu. Přelom 19. a 20. století je též poznamenán objevem radioaktivity a vznikem jaderné fyziky, 
    která tak významným způsobem zasáhla do života celého lidstva. V jaderné fyzice se uplatní další dvě 
    přírodní síly - tzv. silná, která udržuje nukleony v atomových jádrech a slabá, která se projevuje 
    při radioaktivní přeměně beta za vzniku neutrin. Moderní fyzika odhalila v kosmickém záření a pomocí 
    urychlovačů obrovské množství částic, jejichž vlastnosti studuje a snaží se je utřídit a vysvětlit. 
    Mezi všemi těmito částicemi působí čtyři základní síly přírody: gravitační, elektromagnetická, silná 
    a slabá. V nedávné době se podařilo prokázat, že i elektromagnetická a slabá interakce jsou téže 
    podstaty a tvoří jedinou sílu elektroslabou. V průběhu historie fyziky od Newtona a Maxwella k dnešku 
    tak probíhá úsilí o sjednocování interakcí, které pokračuje i dnes. Fyzika se pokouší prokázat, že i 
    silná a elektroslabá interakce jsou téže povahy, a že k nim konečně přistupuje i síla gravitační. Tím 
    by vznikla idea jediné přírodní síly sjednocující všechny přírodní jevy a děje. Fyzika ovšem nemůže k 
    takovému závěru dojít pouhým uvažováním, musí matematicky vypracovat a zdůvodnit příslušnou teorii a 
    její závěry experimentálně ověřit. To vede ke snaze budovat stále větší a větší urychlovače a také k 
    intenzivnímu výzkumu jevů v kosmu. Sjednocování interakcí má totiž těsnou návaznost na vývoj vesmíru 
    podle hypotézy o tzv. ”velkém tresku”. Právě v počátcích vývoje vesmíru by se měly všechny čtyři 
    (resp. tři) interakce uplatňovat rovnocenným způsobem a teprve v průběhu dalšího vývoje a rozpínání 
    vesmíru se postupně oddělovat. Tak jako počátky vzniku vědecké fyziky v 17. století jsou spjaty s 
    astronomickými pozorováními sluneční soustavy, je i dnes fyzika stále více propojena s astrofyzikou. 
    Vesmír zůstává největší fyzikální laboratoří.
\printbibliography[heading=subbibliography]
