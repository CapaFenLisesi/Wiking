\chapter{Základy fyziky}
\minitoc
\newpage
  \section{Úvod}
    V této kapitole budeme zkoumat nejzákladnější myšlenky, s nimiž se ve fyzice setkáváme - budeme hovořit o 
    tom, jak si v současnosti představujeme povahu věcí. Nebudeme však hovořit o tom, jak se poznala 
    správnost těchto představ - o těchto detailech se dozvíte v pravý čas.
    
    Věci, o něž se v naší vědě zajímáme, se nám ukazují množstvím projevů a atributů. Stojíme-li například na 
    břehu a hledíme na moře, vidíme vodu, na vodě pěnu, nad mořem oblaka, slunce, modrou oblohu a vůbec 
    světlo, slyšíme zvuk, nárazy vln, svištění větru, cítíme vzduch. Na břehu je písek a skály, a každá má 
    jinou tvrdost a pevnost, barvu a složení. Jsou tam zvířata a vodní tráva, je tam hlad i nemoc a na břehu 
    je pozorovatel se svými myšlenkami a snad i štěstím. Každé jiné místo v přírodě se vyznačuje podobnou 
    pestrostí věcí a vlivů, podobnou složitostí. Naše zvědavost nás nutí klást otázky, hledat souvislosti a 
    chápat mnohotvárnost věcí jako následek snad relativně malého počtu nejjednodušších věcí a sil působících 
    nekonečně rozmanitě.
    
    Klademe si otázku: Je písek jiný než skály? Není snad písek nic jiného, než velký počet velmi malých 
    kamínků? Je Měsíc velká skála? Kdybychom porozuměli tomu, co jsou skály, znamená to, že bychom pochopili 
    i podstatu písku a Měsíce? Co je to vítr? Jsou to nárazy vzduchu podobné nárazům vody na břeh? Jaké 
    společné rysy mají rozličné druhy pohybu? Co mají společného různé druhy zvuku? Kolik různých barev 
    existuje? A tak dále. Takovým způsobem se snažíme postupně analyzovat všechny věci. Dáváme do souvislostí 
    věci, které na první pohled vzájemně nesouvisí. Děláme to s nadějí, že se nám podaří redukovat počet 
    rozličných věcí a tak je lépe poznat.
    
    Před několika sty lety vznikla metoda hledání částečných odpovědí na uvedené otázky. \emph{Pozorování, 
    usuzování a experiment} vytvářejí to, co nazýváme \emph{vědeckou metodou}. Budeme se muset omezit jen na 
    holý popis našich představ o tom, co se nazývá z\emph{základní fyzikou} nebo základními myšlenkami, které 
    vznikly aplikováním vědecké metody.
    
    Co to znamená něco „pochopit“? Můžeme si představit, že to složité nahromadění pohybujících se věcí, 
    které vytvářejí „svět“, je šachová hra bohů a my vystupujeme jako diváci, kteří neznají pravidla hry, ale 
    je jim dovoleno hru \emph{pozorovat}. Samozřejmě, pozorujeme-li dostatečně dlouho, můžeme nakonec 
    pochytit několik pravidel. \emph{Pravidla hry} představují to, co chápeme jako \emph{základní fyziku}. I 
    kdybychom znali všechna pravidla, nemuseli bychom ještě rozumět každému kroku hry, protože je příliš 
    složitá a možnosti našeho rozumu omezené. Hrajete-li šachy, jistě víte, že je jednoduché naučit se 
    všechna pravidla, ale i tak je velmi těžké zvolit ten správný tah nebo pochopit záměry protihráče. Stejné 
    je to i s přírodou, jen mnohem těžší. Máme však možnost najít alespoň všechna pravidla. Zatím je všechna 
    neznáme. (Každou chvíli se objevuje něco takového jako rošáda, kterou ještě neznáme.) Nejen, že neznáme 
    všechna pravidla, ale pomocí těch, která známe, umíme jen velmi málo vysvětlit. Je tomu tak proto, že 
    téměř všechny situace jsou ohromně složité a známá pravidla nám neumožní sledovat všechny obraty hry, 
    nemluvě o předvídání dalších kroků. Musíme se proto omezit na základnější otázku pravidel hry. Naučíme-li 
    se pravidla, budeme to považovat za „pochopení“ světa.
    
    Jak můžeme rozhodnout, zda pravidla, která vlastně jen „odhadujeme“, jsou skutečně správná, když 
    nemůžeme dokonale analyzovat hru? Existují zhruba tři způsoby. Především nám příroda může poskytnout 
    (nebo my si od přírody vynutíme) jednoduché situace skládající se z malého počtu částí, umožňující 
    přesnou předpověď budoucího dění, a tím i zkoušku pravidel. (V rohu šachovnice zůstalo jen málo figurek, 
    jejichž tahy již umíme přesně určit)
    
    Druhý způsob zkoušky pravidel spočívá v jejich použití k odvození obecnějších pravidel. Například, 
    střelec se na šachovnici pohybuje úhlopříčně. Odtud je možné usuzovat na skutečnost, že určitý střelec 
    bude vždy na bílém poli. Odhlédneme-li od podrobností, můžeme prověřovat naše pravidlo o pohybu 
    uvedeného střelce tak, že sledujeme, jestli se vždy nachází na bílém poli. Po dlouhém čase se samozřejmě 
    může stát, že se náhle objeví na černém poli (v průběhu hry byl vzat, ale jeden pěšec došel na konec 
    šachovnice a proměnil se na střelce na černém poli). Tak to bývá i ve fyzice. Dlouho používáme pravidlo, 
    které ve všech směrech dobře vyhovuje, ačkoliv neznáme detaily, a potom najednou objevíme \emph{nové 
    pravidlo}. Z hlediska základů fyziky probíhají nejzajímavější jevy na nových místech, na místech, kde 
    pravidla neplatí a ne tam, kde pravidla \emph{platí}. To je způsob, jakým objevujeme nová pravidla.
    
    Třetí ze způsobů, kterými se můžeme přesvědčit o správnosti našich myšlenek, je poměrně hrubý, ale snad 
    nejúčinnější. Je to způsob přibližného odhadu. Ačkoliv nejsme schopni říci, proč Aljechin \emph{táhl 
    právě tou figurkou}, můžeme v \emph{hrubých rysech} chápat, že seskupuje figurky okolo krále, aby ho 
    chránil, protože za daných okolností je to nejrozumnější. Podobně je to i s naším chápáním přírody. Často 
    ji více či méně chápeme, aniž bychom byli schopni znát význam tahu \emph{každé jednotlivé figurky}.
    
    Zpočátku se přírodní jevy hrubě rozdělovaly do tříd jako teplo, elektřina, mechanika, magnetizmus, 
    vlastnosti látek, chemické děje, světlo nebo optika, rentgenové paprsky, jaderná fyzika, gravitace, 
    mezonové jevy atd. Cílem je však pochopení \emph{celé přírody} jako různých aspektů \emph{jednoho 
    souboru} jevů. Úkolem základní teoretické fyziky dneška je \emph{nalezení zákonů stojících za 
    experimentem a sjednocení uvedených tříd}. Historicky se nám vždy podařilo sloučit je, ale postupem času 
    se objevovaly nové věci. Když jsme si již vytvořili ucelenou představu, objevily se najednou rentgenové 
    paprsky. Když se i tento jev dostal do jednotného schématu, objevily se mezony. Proto v každém stádiu hry 
    vypadá situace dost chaoticky. Mnohé se objasnilo z jednotného hlediska, ale ještě stále je mnoho volných 
    konců nitek, o nichž nevíme, kam patří. Takový je dnes stav věcí a my se ho pokusíme popsat.
    
    Všimněme si v historii několika příkladů uvedeného sjednocování. Uvažujme nejdříve \emph{teplo a 
    mechaniku}. Jsou-li atomy v pohybu, obsahuje systém tím více tepla, čím více pohybu v něm je, takže 
    \emph{teplo a všechny tepelné efekty je možné vyjádřit pomocí zákonů mechaniky}. Dalším úžasným 
    sjednocením bylo objevení souvislosti mezi \emph{elektřinou, magnetizmem} a světlem, o nichž se zjistilo, 
    že jsou různými aspekty stejné věci, kterou dnes nazýváme \emph{elektromagnetické pole}. Dále chemické 
    děje, rozmanité vlastnosti různých látek a chování atomových částic byly sjednoceny do \emph{kvantové 
    chemie}.
    
    Zůstává zde však otázka, zda bude možné vše sjednotit tak, abychom mohli prohlásit, že svět představuje 
    rozmanité aspekty jediné věci? To nikdo neví. Víme pouze, že na naší cestě vpřed se nám daří spojovat 
    fragmenty, přičemž vždy nalézáme cosi, co nezapadá do obecného obrazu, a proto se opět pokoušíme doplnit 
    skládačku. Nevíme, zda tato skládačka má konečný počet částí a zda má tato hra vůbec hranice. Dozvíme se 
    to až tehdy, když složíme výsledný obraz, jestli ho vůbec kdy složíme. Chtěli bychom však ukázat, kam až 
    tento proces sjednocování pokročil a jaká je dnešní situace při objasňování základních jevů pomocí co 
    nejmenšího počtu principů. Jednodušeji řečeno: \textbf{z čeho jsou složeny věci a kolik je těch 
    stavebních prvků?} \cite[s.~27]{Feynman02}
    
  \section{Hlavní etapy vývoje}
    Fyzika prošla dlouhým historickým vývojem a znalost tohoto vývoje pomáhá lépe pochopit logiku soustavy
    fyzikálních poznatků a dokonce do\-cházet k poznatkům novým. V krátkosti dějiny fyziky můžeme 
    rozdělit na 
    tři hlavní etapy:
    \begin{itemize}
     	\item Stará fyzika - od starověku do počátku 17. století (orientačně do roku 1600).
     \item Klasická fyzika - 1600 – 1900.
     \item Moderní fyzika - 1900 – dosud.
    \end{itemize}
    Starou fyziku nemůžeme považovat za vědu ve vlastním smyslu, i když se dobrala celé řady významných 
    vědeckých poznatku. První z nich znali již staří Sumerové, Babyloňané, Egypťané a Číňané. Šlo zejména 
    o  poznatky astronomické a geometrické (Pythagorova veta) a také o metody měření některých 
    fyzikálních veličin (délka, hmotnost, čas). Fyzika ve starém Řecku byla jako součást filosofie 
    převážně spekulativní a tento charakter si pod vlivem aristotelismu udržela, až do počátku novověku. 
    Skutečný fyzikální výzkum prováděli až helenističtí Řekové, kdy se centrem vědy a kultury antického 
    světa stala Alexandrie. V Alexandrii studoval největší fyzik starověku Archimédes, který dospěl k 
    důležitým poznatkům o statické rovnováze těles a plování těles a v matematice se těsně přiblížil 
    objevu diferenciálního a integrálního počtu. Alexandrijští Řekové znali také zákon odrazu světla 
    (nikoli lomu) a prováděli první měření teploty. Poznatky antiky byly středověké Evropě 
    zprostředkovány Araby, kteří se též intenzivně zabývali optikou (Alhazen) a určováním měrné hmotnosti 
    látek. Zatímco ve středověku byly hlavní přírodovědné poznatky čerpány z Euklidových ”Základu” 
    (geometrie), ”Almagestu” Klaudia Ptolemaia (geocentrický výklad astronomie sluneční soustavy) a spisu 
    Aristotelových (mj. ”Fysika”), vešly práce Archimédovy v Evropě ve známost až teprve začátkem 
    novověku. Ve starověku a středověku však fyzika neprováděla systematické experimenty, nevyužívala 
    matematický aparát k popisu přírodních jevu a neměla ani přesně definovány základní pojmy (rychlost, 
    zrychlení, síla apod.) Zrod fyziky jako vědy se datuje začátkem 17. století. Na základě  
    astronomických výzkumu Keplerových (1571-1630) a pozemských mechanických experimentů Galileových 
    (1564-1642) mohl Isaac Newton (1643-1727) vytvořit první fyzikální teorii, klasickou mechaniku,   
    využívající matematický aparát diferenciálního a integrálního poctu. Newton přišel s koncepcí   
    všeobecné gravitace a ukázal, že není přehrady mezi nebeskou a pozemskou fyzikou, že síla, která    
    udržuje planety na jejich drahách kolem Slunce je táž jako síla, která nutí jablko padat k zemi. 
    Základní Newtonovo dílo z r. l687 nese název ”Matematické základy přírodní filosofie” (”Philosophiae 
    naturalis principia mathematica”) a představuje pravděpodobně nejvýznamnější vědeckou knihu, která 
    byla kdy napsána. Newton se zabýval též optikou a rozpracoval teorii rozkladu bílého světla do 
    spektra. V té době byl již zásluhou Snellovou a Descartovou znám i zákon lomu světla. Z roku 1600 
    pochází první vědecký spis o elektřině a magnetismu od anglického lékaře a fyzika Gilberta. Výzkumem 
    těchto jevu se v následujících stoletích zabývala celá řada fyziků (Coulomb, Volta, Oersted, 
    Amp\`{e}re a další). Tento výzkum pak završil Faraday (1791-1867) svým objevem zákona 
    elektromagnetické indukce a svou koncepcí siločar elektromagnetického pole. Úlohu Newtona 
    elektromagnetismu pak sehrál James Clerk Maxwell (1831-1879), který ve svém ”Traktátě o elektřině a 
    magnetismu” z r. 1873 sestavil slavné Maxwellovy rovnice popisující vlastnosti elektromagnetického 
    pole. Maxwell zároveň teoreticky zdůvodnil elektromagnetickou povahu světla a ukázal, že jevy spojené 
    s vlastnostmi elektrického náboje (”elektřina”), elektrického proudu (”galvanismus”), magnetického 
    pole a světla (optika), jsou jedné a téže elektromagnetické povahy. V devatenáctém století byl tak 
    dovršen výzkum mechanických jevů a elektromagnetismu a klasická fyzika tím za\-vršena. V přírodě tedy 
    existovaly pouze dvě síly, dva způsoby vzájemné interakce mezi částicemi: gravitační a 
    elektromagnetická. Mezi nimi se však projevoval určitý rozpor. Jak Newtonovy tak Maxwellovy rovnice 
    platí v libovolné inerciální vztažné soustavě. Při přechodu od jedné inerciální soustavy k druhé se 
    však Newtonovy rovnice transformují pomocí tzv. Galileiho transformací a Maxwellovy rovnice pomocí 
    Lorentzových transformací. Fyzika se tak rozdvojila, mechanické a elektromagnetické děje se zdály být 
    neslučitelné. Kromě toho existovaly některé experimenty, jejichž výsledek nedokázala klasická fyzika 
    vysvětlit: průběh spektra rovnovážného elektromagnetického záření (tzv. záření absolutně černého 
    tělesa) a pokus Michelsonův, který svědčil o neexistenci světelného éteru. Tyto zdánlivě nepodstatné 
    rozpory vyústily ve 20. století ve vznik moderní fyziky, tj. fyziky kvantové a relativistické. Právě 
    koncem roku 1900 vyslovil Planck tzv. kvantovou hypotézu, jíž vysvětlil záření absolutně černého 
    tělesa, a v r. 1905 publikoval Einstein práci o speciální teorii relativity. V ní překlenul rozpor 
    mezi Newtonovou a Maxwellovou fyzikou a fyziku opět sjednotil. Předpoklad o existenci světelného 
    éteru se teorií relativity stal zbytečným. V roce 1916 vytvořil Einstein i obecnou teorii
    relativity jako moderní teorii gravitace. Gravitační síly podle této teorie souvisejí se zakřivením 
    prostoročasu. Jak speciální, tak obecná teorie relativity přecházejí při rychlostech objektu 
    podstatně menších než je rychlost světla ve vakuu a při slabých gravitačních polích v teorii 
    Newtonovu. Přelom 19. a 20. století je též poznamenán objevem radioaktivity a vznikem jaderné fyziky, 
    která tak významným způsobem zasáhla do života celého lidstva. V jaderné fyzice se uplatní další dvě 
    přírodní síly - tzv. silná, která udržuje nukleony v atomových jádrech a slabá, která se projevuje 
    při radioaktivní přeměně beta za vzniku neutrin. Moderní fyzika odhalila v kosmickém záření a pomocí 
    urychlovačů obrovské množství částic, jejichž vlastnosti studuje a snaží se je utřídit a vysvětlit. 
    Mezi všemi těmito částicemi působí čtyři základní síly přírody: gravitační, elektromagnetická, silná 
    a slabá. V nedávné době se podařilo prokázat, že i elektromagnetická a slabá interakce jsou téže 
    podstaty a tvoří jedinou sílu elektroslabou. V průběhu historie fyziky od Newtona a Maxwella k dnešku 
    tak probíhá úsilí o sjednocování interakcí, které pokračuje i dnes. Fyzika se pokouší prokázat, že i 
    silná a elektroslabá interakce jsou téže povahy, a že k nim konečně přistupuje i síla gravitační. Tím 
    by vznikla idea jediné přírodní síly sjednocující všechny přírodní jevy a děje. Fyzika ovšem nemůže k 
    takovému závěru dojít pouhým uvažováním, musí matematicky vypracovat a zdůvodnit příslušnou teorii a 
    její závěry experimentálně ověřit. To vede ke snaze budovat stále větší a větší urychlovače a také k 
    intenzivnímu výzkumu jevů v kosmu. Sjednocování interakcí má totiž těsnou návaznost na vývoj vesmíru 
    podle hypotézy o tzv. ”velkém tresku”. Právě v počátcích vývoje vesmíru by se měly všechny čtyři 
    (resp. tři) interakce uplatňovat rovnocenným způsobem a teprve v průběhu dalšího vývoje a rozpínání 
    vesmíru se postupně oddělovat. Tak jako počátky vzniku vědecké fyziky v 17. století jsou spjaty s 
    astronomickými pozorováními sluneční soustavy, je i dnes fyzika stále více propojena s astrofyzikou. 
    Vesmír zůstává největší fyzikální laboratoří.
  
  \section{Fyzika před rokem 1920}
    Je dost těžké začít hned se současnými představami, a proto se podívejme, jak se jevil svět v roce 1920 a 
    potom na tomto obrázku něco změníme. Naše představa světa byla před rokem \textbf{1920} následující: 
    „Scénou“, na které vystupuje vesmír, je \emph{trojrozměrný geometrický prostor} popsaný ještě Euklidem a 
    věci se mění v prostředí, které nazýváme časem. Prvky vystupující na scéně jsou \emph{částice}, například 
    atomy, které mají určité vlastnosti. Především vlastnost setrvačnosti: pohybuje-li se částice, zachová si 
    pohyb v původním směru, pokud na ni nepůsobí \emph{síly}. Druhým prvkem jsou tedy síly, o nichž se tehdy 
    předpokládalo, že jsou dvojího druhu. K prvnímu, velmi složitému druhu, patřila síla vzájemného působení, 
    která udržovala atomy v jejich různých kombinacích komplikovaným způsobem a byla zodpovědná za to, jestli 
    se sůl při zvyšování teploty rozpouští rychleji nebo pomaleji. Druhou známou silou byla interakce 
    dalekého dosahu - hladké a klidné přitahování. Tato síla, měnící se nepřímo úměrně čtverci vzdálenosti, 
    byla nazvána \emph{gravitací}. Její zákon byl známý a byl velmi jednoduchý. Proč věci zůstávají v pohybu, 
    když se už začaly pohybovat, nebo proč existuje gravitační zákon, bylo, samozřejmě, neznámé.
    
    Zabýváme se popisem přírody. Z tohoto hlediska je plyn a právě tak všechna hmota myriádou pohybujících se 
    částic. Takto se dostávají do souvislosti mnohé věci, které jsme viděli na mořském břehu. \emph{Tlak} 
    pochází od \emph{srážek atomů} se stěnami nebo s čímkoliv jiným; atomy pohybující se převážně jedním 
    směrem vytvářejí vítr; \emph{chaotické vnitřní pohyby} představují \emph{teplo}. Známe vlny zvýšené 
    hustoty, kde se shromáždilo příliš mnoho částic, které při rozletu stlačují další shluky částic a pohyb 
    se tak předává dál. Tyto vlny vyšší hustoty představuj í \emph{zvuk}. Pochopení tolika věcí je možno 
    považovat za úžasný úspěch. O některých z těchto věcí jsme hovořili v předcházející kapitole.
    
    Jaké druhy částic existují? Tehdy předpokládali, že je jich 92. Nakonec bylo objeveno 92 různých druhů 
    atomů. Měly různá jména podle svých chemických vlastností.
    
    Byl tu ještě problém \emph{povahy sil krátkého dosahu}. Proč uhlík přitahuje jeden kyslík, případně dva, 
    ale ne víc? Jaký je mechanizmus vzájemného působení mezi atomy? Je to gravitace? Na tuto otázku musíme 
    odpovědět záporně, protože gravitace je na to příliš slabá. Představme si však sílu podobnou gravitaci, 
    měnící se nepřímo úměrně čtverci vzdálenosti, ale mnohem silnější a odlišnou ještě v jednom směru. V 
    případě \emph{gravitace jde vždy o přitahování}. Představme si však, že existují dva druhy „věcí“ a tato 
    nová síla  (samozřejmě elektrické povahy) má tu vlastnost, že věci stejného druhu se odpuzují a věci 
    různého druhu se přitahují. „Předmět“, jenž je nositelem tohoto silného vzájemného působení, se nazývá 
    \emph{náboj}.  
    
    K čemu jsme došli? Předpokládejme, že máme dvě věci různého druhu, jež se vzájemně  
    přitahují (plus a minus) a které drží těsně u sebe. Předpokládejme, že v určité vzdálenosti od uvedené 
    dvojice máme další náboj. Bude tento náboj pociťovat přitažlivost? Mají-li první dva náboje stejnou 
    velikost, neměl by pocítit \emph{prakticky žádnou přitažlivost}, protože přitahování jedním nábojem a 
    odpuzování druhým nábojem se vykompenzují. Ve velkých vzdálenostech je tedy síla velmi malá. Když třetí 
    náboj \emph{hodně přiblížíme} k prvním dvěma, objeví se přitahování, protože odpuzování stejných nábojů a 
    přitahování různých se snaží oddálit stejné náboje a přiblížit různé. Odpuzování bude nakonec 
    \emph{slabší} než přitahování. To je příčina, proč atomy, které se skládají z kladných a záporných 
    elektrických nábojů, na sebe téměř nepůsobí (zanedbáme-li gravitaci), jsou-li od sebe dost vzdáleny. Když 
    se ale přiblíží, mohou „\emph{vidět jeden do druhého}“, přeskupit své náboje a velmi silně vzájemně 
    působit. Podstatou interakce mezi atomy je \emph{elektrické} působení. Tato síla je tak veliká, že 
    všechny plusy a minusy se obvykle dostávají do tak těsné kombinace, jak je to jen možné. Všechny věci, 
    včetně nás samotných, se skládají z drobných, velmi silně interagujících kladných a záporných částic, 
    které jsou velmi přesně vyvážené. Na okamžik je možné náhodou odstranit několik minusů nebo plusů 
    (obvykle je jednodušší odstranit minusy), v tu chvíli jsou elektrické síly \emph{nevyvážené} a můžeme 
    pozorovat působení elektrické přitažlivosti.
    
    Abychom si vytvořili představu o tom, o kolik je elektrické působení silnější než gravitace, představme 
    si dvě zrnka písku, která mají jeden milimetr v průměru a jsou vzdálená třicet metrů. Kdyby elektrické 
    síly mezi nimi nebyly vyvážené, kdyby nebylo odpuzování a vše se navzájem přitahovalo a nic se 
    nekompenzovalo, jakou silou by se zrnka přitahovala? Byla by to síla tří miliónů tun. Jistě chápete, že 
    pro vytvoření značného elektrického působení stačí velmi malý přebytek nebo nedostatek záporných nebo 
    kladných nábojů. Proto není vidět rozdíl mezi elektricky nabitým a nenabitým předmětem - pro nabití 
    předmětu je třeba tak málo částic, že se téměř neprojeví na jeho hmotnosti, či rozměru.
    
    S těmito poznatky bylo jednodušší pochopit atomy. Předpokládalo se, že mají uprostřed „\emph{jádro}“, 
    které je kladně elektricky nabité a velmi těžké, a toto jádro je obklopeno určitým počtem „elektronů“, 
    jež jsou velmi lehké a záporně nabité. Teď trochu pokročíme v našem výkladu a poznamenáme, že v samotných 
    jádrech byly objeveny dva druhy částic - \emph{protony} a \emph{neutrony}, které mají téměř stejnou, 
    velmi velkou hmotnost. Protony jsou elektricky nabité a neutrony jsou neutrální. Máme-li atom se šesti 
    protony v jádře, které je obklopeno šesti elektrony (záporné částice obyčejného světa jsou všechno 
    elektrony a ty jsou velmi lehké v porovnání s protony a neutrony, které tvoří jádra), půjde o atom číslo 
    šest v chemické tabulce a tento atom se nazývá uhlík. Atom číslo osm se nazývá kyslík atd. Chemické 
    vlastnosti závisí na vnějších elektronech, ve skutečnosti jen na tom, kolik má atom elektronů. 
    \emph{Chemické vlastnosti} látek tedy závisí na jediném čísle, na \emph{počtu elektronů}. (Seznam prvků 
    sestavený chemiky by se mohl nahradit očíslováním 1, 2, 3, 4, 5 atd. Místo toho, abychom říkali „uhlík“, 
    stačilo by říci „prvek číslo šest“, což by znamenalo, že prvek má šest elektronů. Při objevování prvků 
    však tato skutečnost nebyla známa a dále, při číslování by vše vypadalo velmi složitě. Proto je lepší 
    ponechat prvkům názvy i symboly a nedožadovat se pouhého očíslování.)
    
    O elektrické síle bylo získáno mnoho dalších poznatků. Bylo by přirozené předpokládat, že elektrická 
    interakce je jednoduché přitahování dvou předmětů: kladného a záporného. Zjistilo se však, že toto není 
    úplně vhodná představa. Situaci lépe vystihuje představa, že existence kladného náboje v prostoru 
    způsobuje jeho jisté \emph{zakřivení}, vytváří v něm určitou „podmínku“, aby záporný náboj vložený do 
    tohoto prostoru cítil působení síly. Tato možnost vzniku síly se nazývá \emph{elektrické pole}. 
    Dostane-li se elektron do elektrického pole, je jakoby „tažen“. Přitom platí dvě pravidla: a) 
    \emph{náboje vytvářejí pole}, b) \emph{v poli působí na náboje síly a náboje se pohybují}. Příčina 
    takového chování se stane jasnější, jakmile rozebereme následující jev: Nabijeme-li těleso 
    elektricky, například hřeben, a do určité vzdálenosti položíme nabitý ústřižek papíru, přičemž začneme 
    hřebenem pohybovat sem a tam, bude se papír natáčet směrem k hřebenu. Zrychlíme-li pohyb hřebenu, 
    zjistíme, že papír zaostává, působení se opožďuje. (V prvním stádiu, když pohybujeme hřebenem poměrně 
    pomalu, zkomplikuje nám situaci \emph{magnetizmus}. Magnetické vlivy se projevují, když jsou \emph{náboje 
    v relativním pohybu}, takže magnetické a elektrické síly je možné skutečně připsat jedinému poli jako dvě 
    stránky jedné věci. Měnící se elektrické pole nemůže existovat bez magnetizmu.) Oddálíme-li nabitý papír, 
    zpoždění je větší. V tu chvíli pozorujeme zajímavou věc. Ačkoliv se síly působící mezi dvěma nabitými 
    předměty mění nepřímo úměrně čtverci vzdálenosti, při kmitání náboje zjišťujeme, že jeho působení se 
    rozprostírá mnohem dále, než by se dalo očekávat. Pokles tohoto působení je mnohem pomalejší než při 
    nepřímé úměrnosti čtverci vzdálenosti.
    
    S analogickou situací se setkáváme, když na vodě plave splávek a my ho uvedeme do pohybu „přímo“ tím, že 
    způsobíme pohyb vody jiným splávkem. Kdybychom se dívali jen na dva splávky, pozorovali bychom pouze to, 
    že jeden se dává do pohybu jako odezva na pohyb druhého, že mezi nimi existuje určitá „  interakce“. Ve 
    skutečnosti jsme ale rozčeřili vodu a voda posunula druhý splávek. Mohli bychom zformulovat „zákon“, že i 
    při slabém zčeření vody se na vodě budou pohybovat předměty nacházející se blízko zdroje zčeření. Kdyby 
    byl druhý splávek dost daleko, sotva by se dal do pohybu, neboť jsme uvedli vodu do pohybu jen v jednom 
    místě. Bude-li však druhý splávek pravidelně kmitat, vznikne nový úkaz, při kterém se pohyb vody přenáší 
    dál, vzniká \emph{vlnění} a vliv poskakujícího splávku již nemůžeme chápat jako přímé působení mezi 
    splávky. Myšlenku přímé interakce tedy musíme nahradit předpokladem o existenci vody nebo v případě 
    elektrických nábojů tím, co nazýváme \emph{elektromagnetickým polem}.

    \begin{figure}[ht!] % \ref{FYZ:fig_base01}
      \centering
      \includegraphics[width=1\linewidth]{wiki_EM_Spectrum.pdf}
      \caption{Elektromagnetické spektrum (někdy zvané Maxwellova duha) zahrnuje elektromagnetické záření 
      všech možných vlnových délek. Srovnání délek elektromagnetických vln s běžnými předměty a odpovídající 
      teplotní stupnice umožňuje lépe získat představu o jejich rozměrech a energiích.}
      \label{FYZ:fig_base01}
    \end{figure}
    
    Elektromagnetické pole může přenášet vlny. Některé z těchto vln jsou světlo jak je znázorněno na obrázku 
    \ref{FYZ:fig_base01}, jiné se používají při rádiovém vysílání, ale obecně se nazývají 
    \emph{elektromagnetickými vlnami}. Tyto vlny mohou mít rozmanité \emph{frekvence}. Jediné, čím se jedna 
    vlna liší od druhé, je právě frekvence vlnění. Kdybychom pohybovali nábojem sem a tam a dělali bychom to 
    stále rychleji a rychleji, objevovala by se celá řada různých jevů, které je možné systematizovat udáním 
    čísla vyjadřujícího počet kmitů za sekundu. Frekvence, s nimiž přicházíme do styku prostřednictvím 
    běžných rozvodových elektrických sítí v domech, jsou řádově sto kmitů za sekundu. Zvýšíme-li frekvenci na 
    \SI{500}{\kHz} nebo \SI{1000}{\kHz} (\SI{1}{\kHz} = 1000 kmitů za sekundu), dostáváme se z domů ven, „na 
    vzduch“, neboť máme co činit s frekvencemi používanými při rozhlasovém vysílání. (Se vzduchem to ale nemá 
    co dělat! Rádiové vlny se mohou šířit i v prostoru, v němž není vzduch.) Zvyšujeme-li frekvenci, 
    dostáváme se do oblasti \emph{VKV} a televizního vysílání. Při ještě vyšších frekvencích máme velmi 
    krátké vlny, které se využívají např. v \emph{radiolokaci}. Kdybychom šli ještě výše, nepotřebovali 
    bychom už zařízení na registraci takových vln, protože bychom je viděli naším zrakem. Kdybychom dokázali 
    pohybovat nabitým hřebenem tak rychle, aby kmital s frekvencemi od \SI{5e14}{\Hz} do \SI{5e15}{\Hz}, 
    viděli bychom toto kmitání jako červené, modré nebo fialové světlo v závislosti na frekvenci. Frekvence 
    pod touto oblastí nazýváme \emph{infračervenými} a nad touto oblastí \emph{ultrafialovými}. Skutečnost, 
    že naše vidění je omezeno na určitou frekvenční oblast, nedělá tuto oblast elektromagnetického spektra z 
    fyzikálního hlediska důležitější než jiné oblasti, avšak z lidského hlediska je tato oblast přece jen 
    zajímavější. Kdybychom frekvenci ještě zvýšili, dostali bychom \emph{rentgenové paprsky}. Tyto paprsky 
    nejsou nic jiného, než světlo s velmi vysokou frekvencí. Ještě vyšším frekvencím odpovídá \emph{záření 
    gama}. Výrazy rentgenové paprsky a záření gama jsou téměř synonyma. Zářením gama nazýváme obvykle 
    elektromagnetické vlny pocházející z jader a rentgenovými paprsky vlny pocházející z atomů; při shodě 
    jejich frekvencí jsou však fyzikálně nerozlišitelné, bez zřetele na jejich původ. Vlny ještě vyšších 
    frekvencí, řekněme \SI{10e24}{\Hz}, lze získat uměle, například na \emph{synchrotronu} v Caltechu. 
    Elektromagnetické vlny úžasně vysokých frekvencí (až tisíckrát vyšších) je možné najít ve vlnách 
    \emph{kosmického záření}. Tyto vlny však neumíme ovládat. \cite[s.~29]{Feynman02}
  
  \section{Kvantová Fyzika}
    Když jsme načrtli představu elektromagnetického pole, v němž se mohou šířit vlny, brzy zjistíme, že tyto 
    vlny se chovají nezvykle, jako kdyby to ani vlny nebyly. Při vyšších frekvencích se více podobají 
    \emph{částicím}! Jejich neobvyklé chování vysvětluje \emph{kvantová mechanika}, jejíž vznik je spojován s 
    obdobím těsně po roce 1920. Před rokem 1920 pozměnil Einstein obraz trojrozměrného prostoru a nezávislého 
    času nejdříve na kombinaci, kterou nazýváme \emph{prostoročasem} a potom na \emph{zakřivený} prostoročas, 
    aby vystihl gravitaci. „Scéna“ se změnila na prostoročas a o gravitaci předpokládáme, že je modifikací 
    prostoročasu. Zjistilo se dokonce, že zákony pro pohyb částic jsou nepřesné. Mechanické zákony 
    „setrvačnosti“ a „síly“ jsou \emph{nesprávné} - Newtonovy zákony neplatí ve světě atomů. Zjistilo se, že 
    věci se v malém měřítku chovají úplně jinak než věci ve velkém měřítku. To dělá fyziku obtížnou, ale 
    velmi zajímavou. Obtížnou proto, že chování věcí malých rozměrů je pro nás „nepřirozené“, nemáme v tomto 
    směru přímé zkušenosti. Věci se tu chovají úplně jinak, než jsme zvyklí, a proto není možné popsat jejich 
    chování jinak, než analyticky. Takový popis je těžký a vyžaduje mnoho představivosti.
    
    Kvantová mechanika má mnoho zvláštností. Především vylučuje předpoklad, že částice má určitou polohu a 
    určitou rychlost. Abychom ukázali, do jaké míry je klasická fyzika správná, uvedeme pravidlo kvantové 
    mechaniky, které říká, že není možné současně vědět, kde se něco nachází a jak rychle se to pohybuje. 
    Neurčitost v hybnosti a neurčitost v poloze jsou \emph{komplementární} a jejich součin je konstantní. 
    Můžeme to zapsat následujícím způsobem: \(\Delta x \Delta p \frac{\si{\planckbar}}{2\pi}\). Podrobněji 
    bude o tomto principu mluveno později. Vysvětluje se tím velmi záhadný paradox: jsou-li atomy složeny z 
    kladných a záporných nábojů, proč se záporný náboj prostě neusadí na kladném náboji (tyto náboje se 
    přitahují) a to tak těsně, že by ho úplně vyrušil? \emph{Proč jsou atomy tak velké}? Proč je jádro 
    uprostřed a elektrony okolo něho? Zpočátku se myslelo, že příčinou je velký rozměr jádra; jenže jádro je 
    velmi malé. Atom má průměr okolo \SI{10e-10}{\meter}. Jádro má průměr asi \SI{10e-15}{\meter}. Kdybychom 
    měli atom a chtěli bychom vidět jeho jádro, museli bychom ho zvětšit tak, aby dosáhl velikosti místnosti 
    a i potom by bylo jádro malé jako skvrnka, kterou sotva spatříte okem, ale téměř \emph{všechna hmotnost} 
    atomu připadá na toto nepatrné jádro. Co brání elektronu prostě spadnout na jádro? Právě uvedený princip. 
    Kdyby elektrony byly v jádru, znali bychom přesně jejich polohu a princip neurčitosti by si potom 
    vyžadoval, aby měly velmi velkou (ale \emph{neurčitou}) hybnost, tj. velmi velkou \emph{kinetickou 
    energii}. S takovou energií by se odtrhly od jádra. Dochází proto ke kompromisu: elektrony si ponechají 
    jakýsi prostor pro tuto neurčitost a potom se ve shodě s tímto pravidlem pohybují s jistým minimálním 
    množstvím pohybu. (Vzpomeňte si, že atomy krystalu při ochlazení na absolutní nulu neustaly ve svém 
    pohybu, ale přece jen kmitaly. Proč? Kdyby se přestaly pohybovat, věděli bychom, kde se 
    nacházejí a že mají nulový pohyb a to by bylo v rozporu s principem neurčitosti. Nemůžeme vědět, kde jsou 
    a jak rychle se pohybují; proto atomy musí neustále kmitat!)
    
    Jinou, velmi zajímavou změnou v ideách a filozofii vědy, kterou přinesla kvantová mechanika, je nemožnost 
    přesně předpovědět, co se za jakýchkoli daných okolností odehraje. Například, je možné připravit atom, 
    který bude emitovat světlo, a můžeme zjistit, kdy k této emisi došlo tím, že zachytíme foton (o tomto si 
    brzy řekneme více). Nemůžeme však dopředu předpovědět, kdy se uskuteční emise světla, nebo v případě více 
    atomů, který z nich bude emitovat světlo. Možná se domníváte, že je to proto, že v atomu se nacházejí 
    jakási vnitřní „kolečka“, která jsme ještě nerozeznali. Ne, taková vnitřní kolečka neexistují! Příroda, 
    tak jak ji dnes chápeme, se chová tak, že je principiálně nemožné přesně předpovědět, co se skutečně 
    stane v daném experimentu. 
    
    Opět se vrátíme ke kvantové mechanice a základní fyzice, ale nebudeme zabíhat do podrobností kvantově 
    mechanických principů, protože jsou dost těžké k pochopení. Budeme prostě předpokládat jejich existenci a 
    ukážeme, k jakým následkům vedou. Jedním z následků je, že věci, které jsme považovali za vlny, se 
    chovají jako částice a částice zase jako vlny; ve skutečnosti se tedy všechno chová stejně. Není rozdíl 
    mezi vlnou a částicí. \textbf{Kvantová mechanika sjednocuje myšlenku pole, jeho vln a částic vjedno.} Při 
    nízkých frekvencích je aspekt pole více zřejmý, resp. užitečnější pro přibližný popis vyjádřený řečí naší 
    každodenní zkušenosti. Se vzrůstem frekvence však zařízení, které obvykle používáme v experimentu, 
    poskytuje spíše důkazy o částicích. I když mluvíme o vysokých frekvencích, musíme přiznat, že v oblasti 
    frekvencí nad \SI{10e12}{\Hz} nebyl zatím zjištěn žádný jev přímo související s frekvencí. K existenci 
    vyšších frekvencí docházíme pouze úvahou vycházející z energie částic a předpokladu správnosti 
    \emph{vlnově-korpuskulární představy kvantové mechaniky}.
    
    Takto docházíme i k novému pohledu na \emph{elektromagnetickou interakci}. Kromě elektronu, protonu a 
    neutronu existuje nový druh částice. Tuto částici nazýváme foton. Nový pohled na interakci elektronů a 
    protonů, tj. \emph{elektromagnetickou teorii}, která zároveň \emph{splňuje} zákonitosti \emph{kvantové 
    mechaniky}, nazýváme \emph{kvantovou elektrodynamikou}. Tato základní teorie \emph{interakce světla a 
    hmoty}, nebo \emph{elektrického pole a nábojů}, je dosud největším úspěchem fyziky. V této jediné teorii 
    máme základní zákony, jimiž se řídí všechny známé jevy s výjimkou gravitace a jaderných procesů. Pomocí 
    kvantové elektrodynamiky můžeme vysvětlit všechny známé zákony mechaniky, elektřiny a chemie. Plynou, zní 
    zákony srážek kulečníkových koulí, pohyb vodičů v magnetickém poli i tepelná kapacita oxidu uhelnatého, 
    barva neonových reklam, hustota soli, reakce vodíku a kyslíku při vzniku vody - to vše jsou následky 
    jediného zákona. Všechny tyto detaily je možné získat, je-li situace dost jednoduchá na to, abychom ji 
    mohli přibližně popsat. To sice není splněno téměř nikdy, často však můžeme pochopit více či méně, co se 
    vlastně děje. Dosud se neobjevily žádné výjimky ze zákonů kvantové elektrodynamiky, až na atomová jádra. 
    O jádrech však nemůžeme říci, jestli jde v jejich případě o výjimku, protože vlastně nevíme, jaké procesy 
    v nich probíhají. Při budování teorie jádra musíme překonat tři hlavní problémy:
    \begin{enumerate}
     \item Není znám přesný tvar sil působících mezi nukleony v jádře,
     \item rovnice popisující pohyb nukleonů v jádře jsou velmi komplikované – problém matematického popisu,
     \item jádro má zároveň příliš mnoho nukleonů (nedá se popsat pohyb každé jeho částice) i příliš 
           málo (nedá se popsat jako makroskopické spojité prostředí).   
    \end{enumerate}
    Proto se musíme spokojit pouze s modely atomového jádra. 
    
    V podstatě je kvantová elektrodynamika teorií celé chemie a všech životních procesů, je-li možné život v 
    konečném důsledku redukovat na chemii, nebo vlastně na fyziku, protože chemie vede k fyzice (a ta část 
    fyziky, která se uplatňuje v chemii, je již dobře známá). Navíc, kvantová elektrodynamika - ta úžasná 
    vědní disciplína - předpověděla mnoho nových věcí. Především mluví o vlastnostech fotonů velmi velkých 
    energií, paprscích gama apod. Předpověděla i jinou, velmi pozoruhodnou věc: kromě elektronu musí 
    existovat jiná částice se stejnou hmotností, ale s opačným nábojem, tzv. \emph{pozitron} a elektron s 
    pozitronem mohou při srážce anihilovat, přičemž se vyzáří světlo nebo paprsky gama (což je vlastně totéž, 
    neboť světlo i záření gama se liší polohou ve frekvenční škále elektromagnetických vln). Zobecnění 
    poznatku, že ke každé částici existuje antičástice, se ukazuje být pravdivým. V případě elektronů má 
    antičástice jiné jméno - nazývá se pozitronem, ale u většiny jiných částic mluvíme o anti-tom a tom, 
    např. o antiprotonu nebo antineutronu. Do kvantové elektrodynamiky se vkládají \emph{dvě čísla} a o 
    většině ostatních čísel ve světě se předpokládá, že jsou následkem těchto dvou. Tato dvě vkládaná čísla 
    nazýváme hmotností a nábojem elektronu. Ve skutečnosti to však není úplně tak, neboť máme celý soubor 
    chemických čísel, která hovoří o tom, jak těžká jsou jádra. To nás přivádí k další kapitole.
  
  \section{Jádra a Částice}
    \emph{Z čeho jsou jádra a jak drží pohromadě}? Zjistilo se, že jádra jsou udržována obrovskými silami. 
    Při uvolnění těchto sil se uvolňuje energie, která je obrovská v porovnání s chemickou energií, tak jak 
    je obrovský výbuch atomové bomby v porovnání s výbuchem trinitrotoluenu. U atomové bomby jde totiž o 
    změny uvnitř jádra, zatímco výbuch trinitrotoluenu souvisí se změnami elektronového obalu atomů. Proto si 
    klademe otázku: co jsou to za síly, které udržují protony a neutrony v jádře pohromadě? Tak, jako je 
    možné elektrické působení přisoudit částici - fotonu, předpokládal Yukawa, že i síly mezi neutrony a 
    protony mají svá pole a kmity tohoto pole se chovají jako částice. Kromě neutronů a protonů by proto měly 
    existovat jiné částice a Yukawa odvodil vlastnosti těchto částic z již známých charakteristik jaderných 
    sil. Například, předpověděl, že by měly mít hmotnost dvěstě až třistakrát větší než elektron; a div se 
    světe - v kosmickém záření byly objeveny částice s takovouto hmotností! Později se ukázalo, že to nebyla 
    ta správná částice. Tuto částici nazvali \(\mu\text{-mezon}\) neboli \emph{mion}.
    
    Trochu později, v roce 1947 nebo 1948, byla objevena jiná částice, \(\pi\text{-mezon}\) neboli 
    \emph{pion}, která vyhovovala Yukawovu kritériu. Abychom získali jaderné síly, musíme k protonu a 
    neutronu přidat pion. A teď si řeknete: „Och, jak velkolepé! - pomocí této teorie vybudujeme 
    nukleodynamiku, ve které budou mít piony takovou úlohu, jakou jim přisoudil Yukawa a všechno bude 
    vysvětleno“. Ta věc má však háček! Ukázalo se, že výpočty v této teorii jsou tak složité, že se dodnes 
    nikomu nepodařilo odvodit všechny důsledky této teorie, nebo ji porovnat s experimentem; a to se už táhne 
    spoustu let!
    
    Máme tedy teorii, ale nevíme, jestli je správná nebo nesprávná. Víme však už, že je trochu chybná, nebo 
    aspoň neúplná. Zatím co jsme marnili čas teorií a snažili se odvodit její důsledky, experimentátoři 
    některé věci objevili. Například, objevili \(\mu\text{-mezon}\) neboli mion a my ani nevíme, jaká je jeho 
    úloha. V kosmickém záření se našel velký počet dalších „přebytečných“ částic. Dnes máme přibližně třista 
    takových částic a je velmi těžké porozumět vztahům mezi těmito částicemi a pochopit, na co je příroda 
    potřebuje, nebo která z nich na které závisí. Dnes tyto různé částice nechápeme jako různé aspekty téže 
    věci a skutečnost, že máme tak mnoho nesouvisejících částic, je odrazem toho, že máme tak mnoho 
    nesouvisejících informací bez dobré teorie. Po ohromném úspěchu kvantové elektrodynamiky máme jisté 
    znalosti z jaderné fyziky, ale jen hrubé znalosti, částečně experimentální a částečně teoretické. 
    Vycházíme přitom z charakteru sil působících mezi protony a neutrony a sledujeme, co z toho vyplyne, ale 
    v podstatě nechápeme, odkud ty síly pocházejí. Kromě toho nebylo dosaženo téměř žádného pokroku. 
    Objevili jsme velký počet chemických prvků. Mezi těmito prvky se najednou objevila souvislost, 
    neočekávaná souvislost zakotvená v Mendělejevově periodické tabulce prvků. Například, sodík a draslík 
    jsou téměř shodné ve svých chemických vlastnostech a v Mendělejevově tabulce se nacházejí ve stejném 
    sloupci. Hledala se tabulka Mendělejevova typu pro nové částice. Taková tabulka nových částic byla 
    sestavena nezávisle Gell-Mannem v USA a Nishijimou v Japonsku. Základem jejich klasifikace je nové číslo, 
    jež je možno, podobně jako elektrický náboj, přiřadit každé částici a které se nazývá její „podivností“ S 
    (od anglického slova strangeness). Toto číslo se, podobně jako elektrický náboj, zachovává v reakcích 
    vyvolávaných jadernými silami.  
   
\printbibliography[heading=subbibliography]
