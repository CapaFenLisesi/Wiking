% \documentclass{article}
% \usepackage{tikz}
% \usetikzlibrary{decorations.markings}
% \usetikzlibrary{intersections}
% \usepackage{subfigure}
% \usetikzlibrary{calc}

% \newcommand{\MyXYcross}{%
%           \draw[name path=axeX,->] (\xmin,\zero) -- (\xmax,\zero)   node[right] {$x$} coordinate(x axis);
%           \draw[name path=axeY,->] (\phase,\ymin) -- (\phase,\ymax) node[left]  {$y$} coordinate(y axis);
%           \path[name intersections={of=axeX and axeY, name=pocatek}]; 
%           \node[below left] at (pocatek-1) {$0$};
%           \draw[fill=white] (pocatek-1) circle(2pt);  
% }
% \begin{document}
 
  \begin{figure}[htb]  
    \centering
        \def\xmin{-15}
        \def\xmax{130}
        \def\ymin{-15}
        \def\ymax{+90}
        \def\zero{0}
        \def\phase{0}
        \def\period{48}
      \begin{tikzpicture}
        \begin{scope}[draw=black,line join=round, miter limit=4.00,line width=0.5pt,y=1pt,x=1pt] 
          \MyXYcross;   
        \end{scope}
        
        \begin{scope}[domain=0:2*pi, line join=round, miter limit=4.00,line width=0.5pt, 
                      x=10pt,y=30pt, xshift=25, yshift=45]      
          \draw[name path=sinx, color=blue, smooth]   
              plot[mark=triangle*] (\x,{sin(\x r)}); % r .. radian
          \draw[thin, dashed] 
                  (0,0)    node[left] {$f(a)$}  -- +(0,-1.5) node[below] {$a$} 
                  (2*pi,0) node[right] {$f(b)$} -- +(0,-1.5) node[below] {$b$}
                  ({pi*0.5},{sin(pi*0.5 r)}) coordinate(max) -- +(0,-2.5)  node[below] {$\xi_1$}
                  ({pi*1.5},{sin(pi*1.5 r)}) coordinate(min) -- +(0,-0.5)  node[below] {$\xi_2$};
           \draw[fill=white] (0,0) -- (2*pi,0) circle(1pt);           
           \draw[] (max) ++(-1,0)  -- +(2,0);                 
           \draw[] (min) ++(-1,0)  -- +(2,0);
           \draw[fill=white] (max) circle(1pt);
           \draw[fill=white] (min) circle(1pt);
           \draw[fill=white] (0,0) circle(1pt); 
           \draw[fill=white] (2*pi,0) circle(1pt);         
        \end{scope}  
      \end{tikzpicture}
    \caption{K výkladu Rolleovy věty}\label{MAI:fig_rolle_01}
  \end{figure}
  
%\end{document}