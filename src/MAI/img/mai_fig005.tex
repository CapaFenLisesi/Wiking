\begin{tikzpicture}[thick,scale=0.7, every node/.style={transform shape}]
  \begin{axis}[
    xmin = -3, xmax = 3, ymin = 0, ymax = 2,  % osy
    domain = -3:3,
    restrict y to domain=0:4,
    grid = major,   % both
    grid style={line width=.1pt, draw=gray!20},
    major grid style={dashed, line width=.2pt, draw=gray!40},
    minor tick num=5,
    clip = true,
    clip mode=individual,
    axis x line = middle,
    axis y line = middle,
    xlabel={$x$},
  %  xlabel style={at=(current axis.right of origin), anchor=west},
    ylabel={$y$},
  %  ylabel style={at=(current axis.above origin), anchor=south},
    enlarge y limits={rel=0.13},
    enlarge x limits={rel=0.07},
  %  title={Dirichletova funkce}
  ]
      
      \addplot[domain=-3:3, ultra thick,samples=10,blue] {1};
      \label{plot one}
      \addplot[domain=-3:3, ultra thick,samples=10,red] {0};
      \label{plot two} 
      \node [draw,fill=white] at (rel axis cs: 0.6,0.7) {\shortstack[l]{
      \(f(x) = 
        \left\lbrace\begin{array}{l@{}l@{}}
            1 & \text{ když } x \in \mathbb{Q}\\
            0 & \text{ když } x \in \mathbb{R} \setminus\mathbb{Q}
        \end{array}\right.
      \)
        }
         };
    \end{axis} 
\end{tikzpicture}