% !TeX spellcheck = cs_CZ

\begin{example}\label{mai:exam005}
  \textbf{Příklad s ježibabou:}\newline
  Jeníček a Mařenka kradli ježibabě perník. Dohromady snědli 11 perníkových srdíček. Jeníček jich 
  přitom zkonzumoval o 3 více než Mařenka. Otázka je tradiční — kolik srdíček snědl každý z nich?
  Označíme-li \(M\) počet kousků, které snědla Mařenka a \(J\) počet srdíček, na nichž si pochutnal 
  Jenda, můžeme informace zadané v úloze zapsat takto:
    \begin{equation*}
      M + J = 11 \qquad J = M + 3.
    \end{equation*}
  Řešení není problémem, snadno vidíme, že \(M = 4\) a \(J = 7\).
\end{example}