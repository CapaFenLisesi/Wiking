% !TeX spellcheck = cs_CZ
\wikitextrule
\begin{example}\label{MAI:exam026}
  (Určení vnitřní a vnější složky) Uveďme příklad dvou funkcí \(F(x)=\sqrt{x^2}\) a \(G(x) = 
  (\sqrt{x})^2\). Liší se tyto funkce, nebo jde o tutéž funkci, jen jinak zapsanou? Vidíme, že platí

  \begin{align*}
    \mathcal{D}_F &=\realset, F(x)=\abs{x}\forall x\in\mathcal{D}_F, \mathcal{H}_F = [0, infty),\\
    \mathcal{D}_G &=[0, \infty), G(x)=x\forall    x\in\mathcal{D}_G, \mathcal{H}_G = [0, infty).
  \end{align*}
  Funkce \(F\) a \(G\) mají různé definiční obory, ale na jejich průniku dávají stejné funkční 
  hodnoty. Ani zde však obecně nelze pořadí skládání funkcí zaměňovat.
\end{example}