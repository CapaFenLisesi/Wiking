% !TeX spellcheck = cs_CZ
%     An overview of high school mathematics

{\tikzset{external/prefix={tikz/MAI/}}
 \tikzset{external/figure name/.add={ch01_}{}}
%---------------------------------------------------------------------------------------------------
% intro_MA.tex
%---------------------------------------------------------------------------------------------------
\chapter{Historie matematické analýzy}\label{mai:IchapI}
\minitoc


  Analýza jako nezávislý předmět byla vytvořena v 17. stol. během vědecké revoluce. Kepler, 
  Galilei, Descartes, Fermat, Huygens, Newton a Leibniz, když zmíníme jen několik důležitých jmen 
  těch, kteří přispěli k jejímu vzniku. Otázky z mechaniky, optiky a astronomie hráli roli v jejím 
  raném období, tak jako vnitřní problémy matematiky, jako výpočet obsahů, objemů a analýza 
  komplikovaných křivek. Pohyb po zakřivených drahách působením proměnných sil, které se staly 
  předmětem důkladného zájmu po studiu volně padajících těles Galilea, vedl k počátečnímu úspěchu. 
  Z velké rozmanitosti snah, které se objevily na konci 17. stol. v práci Newtona a Leibnize, se 
  zrodila nová matematická disciplína, jejíž některé poznatky jsou v těchto studijních zápiscích.
  
  Základní myšlenka použití diferenciálních rovnic k získání pohledu na globální chování proměnných kvantit z jejich (infinitezimálních) změn prokázala základní a plodné výsledky daleko za hranicemi matematiky a fyzika a formovala náš souhrnný vědecký pohled na svět, zvláště na představu o kauzalitě. Na konci 18. stol., vskutku, největší vědci došli ke shodě, že procesy v přírodě (a společnosti) jsou determinovány a podřízeny zákonům, které mohou být popsány v podobě 
  diferenciálních rovnic. Laplace, tento mistr matematické fyziky, naznačil obraz nějaké fiktivní 
  vševědoucí inteligence, užívající úplnou znalost zákonů a stavu světa v daný časový okamžik, by 
  mohla předpovídat další vývoj světa navždy a hned. Myšlenka \emph{přírodních zákonů} byla kmotrem 
  při vytvoření matematického pojmu funkce a naopak nebyla by to myšlenka nikdy tak vlivná, kdyby 
  matematická analýza nevyvíjela úspěšné metody pro výzkum funkčních závislostí. 

} % tikzset
%---------------------------------------------------------------------------------------------------
%\printbibliography[title={Seznam literatury}, heading=subbibliography]
\addcontentsline{toc}{section}{Seznam literatury}
              