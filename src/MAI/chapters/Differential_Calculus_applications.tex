%----------------------------------------------------------------------------------------------------
% file: Differential_Calculus_applications.tex
%----------------------------------------------------------------------------------------------------
\chapter{Aplikace diferenciálního počtu}\label{chap:Apl_dif_poc}
\minitoc
\newpage   
%================Kapitola: Aplikace diferenciálního počtu ===========================================
  Diferenciální počet má rozsáhlou oblast užití. V této kapitole ukážeme použití výsledků předchozích kapitol k vyšetřování průběhu funkce a vlastnosti rovinných 
  křivek. 
  \section{Průběh funkce}
    Pomocí derivace můžeme studovat vlastnosti funkce, které usnadní vyšetřování jejího průběhu.  
    \subsection{Monotonie funkcí}
      Jednou z důležitých vlastností funkce je její \textquotedblleft monotonie\textquotedblright, kterou jsme definovali již v odst. 
      \ref{MA1:subsec_vlastnosti_funkce} kap. \ref{MA1:chap_Limita}. Proto je při vyšetřování průběhu funkce důležité určit množiny (často jsou to intervaly), na 
      nichž je funkce monotónní, jinak řečeno, najít \textquotedblleft intervaly monotonie funkce\textquotedblright (viz \cite[s.~208]{Brabec1989}). 
    \begin{enumerate}
      \item Zjistíme \textbf{definiční obor funkce}, vyjádříme jej v intervalech a z nich poznáme, kde je funkce \textbf{spojitá}. Funkce je spojitá v $(a,b)$ pro 
            každý bod tohoto intervalu, když$|f(x)-f(c)|<\varepsilon$, kde $\varepsilon>0$ je libovolně zvolené číslo, a pro všechna $x$ z okolí bodu $c$ je  
            $|x-c|<\delta$, kde $\delta>0$ je na $\varepsilon$ nezávislé.
      \item Určíme, je-li funkce \textbf{lichá} $f(-x)=-f(x)$ nebo \textbf{sudá} $f(-x)=f(x)$. Je-li funkce lichá, je souměrná podle středu souměrnosti (obyčejně to 
            bývá počátek souřadnic $xy$), je-li sudá, je souměrná podle osy $y$.
      \item Určíme \emph{průsečíky křivky s osami pravoúhlých souřadnic}. Body, ve kte\-rých křivka protíná osu $x$ spolu s body, ve kte\-rých není křivka spojitá, 
            rozlišují intervaly, v nichž je graf křivky nad osou $x$ od intervalů, ve kterých je graf křivky pod osou $x$.
      \item V krajních bodech definičních intervalů, ve kterých je funkce spojitá, stano\-víme \emph{limity funkce} a dále $$\lim_{x \to \pm \infty}f(x).$$
      \item Vypočítáme $f'(x)$ a $f''(x)$, abychom zjistily, kde je funkce \emph{rostoucí} $f'(x)>0$, \emph{klesající} $f'(x)<0$ a kde jsou \emph{lokální extrémy}. 
            Dostaneme-li dosazením kořenů rovnice $f'(x)=0$ do $f''(x)$ hodnotu $f''(x)>0$, má funkce lokální minimum, při $f''(x)<0$ má funkce lokální maximum. 
            V intervalech, kde $f''(x)>0$, je křivka \textbf{konvexní (vypuklá)}, kde $f''(x)<0$, je křivka \textbf{konkávní (vydutá)}. Body, v nichž $f''(x)$ mění 
            znaménko, jsou \textbf{inflexní body}. Najdeme je tak, že stanovíme hodnoty $x$, pro které je $f''(x)=0$ nebo neexistuje. Číslo $c$ je inflexní bod, když
            existuje takové okolí bodu $c$, že pro $x>c$ je oblouk křivky konvexní a pro $x<c$ konkávní. Je nutné si uvědomit, že když má $f'(x)$ konečnou derivaci, 
            je inflexní bod $c$ taky nulovým bodem druhé derivace čili kořenem rovnice $f''(x)=0$. Obrácená věta neplatí, tj. z $f''(x)=0$ nevyplývá, že v bodě $c$ 
            má $f'(x)$ extrém a že bod $c$ je inflexním bodem.
      \item \textbf{Asymptota} je tečna křivky $f(x)$, jejíž bod dotyku je v nekonečnu. Platí-li $$\lim_{x \to a}f(x) = \pm\infty,$$ je přímka $x=a$ její asymptotou.
            Jinak asymptoty mají rovnici $y=kx+q$, kde $x$ a $y$ jsou souřadnice bodů na asymptotách. Existují-li konečné limity 
            $$\lim_{x \to \pm\infty}\frac{f(x)}{x}=k$$  a $$lim_{x \to \pm\infty}[f(x)-kx] =q$$ pak je asymptotou přímka $y=kx+q$. Můžeme-li rovnici křivky rozložit 
            (tj. rozložit její pravou stranu, oby\-čejně dělením čitatele jmenovatelem, má-li tvar zlomku) na dvě části, z nichž jedna má tvar $kx+q$ a druhá zbytek 
            $\varphi(x)$, tj. $f(x)=kx+q+\varphi(x)$ a $\varphi(x)_{x\rightarrow\pm\infty}\rightarrow 0$, je přímka $y=kx+q$ asymptotou.
      \item Zpřesnění grafu křivky provedeme sestavením tabulky souřadnic dalších bodů křivky, tj. ke zvoleným hodnotám $x$ (z definičního oboru funkce) vypočítáme 
            hodnoty $y$. Do dalších řádků tabulky zapíšeme hodnoty  $f'(x)$ a $f''(x)$, ve kterých intervalech je funkce \emph{rostoucí}, ve kterých \emph{klesá}, 
            kde je \emph{vypuklá}, kde je \emph{dutá}, kde jsou \emph{lokální extrémy}, \emph{inflexní body} apod., případně sestavíme dílčí tabulky pro jednotlivé 
            \emph{charakteristické vlastnosti} vyšetřované funkce.
    \end{enumerate}
    \begin{example}Vyšetřete průběh funkce
       $$f(x):y=\frac{1+x^2}{1-x^2}$$
       \begin{enumerate}
         \item Definiční obor $D_f=\mathcal{R}-\{±1\}=(-\infty,-1)\cup(-1,1)\cup(1,+\infty)$
         \item Funkce je sudá $$f(-x)=f(x): \frac{1+x^2}{1-x^2}=\frac{1+(-x)^2}{1-(-x)^2}.$$ Funkce není periodická.
         \item Stanovíme funkční hodnoty v krajních bodech definičního obor $1, -1$ a v nevlastních bodech $-\infty,+\infty$.Protože je funkce \textbf{sudá}, omezíme
               se jen na vyšetřování nezáporné části. Nejprve vlastnosti fun\-kce v okolí bodu $1$. Ten nepatří do $D_f$ a proto určíme limity funkce v pravém a 
               levém okolí tohoto bodu. $$\lim_{x\to 1_{-}}=\frac{1+x^2}{1-x^2}.$$ Pro výpočet limity použijeme substituci $y=1-x^2$: 
               $$\lim_{y\to0+}\frac{2-y}{y}=+\infty$$ 
               \footnote{$\lim_{x\to0_+}\frac{1}{x}=\infty$} proto $$\lim_{x\to1_{-}}\frac{1+x^2}{1-x^2}=+\infty.$$ Obdobně dojdeme k 
               $$\lim_{x\to1_+}\frac{1+x^2}{1-x^2}=-\infty.$$ A konečně v nevlastních bodech $±\infty$ je limita 
               $$\lim_{x\to±\infty}\frac{1+x^2}{1-x^2}=\lim_{x\to\pm\infty}\frac{1}{1-x^2}+\lim_{x\to\pm\infty}\frac{x^2}{1-x^2 }=0-1=-1.$$ Výpočtem limit jsme 
               zároveň určili dva absolutní (globální) extrémy a jeden lokální:
               \begin{itemize}
                 \item v intervalu $(-1,1)$ má funkce maximum $\infty$ a minimum $1$,
                 \item v intervalech $(-1,1)\cup(1,+\infty)$ má funkce minimum $-\infty$ a maximum $-1$.
               \end{itemize}
         \item Nyní vyšetříme zda, případně kolik a jaké, má funkce $f(x)$ průsečíky s osami souřadnic.  S osou $x$ nemá funkce žádné průsečíky, protože pro $y=0$ 
               není definována $H_f=\mathcal{R}-\{-1,1\rangle$. Pro $x=0$ je $y=\frac{1+0^2}{1-0^2}=1$, proto má $f(x)$ právě jeden průsečík s osou $y$ a to $[0,1]$.
         \item Zatím jsme zjistili, že naše funkce není definována v bodech $1$ a $-1$ a proto není spojitá v $\mathcal{R}$. Nevíme však, jaký je její průběh v 
               jednotlivých intervalech definičního oboru.  Abychom získali názornější představu o průběhu funkce, zjistíme má-li derivaci.
               \begin{align*}
                 y' &= \frac{(1+x^2)'(1-x^2 )-(1+x^2)(1-x^2 )'}{(1-x^2)^2} \\
                 y' &= \frac{2x(1-x^2 )-(1+x^2 )(-2x)}{(1-x^2 )^2}         \\
                 y' &= \frac{4x}{(1-x^2 )^2}
               \end{align*}
               Protože má vlastní derivaci\footnote{$f(x)$ je spojitá v intervalech $(-\infty,-1),(-1,1),(1,\infty)$ věta s spojité funkci}, můžeme určit její 
               vlastnosti v intervalech $\langle0,1)$ a $(1,\infty)$. V těchto intervalech je $y'>0$ a proto jde o funkci ryze monotónní, rostoucí \footnote{Plyne z 
               věty o postačujících podmínkách ryzí monotónnosti funkce na intervalu} v daných intervalech \footnote{V intervalech $(-\infty,-1),(-1,0\rangle$ je 
               funkce klesající.}. Výpočtem zjistíme druhou derivaci funkce. Ta nám pomůže určit další extrém v intervalu $\langle0,1)$ a zároveň vyšetřit 
               \textbf{konkávnost} a \textbf{konvexnost}.
               \begin{align*}
                 y'' &= \frac{(4x)' (1-x^2 )^2-(4x)(1-2x^2+x^4 )'}{(1-x^2 )^4}  \\
                 y'' &= \frac{4(1-2x^2+x^4 )-4x(-4x+4x^3 )}{(1-x^2 )^4}         \\
                 y'' &= \frac{4(1-x^2 )(3x^2+1)}{(1-x^2 )^4}                    \\
                 y'' &= \frac{4(3x^2+1)}{(1-x^2 )^3}
               \end{align*}
             Abychom mohli určit lokální extrém funkce $f(x)$ v intervalu $\langle0,1)$, pomocí druhé derivace, musíme najít kořeny rovnice $f' (x)=0$. V našem 
             případě $$y'=\frac{4x}{(1-x^2 )^2}\Rightarrow\frac{4x}{(1-x^2 )^2}=0\rightarrow x_0=0,$$ tento kořen \footnote{stacionární bod}  pak dosadíme do druhé 
             derivace, tj. $$y''(0)=\frac{4(3\cdot0^2+1)}{(1-0^2 )^3} =4,$$ protože je $f''(x)>0$, má v bodě $x_0$ lokální minimum. Můžeme rovněž konstatovat, že 
             funkce nemá inflexní body \footnote{Pro existenci inflexního bodu je nutné splnění jedné z podmínek a to buď $f''(x_0)=0$, nebo $f''(x_0)$ neexistuje.}. 
             Konkávnost a konvexnost funkce v intervalech $\langle0,1)$ a $(1,\infty)$ vyšetříme pomocí vlastností druhé derivace funkce. Tedy
             \begin{itemize}
               \item $\langle0,1): y''=\frac{4(3x^2+1)}{(1-x^2 )^3} >0 \Rightarrow$ funkce je v tomto intervalu \textbf{konvexní},
               \item $(1,\infty): y''=\frac{4(3x^2+1)}{(1-x^2 )^3} <0 \Rightarrow$ funkce je v tomto intervalu \textbf{konkáv\-ní}.
             \end{itemize}
           \item Z předchozích výpočtů plyne, že křivka má asymptoty $y=-1,x=\pm1$.
       \end{enumerate}
         %----------------------------------
          % image: MAI_rolle_01.tex label: \label{MAI:fig_diff_app_01}
          \input{../src/MAI/img/MAI_diff_app_01.tex}  
        %----------------------------------       

    \end{example}
%----------------------------------------------------------------------------------------------------
%    
\printbibliography[heading=subbibliography]
\addcontentsline{toc}{section}{Seznam literatury} 