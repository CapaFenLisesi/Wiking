%====================================================================================================
% --------------------------------------------- Definite Integral -----------------------------------
\chapter{Určitý integrál}
\minitoc
\newpage  
\section{Motivace} 
  %----------------------------------
  % image: MAI_animated_integral.tex label: \label{MAI:fig_anim_int}
    \input{../src/MAI/img/MAI_animated_integral.tex}  
  %---------------------------------- 

\subsection{Výpočet integrálu}
    \begin{example}Metodou per partes spočítejte integrály:$\displaystyle\int_1^{ln5}{(x+1)e^xdx}$
      \begin{align*}
        \int{(x+1)e^xdx} &= \int{e^xdx}+\int{x\cdot e^xdx} \\
                         &= e^x + (x-1)e^x = xe^x \\
        \int_1^{ln5}{(x+1)e^xdx} &= [xe^x]_1^{ln5} = 5ln5-e\\
      \end{align*}
      kde integrál
      \begin{equation*}
          \int{xe^xdx}=
            \left[\begin{array}{cc}
              u=x   & dv=e^x \\
              du=dx & v=e^x
            \end{array}\right]=
            xe^x-\int{e^xdx} = xe^x - e^x+C
      \end{equation*}
    \end{example}

\newpage
\section{Vlastnosti určitého integrálu}
  V této kapitole mluvíme o spojitých funkcích $\Rightarrow$ příslušné integrály tedy vždy
  existují. Čerpáno z knih:
  \cite{Knichal}.

  \begin{lemma}
    \textbf{První věta o střední hodnotě integrálního počtu}: Je-li funkce $f(x)$ spojitá v
    intervalu $\langle a, b\rangle$, existuje alespoň jeden takový bod $c\in(a, b)$, že platí

    \begin{equation}\label{MA:eq_av1}
      \int_a^b f(x)dx = (b-a)f(c).
    \end{equation}
  \end{lemma}

  \begin{proof} Použitím Lagrangeovy věty napsané pro funkci $F(x)$, primitivní na intervalu
    $\langle a, b\rangle$ k dané funkci $f(x)$. Podmínky věty jsou zřejmě splněny: $F(x)$ je
    spojitá na intervalu $\langle a, b\rangle$ a má všude derivaci $F'(x)= f(x)$. Tedy existuje
    alespoň jeden bod $c\in(a, b)$,
    
      %----------------------------------
        % image: MAI_rolle_02.tex label: \label{MA:fig_rolle_02} 
        \input{../src/MAI/img/MAI_rolle_02.tex}  
      %----------------------------------   
    
      že $$F(b)-F(a) = (b-a)F'(c),$$ čímž je věta dokázána, neboť $F(b)-F(a) = \int_a^bf(x)dx$ a
      $F'(c) = f(c)$. Funkční hodnotu $f(c)$, danou podle (\ref{MA:eq_av1}) rovnicí  
      \begin{equation}\label{MA:eq_av2}
         f(c) = \frac{1}{b-a}\int_a^b f(x)dx
      \end{equation}
      nazýváme \texttt{střední hodnotou}.
  \end{proof}

  Pro spojitou nezápornou funkci $f(x)$, lze větu o střední hodnotě jednoduše geometricky
  interpretovat dle (obr.\ref{MAI:fig_rolle_02}). Levá strana (\ref{MA:eq_av1}) určuje obsah
  křivočarého lichoběžníka $ABCD$, pravá strana obsah obdélníka $ABEF$. Podle této věty nabývá
  funkce $f(x)$ aspoň v jednom bodě intervalu $(a, b)$ takové hodnoty $f(c)$, že uvažovaný
  křivočarý lichoběžník má stejný obsah jako obdélník o základně $b-a$ a výšce $f(c)$ (str. 155
  knihy \cite{Knichal}).

  \begin{example} Určete střední hodnotu $i_s$ střídavého proudu $$i(t) = I_0\sin\omega t$$ v
    časovém intervalu $\langle 0, \frac{T}{2}\rangle$ (v průběhu jedné poloviny periody). $I_0$ je
    maximální hodnota proudu (obr. \ref{MA:fig_Iav_exam}), perioda $T$ je dána vztahem $T =
    \frac{2\pi}{\omega}$
    %----------------------------------
      % image: MAI_rolle_02.tex label: \label{MA:fig_Iav_exam}
      \input{../src/MAI/img/MAI_Iav_exam.tex}  
    %----------------------------------

    Podle \ref{MA:eq_av2} bude
    \begin{align*}
     i_s &=  \frac{2}{T}
             \int_0^{\frac{T}{2}}I_0\sin\omega t\dd{t} =
             \frac{2I_0}{T}\left[-\frac{\cos\omega t}{\omega}\right]_0^{\frac{T}{2}}        \\
         &=  \frac{2I_0}{T}\frac{1}{\omega}\left(-\cos\frac{\omega T}{2}+ \cos 0\right)     \\
         &=  \frac{2I_0}{2\pi}(-\cos\pi + \cos 0) = \frac{2}{\pi}I_0 \doteq 0,637 I_0.
  \end{align*}

  Tato hodnota se rovná intenzitě elektrického proudu, při kterém by vodičem v průběhu uvažované
  poloviny periody prošel stejný elektrický náboj jako při proudu střídavém.
  \end{example}

  \begin{example} Efektivní hodnota $i_{ef}$ střídavého proudu $$i(t) = I_0\sin\omega t$$ (viz
    předchozí příklad) je definována jako odmocnina ze střední hodnoty funkce $i^2(t)$ v průběhu
    jedné periody $T = \frac{2\pi}{\omega}$. Tedy
    \begin{align*}
      i_{ef}^2 &= \frac{1}{T}\int_0^T I_0^2\sin^2\omega t\dd{t} = 
                  \frac{1}{T}\int_0^T \frac{I_0^2}{2}(1- \cos2\omega t)\dd{t}           \\
               &= \frac{I_0^2}{2T}
                  \left[
                    t-\frac{\sin2\omega t}{2\omega}
                  \right]_0^T = \frac{I_0^2}{2}
    \end{align*}
    neboť $\sin2\omega T=\sin4\pi = 0.$ Odtud $$i_{ef} = \frac{I_0}{\sqrt{2}}.$$ Střídavý proud
    $i(t) = I_0\sin\omega t$ má na témže odporu stejný výkon jako stejnosměrný proud o intenzitě
    $i = 0,707I_0$.
  \end{example}
  Následující věta může být využita k odhadu některých integrálů
  \begin{lemma}
    \textbf{Druhá věta o střední hodnotě integrálního počtu}: Jsou-li funkce $f(x)$ a $g(x)$
    spojité v intervalu $\langle a, b \rangle$ a je-li funkce $g(x)$ v $\langle a, b \rangle$
    nezáporná a nerostoucí, existuje alespoň jeden bod $c\in\langle a, b \rangle$ tak, že platí
    \begin{equation}\label{MA_eq_av3}
        \int_a^b f(x)g(x) = g(a)\int_a^c f(x)dx.
    \end{equation}
  \end{lemma}
  Zcela obdobnou větu lze vyslovit pro případ, že $g(x)$ je v intervalu $\langle a, b \rangle$
  nezáporná a neklesající, tj. na pravé straně \ref{MA_eq_av3} je pak integrál $g(b)\int_c^b
  f(x)dx$

  \begin{example} Odhadněte hodnotu integrálu
    \begin{equation}\label{MA_eq_sinx_x}
        \int_{100\pi}^{1000\pi}\frac{\sin x}{x}dx
    \end{equation}
    Řešení: Funkce $f(x) = \sin x$ a $g(x) = \frac{1}{x}$ jsou v uvažovaném intervalu $\langle
    100\pi, 1000\pi \rangle$ spojité a funkce $g(x)$ je kladná a nerostoucí.
    \begin{equation*}
      \int_{100\pi}^{1000\pi}\frac{\sin x}{x}dx = 
      \frac{1}{100\pi}\int_{100\pi}^c\sin xdx =\frac{1}{100\pi}\left(\cos100\pi - \cos c\right)
    \end{equation*}
    kde $c$ je kladné číslo z intervalu $\langle 100\pi, 1000\pi \rangle$. Dále pro všechna
    $c\in\langle 100\pi, 1000\pi \rangle$ platí $0\leq1-\cos c\leq2$, takže
    \begin{equation*}
        0\leq\int_{100\pi}^{1000\pi}\frac{\sin x}{x}dx\leq \frac{1}{50\pi}.
    \end{equation*}
  \end{example}   
%---------------------------------------------------------------------------------------------------
\printbibliography[heading=subbibliography]
\addcontentsline{toc}{section}{Seznam literatury}