\sisetup{input-complex-roots = j, 
         output-complex-root = \ensuremath{\mathrm{j}}}
         
%==================Kapitola: Teoretické základy radioelektroniky=====================================
\chapter{Teoretické základy radioelektroniky}\label{chap:ra_theory}
\minitoc
\newpage
  %--------------- Základní klasifikace radioelektronických systémů ------------------------------------------
  \section{Základní klasifikace radioelektronických systémů}
    Pod pojmem systém \cite[s.~56]{ZaludRA} 

  \section{Dvojbrany v radioelektronice}
    Pod pojmem \emph{systém} se zde označuje elektrický obvod nebo soustava obvodů, které vytvářejí alespoň 
    jeden výstupní signál jako odezvu na alespoň jeden vstupní signál. Jednou z nejdůležitějších tříd 
    radioelektronických systémů jsou lineární systémy s časově invariantními (neměnnými) parametry, které 
    mají podobu \emph{n-branů}. Mezi nimi potom zaujímají významné místo lineární dvojbrany a trojbrany, 
    jejichž popisem a obecnými vlastnostmi se zabývá tato kapitola. Připomeňme, že pro označení linearity a 
    časové invariance se v teorii obvodů používá zkratka \emph{LTI- (Linear Time Invariant)}, což nijak dále 
    nezdůrazňujeme, nebo tato kapitola je zaměřena právě jen na dvojbrany a trojbrany tohoto typu.
    
    V radioelektronice se často vyskytují dvojbrany a trojbrany, které přísně vzato lineární nejsou, avšak 
    jejich nelinearity jsou tak malé, že je lze v technické praxi zanedbat; do této kategorie patří například 
    všechny obvody s diodami a s tranzistory, pracujícími v režimu malých signálů, kde jsou prakticky 
    lineární a nelinearity se u nich začínají projevovat až při vyšších úrovních signálů. Tyto dvojbrany a 
    trojbrany, označované termínem kvazilineární (tedy téměř lineární) Jsou rovněž zahrnuty do této 
    kapitoly. Zvláštní pozornost je věnována zejména jejich vlastnostem v hraniční oblasti mezi lineárním a 
    nelineárním režimem, která je v radioelektronice velmi důležitá.
    
    \subsection{Admitanční parametry a rozptylové parametry dvojbranů}
      Pod pojmem dvojbran (\emph{two-port}) se rozumí elektrický systém, který má jeden pár vstupních svorek 
      a jeden pár výstupních svorek, tedy jednu vstupní a jednu výstupní bránu (dvojbran zřejmě představuje 
      zvláštní třídu obecnějšího pojmu \emph{čtyřpól}, který má čtyři nezávislé svorky). Linearizované 
      přenosové vlastnosti dvojbranů je možné charakterizovat pomocí jejich admitančních a rozptylových 
      parametrů, kterým je dále věnována náležitá pozornost.
      
      Při definici \emph{admitančních parametrů} dvojbranů, nazývaných také parametry \emph{y (Admittance 
      Parameters)}, vyjdeme z obr. 4. la. Zde je znázorněn lineární časově invariantní dvojbran, který má na 
      vstupu svorkové napětí a proud \(u_l\), \(i_1\) a na výstupu \(u_2\), \(i_2\). K jeho vstupu je 
      připojen generátor s vnitřní admitancí \(Y_g\) a k výstupu zatěžovací admitance \(Y_z\). Admitanční 
      parametry tohoto dvojbranů jsou definovány relacemi
      \begin{align}
        i_1 &= y_{11}u_1 + y_{12}u_2   \\ 
        i_2 &= y_{21}u_1 + y_{22}u_2    
      \end{align}
  
    \subsection{Vektorové měření impedance a přizpůsobení}
      \subsubsection{Činitel odrazu, přepočet na impedanci, poměr stojatého vlnění}
        Komplexní činitel odrazu \(\Gamma_L\) zátěže o impedanci \(Z_L\) připojené na konec homogenního 
        vedení o vlnové impedanci \(Z_0\) je v rovině připojení impedance \(Z_L\) definován jako poměr fázoru 
        harmonické napěťové vlny \(U^-\) na svorkách zátěže \(Z_L\), která se od této zátěže odráží (odražená 
        vlna), ku fázoru napěťové vlny \(U^+\) na svorkách zátěže \(Z_L\), která na zátěži \(Z_L\) přichází 
        po vedení (postupná vlna), tj.
        \begin{equation}\label{RA:eq_smith01}
          \Gamma_L = \frac{U^-}{U^+}
        \end{equation}
        Činitel odrazu je v této rovině funkcí pouze \(Z_0\) a \(Z_L\), přičemž obě tyto impedance mohou být 
        kmitočtově závislé (a také bývají, zejména \(Z_L\)).
  
        Mezi činitelem odrazu \(\Gamma_L\) a impedancemi \(Z_0\) a \(Z_L\) lze odvodit vztah
        \begin{equation}\label{RA:eq_smith02}
          \Gamma_L = \frac{Z_L-Z_0}{Z_L-Z_0}.
        \end{equation}    
        Pokud známe vlnovou impedanci vedení \(Z_0\) a činitel odrazu \(\Gamma_L\), lze ze vztahu 
        \ref{RA:eq_smith02} snadno vyjádřit zatěžovací impedanci \(Z_L\)
        \begin{equation}\label{RA:eq_smith03}
          Z_L = Z_0\frac{1+\Gamma_L}{1-\Gamma_L},
        \end{equation}    
  
        často udávaným parametrem popisujícím míru nepřizpůsobení je poměr stojatého vlnění – \textbf{PSV} 
        \emph{(Voltage Standing Wave Ratio – VSWR)}. Je definován jako poměr maximální a minimální hodnoty 
        amplitudy tzv. stojatého vlnění, které vzniká součtem postupné a odražené vlny na vedení. Poměr 
        stojatého vlnění lze vyjádřit taktéž pomocí vlnové impedance vedení \(Z_0\) a zatěžovací impedance 
        \(Z_L\), a tedy i pomocí činitele odrazu \(\Gamma_L\)
        \begin{equation}\label{RA:eq_smith04}
          PSV = \frac{U_{max}}{U_{min}} 
              = \frac{\abs{Z_L+Z_0} + \abs{Z_L-Z_0}}{\abs{Z_L+Z_0} - \abs{Z_L-Z_0}} 
              = \frac{1+\abs{\Gamma_L}}{1-\abs{\Gamma_L}}.
        \end{equation}
        Poslední výraz umožňuje odpoutat definici poměru stojatého vlnění od vedení a pracovat s ním podobně 
        jako s velikostí činitele odrazu \(\Gamma_L\).
      
      \subsubsection{Impedanční přizpůsobení pomocí obvodů se soustředěnými parametry}
        Přivádíme-li vysokofrekvenční signál z generátoru o vnitřní impedanci \(Z_G\) na zátěž o impedanci 
        \(Z_L\), je obvykle žádoucí dosažení stavu tzv. \emph{impedančního přizpůsobení}. Přitom rozlišujeme 
        dva typy impedančního přizpůsobení. Prvním je \emph{impedanční přizpůsobení pro maximální přenos 
        výkonu}, pro které platí
        \begin{equation}\label{RA:eq_smith05}
          Z_L = Z^*_G,
        \end{equation}  
        kde \(Z^*_G\) je komplexně sdružená hodnota k vnitřní impedanci generátoru \(Z_G\). V tomto stavu 
        obdržíme na zátěži \(Z_L\) největší činný výkon, jaký je generátor vůbec schopen do nějaké zátěže 
        dodat – tzv. \emph{dosažitelný výkon}.
  
        Druhým typem je \emph{bezodrazové impedanční přizpůsobení}, vhodné v případě, kdy na zátěž \(Z_L\) 
        přivádíme signál po vedení o vlnové impedanci \(Z_0\) nezanedbatelné délky vzhledem k vlnové délce 
        signálu. V tomto stavu platí
        \begin{equation}\label{RA:eq_smith06}
          Z_L = Z_0,
        \end{equation}  
        a jeho výhodou je, že při něm na konci vedení nedochází k odrazům, které by jinak mohly kromě 
        zhoršení účinnosti přenosu výkonu způsobit také degradaci kvality signálu (v případě odrazů na obou 
        koncích vedení, např. tzv. „duchy“ v televizním obrazu způsobené právě nepřizpůsobeným anténním 
        napáječem). Jelikož u většiny vedení máme v praxi téměř nulovou imaginární část vlnové impedance 
        \(Z_0\) (přesně nulovou mají bezeztrátová vedení), je bezodrazové impedanční přizpůsobení obvykle 
        současně přizpůsobením pro maximální přenos výkonu (avšak pouze za předpokladu, že je provedeno na 
        obou koncích vedení, tj. jak na straně zátěže, tak na straně generátoru).
  
        V našem případě budeme bezodrazově přizpůsobovat určitou zatěžovací impedanci \(Z_L\) k vedení o 
        reálné vlnové impedanci \(Z_0 = \SI{50}{\ohm}\) (půjde tedy současně i o přizpůsobení na maximální 
        přenos výkonu). To znamená, že budeme navrhovat přizpůsobovací obvod (viz obr. 
        \ref{fyz:fig_RA_smith01}), který zakončen na svém  výstupu impedancí \(Z_L\) bude při daném kmitočtu 
        vykazovat vstupní impedanci rovnou \(Z_0 = \SI{50}{\ohm}\).
        \begin{figure}[ht!] %\ref{fyz:fig_RA_smith01} 
          \centering
          \includegraphics[width=0.5\linewidth]{RA_smith01.png}
          \caption{Umístění přizpůsobovacího obvodu mezi zátěží a zdrojem signálu (vedením)}
          \label{fyz:fig_RA_smith01} 
        \end{figure}
        K tomuto účelu vystačíme s nejjednodušším typem zapojení, které umožňuje dosažení přizpůsobení na 
        jednom kmitočtu s tzv. \(\Gamma\text{-články}\). Jde o kaskádní řazení dvojice reaktančních prvků, z 
        nichž jeden je vždy v podélné a druhý v příčné větvi bez ohledu na pořadí. V rezistivních obvodech 
        totiž dochází ke ztrátám přenášeného výkonu, zatímco obvody složené z reaktancí slibují (alespoň 
        teoreticky) dosažení přizpůsobení beze ztrát. Kombinací obou pořadí s typy reaktančních prvků 
        (kondenzátor, cívka) dostáváme celkem osm různých možností uspořádání \(\Gamma\text{-článků}\), 
        přičemž volba konkrétní struktury závisí především na hodnotě přizpůsobované impedance \(Z_L\). 
        Obrázek \ref{fyz:fig_RA_smith02}  ukazuje jednotlivé vhodné struktury v závislosti na umístění 
        impedance \(Z_L\) v impedančním Smithově diagramu v jedné z osmi oblastí \emph{A} až \emph{H}, při 
        aplikaci požadavku přesunu z bodu \(Z_L\) do bodu \(Z_0\) co nejkratší cestou. Hlavní myšlenka zde 
        spočívá v tom, že do bodu \(Z_0\) se lze dostat jedině pohybem po kružnici jednotkové reálné části 
        impedance (\(r = 1\)) nebo admitance (\(y = 1\)), a to buď po směru nebo proti směru hodinových 
        ručiček. To zajišťuje první prvek \(\Gamma\text{-článku}\) – sériově nebo paralelně řazený induktor 
        nebo kapacitor. Z toho plynou čtyři varianty označené malými písmeny „\emph{a}“ až „\emph{d}“. Úkolem 
        druhého stupně je přesun z bodu \(Z_L\) na jednu (obvykle tu nejbližší) z obou zmíněných kružnic opět 
        sériově nebo paralelně řazeným induktorem nebo kapacitorem – odtud tedy osm variant přizpůsobení v 
        oblastech \emph{A} až \emph{D}. Způsob návrhu \(\Gamma\text{-článku}\) pomocí Smithova diagramu 
        nejlépe objasníme na číselném příkladu.
  
        \begin{figure}[ht!] %\ref{fyz:fig_RA_smith02} 
          \centering
          \includegraphics[width=\linewidth]{RA_smith02.png}
          \caption{Vhodné způsoby přesunu ve Smithově diagramu a odpovídající struktury  
          \(\Gamma\text{-článků}\)}
          \label{fyz:fig_RA_smith02} 
        \end{figure}
  
        \begin{example}
          Navrhněte graficko-početní metodou ve Smithově diagramu přizpůsobovací obvod pro 
          \(Z_L = \SI{30 - 65j}{\ohm}\) , \(Z_0 = 50\)  a \(f = \SI{100}{\MHz}\).
  
          \textbf{Řešení:}
            Platí \(z_L = \dfrac{Z_L}{Z_0} = \num{0.6 - 1.3j}\), daný bod se tedy nachází ve 4. kvadrantu, 
            vně kružnice konstantní reálné části impedance \(r = 1\). Proto volíme strukturu typu \(H\). 
            Začínáme v admitančních souřadnicích. Pro posun z bodu \(y_L = \frac{1}{z_L} = \num{0.29 + 
            0.64j}\) nejkratší cestou na kružnici \(r = 1\), tj. do bodu \(y_1 = \num{0.29 + 0.45j}\), musíme 
            z admitance \(y_L\) ubrat normovanou susceptanci \(\num{0.64} - \num{0.45} = \num{0.19} = \Delta 
            b = \dfrac{\omega L_2}{Z_0}\). Paralelní indukčnost \(L_2\) tedy bude
            \MULTIPLY{2}{\numberPI}{\nmbrA}
            \MULTIPLY{\nmbrA}{100}{\nmbrB}
            \MULTIPLY{\nmbrB}{0.19}{\nmbrC}
            \DIVIDE{50}{\nmbrC}{\nmbrD}
            \MULTIPLY{1000}{\nmbrD}{\nmbrE}
            \ROUND[2]{\nmbrE}{\sol}
            \begin{equation*}
              L_2 = \frac{Z_0}{\omega\abs{\Delta b}} = 
              \frac{50}{2\cdot\pi\cdot\num{100e6}\cdot\num{0.19}} = \SI{\sol}{\nano\henry}
            \end{equation*}
            Pro posun z bodu \(z_1 = \dfrac{1}{y_1} = \num{1 - 1.59j}\) do bodu \(z_0 = \num{1 + 0j}\) musíme 
            k \(z_1\) přidat normovanou reaktanci \(1.59 = \Delta x = \frac{\omega L_2}{Z_0}\). Sériová 
            indukčnost \(L_1\) tedy bude
            \MULTIPLY{2}{\numberPI}{\nmbrA}
            \MULTIPLY{\nmbrA}{100}{\nmbrB}
            \MULTIPLY{50}{1.59}{\nmbrC}
            \DIVIDE{\nmbrC}{\nmbrB}{\nmbrD}
            \MULTIPLY{1000}{\nmbrD}{\nmbrE}
            \ROUND[2]{\nmbrE}{\sol}
            \begin{equation*}
              L_1 = \frac{\Delta x Z_0}{\omega} = \frac{\num{1.59}\cdot\num{50}}{2\cdot\pi\cdot\num{100e6}} 
                  = \SI{\sol}{\nano\henry}
            \end{equation*}  
        \end{example}
        
        Kromě přesunu nejkratší cestou přicházejí v úvahu ještě další varianty spojené s přechodem do jiného 
        segmentu. Např. pro \(Z_L\) v oblasti \(A\) lze použít též \(\Gamma\text{-článek}\) typu \(B\); pro 
        \(Z_L\) v oblasti \(E\) lze použít též typy \(B\), \(D\) nebo \(G\) apod. Detailnější grafické 
        znázornění oblastí použitelnosti všech osmi typů \(\Gamma\text{-článků}\) naleznete v příloze. Pro 
        dosažení nejmenší kmitočtové závislosti lze však obvykle doporučit pouze přesun nejkratší cestou.
  
      \subsubsection{Smithův diagram}
        \emph{Impedanční Smithův diagram} je soustava křivek (přesněji kružnic nebo jejich částí) tvořených 
        body normované impedance \(z\) s konstantní reálnou či imaginární částí zakreslených v komplexní 
        rovině činitele odrazu \(\Gamma\) definovaná konformním zobrazením
        \begin{equation}\label{RA:eq_smith07}
          \Gamma = \frac{z-1}{z+1}.
        \end{equation} 
  
        Každému bodu v Smithově diagramu odpovídá jedna hodnota normované impedance z = r + jx, a odnormované 
        impedance Z = z Z0 = R+jX. Otočením o 180±, popř. středovou souměrností, získáme admitanční Smithův 
        diagram y = g + jb, resp. Y = y=Z0 = G + jB. Kružnice konstantních reálných částí se sbíhají v bodě 
        1a mají středy na reálné ose. Kružnice (části kružnic) konstantní imaginární části se taktéž sbíhají 
        v bodě 1a mají středy na přímce Re [] = 1.
  
        Jelikož pro převod mezi poměrem stojatých vln a modulem činitele odrazu j􀀀j platí stejný vztah jako 
        mezi činitelem odrazu 􀀀 a normovanou impednací z, tj.
        \begin{equation}\label{RA:eq_smith08}
          \Gamma = \frac{z-1}{z+1} \qquad \Leftrightarrow 
          \abs{\Gamma} = \frac{\abs{PSV}-1}{z+\abs{PSV}}.
        \end{equation} 
        s rozdílem, že PSV je vždy reálné číslo větší nebo rovno jedné, lze PSV pro daný bod ve Smithově 
        diagramu určit z průsečíku kružnice se středem v bodě z = 0 a poloměrem j􀀀j a části reálné osy 
        normované impedance mezi body z = 1 a z ! 1.
        
      \subsubsection{Měření činitěle odrazu} 
        Abychom mohli měřit činitel odrazu podle definice (1), musíme mít možnost
        \begin{itemize}
          \item oddělit vlnu odraženou (U¡) od vlny dopadající (U+)
          \item měřit komplexní poměr fázorů napětí na příslušném kmitočtu.
        \end{itemize} 
        
        Splnění prvního požadavku nám zajišťuje tzv. směrový vazební člen (SVČ), splnění druhého požadavku 
        pak vektorový voltmetr. Celkové uspořádání měřicí sestavy využívající směrového vazebního členu a 
        vektorového voltmetru znázorňuje obr. 3. Harmonický signál potřebného kmitočtu se rozděluje pomocí 
        symetrického přizpůsobeného T-článku (splitteru) do tzv. referenční a měřicí větve. Obě větve jsou 
        zakončeny terminátory o impedanci rovné vlnové impedanci Z0 vedení a konektorů, aby zde nedocházelo k 
        nežádoucím odrazům a stojatému vlnění. Ke snížení stojatého vlnění též přispívají pevné útlumové 
        články v obou větvích (20, 14 a 6 dB). Napětí v obou větvích jsou snímána sondami vektorového 
        voltmetru zasunutými do průchozích držáků. Měřená zátěž ZL je připojena na konec hlavního vedení 
        směrového vazebního členu SVč . Odražená vlna se potom vyvazuje na bránu 4 (přes vazbu K2 SVČ) a její 
        amplituda a fáze je měřena sondou B vektorového voltmetru. Sonda A v referenční větvi poskytuje 
        fázovou referenci pro stanovení fáze činitele odrazu.
  
        Měření činitele odrazu probíhá ve dvou krocích:
        \begin{itemize}
          \item Nastavení amplitudové a fázové referenční hodnoty. Měřicí aparaturu je třeba zkalibrovat tak, 
          aby udávaný modul a fáze činitele odrazu pro nějakou zátěž, jejíž činitel odrazu předem známe, této 
          hodnotě přesně odpovídal. K tomuto účelu se obvykle používá zkrat (Z = 0, 􀀀 = ¡1;0), ale v našem 
          konkrétním případě se jako vhodnější ukazuje otevřený konec vedení (Z ! 1, 􀀀 = +1;0), neboť 
          naměřená odražená vlna při něm vykazuje o něco větší amplitudu než při zakončení zkratem. Bránu 3 
          SVč (konec hlavního vedení) tedy necháme nezakončenou a na generátoru nastavíme příslušný kmitočet 
          f a vhodnou amplitudu (zpravidla největší přípustnou, pro dobrý odstup signálu od šumu ve 
          vektorovém voltmetru). Amplitudu napětí naměřenou v této situaci sondou B uložíme na přístroji 
          BM553 jako referenční stlačením tlačítek <B>, <LEVEL REF STORE> a fázový rozdíl 'B ¡'A uložíme jako 
          referenční tlačítkem <',T REF STORE>. Aplikaci uložené amplitudové referenční hodnoty poté 
          aktivujeme tlačítkem <LIN REF>. Přístroj by měl zobrazovat správnou hodnotu 􀀀 pro zakončení 
          naprázdno, tj. modul 1;0 a fázi 0±.
  
          \item Vlastní měření činitele odrazu 􀀀. Na bránu 3 směrového vazebního členu připojíme měřenou 
          zátěž. Bylo-li provedeno nastavení podle předchozího bodu, bude již přístroj ukazovat správnou 
          hodnotu činitele odrazu 􀀀 v polárních souřadnicích.   
        \end{itemize}
        
        Z uspořádání měřicího systému je zřejmé, že při měření činitele odrazu na více kmitočtech je třeba po 
        každé změně kmitočtu signálu celý první krok postupu zopakovat.
  
      \subsubsection{Směrový vazební člen}     
        Směrový vazební člen (SVČ, též směrová vazební odbočka, směrová vazební odbočnice, směrová vazba) je 
        vysokofrekvenční čtyřbran umožňující oddělení a měření složek signálu, které se šíří po vedení pouze 
        jedním směrem. Mezi jeho četné aplikace patří měření činitele odrazu, oddělení signálního generátoru 
        od měřicích obvodů, rozdělení výkonu a připojení dalších přístrojů (vlnoměrů, analyzátorů, wattmetrů 
        apod.).
  
        Princip SVČ je založen na vlastnostech obvodů s rozprostřenými parametry a jeho vnější funkce je 
        patrná z obrázku 4. Dva úseky vedení ‚- hlavní (zakončené branami 1 a 3) a vedlejší/vazební (mezi 
        branami 2 a 4) ‚- jsou spojeny tak, aby mezi nimi vznikla vazba. Tato vazba způsobuje, že se signál 
        zavedený na bránu 1 hlavního vedení rozdělí v určitém poměru na dvě vzájemně fázově posunuté složky. 
        Jedna složka vystupuje na bráně 3 a druhá bud’ na bráně 2, přičemž brána 4 je dokonale izolována 
        (protisměrný SVČ ), nebo na bráně 4, přičemž brána 2 je izolována (souměrný SVČ ). Zmíněná vazba je 
        reciprocitní a vykazuje obdobné chování i ze směru od vedlejšího vedení k hlavnímu.      
  
        \begin{figure}\ref{fyz:fig_RA_smith03} 
          \centering
          \includegraphics[width=\linewidth]{RA_smith03.png}
          \caption{Směrový vazební člen jako čtyřbran}
          \label{fyz:fig_RA_smith03} 
        \end{figure}
  
        SVČ lze popsat maticí rozptylových parametrů o rozměru 4 Q 4, kde parametry skk na hlavní diagonále 
        jsou činitele odrazu na jednotlivých branách, ostatní parametry jsou činitele přenosu mezi branami. U 
        ideálního protisměrného (obr. 4) SVč jsou všechny činitele odrazu rovny nule, přenos z brány 1 na 
        bránu 4 je též nulový a s ohledem na symetrii SVč jsou nulové i přenosy mezi branami 4–1, 2–3 a 3–2. 
        Rozptylová matice ideálního SVč potom tedy je
  
        \begin{equation}\label{RA:eq_smith09}
          s = \left[
            \begin{matrix}
                0    & s_{12} & s_{13} &  0       \\
              s_{21} &   0    &   0    & s_{24}   \\
              s_{31} &   0    &   0    & s_{34}   \\
                0    & s_{42} & s_{43} &  0     
            \end{matrix}
              \right]
        \end{equation} 
  
        Vzhledem k aplikaci SVč a symetrii matice s-parametrů se SVč často charakterizují několika skalárními 
        parametry vycházejícími z s-parametrů resp. výkonové bilance na branách SVč . Jsou to tyto parametry: 
        vazba K (K1, K2, též tzv. přeslechový útlum), směrovost D, izolace I a taktéž kmitočtový rozsah.
  
        Vazba K (též K1) je definována jako poměr výkonu P1 přiváděného na vstup hlavního vedení 1 a výkonu 
        P2 vystupujícího z brány 2 při bezodrazovém zakončení bran 2 až 4,
        \begin{equation}\label{RA:eq_smith10}
          K = K_1 = 10\log\frac{P_1}{P_2} 
                  = 10\log\frac{1}{\abs{s_{21}}^2} = -20\log\abs{s_{21}},  
        \end{equation} 
        a vyjadřuje tedy vazební útlum přímé vlny na hlavním vedení (směřující od brány 1 k bráně 3) na vstup 
        vedlejšího vedení 2.
  
        Podobně vazba K2 je definována jako poměr výkonu P3 přiváděného na výstup hlavního vedení 3 a výkonu 
        P4 vystupujícího z brány 4 při bezodrazovém zakončení bran 1, 2 a 4,
        \begin{equation}\label{RA:eq_smith11}
          K_2 = 10\log\frac{P_3}{P_4} 
              = 10\log\frac{1}{\abs{s_{43}}^2} = -20\log\abs{s_{43}},  
        \end{equation} 
        a vyjadřuje tedy vazební útlum zpětné vlny na hlavním vedení (směřující od brány 3 k bráně 1) na 
        výstup vedlejšího vedení 4.
  
        Směrovost D je definována jako poměr výkonu P2 vystupujícího z brány 2 a výkonu P4 vystupujícího z 
        brány 4 při přivádění signálu na bránu 1 a bezodrazovém zakončení bran 2 až 4,
        \begin{equation}\label{RA:eq_smith12}
          D = 10\log\frac{P_2}{P_4} 
            = 10\log\frac{1}{\abs{s_{42}}^2} = -20\log\abs{s_{42}},  
        \end{equation} 
  
        Izolace I je definována jako poměr výkonu P1 přiváděného na vstup hlavního vedení 1 a výkonu P4 
        vystupujícího z brány 4 při bezodrazovém zakončení bran 2 až 4,
        \begin{equation}\label{RA:eq_smith13}
          I = 10\log\frac{P_1}{P_4} 
            = 10\log\frac{1}{\abs{s_{41}}^2} = -20\log\abs{s_{41}}, 
        \end{equation} 
        a vyjadřuje tedy vazební přeslechový útlum mezi přímou vlnou na hlavním vedení (směřující od brány 1 
        k bráně 3) a výstupem vedlejšího vedení 4.
  
        Z uvedených definic plyne, že směrovost D je rovna rozdílu izolace I a vazby K (resp. K1)
        \begin{equation}\label{RA:eq_smith14}
          D = I - K = I - K_1 = 20\log\frac{\abs{s_{21}}}{\abs{s_{41}}}
        \end{equation}  
  
        Směrovost D (v souladu se svým názvem) popisuje schopnost směrové selekce – čím je větší, tím méně se 
        při měření vlny signálu v jednom směru pomocí SVč nežádoucím způsobem projevuje vlna šířící se v 
        opačném směru.
  
      \subsection{Vektorový voltmetr}
        Vektorový voltmetr je schopen měřit amplitudy dvou harmonických signálů UA aUB a jejich fázový rozdíl 
        'B ¡'A. Dále bývá obvykle schopen měřit poměr amplitud UB=UA. Vektorový voltmetr v modernějším 
        provedení, jako je TESLA BM553, umožňuje i relativní měření vůči uložené referenci – UA=Uref , 
        UB=Uref a ('B ¡ 'A)='ref – a různá odvozená měření (např. s-parametrů, R, L, C prvků apod.) 
        využívající zvláštní příslušenství.
  
        Princip vektorového voltmetru je založen na vzorkování obou měřených signálů a jejich současném 
        převodu na nízký mezifrekvenční kmitočet 20 kHz. Takto nízký kmitočet lze již snadno a přesně 
        zpracovat, jednak detektory a A/D převodníky pro zjištění amplitudy, jednak logickými obvody pro 
        zjištění fázového rozdílu. Vzorkování se provádí velmi rychlými analogovými spínači se Schottkyho 
        diodami umístěnými přímo v sondách. Vzorkovací kmitočet se musí od kmitočtu měřeného signálu lišit 
        přesně o 20 kHz, což je zajištěno pomocí smyčky automatické kmitočtové a fázové synchronizace uvnitř 
        voltmetru.
      

  
\printbibliography[heading=subbibliography]