%======================================Kapitola: Preprocesor jazyka C ===================================================================  
\chapter{Preprocesor jazyka C}
\minitoc
\newpage
  Preprocesor interpretuje jednoduché direktivy pro vložení zdrojového kódu z jiného souboru (\lstinline[basicstyle=\ttfamily]!#include!), definice maker (\lstinline[basicstyle=\ttfamily]!#define!) a podmíněné vložení kódu (\lstinline[basicstyle=\ttfamily]!#if!).\texttt{C} preprocesor přijímá tyto direktivy:
  
  \begin{table}[ht!]
    \centering
    \begin{tabular}{c c c c}
      \hline
      \lstinline[basicstyle=\ttfamily]!#define! & \lstinline[basicstyle=\ttfamily]!#elif! & \lstinline[basicstyle=\ttfamily]!#else! & \lstinline[basicstyle=\ttfamily]!#endif! \\
      \lstinline[basicstyle=\ttfamily]!#error! & \lstinline[basicstyle=\ttfamily]!#if! & \lstinline[basicstyle=\ttfamily]!#ifdef! & \lstinline[basicstyle=\ttfamily]!#ifndef! \\
      \lstinline[basicstyle=\ttfamily]!#include! & \lstinline[basicstyle=\ttfamily]!#line! & \lstinline[basicstyle=\ttfamily]!#pragma! & \lstinline[basicstyle=\ttfamily]!#undef! \\
      \hline            
    \end{tabular}
    \caption{Seznam platných direktiv jazyka \texttt{C}}\label{S4101C1:C_tab_direktiva}
  \end{table} 
  
 \section{Připojení externích souborů}
 
 \section{Definice maker}
   Definice maker ve významu rozsahů polí je typickým příkladem použití preprocesoru. Ve zdrojovém textu se neodvoláváme na magická čísla, ale na vhodně symbolicky pojmenovaná makra, která zvýší čitelnost programu.

   Pro větší přehlednost rozdělme makra na 
   \begin{itemize}
     \item symbolické konstanty,
     \item makra
   \end{itemize}
   Klíčem nechť je skutečnost, že makro na rozdíl od symbolické konstanty má argumenty.
   \subsection{Symbolické konstanty}
   \subsection{Makra}   
   
 \section{Podmíněný překlad}  
   Preprocesor může během své činnosti vyhodnocovat, je-li nějaké makro definováno, či nikoliv. Při použití klíčového slova preprocesoru \texttt{defined} pak může spojovat taková vyhodnocení do rozsáhlejších logických výrazů. Argument defined nemusí být uzavřen do závorek. Může se však vyskytnout jen za \lstinline[basicstyle=\ttfamily]!#if! nebo \lstinline[basicstyle=\ttfamily]!#elif!. Například si ukažme složitější podmínku: