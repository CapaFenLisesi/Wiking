%\documentclass{article}
%  \usepackage{circuitikz}

%\begin{document}
  \begin{wrapfigure}{r}{2.8in}
    \centering
	\begin{circuitikz} 
	  \draw
		(0,0) node[transformer] (T) {}
		node[ocirc] (A) at ([xshift=-1.5cm]T.A1) {}
		node[ocirc] (B) at ([xshift=-1.5cm]T.A2) {}
		node[ocirc] (C) at ([xshift=1.5cm]T.B1) {}
		node[ocirc] (D) at ([xshift=1.5cm]T.B2) {}
		node[circ]  (E) at ([xshift=0.4cm,yshift=-5pt]T.A1)  {}
		node[circ]  (F) at ([xshift=-0.4cm,yshift=-5pt]T.B1) {}
		(T.A1) to [-o] (A)
		(T.A2) to [-o] (B) 
		(T.B1) to [-o] (C)
		(T.B2) to [-o] (D)
		([yshift=+.25cm]T.west) node{$L_1$}
		([yshift=-.25cm]T.west) node{$N_1$}
	    ([yshift=+.25cm]T.east) node{$L_2$}
		([yshift=-.25cm]T.east) node{$N_2$} 
	  ;
	  \coordinate (X) at ([xshift=-0.6cm]B);
	  \draw (A) --+(-0.6,0) to [sV] (X) -- (B); 
		
	  \begin{scope}[shorten >= 10pt,shorten <= 10pt,]
		\draw[->] (A) -- node[right] {$u_1(t)$} (B); 
	  \end{scope}
		
	  \draw[->] ([xshift=-0.9cm,yshift=10pt]T.A1) -- node[above] {$i_1(t) = i_\mu(t)$} +(20pt,0);
	\end{circuitikz}	 
	\caption[Transformátor naprázdno.]{Transformátor naprázdno.}
    \label{es:fig_MJ_patocka_trf_naprzdn}
    \vspace*{-1\baselineskip}
  \end{wrapfigure} 
%\end{document}