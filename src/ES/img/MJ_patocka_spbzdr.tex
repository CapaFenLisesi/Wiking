% \documentclass{article}
%   \usepackage{circuitikz}  
%   \usepackage{tikz}
%     \usetikzlibrary{decorations.markings}
% 	\usetikzlibrary{arrows.new}  
%   \usepackage{wrapfig}
%   \usepackage{subfig}
%   
% \begin{document}
  \begin{figure}[htp]
    \centering
    \subfloat[ ]   {
        \begin{circuitikz}[scale=1, every node/.style={scale=1}]
          \draw (0,0) node[ocirc] {} --+ (1,0) coordinate(A); 
          \coordinate (B) at ([yshift=-2.5cm]A);
          \draw (A) to[R] (B) --+ (-1,0) coordinate(D) node[ocirc] {}; 
		  
		  \begin{scope}[shorten >= 10pt,shorten <= 10pt]
		  
            \draw[-open triangle 45 new,arrow head=5pt]
			     ([xshift=.6cm,yshift=-0.5cm]A) -- node[right] {$i(t)$} 
			     ([xshift=.6cm,yshift=0.5cm]B); 
	        \draw[->]  (0,0) -- node[left] {$u(t)$} (D);			
	      \end{scope}		  
        \end{circuitikz}    
    }
    \subfloat[ ]   {
        \begin{circuitikz}[scale=1, every node/.style={scale=1}]
          \draw (0,0) node[ocirc] {} --+ (1,0) coordinate(A); 
          \coordinate (B) at ([yshift=-2.5cm]A);
          \draw (A) to[L] (B) --+ (-1,0) coordinate(D) node[ocirc] {}; 
	      
		  \begin{scope}[shorten >= 10pt,shorten <= 10pt]
	        \draw[-open triangle 45 new,arrow head=5pt] 
  			     ([xshift=.6cm,yshift=-0.5cm]A) -- node[right] {$i(t)$} 
  			     ([xshift=.6cm,yshift=0.5cm]B); 
	        \draw[->]  (0,0) -- node[left] {$u(t)$} (D);
	      \end{scope}		  
        \end{circuitikz}
    }  
    \subfloat[ ]   {
        \begin{circuitikz}[scale=1, every node/.style={scale=1}]
          \draw (0,0) node[ocirc] {} --+ (1,0) coordinate(A); 
          \coordinate (B) at ([yshift=-2.5cm]A);
          \draw (A) to[L] (B) --+ (-1,0) coordinate(D) node[ocirc] {}; 
	      
		  \begin{scope}[shorten >= 10pt,shorten <= 10pt]
	        \draw[open triangle 45 new-,arrow head=5pt]  
			     ([xshift=.6cm,yshift=-0.5cm]A) -- node[right] {$i(t)$} 
			     ([xshift=.6cm,yshift=0.5cm]B); 
	        \draw[->]  (0,0) -- node[left] {$u(t)$} (D);
	      \end{scope}		  
        \end{circuitikz}
    } 	 

	\caption{Vzájemná orientace okamžité hodnoty proudu a napětí ve spotřebičovém a
             zdrojovém režimu: a) Odpor je vždy spotřebičem. b) Cívka ve spotřebičovém režimu. 
             c) Cívka ve zdrojovém režimu.}
    \label{es:fig_MJ_patocka_spbzdr}
  \end{figure} 
% \end{document} 