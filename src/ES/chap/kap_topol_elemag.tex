%file: kap_topol_elemag.tex
%===========Kapitola: Topologické vlastnosti elektromagnetického pole===============================
\chapter{Topologické vlastnosti elektromagnetického pole}\label{ES:kap_topovlelmagp}
\minitoc
\newpage    
  V předchozích kapitolách byly na mnohých místech zdůrazňovány některé topologické souvislosti.
  Kapitola o topologii je úmyslně zařazena až následně, jednak aby shrnula získané poznatky a
  vtiskla jim určitý řád, jednak aby čtenář již měl předchozí konkrétní představy o některých
  abstraktních pojmech \cite[s.~40]{Patocka4}. 
  
  Topologie je matematická disciplína, patřící do vyšších pater v hierarchii matematiky. Topologie
  se zaýává prostorovými útvary, podobně jako geometrie. Na rozdíl od geometrie ji však nezajímají 
  \emph{kvantitativní} ukazatele zkoumaných geometrických útvarů, nýbrž určité vyšší obecnější
  \emph{kvalitativní} ukazatele. S nadsázkou lze říci, že je to ''geometrie, která nic neměří''.
  Topologie se dělí na dvě zdánlivě odlišné disciplíny: 
  \begin{itemize}
    \item \emph{Topologie diskrétních útvarů} neboli \emph{teorie grafů} - zabývá se diskrétními
          prostorovými útvary, tj. \emph{grafy}. Graf sestává z \emph{uzlů} propojených
          \emph{hranami}. Hrany mohou být \emph{orientované} nebo \emph{neorientované}. Všechny
          diskrétní elektrické obvody jsou \emph{neorientovanými} grafy. Proto i všechny známe
          metody řešení diskrétních elektrických obvodů (metoda Kirchhoffových zákonů, metoda
          smyčkových proudů, metoda uzlových napětí) podléhají zákonitostem diskrétní topologie. 
    \item \emph{Topologie spojitých útvarů} - zabývá se spojitými prostorovými útvary, tj.
          \emph{plochami a křivkami} v prostorech libovolné dimenze. Patří sem všechny elektrické
          obvody s \emph{parametry spojitě rozprostřenými} v 3D prostoru. Takovým útvarem je např.
          krabice naplněná elektricky vodivým grafitovým práškem, do níž umístíme v libovolných
          místech dvě nebo více elektrod. Intuitivně tušíme, že v prostoru krabice budeme pracovat s
          ekvipotenciálami v podobě \emph{ploch} nebo se siločárami v podobě \emph{křivek} atd. 
  \end{itemize}
  
  Elektrické obvody se chovají poněkud jinak než obvody magnetické: 
  \begin{itemize}
    \item V \emph{elektrických} obvodech je poměr mezi měrnou elektrickou vodivostí \emph{vodičů} a
          \emph{izolantů} minimálně $10^{12}$, obvykle i větší. Proto lze snadno pomocí vodiče
          obaleného izolantem docílit toho, že elektrický proud teče pouze prostorově vymezenými
          \emph{diskrétními} cestami. Pak je pochopitelné, že vhodným a \emph{absolutně přesným}
          nástrojem k analýze elektrického obvou je \emph{topologie diskrétních útvarů}.
    \item V \emph{magnetických} obvodech je poměr mezi měrnou magnetickou vodivostí (permeabilitou)
          \emph{vodičů} a \emph{izolantů} typicky $10^{3}$, což je dáno relativní permeabilitou
          feromagnetik vůči vakuu. Vakuum je tedy \emph{velmi špatný} magnetický izolant a lepší v
          přírodě bohužel neexistuje. V této situaci je obtížné realizovat ryze diskrétní magnetický
          obvod, protože železo neumíme ''obalit'' kvalitním magnetickým izolantem. U cívky se
          železným jádrem podle obr. \ref{es:fig_Bzv} v kapitole \ref{es:tokcvk_frmj} jsme ukázali,
          že rozptylový tok vzdušných cest činí řádově 1 \% až 5 \% z toku celkového. Takový obvod
          je sice již řešitelný metodami \emph{diskrétní} topologie, ale pouze přibližně. Je to
          běžný inženýrský postup, který se v praxi velmi úspěšně používá. Pokud však rozptylový tok
          nehceme nebo nemůžeme zanedbat, je nezbytné pracovat metodami \emph{topologie spojitých
          útvarů\footnote{Všimněme si ale, že ke zjednodušeným výrazům pro výpočet spřaženého toku v
          \emph{diskrétním} obvodu jsme dospěli pomocí integrálních metod, které používá topologie
          \emph{spojitých} útvarů}. Berme to jako ukázku, že mezi oběma topologiemi je hluboký
          vztah, i když není na první pohled patrný.}
  \end{itemize} 
  %----------------------------------
  % image: resistor_grid.tex label: \label{ES:fig_res_grid}
      \input{../src/ES/img/resistor_grid.tex}  
  %----------------------------------
  
  Na první pohled se zdá, že obě topologie využívají natolik odlišných matematických postupů, že
  spolu tyto disciplíny nijak nesouvisí. Opak je pravdu. Mezi oběma panuje hluboký vztah, obě
  vycházejí ze stejných základů. Vysvětlení lze hledat na obr. \ref{ES:fig_res_grid}. Je zde
  nakreslen \emph{přenosový dvojbran} se zcela obecnou vnitřní strukturou, která může mít např.
  podobu husté vodivostní\footnote{V magnetických obvodech je psychologicky výhodnější pracovat s
  magnetickými vodivostmi než s magnetickými odpory (reluktancemi). Permeabila má totiž význam
  \emph{měrné magnetické vodivosti}} sítě, ve které mají jednotlive vodivosti nahodile různé
  hodnoty. S ohledem na dobře známé analogie je lhostejné, zda se jedná o \emph{elektrický} nebo
  \emph{magnetický} obvod. Nakreslený obvod je zcela určitě \emph{diskrétní}, bude tedy řešen
  některou klasickou diskrétní metodu, např. metodou smyčkových proudů nebo metodou uzlových
  napětí. Výpočtem zjistíme, že z pohledu vstupní a výstupní brány má dvojbran konkrétní přenosové
  parametry (napěťový přenos naprázdno, proudový přenos nakrátko, vstupní impedanci naprázdno,
  nakrátko, atd.) Učiňme následující myšlenkový pokus: vodivostní síť budem neustále zjemňovat.
  Tj., ve smšru vodorovném i svislém budeme zvyšovat počty prvků, ale tak, aby celková vodivost na
  jednotku délky zůstávala v dané oblasti \emph{konstantní}. Výsledkem zjemňování bude v limitním
  případě vznik \emph{spojité} vodivé desky (např. izolační podložka nastříkaná elketricky vodivým
  odporovým lakem). Mezi původním diskrétním obvodem a deskou zřejmě platí následující souvislosti:
  \begin{itemize}
    \item Původní diskrétní vodivosti byly nahodile různé \(\longrightarrow\) deska bude
          \emph{nehomogenní, anizotropní}.
    \item Původní diskrétní vodivosti byly stejně velké ve směru \emph{x} a stejně velké (ale s
          jinou hodnotou ve směru \emph{y}) \(\longrightarrow\) deska bude \emph{homogenní
          anizotropní}.
    \item Všechny diskrétní vodivosti měly stejnou hodnotu \(\longrightarrow\) deska bude
          \emph{homogenní, izotropní}.
  \end{itemize}
  
  Intuitivně tušíme, že vytvořená \emph{spojitá} deska\footnote{Uvedený příklad se týká
  dvojrozměrné desky. Příklad lze jistě zobecnit na trojrozměrné objekty (lze si představit
  krabici naplněnou vodivým grafitovým práškem, do které zavedeme čtyři bodové elektrody).} bude mít
  všechny přenosové parametry číselně shodné s původním \emph{diskrétním} obvodem. Přitom ale u
  desky nelze tyto parametry určit klasickými diskrétními metodami (nelze určit matici obvodu). Je
  nutný přechod od diskrétních operací k operacím integrálním, tedy od topologie \emph{diskrétních
  útvarů} k topologii \emph{spojitých útvarů}. Z uvedeného myšlenkového pokusu plyne, že v
  \emph{limitním případě} velmi jemné sítě musí dát diskrétní i spojité operace stejný kvantitativní
  výsledek. Na tomto poznatku je založeno přibližné řešení spojitých prostorových polí
  \emph{metodami konečných prvků}.
  
  Především jsme ovšem chtěli ukázat, že mezi diskrétními a spojitými topologickými metodami není
  zásadního rozdílu, obě vycházejí ze stejných základů a v limitním případě spolu splývají. 
  
  \section{Topologie diskrétních útvarů}  
    Cílem této kapitoly je především vysvětlit \emph{pricnip reciprocity} v pasivních elektrických
    obvodech a pomocí něho odvodit \emph{počet stupňů volnosti} elektrických obvodů. Zvláštním
    případem obvodu je \emph{pasivní přenosový dvojbran}, u kterého bude dokázáno, že má vždy
    \emph{tři stupně volnosti}. Tento poznatek má totiž mimořádný význam v teorii
    \emph{transformátorů}, který je právě typickým představitelem přenosového dvojbranu. V
    kapitolách zabývajících se transformátorem - především jeho náhradním zapojením - se budeme
    odvolávat na výsledky získané v této kapitole. 
    
    \subsection{Základní pojmy teorie grafů}
      Názvosloví a základní pojmy teorie grafů lze shrnout do následujících bodů:
      \begin{itemize}
        \item Základním pojmem je \emph{graf} (graf orientovaný, neorientovaný). Graf je vlastně
              „schéma“ příslušného obvodu s vynechanými obvodovými prvky.
        \item Graf sestává z \emph{uzlů} a \emph{hran}.
        \item Uzel je spojení alespoň tří hran\footnote{Spojení dvou hran je elektrický bod, nikoli
              uzel.}.
        \item Hrana může být \emph{orientovaná} (je jí přiřazen směr), \emph{neorientovaná} (nemá
              přiřazen směr). V elektrotechnice se používají výhradně neorientované hrany - tedy i
              grafy (vlastnosti obvodových prvků R, L, C jsou nezávislé na směru proudu).
        \item \emph{Úplný strom}: nepřerušená celistvá soustava nejmenšího počtu hran, která spojuje
              všechny uzly grafu.
        \item \emph{Nezávislá hrana}: hrana nepatřící do úplného stromu.
        \item \emph{Nezávislá smyčka}: uzavřená smyčka, která musí obsahovat nezávislou hranu, tj.
              hranu nepatřící do úplného stromu.
        \item Nezávislých smyček je tolik, kolik je nezávislých hran.
      \end{itemize}
     
     Označme v grafu:
       \begin{itemize}
         \item Počet uzlů:	\(q +1\)
		 \item Počet hran úplného stromu:	\(q\)
		 \item Počet hran (počet neznámých proudů):	\(p\)
		 \item Počet nezávislých hran (nezávislých smyček):	\(n=p-q\)
      \end{itemize}

     U složitých obvodů je hledání \(n\) nezávislých smyček obtížné. Proto se k tomuto účelu používá
     úplný strom, jehož nalezení je snadné. Nezávislé hrany jsou ty, které \emph{nepatří} do úplného
     stromu. Každou nezávislou hranou pak musí procházet alespoň jedna nezávislá smyčka. Všechny
     pojmy budou ukázány na konkrétním příkladu.

     Řešením obvodu se rozumí: Nalezení všech \(p\) neznámých proudů ve všech \(p\) hranách.
     Principiálně se vždy jedná o řešení soustavy \(p\) rovnic o \(p\) neznámých proudech.

     K řešení lze použít tři metody:
     \begin{itemize}
       \item Metoda založená na přímém použití I. a II. Kirchhoffova zákona\footnote{Gustav Robert
             Kirchhoff (1824-1887), německý fyzik, působil na univerzitách v Heidelbergu a v
             Berlíně. I. a II. KZ objevil r. 1845 ještě jako student. Dále se zabýval spektroskopií,
             tepelnou radiací černého tělesa, spoluobjevitel Cesia a Rubidia. Žák F. E. Neumanna.}.
             Je nejméně efektivní, vede na nejrozsáhlejší soustavu \(p\) rovnic o \(p\) neznámých.
       \item Metoda smyčkových proudů (Mesh Currents Matrix Method), Maxwellova metoda. Vede na
             soustavu pouze \(n\) rovnic o \(n\) neznámých smyčkových proudech\footnote{Ze
             smyčkových proudů lze skutečné proudy snadno vyřešit pomocí doplňkových rovnic
             sestavených pomocí I. KZ.}. Vezmeme-li v úvahu nejsložitější obvod, ve kterém je každá
             dvojice uzlů spojena hranou, pak bude:
             \begin{equation}
               n=p-q=\frac{q(q-1)}{2},
             \end{equation}
             což je	podstatně méně rovnic než \(p\).            
     \end{itemize}    
            
\printbibliography[heading=subbibliography]