\begin{example}
  Určete vlastní čísla a odpovídající vlastní vektory následují\-cích matic:
  \begin{displaymath}
    \mathbf{A}=\left(\begin{array}{rr}1&0.5\\3.5&4\end{array}\right), \quad
    \mathbf{B}=\left(\begin{array}{rr}3&-1 \\2.5&4\end{array}\right)
  \end{displaymath}
  \textbf{Řešení}: Vlastní čísla určíme z charakteristické rovnice: 
    $\det(\mathbf{A}-\lambda\mathbf{I})=0$. Vlastní vektory $\mathbf{x_i}$
    odpovídající vlastním číslům $\lambda_i$, jsou řešením homogenní
    soustavy rovnic $(\mathbf{A}-\lambda_i\mathbf{I})\mathbf{x_i}=0$.
    \begin{itemize}
      \item Vlastní čísla matice \textbf{A}:
        \begin{equation*}
             \textbf{A} - \lambda\textbf{I}=
               \left(\begin{array}{cc}
                        1-\lambda  &  0.5          \\
                       -3.5        &  4-\lambda
                     \end{array}
               \right)
        \end{equation*}
        \begin{align*}
           \det(\mathbf{A}-\lambda\mathbf{I}) &= 0 \\
           (1-\lambda)(4-\lambda)-\frac{7}{4} &= 0 \\
           \lambda^2-5\lambda+\frac{9}{4}     &= 0
        \end{align*}
        \begin{equation*}
           \lambda_1 = 4.5,\quad \lambda_2 = 0.5
        \end{equation*}     
    \end{itemize}

    \begin{itemize}
      \item Vlastní čísla matice \textbf{B}:
        \begin{equation*}
             \textbf{B} - \lambda\textbf{I}=
               \left(\begin{array}{cc}
                       3-\lambda  & -1             \\
                       2.5        &  4-\lambda
                     \end{array}
               \right)
        \end{equation*}
        \begin{align*}
           \det(\mathbf{B}-\lambda\mathbf{I}) &= 0 \\
           (3-\lambda)(4-\lambda)+\frac{5}{2} &= 0 \\
           \lambda^2-7\lambda+\frac{29}{2}    &= 0
        \end{align*}
        \begin{equation*}
           \lambda_1 = \frac{7+3i}{2},\quad \lambda_2 = \frac{7-3i}{2}
        \end{equation*}
    \end{itemize}
  % matice A
  Vlastní vektor matice $\mathbf{A}$ pro $\lambda_1=4.5:
  (\mathbf{A}-\lambda_1\mathbf{I})\mathbf{x_1}=0 \Rightarrow$
  \begin{align*}
    \left(
      \begin{array}{cc}
         1  -4.5  &  0.5   \\
        -3.5      &  4-4.5
      \end{array}
    \right) &\sim
    \left(
      \begin{array}{cc}
        -3.5  &  0.5       \\
        -3.5  & -0.5
      \end{array}
    \right)                         \\
    \Rightarrow\mathbf{x_1} &=
      \left(
        \begin{array}{c}
          1 \\ 7
        \end{array}
      \right)\, r, r\in\mathbb{R}, r\neq0
  \end{align*}
  Vlastní vektor matice $\mathbf{A}$ pro $\lambda_2=0.5:
  (\mathbf{A}-\lambda_1\mathbf{I})\mathbf{x_2}=0 \Rightarrow$
  \begin{align*}
    \left(
      \begin{array}{cc}
         1  -0.5  &  0.5   \\
        -3.5      &  4-0.5
      \end{array}
    \right) &\sim
    \left(
      \begin{array}{cc}
         0.5  &  0.5   \\
         3.5  &  3.5
      \end{array}
    \right)                         \\
    \Rightarrow\mathbf{x_2} &=
    \left(
      \begin{array}{c}
        -1 \\ 1
      \end{array}
    \right)\, r, r\in\mathbb{R}, r\neq0
  \end{align*}
  % matice B
  Vlastní vektor matice $\mathbf{A}$ pro $\lambda_1=\frac{7+3i}{2}:
  (\mathbf{B}-\lambda_1\mathbf{I})\mathbf{x_1}=0 \Rightarrow$
  \begin{align*}
    \left(
      \begin{array}{cc}
         3 - \frac{7+3i}{2}            &  -1                                     \\
         \frac{5}{2}                   &  4 - \frac{7+3i}{2}
      \end{array}
    \right)\sim
    \left(
      \begin{array}{cc}
        -\frac{1}{2}-\frac{3}{2}i      &  -1                                     \\
        \frac{5}{2}                    & \frac{1}{2}-\frac{3}{2}i
      \end{array}
    \right)\sim \\
    \left(
      \begin{array}{cc}
        -\frac{10}{4}                  &-\left(\frac{1}{2} -\frac{3}{2}i\right)  \\
        \frac{5}{2}                    & \frac{1}{2}-\frac{3}{2}i
      \end{array}
    \right)\sim
      \left(
        \begin{array}{cc}
          -5                           &-\left(1-3i\right)                       \\
           5                           & \left(1-3i\right)
        \end{array}
    \right)\rightarrow \\
    \mathbf{x_1}=
      \left(
        \begin{array}{c}
          -1+3i \\ 5
        \end{array}
      \right)\, r, r\in\mathbb{C}, r\neq0
  \end{align*}
  Vlastní vektor matice $\mathbf{B}$ pro $\lambda_2=\frac{7-3i}{2}:
  (\mathbf{B}-\lambda_1\mathbf{I})\mathbf{x_2}=0 \Rightarrow$
  \begin{align*}
    \left(\begin{array}{cc}
             3  - \frac{7-3i}{2}       &  -1                                     \\
            \frac{5}{2}                &  4 - \frac{7-3i}{2}
          \end{array}
    \right)\sim
    \left(\begin{array}{cc}
            -\frac{1}{2}+\frac{3}{2}i  &  -1                                     \\
            \frac{5}{2}                & \frac{1}{2}+\frac{3}{2}i
          \end{array}
    \right)\sim \\
    \left(\begin{array}{cc}
            -\frac{10}{4}              &-\left(\frac{1}{2} +\frac{3}{2}i\right)  \\
            \frac{5}{2}                & \quad\frac{1}{2}+\frac{3}{2}i
          \end{array}
    \right)\sim
      \left(\begin{array}{cc}
            -5                         &-\left(1+3i\right)                       \\
             5                         & \quad\left(1+3i\right)
          \end{array}
    \right)\rightarrow\\
    \mathbf{x_2}=
      \left(
        \begin{array}{c}
          -1-3i \\ 5
        \end{array}
      \right)\, r, r\in\mathbb{C}, r\neq0
  \end{align*}
\end{example}
%---------------------------------------------------------------
\lstinputlisting{../src/LA/img/vlastni_cisla_01.m}
\begin{lstlisting}[caption=Výpis programu pro ověření výpočtu vlastních čísel matic programem
  Matlab.]
\end{lstlisting}
%---------------------------------------------------------------