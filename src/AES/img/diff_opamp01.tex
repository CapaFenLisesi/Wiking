% Diferenciální zesilovač
% pozn: symbol země (ground) je předefinován v knihovně wiking.sty

% Define box and box title style
% http://www.texample.net/tikz/examples/boxes-with-text-and-math/
\tikzstyle{mybox} = [draw=red, fill=blue!20, very thick,
    rectangle, rounded corners, inner sep=5pt, inner ysep=10pt, inner xsep=10pt ]
\tikzstyle{fancytitle} =[fill=red, text=white]

\begin{figure}[htp]
  \centering
  \begin{circuitikz}[scale=1, every node/.style={scale=1}]
    \draw (0,0) node[op amp] (opamp) {};
    \node [left=0.1cm of opamp.-]       (p1) {};
    \node [left=0.1cm of opamp.+]       (p2) {};
    \node [above=0.8cm of p1]           (A)  {};
    \node [left=1.5cm of A]             (p3) {};
    \node [right=2.7cm of A, coordinate](C)  {};
    \node [right=0.8cm of C]            (p6) {};
    \node [below=0.8cm of p2]           (B)  {};
    \node [left=1.5cm of B]             (p4) {};
    \node [right=2.7cm of B]        (ground) {};
    \node [right=0.1cm of opamp.out, coordinate] (vo) {};
    %%% Draw connections
    \draw (p3) node[left] {$u_1$} to[R=$R_1$,o-] (A)  to[R=$R_2$,*-] (C); 
    \draw (opamp.-)   to[short] (p1) to[short] (A);
    \draw (opamp.out) to[short] (vo);
    \draw (C) to[short,*-] (C) |- (vo); 
    \draw (p4) node[left] {$u_2$} to[R=$R_3$,o-] (B)  to[R=$R_4$,*-] (ground) node [ground] {};  
    \draw (opamp.+) to[short] (p2) to[short] (B); 
    \draw (C) to[short,-o] (p6) node[right] {$u_{0}$};
    \node[above] at (A) {A}; % text 
    \node[below] at (B) {B}; % text 
    %%% minipage
    \node [mybox, right=5cm of p2] (box){%
       \begin{minipage} [l] {0.3\textwidth}
          \begin{align*}
              \frac{R_4}{R_3} &= \frac{R_2}{R_1} \\
              u_0             &= \frac{R_2}{R_1}(u_2 - u_1) \\
              R_{|1|}         &= R_1 \\
              R_{|2|}         &= R_3 + R_4 \\
              R_{\mathsf{O}|} &= 0
          \end{align*}
       \end{minipage}
    };
    \node[fancytitle, right=10pt, rounded corners] at (box.north west) {rovnice:};
   %\node[fancytitle, rounded corners] at (box.east) {$\clubsuit$};
    %%% text R_{|1|} and R_{|2|}
    \draw[->] (-4.0,0.5) -- +(0,0.5) node[left] {$R_{|1|}$} -- +(0.5,0.5);
    \draw[->] (-4.0,-0.5) -- +(0,-0.5) node[left] {$R_{|2|}$} -- +(0.5,-0.5);
  \end{circuitikz} 
  \caption{Rozdílový zesilovač. Podmínky potlačení souhlasné složky vstupních napětí $u_1$ a $u_2$ je 
           poměrové vyvážení zpětnovazebních rezistorů, $R_4/R_3 = R_2/R_1$. S ohledem na ofset se obvykle
           volí uplná symetrie, tj. $R_4 = R_2$ a $R_3 = R_1$.}
  \label{AES:fig_diff_opamp01}         
\end{figure}