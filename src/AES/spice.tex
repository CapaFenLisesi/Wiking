%==================================Kapitola: Počítačová simulace v elektrotechnice========================
\chapter{Počítačová simulace v elektrotechnice}
\minitoc
\newpage
  \section{Historie}
    V roce 1971 vytvořil student „University of California“, Berkeley, USA \emph{Larry Nagel} program \texttt{SPICE1} (\texttt{SPICE} =
    \emph{Simulation Program with Integrated Circuit Emphasis}). Program umožňoval analýzu dějů v obvodech, obsahujících zejména bipolární a
    unipolární tranzistory. O věrohodnost výsledků bylo usilováno propracovaností modelů i matematických algoritmů řešení rovnic. Uživatel měl navíc
    možnost roz\-ši\-řo\-vá\-ní sortimentu analyzovaných součástek technikou makromodelů zakládáním tzv. \emph{pod\-ob\-vo\-dů}
    (\texttt{subcircuits}) \texttt{SPICE}. Protože program byl v podstatě volně šiřitelný, stal se brzo standardním simulačním nástrojem pro
    elektrotechnické úlohy. Usilovně se pracovalo na jeho zdokonalování.

    V roce 1975 byla představena verze \texttt{SPICE2} s podstatně vylepšenými modely i numerickými algoritmy. Tato verze byla v průběhu téměř 20 let
    postupně zdokonalována na Berkeleyské univerzitě až do dnes všeobecně známého standardu \texttt{SPICE2G.6}, který byl v r. 1983 zpřístupněn k
    volnému používání. Zdrojové texty \texttt{SPICE1} a \texttt{SPICE2} byly napsány ve Fortranu. Vzhledem k zvýšenému využívání unixových pracovních
    stanic padlo v Berkeley rozhodnutí přepsat SPICE2 do jazyka C. Tak začala vznikat verze \texttt{SPICE3}. Dnes je rozšířena verze
    \texttt{SPICE3F.2}. Oproti SPICE2G.6 se vyznačuje řadou vylepšení, ovšem z různých důvodů došlo k ztrátě zpětné kompatibility se SPICE2G.6.

    S růstem výkonnosti počítačů PC byly programy, dosud běžící na výkonných pracovních stanicích, přepisovány na programy spustitelné na „PCčkách“.
    Tak vznikl standard \texttt{PSpice}. Dnes existuje více simulačních programů, které využívají v podstatě tři ne zcela kompatibilní standardy:
    SPICE2, SPICE3, PSPICE. Všechny lze rozdělit na tzv. „\emph{Spice-like}“ a „\emph{Spice-compatible}“ simulátory.

    Označení „\emph{Spice-like}“ znamená, že simulátor je schopen generovat podobné výsledky analýzy jako SPICE, avšak nemusí být schopen číst
    standardní vstupní soubory SPICE. Typickými příklady jsou staré verze programů \texttt{Micro-Cap} nebo \texttt{TINA}, program apod. Termínem
    „\emph{Spice-compatible}“ se označují simulační programy, které dokáží číst standardní vstupní soubory SPICE, provádět klasické SPICE analýzy, a
    generovat výsledky v standardním SPICE2G.6 tvaru. Ze současných programů jsou to například \texttt{PSpice}, \texttt{HSpice} (standard SPICE3),
    \texttt{WINSpice} (standard SPICE3), \texttt{MicroCap} od verze IV, \texttt{Multisim}, \texttt{LTspice} (standard SPICE3) a další.

    Kromě toho existují programy pro simulaci obvodů, které nemají s výše uvedenými skupinami programů mnoho společného. Jedná se zejména o
    jednoúčelové programy, specializované na analýzy obvodů, které nelze realizovat programy typu SPICE. Programy typu „SPICE-compatible“ jsou široce
    využívány mimo jiné proto, že umožňují neomezené rozšiřování sortimentu modelovaných součástek o nové typy, jejichž modely se průběžně objevují na
    webu a následně i v inovovaných knihovnách nových verzí programů. Na akademických pracovištích i v průmyslu je oblíbeným produktem
    OrcadPSpice.\cite[s.~10]{Biolek2}

  \section{Simulace a analýza v programu LTspice IV}   