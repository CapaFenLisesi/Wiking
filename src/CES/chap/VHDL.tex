% !TeX spellcheck = cs_CZ
{\tikzset{external/prefix={tikz/CES/}}
 \tikzset{external/figure name/.add={ch11_}{}}
%---------------------------------------------------------------------------------------------------
% file VHDL.tex
%---------------------------------------------------------------------------------------------------
\lstset{ %
  language=VHDL,                         % choose the language of the code
  basicstyle=\small,                     % the size of the fonts that are used for the code: 
                                         % basicstyle=\small, \footnotesize
  backgroundcolor=\color{White},         % choose the background color. You must add  
                                         % \usepackage{color}
  commentstyle=\color{help}\textit,
  stringstyle=\ttfamily,                 % typewriter type for strings
  keywordstyle=\color{keyword}\textbf,
  breaklines=true,                       % sets automatic line breaking
  breakatwhitespace=true,                % sets if automatic breaks should only happen at whitespace
  showspaces=false,                      % show spaces adding particular underscores
  showstringspaces=true,                 % underline spaces within strings
  showtabs=true,                         % show tabs within strings adding particular underscores
  frame=none,	                           % adds a frame around the code - none, single
  tabsize=8,                             % sets default tabsize to 8 spaces
  captionpos=b,                          % sets the caption-position to bottom
  numbers=left,                          % where to put the line-numbers -none, left, right
  numberstyle=\footnotesize,             % the size of the fonts that are used for the line-numbers
  stepnumber=1,                          % the step between two line-numbers. If it's 1 each line
                                         % will be numbered
  xleftmargin=3em,                       % adjust left margin
}
%===========================Kapitola: Jazyk VHDL==================================================
\chapter{Jazyk VHDL}
\minitoc

  \section{Návrh číslicového obvodu}
    \subsection{Popis číslicové funkce}
      Máme-li představu o funkci a struktuře budoucího číslicového obvodu, nastupuje proces 
      \emph{zachycení návrhu} (design entry, design capture), při kterém je nutné naše představy 
      přenést do počítačem zpracovatelné formy. Tuto úlohu, lze splnit na různých úrovních 
      abstrakce \cite[s.~19]{Stastny2010}:
      \begin{itemize}
        \item \textbf{hradlové/schématické} - navrhujeme přímo kreslením schématu budoucího 
              obvodu. Výhodou tohoto postupu je jeho srozumitelnost a zachycení skuteč\-né podoby 
              návrhu  - co máme ve schématu je to, co se realizuje. Nevýhody nicméně převyšují 
              výhody. Kreslení schéma je obvykle specifické pro zvolený obvodu, protože často 
              používáme struktury, které jsou k dispozici jen na příslušném PLD obvodu. Konverze do 
              jiného obvou, znamená překreslení schématu. Vlastní proces kreslení je pomalý a 
              únavný, protože pracujeme na nízké úrovni abstrakce - kreslíme obvod hradldo po 
              hradle. Chceme-li například realizovat stavový automat, musíme nejprve zminimalizovat 
              jeho přechodovou a výstupní funkci a pak nakreslit schéma. Snadno se můžeme dostat do 
              situace, kdy je nutné kompletní překreslení. 
        \item \textbf{meziregistrových přenosů} - tzv. \texttt{RTL} (\emph{Register Transfer  s  
              Level}). \texttt{RTL} popis je dnes standardním prostřekdem pro popis číslicové 
              funkce. Číslicové synchronní obvody se skládají ze dvou základních typů         
              logických bloků: pamě\-ťo\-vých prvků (registrů) a kombinačních funkcí. Na úrovni 
              abstrakce, číslicový obvod popisujeme tak, že jednotlivé struktury popíšeme pomocí 
              těchto dvou typů logiky a doplníme informací o jejich vzájemném propojení          
              (odkud, kam a přes jaké kombinační logické funkce jsou přelévána data mezi registry). 
              Popis obvodu je realizován v textové podobě pomocí zápisu ve speciálním 
              progra\-mo\-va\-cím jazyku (\texttt{HDL} - \emph{Hardware Description         
              Language}). Pro popis se používají nejčastěji jazyk \texttt{VHDL} a \texttt{Verilog}. 
              Použití \texttt{RTL} úrovně má nesporné výhody: získáme technologicky nezávislý popis 
              obvodu na relativně vysoké úrovni abstrakce, přičemž jeden řádek zdrojového kódu je v 
              hardware reprezentován typicky desítkami/stovkami hradel. To zvyšuje produktivitu 
              práce, zpřehledňuje vlastní návrh, zjednodušuje přenos návrhu mezi různými 
              technologiemi a zrychluje jak vlastní návrh, tak pozdější opravy. Jedinou 
              nevý\-ho\-dou je nevhodnost pro ryze asynchronní návrh, to ale není při práci s     
              hrad\-lo\-vý\-mi poli omezující, neboť hradlová pole jsou určena právě pro synchronní 
              číslicové obvody.
        \item \textbf{algoritmické} - neustále se zkracující délka návrhového cyklu spolu s  
              rostoucí komplexitou navrhovaných systémů nutí návrháře používat stále vyšší úrovně 
              abstrakce. Architektura na \texttt{RTL} úrovni je navrhována vždy s ohledem ke 
              skutečnému časování obvodu a použitém paralelizmu. Systém je na této úrovni narvržen 
              s odpovídajícím počtem výpočetních jednotek, řídicích bloků, sběrnic, apod. Problém 
              nastává v okamžiku, kdy v pozdějších fázích návrhu zjistíme, že navržená architektura 
              nesplňuje očekávání. Právě od\-stra\-ně\-ní informace o paralelismu a časování ze 
              zdrojového kódu je přínosem \emph{algoritmické syntézy}. Funkce bloku je popsána v 
              některém z jazyků na ''vyšší úrovni'' - např. \texttt{ANSI C}, \texttt{Handel-C}, 
              \texttt{SystemC}, nebo \texttt{System Verilogu}. Příslušný syntézní nástroj pak    
              dostane informaci o počtu funkčních jednotek a časování ve formě jednoduchých oemzení 
              (například povolíme použití nejvýše dvou násobiček a dvou sčítaček) a na základě 
              pžedloženého algoritmu vygeneruje \texttt{RTL} kód výsledného systému (datových cest 
              i řídicích bloků). Kaž\-dá změna v architektuře je triviální - zjistíme-li, že 
              výpočetní výkon systému je příliš nízký, stačí jen znovu spustit syntézní proces s 
              jiným počtem aritmetických jednotek. Rychlost celého procesu umožňuje vyzkoušet celou 
              řadu alternativ architektury a najít nejvhodnější kompromis mezi plochou a rychlostí 
              obvodu. Zjednodušuje se i verifikace. Použitelnost algoritmické syntézy zatím omezuje 
              fakt, že výsledek není tak dobrý v porovnání s návrhem od profesionála. 
      \end{itemize}
  \section{Úvod}
    Název VHDL představuje akronym — \texttt{VHSIC} \texttt{Hardware Des\-cription Language}. Samo 
    označení \texttt{VHSIC} je další akronym představující název projektu, v rámci něhož byl jazyk 
    VHDL zpracován, a znamená \texttt{Very High Speed Integrated Circuits}. I když označení 
    \texttt{VHDL} v tomto kontextu není příliš přiléhavé, vžilo se a obecně se používá. Jazyk VHDL 
    byl původně vyvinut především pro modelování a simulaci rozsáhlých systémů. Na mnoha jeho 
    konstruktech je to znát, některé z nich nemají pro syntézu vůbec význam. Zde se však budeme 
    zabývat především použitím jazyka VHDL k vytváření modelů určených pro syntézu číslicových 
    systémů. České termíny budou v prvním výskytu zapsány tučně. Často tyto termíny nejsou 
    ustálené, a proto budeme uvádět i jejich anglické ekvivalenty, které již většinou mají 
    ustálenou podobu.

  % ------------- Základní vlastnosti jazyka VHDL --------------------------------------------------
  \section{Základní vlastnosti jazyka VHDL}
     \begin{itemize}
       \item Je to otevřený standard (\emph{open standard}). K jeho použití pro sestavení  
             návrhových systémů není třeba licence jeho vlastníka, jako je tomu u jiných jazyků HDL 
             (například u jazyka \texttt{ABEL}). To je jeden z důvodů, proč je tento jazyk v 
             návrhových systémech často používán.
       \item Umožňuje pracovat na návrhu, aniž je předtím zvolen cílový obvod. Ten může být zvolen 
             až v okamžiku, kdy jsou známy definitivní požadavky na prostředí, v němž má navrhovaný 
             systém pracovat, a je možno cílový obvod měnit podle potřeby při zachování textu 
             popisujícího systém, může být zvolen obvod \texttt{PLD} nebo \texttt{FPGA} 
             (\emph{Device-independent design}).
       \item Je možno provést simulaci navrženého obvodu na základě téhož zdrojového textu, 
             který pak bude použit pro syntézu a implementaci v cílovém obvodu. Zdrojový text je 
             možno zpracovávat v různých simulátorech a v syntetizérech různých výrobců. 
             Odsimulovaný text může být použit v dalších projektech s různými cílovými obvody, což 
             je podporováno hierarchickou strukturou jazyka. Této vlastnosti jazyka se říká 
             přenositelnost (\emph{portability}) kódu.
       \item V případě úspěšného zavedení výrobku na trh lze popis modelu systému v jazyku VHDL   
             použít jako podklad pro jeho implementaci do obvodů \texttt{\textbf{ASIC}} vhodných 
             pro velké série.
     \end{itemize}
     \textbf{Některé námitky proti VHDL:}
     \begin{itemize}
       \item Jazyk VHDL je dosti „upovídaný", jazykové konstrukty nejsou navrženy tak, aby 
             zdrojový text byl stručný a při popisu modelu určitého systému se setkáme s 
             opakováním bloků stejného znění. Ty je však možno snadno vytvářet využitím        
             kopírování a podobných možností současných editorů.
       \item V jazyku VHDL je možno vytvořit neefektivní konstrukce, efektivnost nebo její   
             nedostatek nemusí být na první pohled ze zdrojového textu patrné. To je však vlastnost 
             i jiných jazyků vyšší úrovně a výsledná efektivnost konstrukce závisí nejen na kvalitě 
             programových návrhových prostředků, ale také na zkušenosti konstruktéra (návrháře).
     \end{itemize}
  
    Základní verze jazyka VHDL byla přijata jako standard \texttt{IEEE} číslo 1076 v roce 1987. 
    Konstrukty odpovídající tomuto standardu se označují jako konstrukty jazyka \texttt{VHDL-87}. 
    Podobně jako další standardy \texttt{IEEE}, i tento standard se v pravidelném pětiletém 
    intervalu aktualizuje. Upravená verze standardu byla přijata v roce 1993, odkazuje se na ni jako
    na standard \texttt{VHDL-93}.
  
    Vedle jazyka \texttt{VHDL} se setkáme také s jazykem \texttt{Verilog}, který má podobné 
    použití. Uvádí se, že jazyk \texttt{VHDL} je rozšířený zejména v Evropě, zatímco 
    \texttt{Verilog} se používá hlavně v asijských zemích. V USA se používají oba tyto jazyky.
    
    Vyjadřovací schopnosti jazyku VHDL jsou dány příkazy, jenž mají souběžný nebo sekvenční 
    charakter. Některé příkazy jsou jen jednoho druhu, jiné mohou být obojího druhu. Toto rozlišení 
    se týká toho, ve které části popisu se příkazy mohou používat. Pro stručnost budeme dále mluvit 
    o souběžných a o sekvenčních příkazech, i když jde spíše o to, kde se tyto příkazy nacházejí
    či mohou nacházet. V následujících kapitolách jsou příkazy rozděleny do dvou velkých skupin: 
       \begin{itemize}
         \item \hyperref[Section:VHDL_soubezne_prikazy]{Souběžné příkazy} (\emph{concurrent  
               statements}): zapisují se v textu jazyka mimo procesy, definice funkcí a procedur.
         \item \hyperref[Section:VHDL_sekvencni_prikazy]{Sekvenční příkazy} (\emph{sequential  
               statements}): slouží k algoritmickému vyjádření popisu. Tyto příkazy mohou být 
               zapsány jen v procesech, v definicích funkcí a procedur.
       \end{itemize}
  % ------------------------------------------------------------------------------------------------------------------------------
  \section{Logické úrovně}
   Od jazyka určeného k návrhu a modelování integrovaných obvodů očekáváme schopnost modelovat 
   základní logické úrovně - \emph{log. 0} a \emph{log. 1}. Ty jsou pro jednoduché simulace 
   postačující, ale již například pro návrh a modelování třístavových budičů sběrnic potřebujeme 
   mít možnost pracovat se stavem vysoké impedance \emph{Z}. Dalším jednoduchým příkladem může byt 
   modelování zkratu na sběrnici, který může vzniknout v situaci, kdy dva budiče budí jeden spoj
   opačnými logickými úrovněmi. Balíček \texttt{std\_logic\_1164} knihovny \texttt{IEEE} pro tyto 
   účely zavádí \emph{devítistavovou logiku}. 
   
   \begin{itemize}\addtolength{\itemsep}{-0.5\baselineskip}
     \item 'U': uninitialized. This signal hasn't been set yet.
     \item 'X': unknown. Impossible to determine this value/result.
     \item '0': logic 0
     \item '1': logic 1
     \item 'Z': High Impedance
     \item 'W': Weak signal, can't tell if it should be 0 or 1.
     \item 'L': Weak signal that should probably go to 0
     \item 'H': Weak signal that should probably go to 1
     \item '-': Don't care.
   \end{itemize}
   
   Všimněme si, že tato knihovna je připojena všemi ukázkovými kódy. Logický signál, který má 
   popsaných hodnot nabývat, je pak definován jako signál typu \texttt{std\_logic}, nebo 
   \texttt{std\_logic\_vector} pokud se jedná o sběrnici. Pomocí nich modelujeme standardní 
   propojení logických prvků, tj. ''obyčený kus drátu''.
   \cite[s.~51]{Stastny2010}
  \section{Souběžné příkazy} \label{Section:VHDL_soubezne_prikazy}
  \newpage
  \section{Sekvenční příkazy}\label{Section:VHDL_sekvencni_prikazy} 
  \section{Technologicky nezávislá část návrhu}
    V následujícím textu jsou uvedeny základní způsoby popisu chování sekvenčních (např. klopné 
    obvody) a kombinačních obvodů. Klopné obvody jsou rozdělovány na \textbf{hranově citlivé} a 
    \textbf{úrovňově citlivé}. Je možné je popsat jako \emph{jednobitové paměti}. Hranově citlivý 
    klopný obvod je obvod řízený změnou na vstupu synchronizace (\emph{clock input}).
    Bývá označován "\emph{flip-flop}".  Úrovňově citlivý klopný obvod je nazýván v české 
    terminologii \emph{zdrž}. Obvykle je však srozumitelný anglický název "\emph{latch}".
  
    \subsection{Dynamicky řízené sekvenční obvody}
      Obvody řízené změnou signálu na vstupu synchronizace jsou ve VHDL popisovány použitím příkazu 
      process a podmíněného příkazu  \lstinline[basicstyle=\ttfamily]!if!. V podmíně\-ném příkazu 
      jsou rozlišovány události (\lstinline[basicstyle=\ttfamily]!events!), které znamenají 
      vzestupnou hranu nebo sestupnou hranu signálu na vstupu synchronizace. Při popisu je možné 
      použít dvou různých zápisů, ve kterých se objevuje atribut události synchronizačního
      signálu \lstinline[basicstyle=\ttfamily]!clk clk'event! nebo volání funkce.
      \begin{itemize}\addtolength{\itemsep}{-0.5\baselineskip}
        \item \lstinline[basicstyle=\ttfamily]!(clk'event and clk='1')!  	 vzestupná hrana signálu
        \item \lstinline[basicstyle=\ttfamily]!(clk'event and clk='0')!    sestupná hrana signálu
        \item \lstinline[basicstyle=\ttfamily]!rising_edge(clk)!		     voláni funkce vzestupné hrany
        \item \lstinline[basicstyle=\ttfamily]!falling_edge(clk)!		     voláni funkce sestupné hrany
      \end{itemize}
      Uvedené příklady ukazují možnosti vyjádření vzestupné a sestupné hrany ve VHDL. Vyjádření 
      pomocí atributu je častěji používané, protože tento konstrukt je rozeznatelný při syntéze 
      obvodového řešení. Nicméně použití volání funkce je výhodnější při simulaci, protože nastává 
      pouze při změně signálu \lstinline[basicstyle=\ttfamily]!clk! z $0\rightarrow1$ a
      z $1\rightarrow0$, ale ne z $X\rightarrow1$ nebo z $0\rightarrow X$, které nepředstavují 
      platný přechod z jednoho stavu do druhého. Devět hodnot signálu, které jsou označované jako 
      \lstinline[basicstyle=\ttfamily]!std_logic! jsou určeny k modelování poruchových stavů 
      logické sítě. Jsou to hodnoty \emph{'U','X','0','1'¸'Z','W','L','H','-'}.
 
        \subsection{Staticky řízené sekvenční obvody}
          V následujících odstavcích jsou popisovány klopné obvody řízené úrovní synchronizačního signálu, které jsou známé pod
          názvem \textbf{zdrž} (\emph{latch}).
    
    \subsection{Kombinační obvody} 

  \section{Knihovna LPM}
    Knihovna \textbf{LPM} (angl. \emph{Library of Parametrized module}) obsahuje parametrizovatelné 
    moduly jako jsou hrada, čítače, multiplexory, klopné obvody, aritmetické a paměťové funkce. 
  
    Standard \texttt{LPM} byl navržen v roce 1990 jako jedna z monžností pro efektivní návrh 
    číslicových systémů do odlišných technologií, jako jsou např. obvody PLD, hradlová pole a 
    standardní buňky. Předběžná verze standardu vyšla v roce 1991, další úprava předběžné verze pak 
    v roce 1992. Standard byl přijat organizací \texttt{EIA} (angl. \emph{Electronic Industires 
    Alliance}) v dubnu roku 1993 jako doplněk do standardu \texttt{EDIF}.
  
    \texttt{EDIF} je formát pro přenos návrhu mezi návrhovými nástroji různých výrobců. Formát 
    \texttt{EDIF} popisuje syntaxi, která reprezentuje logický netlist. \texttt{LPM} do něj pak 
    přidává množinu funkcí, která popisuje logické operace netlistu. Před rozšířením o \texttt{LPM} 
    musel každý \texttt{EDIF} netlit typicky obsahovat technologicky specifické logické funkce,
    které zabraňovaly tomu, aby byl návrh ve větší míře nezávislý na cílové technologii \cite[s.~72]{Pinker2006}. 
  
    \subsection{Posuvný registr - lpm shiftreg}
  
    \begin{table*} 
      \centering
      \begin{tabular}{|c|p{3.5cm}|p{8cm}|}
        \hline
           \rowcolor{CornflowerBlue} {\textbf{Jméno}} & {\textbf{Popis}} & {\textbf{komentář}} \\
        \hline\hline       
           \texttt{data[]}   & Data input to the shift register    
                             & Šířku registru určuje parametr \texttt{LPM\_WIDTH}. \\
        \hline   
           \texttt{clock}    & Positive-edge-triggered clok        
                             & Vstup hodinového signálu. \\
        \hline     
           \texttt{enable}   & Clock enable input                  
                             & Blokuje hodinový signál. \\
        \hline      
           \texttt{shiftin}  & Serial shift data input             
                             & Pro funkci je nutné použit alespoň jeden ze signálů
                               \texttt{data[]}, \texttt{aset}, \texttt{aclr}, \texttt{sset}, 
                               \texttt{sclr}, a/nebo \texttt{shiftin}.  \\
        \hline    
           \texttt{load}     & Synchronous parallel load           
                             & (1) load operation (podmínka: \texttt{enable} = 1); (0) shift
                               operation (výchozí). \\
        \hline    
           \texttt{aclr}     & Asynchronnous clear input          
                             & Signál \texttt{aclr} má vyšší prioritu než signál
           \texttt{aset}.  \\
        \hline    
           \texttt{aset}     & Asynchronnous set input             
                             & Naplní registr \texttt{g[]} hodnotou \texttt{LPM\_AVALUE} \\
        \hline 
           \texttt{sclr}     & Synchronous clear input             
                             & Signál \texttt{sclr} má vyšší prioritu než signál
           \texttt{sset}.  \\
        \hline  
           \texttt{sset}     & Syncrhonous set input              
                             & Naplní registr \texttt{g[]} hodnotou \texttt{LPM\_SVALUE} \\
        \hline  
           \texttt{q[]}      & Data output from the shift register 
                             & Šířku registru určuje parametr \texttt{LPM\_WIDTH}. Vyžaduje
           \texttt{shiftout}.  \\
        \hline 
           \texttt{shiftout} & Serial shift data output          
                             & Vyžaduje registr \texttt{q[]}. \\
        \hline                                                           
      \end{tabular}
      \caption{Popis portů komponenty \texttt{lpm\_shiftreg}.}
      \label{VHDL:tab_lpm_shiftreg}       
    \end{table*}
    
    %---------------------------------------------------------------
    \lstinputlisting{../src/CES/VHDL/lpm_shiftreg.vhd}
    %---------------------------------------------------------------

} % tikzset
%---------------------------------------------------------------------------------------------------
\printbibliography[title={Seznam literatury}, heading=subbibliography]
\addcontentsline{toc}{section}{Seznam literatury} 