%==============================Kapitola: Spojitě regulované napájecí zdroje===========================================  
\chapter{Spojitě regulované napájecí zdroje}
\minitoc
\newpage
  \section{Metody snímání proudu v napájecích zdro\-jích}
  \section{Neřízené usměrňovače}
    \subsection{Usměrňovače s nesetrvačnou zátěží}
    \subsection{Usměrňovače se sběrným kondenzátorem (s RC zátěží)}
    \subsection{Usměrňovače s nárazovou tlumivkou (s RL zá\-tě\-ží)}
      Základní schéma zapojení je na obr. \ref{enz:fig_usm_1f_RLz}. Obsah vyšších rušivých harmonických produktů lze snížit sériově se zátěží zapojeným obvodem typu hornofrekvenční zádrž. I zde je filtrační účinek závislý na poměru zatěžovací konstanty $\tau_L=\frac{L}{R_L}$ a doby periody $T=\frac{1}{f}$ - je to činitel $K=\tau_L/T$. $(K \uparrow: \tau_L\uparrow,T\downarrow\Rightarrow L\uparrow,R_L\downarrow)$. Tyto usměrňovače jsou tedy výhodné pro zátěže typu "malé napětí x velký proud". I tady lze elegantně podpořit velkou hodnotu koeficientu $K$ tím, že obvod budeme napájet signálem o vysokém kmitočtu. Zatím co v případě RC zátěže byl kondenzátor pamětí napětí, tady použitá tlumivka je naopak pamětí proudu. Z toho důvodu je např. jednocestné zapojení s nárazovou tlumivkou fyzikálně nevhodné, protože mohou nastat jen dva krajní (a oba špatné) případy.
      \begin{figure}[ht!]
        \centering
        \includegraphics[scale=0.6]{patocka_jednocestny_1f_usm_LRzatez.pdf}
        \caption{Jednocestný jednofázový usměrňovač RL zátěží.}
        \label{enz:fig_usm_1f_RLz}
      \end{figure}

      \begin{itemize}
        \item Bude-li indukčnost veliká (v limitě nekonečná), pak podle Lenzova pravidla udrží proud v obvodu (je celý v sérii) na stálé hodnotě i co do směru, dioda se nemůže vůbec uzavřít a obvod nemůže usměrňovat - nebude mít střídavou složku.
        \item Naopak při malé hodnotě (proti jakési kritické) dioda zavře, obvod se přeruší a energie magnetického pole cívky (je vázána existencí proudu) se nemůže uplatnit v překlenutí mezer dodávky energie na výstup. Při dalším otevření diody můžou navíc nastat přechodné děje.
      \end{itemize}

      Existuje však velice jednoduchá a plně funkční úprava a tou je doplnění jednocestného usměrňovače tzv. \textbf{rekuperační (nulovou) diodou} -
      $D_R$ (kreslena čárkovaně viz obr. \ref{enz:fig_usm_1f_RLz}). Při uzavření hlavní diody $D_1$ se tlumivka snaží držet proud obvodem ve stejné
      velikosti a ve stejném směru a tento proud otevře rekuperační diodu $D_R$.

      Mezi filtrací se sběrným kondenzátorem (RC zátěží) a s nárazovou tlumivkou (RL zátěží) je ještě zajímavý rozdíl: filtrace paralelním
      kondenzátorem pracuje s hyperbolicky se měnící impedancí $X_C=\frac{1}{\omega C}$ a potlačení vysokých čísel harmonických je čím dál tím menší.
      Obvod se sériovou tlumivkou pracuje s impedancí $X_L=\omega L$ a potlačující efekt lineárně roste. \emph{Zvlnění na RC zátěži má proto obvykle
      dosti značný obsah vysokých harmonických a je pilovitého průběhu. Zvlnění na RL zátěži je za stejných podmínek $K$ harmonicky čistší a má
      charakter sinusovky.}
      \begin{example}
        Proveďte simulaci vyznačených obvodových veličin obvodu na obrá\-zku \ref{enz:fig_SIM002_sch} se zadanými hodnotami prvků v programu
        PSpice\footnote{Simulace je provedena v programu OrCAD PSpice ver. 16.3}.
        \begin{figure}[ht!]
          \centering
          \includegraphics[scale=0.8]{patocka_jednocestny_1f_usm_PSpice.pdf}
          \caption{Neřízený Jednofázový usměrňovač s nulovou diodu. Simulované veličiny jsou vyznačeny barevnými markery.}
          \label{enz:fig_SIM002_sch}
        \end{figure}

        \begin{figure}[ht!]
          \centering
          \includegraphics[scale=0.5]{PSPice_SIM002_usm1f_RL_6mH_5R.pdf}
          \caption[Jednofázový neřízený usměrňovač s RL zátěží]{Průběhy vyznačených veličin jednofázového neřízeného usměrňovače s RL zátěží (6mH, 5$\Omega$) a nulovou diodou [ENZ/SIM002]}
          \label{enz:fig_exam_PSPICE_SIM002}
      \end{figure}

      \end{example}


  \section{Stabilizátory stejnosměrného napětí}
    Stabilizátory napětí na svém výstupu konstantní napětí v pokud možno co nejširším rozsahu odebíraného výstupního proudu a dodávaného vstupního napětí \cite{Zahlava}
    \begin{itemize}
      \item nelineární (parametrické) stabilizátory napětí,
      \item lineární spojité stabilizátory napětí
    \end{itemize}

    \subsection{Nelineární (parametrické) stabilizátory}
      Využívají vlastností VA charakteristik některých jako je otevřený PN přechod, Zenerovy diody, termistory a jiné. Pro tyto účely potřebujeme tzv. prvky triodového typu u kterých platí, že \emph{dynamický vnitřní odpor je podstatně nižší jak statický}. Tedy $R_{dyn} < R_{stat}$. Pro naše účely jsou nejčastěji používané \emph{Zenerovy diody} a dvojpólové integrované \emph{napěťové referenční obvody}. Zenerovy diody jsou vyráběny jako malovýkonové ( anodová ztráta do 1W) a výkonové (obvykle 10W a více). Pro referenční účely jsou často doplněny dalšími pomocnými kompenzačními prvky. Vlastní princip nelineární\-ho spojitého stabilizátoru ( podivný název „parametrický“ nebudeme použí\-vat) je velice prostý: obvod dle obr. \ref{enz:fig_sch_ZD_stab} tvoří dělič s horním odporem lineárním a dolním (je paralelně k zátěži) tvořeným popsaným nelineárním odporem triodového typu. Za těchto okolností má tento obvod pochopitelně přenos dynamický podstatně menší jak statický a tedy je to stabilizátor napětí.

      \begin{figure}[ht!]
         \centering
         \includegraphics[scale=0.6]{patocka_stabilizator_ZD_sch.pdf}
         \caption{Nelineární spojitý stabilizátor napětí}
         \label{enz:fig_sch_ZD_stab}
       \end{figure}

      \begin{figure}[ht!]
         \centering
         \includegraphics[scale=0.5]{patocka_stabilizator_ZD_VA.pdf}
         \caption{Grafické řešení nelineárního stabilizátoru}
         \label{enz:fig_graf_res_ZD_stab}
       \end{figure}

       Řešení je výhodné v grafické podobě - obr. \ref{enz:fig_graf_res_ZD_stab}. Ve třetím kvadrantu je nakreslena VA charakteristika\footnote{Je typická určitým Zenerovým napětím $U_z$, sklonem pracovní části VA charakteristiky (dynamickým vnitřním odporem  $r_z$) a dovolenou anodovou ztrátou $P_{dov}$. Tato ztráta závisí na způsobu chlazení.} Bod \texttt{B} odpovídá zvolenému vstupnímu napětí $U_1$ a výstupnímu proudu $I_2$ a tedy i velikosti odporu $R_L$. Úloha může být nyní dána např. kolísáním vstupního napětí od $U_{1_{max}}$ do $U_{1_{min}}$ (body \texttt{A} a \texttt{C}), nebo kolísáním zátěže nebo proudu $I_2$ ( body \texttt{E'}a \texttt{D'}). Při současném působení změn vznikne obrazec (přibližně obdélník) \texttt{X}, \texttt{E}, \texttt{D}, \texttt{H}, což je geometrické místo možných stavů obvodů. Na vlastní VA charakteristice prvku pro volený "předřadný" odpor \emph{R} vzniknou body \texttt{G}, \texttt{P} a \texttt{F} a to je grafické řešení. Vidíme, kdy hrozí "zhasnutí" nebo přetíženi Zenerovy diody. Projekcí bodu \texttt{G}, \texttt{P} a \texttt{F} na vodorovnou osu zjistíme okamžité hodnoty výstupního napětí stabilizátoru $U_2$ a jeho kolísání $\Delta U_2$. Lze snadno odečíst zvládnutelné kolísání vstupního napětí či velikosti zátěže atd. Z obrázku je také vidět, že zlepšení stabilizačního účinku obvodu lze dosáhnout zvětšením vstupního napětí (větší odpor \emph{R}) nebo výběrem diody s menším dynamickým odporem $r_z$ . Velice vtipná možnost zlepšení přenosových vlastností stabilizátoru je při náhradě lineárního odporu \emph{R} nelineárním prvkem s pentodovým charakterem VA charakteristiky. Může to být třeba bipolární nebo lépe unipolární tranzistor. Pak vlastně Zenerovu diodu napájíme zdrojem konstantního proudu a to je hojně využívané v integrovaných stabilizátorech.

       Z obr. \ref{enz:fig_sch_ZD_stab} lze snadno odvodit činitel napěťové stabilizace
       \begin{equation}\label{enz:eq_Su_ZD}
         S_u = \frac{\Delta u_{vst}}{\Delta u_{v\acute{y}st}} = \frac{R+r_z\parallel R_L}{r_z\parallel R_L} \cong \frac{R+r_z}{r_z}, \mathrm{kde} r_z\ll R_L
       \end{equation}

        \begin{example} Vliv nenulového dynamického odporu Zenerovy diody na zvlnění výstupního napětí \cite{PAEE} str. 369.
         \newline
         \textbf{Zadání}: $U_S = 14 V; u_{ripple} = 100 mV; U_Z = 8 V; r_Z = 10 \Omega; R_S = 50 \Omega; R_L = 150 \Omega$.
         \begin{figure}[ht!]
           \centering
           \subfloat[ ]{
             \includegraphics[width=0.4\textwidth]{stabilizator_ZD_ripple_effect.pdf}}
           \subfloat[ ]{\label{enz:fig_ZD_ripple_graph}
             \includegraphics[width=0.4\textwidth]{stabilizator_ZD_ripple_graph.pdf}}
           \caption[ Vliv $r_z$ ZD na přenos zvlnění]{K příkladu nenulového dynamického odporu Zenerovy diody na přenos zvlnění ze vstupního napětí na výstupní napětí}
           \label{enz:fig_ZD_ripple}
         \end{figure}
         \newline
         \textbf{Řešení}: Abychom stanovili velikost výstupního napětí, amplitudu zvlnění napětí na zátěži a mohli také určit vliv velikosti dynamického odporu $r_z$, vyjděme z náhradního lineárního obvodu na obrázku \ref{enz:fig_ZD_NLO}.
         \begin{enumerate}
           \item Stejnosměrný ekvivalentní obvod:
             \begin{equation}\label{enz:eq_ZD_DC_UL}
                U_L = U_S\frac{R_L\parallel r_z}{R_S + R_L\parallel r_z} + U_z\frac{R_S\parallel R_L}{r_z + R_S\parallel R_L}= 2.21 + 6.32 = 8.53V
             \end{equation}
           \item Střídavý ekvivalentní obvod:
             \begin{equation}\label{enz:eq_ZD_AC_UL}
                u_L = v_{ripple}\frac{r_z\parallel R_L}{R_S + r_z\parallel R_L} = 0.016V
             \end{equation}
             Tedy jedna šestina zvlnění vstupního napětí se přenese na výstupní svorky stabilizátoru.
         \end{enumerate}
         \begin{figure}[ht!]
           \centering
           \subfloat[Stejnosměrný náhradní obvod ]{\label{enz:fig_ZD_DC_equival}
             \includegraphics[scale=1.5]{stabilizator_ZD_DC_equival.pdf}}
           \subfloat[Střídavý náhradní obvod]{\label{enz:fig_ZD_AC_equival}
             \includegraphics[scale=1.5]{stabilizator_ZD_AC_equival.pdf}}
           \caption{Stabilizátor se ZD lze pro výpočet jeho ss chování v okolí pracovního bodu linearizovat pomocí NLO }
           \label{enz:fig_ZD_NLO}
         \end{figure}
         Schopnost stabilizace je horší, čím větší má Zenerova dioda dynamický odpor $r_z$. Proto musí být $r_z$ výrazně nižší, než hodnoty rezistorů $R_S$ a $R_L$ (viz rov. \ref{enz:eq_ZD_AC_UL}).
        \end{example}

    \subsection{Lineární spojité stabilizátory}
  \newpage
  \section{Násobiče napětí}
  \section{Ochranné a signalizační obvody zdrojů}
    \subsection{Pojistky}
       \subsubsection{Signalizace přerušené pojistky}
         Rozsvícením svítivé diody $D_1$, je uživatel upozorněn na přerušenou tavnou pojistku $PO_1$ v zařízení napájeném malým napětím. Je-li pojistka v pořádku, je při zapnutém vypínači $S_1$ na svítivé diodě napětí tvořené úbytky na pojistce a otevřené diodě $D_2$, jenž nestačí pro její rozsvícení. Jakmile se však pojistka přeruší, dioda $D_1$ se rozsvítí. Průchodu proudu spotřebičem přes $D_2$ při sepnutém vypínači $S_1$ brání její polarizace.

         \begin{figure}[ht!]
           \centering
           \includegraphics[scale=0.3]{sig_cir_fuse_failure.pdf}
           \caption{Obvod signalizující přerušení pojistky v nízkonapěťovém obvodu.}
           \label{enz:fig_fuse_failure}
         \end{figure}

