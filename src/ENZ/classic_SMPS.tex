%===========Kapitola: Impulzně regulované napájecí zdroje=========================================== 
\chapter{Impulzně regulované napájecí zdroje}
\minitoc
\newpage
  \section{Úvod}
    Spínané napájecí zdroje plní funkci stejnou jako zdroje se spojitou regulací. Vý\-ko\-no\-vý
    člen spínacích zdrojů je však zatěžován impulzně, tj. střídavě spínán a rozepínán. Lze tedy
    využít výhody impulzního režimu, tj. odebírat impulzní výkon podstatně větší, než je trvalý
    výkon při lineárním režimu regulátoru s týmž výkonovým členem. Spínací zdroje mají obecně větší
    účinnost než zdroje se spojitou regulací. Jsou výhodné zvláště tam, kde je velký rozdíl napětí
    na vstupu a výstupu regulátoru a kde jsou požadované malé rozměry. Impulzní regulace zajistí
    stabilizované výstupní napětí i pro velké změny vstupního napětí; účinnost zdroje se při tom
    téměř nemění. I přes větší obvodovou složitost jsou ekonomicky výhodnější, neboť jejich použití
    vede k podstatné energetické úspoře.
  
    Impulzně regulované zdroje však mají v porovnání se zdroji s lineární regulací i některé
    nevýhodné vlastnosti, například pomalejší reakci výstupního napětí na rychlé změny zatěžovacího
    výstupního proudu. Při požadavku malého zvlnění výstupního napětí se nesmí zanedbat vliv
    impulzního charakteru těchto zdrojů. Impulzně regulované zdroje jsou také zdrojem rušivých
    signálů, které jsou generovány spínacími prvky \cite[s.~112]{Hammembauer}.
       
    % \begin{table}[ht!]
      % \centering
      % \setlength{\tabcolsep}{5pt}
      % \begin{tabular}{p{1.2cm}p{7cm}}
        % \hline
        % \multirow{2}{*}{\googlepdflink{AN556}{https://docs.google.com/file/d/0BzP8spWKnOvnLXZBV3YzLWloR0U/edit}}
         % &  \textbf{Introduction to Power Supplies} \pdfcomment[icon=Note, color=Melon]{ahoj}\\
         % &  2002 National Semiconductor    \\
         % \hline
       % \end{tabular}
    % \end{table}    
   
  \section{Impulzní regulace ve výkonové elektronice}\label{ENZ:kap_Imp_Reg}
    Základním principem a současně odlišností impulzní regulace od regulace klasické je v její
    \emph{nespojitosti}. To znamená, že nehledě na detailní realizaci, je výstupní napětí 
    stabilizováno zásahy regulačního členu pouze v určitých, časově omezených intervalech. Podstata 
    regulačního členu (regulátoru) tedy spočívá v řízení vzájemných časových relací aktivního a 
    pasivního intervalu pracovního cyklu v závislosti na velikosti zesílené regulační odchylky.
    
    Akční člen je tedy řízen dvouhodnotovým signálem, mající význam \emph{zapnutí} nebo 
    \emph{vypnutí} výkonové součástky. Následující příklad demonstruje, jak lze tento signál 
    vytvořit pomocí \textbf{pulzně-šířkové modulace} v simulátoru \ltLtspiceSW. V simulacích 
    některých topologií spínaných zdrojů bude místo zdroje s lineárně narůstajícím výstupním napětím 
    viz obr. \ref{enz:fig_pwm_wave} použita regulační odchylka.
    
    \begin{figure}[ht!]
      \centering
      \includegraphics[width=\linewidth]{hamm_schema_imp_reg.pdf}
      \caption[Schéma impulzního regulátoru]{Základní schéma impulzního regulátoru}
      \label{enz:fig_imp_reg_basic}
    \end{figure}
    
    Srovnáme-li pro názornost klasický a impulzní regulátor na úrovni blokových schémat, vidíme, že 
    obě jsou formálně dosti podobná. U obou nacházíme napěťový normál \texttt{Uref}, zesilovač 
    regulační odchylky \texttt{Au}, budící obvod i výkonový regulační člen a samozřejmě i 
    zpětnovazební smyčku. Tím však, snad až na základní podstatu regulační smyčky podobnost končí. 
    Funkčně jsou oba regulátory naprosto odlišné.
    
    U spojitého lineárního regulátoru ovládá odchylka výstupního napětí od jmenovité velikosti 
    spojitě okamžitý odpor výkonového regulačního členu v libovolném o\-kam\-ži\-ku tak, aby 
    výstupní napětí bylo konstantní. Z toho, jak je již známo, vyplývá velká poměrná výkonová ztráta 
    na regulačním členu a tedy i malá účinnost spojité regulace za běžných provozních podmínek.
    
    Impulzní regulace obr. \ref{enz:fig_imp_reg_basic} umožňuje výrazně snížit výkonovou ztrátu na
    regulačním členu. V tomto případě pracuje regulační prvek (tranzistor) jako řízený spínač. Proud 
    jím tedy prochází pouze po určitý interval pracovního cyklu. Přitom okamžitá výkonová ztráta v 
    aktivním (sepnutém) stavu je vzhledem k $U_{CES}\rightarrow 0$ řádově menší, než u lineárního 
    regulátoru. Další předností je, že velikost ztráty v podstatě nezávisí na rozdílu vstupního a 
    výstupního napětí, ale prakticky pouze na kolektorovém proudu tranzistoru.
    
    Možnost použít spínací regulační člen při stabilizaci stejnosměrného napětí je podmíněna jeho 
    vzájemnou součinností s filtračním členem, který na rozdíl od aplikace ve spojitém regulátoru 
    musí mít výrazný akumulační charakter. Uspořádání filtru, který je pro větší výkony vždy typu 
    LC, je podřízeno topologii měniče. Princip činnosti nerozlučně vázané dvojice spínač - 
    akumulační výstupní filtr spočívá v akumulaci energie, která je v aktivním intervalu odebrána ze 
    zdroje, aby mohla být v následujícím pasivním intervalu (spínač vypnut) dodávána z filtru do 
    zátěže \cite[s.~121]{Hammembauer}.
           
    \begin{example} 
      Na obr. \ref{enz:fig_pwm_gen} je realizován generátor šířkově modulovaného signálu pro
      simulátor \texttt{LTSpice}, jenž s výhodou využívá komponenty \texttt{B-source}, umožňující
      behaviorální popis požadovaného průběhu.
      \begin{figure}[ht!]
        \centering
        \includegraphics[width=\linewidth]{LTspice_pwm_gen.pdf}
        \caption[LTSpice - PWM generátor]{Realizace PWM generátoru pomocí komponenty B-source
                \emph{(Arbitrary behavioral voltage or current source)} v LTSpice (soubor
                \texttt{pwm.asc})}
        \label{enz:fig_pwm_gen}
      \end{figure}
      Podrobnějším pohledem na zápis rovnic dle obr. \ref{enz:fig_pwm_gen}, lze dojít k závěru, že
      zdroj \texttt{B1} na svůj výstup vnutí hodnotu parametru \texttt{Vhigh}, nebo \texttt{Vlow},
      podle výsledku rozhodovací funkce \texttt{if}. Tj. jeli
      \texttt{Time-floor(Time*f)/f)*Range*f)} větší než \texttt{V(input)}, bude na výstupu $V_{high}
      = 5V$, v opačném případě $V_{low} = 0V$. Funkce \texttt{floor} zaokrouhluje hodnotu svého
      argumentu na celé číslo (\texttt{integer}), což vede na schodovitý průběh a funkce
      \texttt{Time} umožňuje do vztahu vnést okamžitou hodnotu simulačního času. Vzájemný odečtením
      získáme pilový průběh, kterým se komparuje s okamžitou hodnotou zdroje \texttt{V(input)}.
       
       \begin{figure}[ht!]
         \centering
         \includegraphics[width=1\linewidth]{LTspice_pwm_wave.pdf}
         \caption[LTSpice - PWM generátor: průběhy]{Výstupní signál \texttt{V(pwm)} z PWM
                  generátoru na obr. \ref{enz:fig_pwm_gen}  má-li rozhodovací napětí \texttt{V(in)}
                  lineární charakter}
         \label{enz:fig_pwm_wave}
       \end{figure}       
    \end{example} 
   
    \newpage
    %------------------------------------------------------
    % Section: DC/DC měniče s transformátorem
      % DC_DC_converter.tex
%===============================Podkapitola: DC/DC měniče bez transformátoru========================
  \section{DC/DC měniče bez transformátoru}\label{ENZ:kap_DC_DC}
    \subsection{Vymezení pojmů a základních požadavků}
      DC - DC měniče jsou obvody sloužící k regulaci elektrické energie, které mění vstupní
      stejnosměrně napětí U1 na jiné výstupní stejnosměrné napětí U2. Budeme se přitom zabývat
      měniči tzv. \emph{napěťového typu}, což jsou měniče napájené konstantním vstupním napětím z
      napěťového zdroje, nikoliv proudem, z proudového zdroje. V této kapitole se omezíme pouze na
      měniče bez transformátoru, které tedy neumožňuji galvanické oddělení výstupu od vstupu
      \cite{Patocka}.

      Každý měnič sestává z vlastního silového obvodu a řídicí elektroniky (regulačních obvodů).
      Silové obvody nesmí využívat při regulaci energie rezistorů a proto se mohou skládat jen ze
      \textbf{spínačů} a \textbf{akumulačních prvků}, tj. \emph{indukčnosti} a \emph{kapacit}.

      \subsubsection{Napájecí zdroj a zátěž měniče}
        \begin{figure}[b]
          \centering
          \subfloat[náhradní schéma akumulátoru]{\label{enz:fig_nahr_sch_aku}
            \includegraphics[width=0.4\linewidth]{patocka_aktivni_zatez_nahrad_sch.pdf}}
           \hfill 
           \subfloat[náhradní schéma stejnosměrného elektromotoru s cizím
                     buzením]{\label{enz:fig_nahr_sch_ss_motor}
            \includegraphics[width=0.4\linewidth]{patocka_aktivni_zatez_ss_mot_nahrad_sch.pdf}}
          \caption{Aktivní zátěž.}
          \label{enz:fig_aktivni_zatez}
        \end{figure}
        DC/DC měniče mohou přenášet energii z principu oběma směry. Mohou tedy čerpat energii ze
        zdroje a dodávat ji do zátěže nebo také opačně energii čerpat ze zátěže a dodávat ji do
        zdroje. Pojmy zátěž a zdroj je proto nutné chápat v širším slova smyslu.
        \begin{itemize}
          \item Zdrojem s konstantním napětím $U_1$, schopným dodávat i akumulovat energii, je
                akumulátor. Použijeme-li jako zdroj např. usměrňovač se sběrným kondenzátorem, pak
                není schopen dlouhodobě jímat energii z měniče, tj. dlouhodobě nesmí ve střední
                hodnotě převládat směr proudu do kladné svorky zdroje (krátkodobě, v okamžité
                hodnotě, je takový směr možný). Nabíjením sběrného kondenzátoru by totiž rostlo
                napětí $U_1$. Tomu lze zabránit přeměnou dodávané energie na teplo ve vybíjecím
                rezistoru, či na Zenerově diodě, zapojené paralelně ke sběrnému kondenzátoru.
          \item Z hlediska schopnosti spotřeby či dodávky energie, lze rozlišovat zátěž
                \emph{aktivní} a \emph{pasivní}. Aktivní zátěži je opět např. akumulátor, ale třeba
                i stejnosměrný motor. Jeho náhradní zapojeni, platné v ustáleném stavu, je uvedeno
                na obr. 1 1). Vnitřní rotační (pohybové) indukované napětí je úměrně otáčkám, proud
                pak momentu na hřídeli a to včetně znamének.
        \end{itemize}

        Teče-li proud ve  střední hodnotě do zátěže (+I), pak motor pohání, tj. mění elektrickou
        energii na mechanickou (pracuje v \emph{motorickém režimu}). Teče-li ze zátěže (-I), pak
        motor brzdí, tj. mění z vnějšku dodávanou mechanickou energii na energii elektrickou
       (pracuje v \emph{generátorickém režimu}).
       
      \subsubsection{Pracovní kvadranty}
         \begin{figure}[ht!]
           \centering
           \includegraphics[width=0.7\linewidth]{patocka_pracovni_kvadranty_sch.pdf}
           \caption{Označení vstupních a výstupních veličin DC/DC měniče.}
           \label{enz:fig_prac_kvadranty_sch}
         \end{figure}

         Označme si vstupní a výstupní napětí a proud měniče podle obr.
         \ref{enz:fig_prac_kvadranty_sch}. Podle polarity výstupního napětí $U_2$ a výstupního
         proudu $I_2$ může měnič pracovat ve čtyřech kvadrantech tzv.\textbf{ VA-roviny} (viz obr.
         \ref{enz:fig_prac_kvadranty}).

         V kvadrantech \emph{1} i \emph{3} dodává měnič energii do zátěže. Je-li zátěží motor, tak
         pohání. Pasivní zátěže mohou pracovat pouze v těchto kvadrantech. V kvadrantech \emph{2} a
         \emph{4} dodává aktivní zátěž energii zpět do měniče. Jde-li o motor\footnote{Velikost
         napětí ss. motoru je úměrná otáčkám (rychlosti), polarita je dána směrem otáčení
         (uvažujeme motor s cizím buzením, např. s permanentními magnety). Velikost proudu je
         úměrná momentu na hřídeli, polarita je opět dána směrem momentu, tj. zda motor brzdí či
         pohání. Je třeba si povšimnout, že přechod mezi generátorickým a motorickým režimem mezi
         kvadranty 2 a 1 nebo mezi 3 a 4 (tj. takový, kdy se nemění polarita napětí, ale jen
         proudu) vůbec nemusí být na hřídeli motoru opticky pozorovatelný, neboť v dané chvíli
         přechodu se změní jen znaménko momentu (proudu) a přesto otáčky hřídele mohou být
         konstantní.}, pak brzdí.

      \subsubsection{Možnosti zapojení silového obvodu}\label{enz_kap_moznosti_zapojeni}
         \begin{figure}
           \centering
           \includegraphics[width=0.5\linewidth]{patocka_pracovni_kvadranty.pdf}
           \caption{Pracovní kvadranty ve VA rovině.}
           \label{enz:fig_prac_kvadranty}
         \end{figure}
         Na první pohled jsou zřejmá určitá omezení:
         \begin{itemize}
       	   \item Indukčnost nikdy nesmí být zapojena paralelně ke vstupu či výstupu (protože tam je
                 napětí s nenulovou střední hodnotou).
           \item Kapacita nikdy nesmí být zapojena do série se vstupní nebo výstupní svorkou měniče
                 (protože tudy prochází proud s nenulovou střední hodnotou).
           \item Jako akumulační prvek nelze použít samostatně kapacitu, není-li v obvodu použita
                 ještě indukčnost (protože by v měniči napěťového typu docházelo k nepřípustnému
                 nárazovému nabíjení kondenzátoru zkratovým proudem). Čili měnič napěťového typu
                 musí obsahovat alespoň jednu indukčnost.
           \item Žádný spínač nesmí zkratovat vstup ani výstup měniče.
         \end{itemize}
      \subsubsection{Nejjednodušší měniče s jediným akumulačním 
                     prvkem}\label{ENZ:tit_menice_s_1_aku_prvkem} 
         Pro výchozí představu, vysvětlující princip činnosti, vytvoříme silový obvod měniče ze
         dvou prvků. Bude to indukčnost  \emph{L} a ideální přepínač. Vezmeme-li v úvahu omezení z
         kap. \ref{enz_kap_moznosti_zapojeni}, existují podle obr. \ref{enz:fig_DCDC_princip} jen
         tři způsoby, jak takový měnič zapojit \cite{Patocka}.

        \begin{figure}[ht!]
          \centering
          \begin{tabular}{c}
            \subfloat[$U_x=U_1\frac{t_1}{t_1+t_2}<U_1$]{\label{enz:fig_stepdown}
              \includegraphics[width=0.8\linewidth]{patocka_step_down_princip.pdf}}     \\
            \subfloat[$U_x=U_2\frac{t_1}{t_1+t_2}>U_1$]{\label{enz:fig_stepup}
              \includegraphics[width=0.8\linewidth]{patocka_step_up_princip.pdf}}       \\
            \subfloat[$U_x=\frac{U_1\cdot t_1 + U_2\cdot
                      t_2}{t_1+t_2}<>-U_1$]{\label{enz:fig_buckboost}
              \includegraphics[width=0.7\linewidth]{patocka_buck_boost_princip.pdf}}
          \end{tabular}  
          \caption{Principiální schémata DC/DC měničů s jediným akumulačním prvkem.}
          \label{enz:fig_DCDC_princip}
        \end{figure}

        Označme střední hodnotu napětí mezi společným uzlem přepínače \texttt{3} a zemí jako 
        $U_x$. Předpokládejme, že přepínač je ovládán periodickým signálem s periodou $T$ a s
        nastavitelnou střídou, takže po dobu  $t_1$ spojuje svorky \texttt{3 - 1} a po dobu  $t_2 =
        T – t_1$ pak svorky \texttt{3 - 2}. Popišme nyní nejzákladnější vlastnosti tří měničů z
        obr. \ref{enz:fig_DCDC_princip}.

        \begin{enumerate}
          \item Střední hodnota $U_x$ na obr.\ref{enz:fig_stepdown} musí vzhledem k činnosti
                přepínače být:
                \begin{equation}\label{enz:stepdown_Ux}
                  U_x = U_1\frac{t_1}{t_1 + t_2} < U_1
                \end{equation}
                Výstupní napětí je rovno $U_x$, neboť střední hodnota napětí na indukčnosti L musí
                být nulová. Platí proto:
                \begin{equation}\label{enz:stepdown_U2}
                  U_2 = U_x = U_1\frac{t_1}{t_1 + t_2} < U_1
                \end{equation}
                Výstupní napětí je vždy menší než vstupní a má stejnou polaritu. Jde tedy o měnič
                \emph{snižující} a \emph{neinvertující}. Jeho jiné názvy jsou: \textbf{step-down,
                chopper, buck, propustný měnič}. Možné pracovní kvadranty jsou 1 a 2. Čili měnič je
                schopen dávat napětí $U_2$ jediné polarity, ale proud $I_2$ muže téci oběma směry
                (je-li to umožněno - aktivní zátěž).
          \item Střední hodnota $U_x$ na obr.\ref{enz:fig_stepup} musí vzhledem k činnosti
                přepínače být:               
                \begin{equation}\label{enz:stepup_Ux} 
                  U_x = U_2\frac{t_1}{t_1 + t_2} > U_1
                \end{equation}
                Vstupní napětí $U_1$ je rovno $U_x$ (nulová střední hodnota napětí na indukčnosti
                L). Odsud pro $U_2$ platí:
                \begin{equation}\label{enz:stepup_U2}
                  U_2 = U_1\frac{t_1 + t_2}{t_1} > U_1
                \end{equation}
                Střední hodnota výstupního napětí je vyšší než vstupní napětí a má stejnou polaritu.
                Jde tedy o \emph{zvyšující} a \emph{neinvertující} měnič. Jiný název je měnič
                \textbf{step-up, boost}. Možné pracovní kvadranty\footnote{Měnič
                \ref{enz:fig_stepdown} pracující v kvadrantu 1 je měničem \ref{enz:fig_stepup}
                pracujícím v kvadrantu 2. Naopak \ref{enz:fig_stepdown} v kvadrantu 2 je
                \ref{enz:fig_stepup} v kvadrantu 1. Čili \ref{enz:fig_stepdown} a
                \ref{enz:fig_stepup} je vlastně týž obvod, pouze zaměňuje vstup a výstup.} jsou opět
                1 a 2.
          \item Střední hodnota $U_x$ na obr.\ref{enz:fig_buckboost} musí vzhledem k činnosti
                přepínače být:
                \begin{equation}\label{enz:buckboost_Ux}
                  U_x = \frac{U_1t_1 + U_2t_2}{t_1 + t_2} <> - U_1
                \end{equation}
                Protože $U_x$ je střední hodnota napětí na indukčnosti L, musí platit $U_x =0$ tj.
                \begin{equation}\label{enz:buckboost_U2}
                  U_1 = - \frac{t_1}{t_2}U_1 <> - U_1
                \end{equation}
                Výstupní napětí má opačnou polaritu než vstupní, jde tedy o měnič
                \emph{invertující}. Velikost výstupního napětí může být větší i menší než vstupní.
                Vžité názvy jsou měnič \textbf{buck-boost, měnič se společnou tlumivkou, blokující
                měnič}. Možné pracovní kvadranty jsou 3 a 4.
        \end{enumerate}

      \subsubsection{Prakticky realizované silové obvody}\label{ENZ:subkap_silove_obvody}
        Kap. \ref{ENZ:tit_menice_s_1_aku_prvkem} ukazuje, že elektronicky ovládaný přepínač tvoří
        základní stavební kámen každého měniče. Tyto přepínače se ve skutečných obvodech realizují
        pomocí tzv. horních a dolních spínačů, což jsou \emph{trojpóly} podle obr.
        \ref{enz:fig_silove_obvody}.
        \begin{figure}
          \centering
          \begin{tabular}{c}
            \subfloat[Horní spínač]{\label{enz:fig_HS}
              \includegraphics[width=0.3\linewidth]{patocka_horni_spinac.pdf}}     \\
            \subfloat[dolní spínač]{\label{enz:fig_LS}
              \includegraphics[width=0.3\linewidth]{patocka_dolni_spinac.pdf}}     \\ 
            \subfloat[Větev - paralelní kombinace horního a dolního spínače]{\label{enz:fig_arm}
              \includegraphics[width=0.4\linewidth]{patocka_vetev.pdf}}
          \end{tabular}  
          \caption{Horní a dolní spínač.}
          \label{enz:fig_silove_obvody}
        \end{figure}

        \begin{figure*}
          \centering
          \includegraphics[width=0.9\textwidth]{patocka_silove_obvody_kvadranty.pdf}
          \caption[Skutečné silové obvody měničů a jejich kvadranty]{Skutečné silové obvody měničů
                   z obr. \ref{enz:fig_DCDC_princip} a jejich pracovní kvadranty: a) měnič snižující
                   neinvertující (step-down), b) měnič zvyšující neinvertující (step-up), c) měnič
                   invertující (buck-boost)}
          \label{enz:fig_silove_obv_kvadranty}
        \end{figure*}
        
\subsection{Step-down converter \newline(snižující neinvertující měnič)}\label{ENZ:kap_step_down}
  Jedná se o měnič s horním spínačem. Další jeho používané názvy jsou: propustný měnič, chopper,
  buck. \emph{Pracuje v 1. kvadrantu}.
  \begin{figure}[ht!]
    \centering
    \includegraphics[width=1\linewidth]{patocka_step_down_sch1.pdf}
    \caption[Snižující měnič]{Snižující měnič pracující v prvním kvadrantu s aktivní zátěží
             typu stejnosměrný motor nebo s LC filtrem}
    \label{enz:fig_StepDown_sch1}
  \end{figure}

  \begin{figure}[ht!]
    \centering
    \includegraphics[width=1\linewidth]{patocka_step_down_waveform.pdf}
    \caption[Snižující měnič - průběhy]{Průběhy napětí a proudů snižujícího měniče}
    \label{enz:fig_StepDown_wave1}
  \end{figure}

\subsection{Step-up converter (zvyšující neinvertující měnič)}\label{ENZ:kap_step_up}
  \begin{figure}[ht!]
    \centering
    \includegraphics[width=0.8\linewidth]{patocka_step_up_sch1.pdf}
    \caption[Zvyšující měnič]{Zvyšujícího měnič pracující v prvním kvadrantu - Schéma zapojení}
    \label{enz:fig_StepUp_sch1}
  \end{figure}

\subsection{Buck-boost converter \newline(Invertující měnič se společnou tlumivkou)}\label{ENZ:kap_buck_boost}
\subsection{Cuk converter \newline(Měnič se společným konden\-zá\-to\-rem)}\label{ENZ:kap_cuk}
\subsection{SEPIC converter \newline(Single-ended primary inductor converter)}\label{ENZ:kap_sepic}         
    %------------------------------------------------------
    \newpage
    %------------------------------------------------------
    % Section: DC/DC měniče s transformátorem
      \input{../src/ENZ/section/DC_DC_with_transformer.tex}
    %------------------------------------------------------
    %------------------------------------------------------
    % Section: Metody regulace spínanich zdrojů
      % file: Control_Techniques.tex
%============================= Podkapitola: Metody regulace spínaných zdrojů =======================
\section{Metody regulace spínaných zdrojů}
      \subsection{Základy impulzní regulace}
        Základním principem a současně odlišností impulsní regulace od regulace klasické je její
        nespojitost. To v zásadě znamená, že nehledě na detailní realizaci, je výstupní napětí
        $U_out$ stabilizováno zásahy regulačního členu pouze v určitých, časově omezených
        intervalech $T_a$. \cite{Hammembauer}

        Srovnejme pro názornost klasický a impulsní regulátor na úrovni blokových schémat. (obr.
        4.1 a obr. 5.9 ). Vidíme, že obě jsou formálně dosti podobná. U obou nacházíme napěťový
        normál $U_{REF}$, zesilovač regulační odchylky $A_u$, budící obvod i výkonový regulační
        člen a samozřejmě i zpětnovazební smyčku. Tím však, snad až na základní podstatu regulační
        smyčky podobnost končí. Funkčně jsou oba regulátory naprosto odlišné.

        U spojitého lineárního regulátoru ovládá odchylka výstupního napětí od jmenovité velikosti
        spojitě okamžitý odpor výkonového regulačního členu v libovolném okamžiku tak, aby výstupní
        napětí bylo konstantní.

      \subsection{Regulační smyčka}

      \subsubsection{Porovnání regulátoru s napěťovým a proudovým řízením}
        \begin{figure*}
          \centering
          \includegraphics[width=0.6\textwidth]{unitrode_voltage_mode_control.pdf}
          \caption[Regulátor s napěťovým řízením]{Regulátor s napěťovým řízením - Voltage mode
                   control [\cite{SLUA119}]}
          \label{ENZ:fig_V_mode_cntrl}
        \end{figure*}

         The current mode control method uses two control loops --an inner, current control loop
         and an outer loop for voltage control. Figure  1 shows a forward converter (buck family)
         using current mode control. When the switching transistor is on, current through Rsense is
         proportional to the upward ramping filter inductor current. When the ramp voltage Vs
         reaches Ve (the amplified  output  voltage  error), the switching transistor turns  off.
         Thus, the outer voltage control loop defines the level at which the inner loop regulates
         peak current through the switch and through the filter inductor. \cite{SLUP075}

        \begin{figure*}
          \centering
          \includegraphics[width=0.6\textwidth]{unitrode_current_mode_control.pdf}
          \caption[Regulátor s proudovým řízením]{Regulátor s proudovým řízením - Current mode
                   control [\cite{SLUA119}]}
          \label{ENZ:fig_I_mode_cntrl}
        \end{figure*}

        Výhody:
        \begin{itemize}
          \item Input voltage feed-forward, resulting in good open-loop line regulation.
          \item Simplified loop --inductor pole and 2nd order characteristic eliminated.
          \item Optimum large-signal behavior.
          \item No conditional loop stability  problems.
          \item Flux balancing (symmetry correction) in push-pull circuits.
          \item Automatic pulse-by-pulse current limiting.
          \item Current sharing of paralleled supplies for modular power systems.
          \item Less complexity/cost (current sense/amp is not an added complication).
        \end{itemize}

        Nevýhody (continuous  mode  only):
        \begin{itemize}
          \item Peak/avg. current error and instability --slope compensation
          \item Noise immunity is  worse because of  shallower ramp.
          \item Half Bridge runaway
          \item DC open loop load regulation is worse.
          \item (1-D) current error in Boost or Flyback circuits.
          \item Loop irregularities with multiple output buck circuits.
        \end{itemize}

    %------------------------------------------------------

  \section{Sbírka katalogových zapojení neizolo\-va\-ných měničů}
    Existují dvě možnosti, jak provádět řízení pomocí PWM odlišující se \emph{typem zpětné
    vazby}, která je buď čistě \textbf{napěťovou vazbou} (\emph{voltage mode control}), nebo
    \emph{napěťovou vazbou s vnitřní proudovou smyčkou} (\emph{current mode control}). V
    následující diskusi se pokusíme konzistentním způsobem vysvětlit vlastnosti obou řídících
    algoritmů (slua119)
    
    \subsection{Zdroj symetrického napětí s jedním induktorem}
      \begin{figure}
        \centering
        \includegraphics[width=\linewidth]{LT1376_dual_output_reg.pdf}
        \caption[Spínaný zdroj napětí $\pm5 V$ vystačí s jedinou indukčností s dvojím
                 vinutím]{Spínaný zdroj napětí $\pm5 V$ vystačí s jedinou indukčností s dvojím
                 vinutím. \cite{DN100}. Linear Technology Corp. (Dual Output Regulator Uses Only
                 One Inductor)}
        \label{enz:fig_LT1376_cir1}
      \end{figure}
      Toto řešení na obr. \ref{enz:fig_LT1376_cir1} nabízí spínaný zdroj symetrického napětí za
      použití několika dalších součástek a induktoru s dvojím vinutím. Základní části zdroje je
      napěťový regulátor snižující vstupní kladné napětí založený na obvodu \emph{LT1376-5} se
      spínacím kmitočtem 500 kHz a možností zatížení proudem až 1,5 A.

      \begin{figure}
        \centering
        \includegraphics[width=\linewidth]{LT1376_dual_output_reg_performance.pdf}
        \caption{Zatěžovací charakteristika záporné větve.}
        \label{enz:fig_LT1376_cir1_perform}
      \end{figure}

      Druhá polovina induktoru $L_1$ společně s $D_3$, $C_5$ a $C_4$ je určena pro tvorbu
      záporného napětí pomocí \textbf{SEPIC topologie} - \emph{Single Ended Primary Inductance
      Converter}. Kondenzátor $C_4$ vnucuje oběma vinutím stejné napětí. Bez něho pracuje tato
      část jako blokující měnič (\textbf{flyback}), která by sice poskytla -5V, ale jen naprázdno
      se značnou závislostí na zátěži (nedokonalá vazba mezi vinutími).
