% !TeX spellcheck = cs_CZ
%---------- Velikost náboje v minvi:
\begin{example}
  Elektricky neutrální měděná mince o hmotnosti \(m = \SI{3.11}{\g}\) obsahuje stejné množství 
  kladného a záporného náboje. Jaké je velikost kladného (nebo záporného) náboje obsaženého v 
  minci?\newline  
  \textbf{Řešení:}\newline
  Neutrální atom má záporný náboj \(Z\cdot e\), představovaný jeho elektrony a kladný náboj o 
  stejné velikosti představovaný protony v jádře. Pro měď je atomové číslo \(Z\) rovno \num{29}, 
  tj. atom mědi má \num{29} protonů, a je-li elektricky neutrální, také \num{29} elektronů.
  
  Náboj o velikosti \(Q_v\), který hledáme je roven \(N\cdot Z\cdot e\), kde \(N\) je počet atomů 
  obsažených v  jednom molu (Avogadrova konstanta: \(N_A = \SI{6.0221e23}{\per\mole}\)). Počet 
  molů mědi v minci \(\frac{m}{M}\), kde \(M = \SI{63.5}{\g\per\mole}\) je molární hmotnosti mědi: 
  \begin{equation*}
    N = N_A\cdot\frac{m}{M} = \SI{6.0221e23}{\per\mole}
           \frac{\SI{3.11}{\g}}{\SI{63.5}{\g\per\mole}} 
      = \num{2.95e22}.
  \end{equation*}
 Velikost celkového kladného (záporného) náboje v minci je pak 
  \begin{equation*}
    Q_v = N\cdot Z\cdot e = \num{2.95e22}\cdot\num{29}\cdot\SI{1.602e-19}{\coulomb} 
        = \SI{137039}{\coulomb}
  \end{equation*}
  To je obrovský náboj. Pro srovnání: třeme-li ebonitovou tyč vlněnou látkou, můžeme na tyč 
  přemístit stěží náboj o velikosti \SI{1e-9}{\coulomb}.
\end{example} 
