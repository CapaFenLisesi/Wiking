% !TeX spellcheck = cs_CZ
{\tikzset{external/prefix={tikz/TEO/}}
 \tikzset{external/figure name/.add={ch11_}{}}
%==========================Kapitola: Dynamické pochody v lineárních obvodech========================
\chapter[Harmonické obvody]{Lineární obvody v harmonickém ustáleném stavu}
\minitoc

  V této kapitole se seznámíme se \emph{symbolicko-komplexní metodou} (\texttt{SKM}), jež má
  základní důležitost pro teorii obvodů v harmonickém ustáleném stavu. Potom prozkoumáme vlastnosti
  jednodušších obvodů v tomto stavu a metody jejich analýzy. Posléze pojednáme o elektrickém výkon
  v obvodech a o nejdůležitějších otázkách přenosu energie \cite[s.~60]{Mayer1975}.
  
  \section{Periodické veličiny a jejich charakteristické hodnoty}
    \emph{Periodickou veličinou} nazýváme takovou veličinu $v$, jejíž závislost na čase lze
    vyjádřit periodickou funkcí, pro níž existuje konstanta $T>0$ taková, že pro každé $t$ platí
    vztah
    \begin{equation}\label{TEO:eq_harm01}
      v(t+T) = v(t),    
    \end{equation}  
    Konstanta $T$ se nazývá \textbf{perioda} resp. \emph{doba kmitu}. V aplikacích se zpravidla
    používá nejmenší kladná perioda, tzv. \emph{základní perioda}; pro stručnost budeme hovořit
    pouze o periodě. Je-li dána periodická veličina na jakémkoliv intervalu $(t_0, t_0+T)$, je tím
    zřejmě definována pro všechna $t>t_0$. Průběh veličiny $v$ na jakémkoliv intervalu délky $T$ se
    nazývá \emph{cyklem}. Počet cyklů za jednotku času (za sekundu) udává \textbf{kmitočet}, nebo
    též \emph{frekvenci} periodické veličiny
    \begin{equation}\label{TEO:eq_harm02}
      f = \frac{1}{T},    
    \end{equation}       
    V elektrotechnice rozdělujeme periodické veličiny do dvou skupin:
      %----------------------------------
      % image: TEO_fce01.tex label: \label{TEO:fig_01}
        % \documentclass{book}
% \usepackage{tikz}
% \usepackage{pstricks}
%   \usetikzlibrary{intersections}
%   \usetikzlibrary{calc}

% \begin{document}
\begin{figure}[htb]
  \centering
  \begin{tikzpicture}[scale=0.8]
    \def\xmin{-30}
    \def\xmax{300}
    \def\ymin{-30}
    \def\ymax{150}
    \def\nula{62.68476}
    \def\phase{20}
    \def\delta{80}
	
        
        \begin{scope}[draw=black,line join=round,miter limit=4.00,line width=0.5pt, y=1pt,x=1pt]
          \draw[->] (\xmin,\nula) -- (\xmax,\nula) node[right] {$t$} coordinate(x axis);
          \draw[->] (\phase,\ymin) -- (\phase,\ymax) node[left]  {$v(t)$} coordinate(y axis);
          \path[name path=fce,draw=black,line join=round,even odd rule,line cap=butt,miter
            limit=4.00] (0,62.68476) .. controls
            (6.288991,93.43519)   and (6.900971,94.37507)   .. (27.182131,102.49149) .. controls
            (41.727261,108.31238) and (35.032981,125.42937) .. (51.364001,125.36952) .. controls
            (67.695031,125.30962) and (67.928601,83.44326)  .. (69.419751,64.22647)  .. controls
            (70.630951,48.61748)  and (74.370481,32.72818)  .. (94.65164,24.61178)   .. controls
            (109.19677,18.79088)  and (101.39868,1.81368)   .. (117.97023,1.87438)   .. controls
            (134.06073,1.93338)   and (134.11602,42.07748)  .. (136.03504,63.349)    .. controls
            (136.94551,73.44118)  and (139.86205,86.13389)  .. (143.15997,91.12848)  .. controls
            (146.45789,96.12308)  and (151.68117,98.38715)  .. (161.82175,102.44536) .. controls
            (176.36688,108.26625) and (169.6726,125.38324)  .. (186.00362,125.32339) .. controls
            (202.33465,125.26349) and (199.36351,96.7115)   .. (204.05937,64.18034)  .. controls
            (208.75523,31.64918)  and (209.0101,32.68208)   .. (229.29126,24.56568)  .. controls
            (243.83639,18.74478)  and (236.0383,1.76758)    .. (252.60985,1.82828)   .. controls
            (268.70035,1.88728)   and (268.96066,39.13088)  .. (270.65931,63.2234);
            
          \draw[dashed, <->](51.364001,125.36952) coordinate(p4) -- ++(0,-\nula) coordinate(p6); 
          \draw[dashed, <->](117.97023,1.87438) coordinate(p3)   -- ++(0,60.81038) coordinate(p7);
          \path[name path=osax] (\xmin,\nula) -- (\xmax,\nula);
          \path[name path=osay] (\phase,\ymin) -- (\phase,\ymax);         
          % Intersections
          \path[name intersections={of=osay and fce, name=point}];  % prusecik osy y s funkcí - posun
          \path[name path=line1] (point-1) --+(200,0);
          \path[name intersections={of=line1 and fce, name=cross}]; % urceni pruseciku od ktero beru periodu
          \draw[name path=line2,dotted] (cross-2) -- ++(0,-120);              
          \path[name intersections={of=osax and osay, name=pocatek}]; 
          \path[name path=line3] (pocatek-1) ++ (0,-\delta) coordinate(p1) -- +(300,0);
          \path[name intersections={of=line2 and line3, name=p8}];
          \draw[<->] (p8-1) -- (p1); % kota perioda T
          % text
          \node[above] at ($ (p1)!0.5!(p8-1) $) {$T$};
          \node[left]  at ($ (p7)!0.5!(p3) $) {$V_{min}$}; 
          \node[left]  at ($ (p6)!0.5!(p4) $) {$V_{max}$}; 
          \node[below left] at (pocatek-1) {$0$}; 
          \draw[fill=white] (pocatek-1) circle(2pt);          
        \end{scope}     
  \end{tikzpicture}
  \caption{Příklad \textbf{periodické veličiny} $v=v(t)$ pro kterou platí $v(t+T)=v(t)$}\label{TEO:fig_01}
\end{figure}
  
% \end{document}  
      %---------------------------------- 
      %----------------------------------
      % image: TEO_fce02.tex label: \label{TEO:fig_02}
        %\documentclass{article}
%\usepackage{tikz}
%\usetikzlibrary{decorations.markings}
%\usetikzlibrary{intersections}
%\usetikzlibrary{calc}



% \begin{document}
  \begin{figure}[htb]  
        \newcommand{\MyPath}{%
            (18.309015,50.53919507)  .. controls
            (18.309015,50.53919507)  and (21.791754,59.15092207)  ..
            (28.507151,63.26291307)  .. controls
            (31.391329,65.02896007)  and (35.896102,67.50958007)  ..
            (39.847327,68.71973407)  .. controls
            (43.336244,69.78829507)  and (45.883,69.20748807)     ..
            (48.387485,71.78723507)  .. controls
            (51.186791,74.67066107)  and (53.141632,78.26714507)  ..
            (55.148632,82.06152507)  .. controls
            (58.500874,88.39917807)  and (60.888255,94.05049107)  ..
            (66.110428,100.96156507) .. controls
            (71.332601,107.87263907) and (75.976982,111.80666507) ..
            (83.867952,112.10204307) .. controls
            (91.758922,112.39742207) and (100.54114,97.81547307)  ..
            (107.15399,84.24536207)  .. controls
            (109.7867,78.84282207)   and (116.0911,65.30681007)   ..
            (123.19123,50.92337507)  .. controls
            (136.46247,24.03854207)  and (147.18658,6.42393707)   ..
            (159.27196,2.81113707)   .. controls
            (184.95662,-4.86702293)  and (208.53361,32.31924407)  ..
            (217.37186,52.05019707);
        }%
        \newcommand{\MyWaveB}{%
            (200.02545,22.85649707)  .. controls
            (200.02545,22.85649707)  and (209.79787,34.53519707)  ..
            (217.54727,52.29679707)  .. controls
            (221.57621,61.53106707)  and (231.76264,67.80275707)  ..
            (242.62012,69.40393707)  .. controls
            (253.4776,71.00511707)   and (260.28174,112.23421707) ..
            (282.13712,111.96387707) .. controls
            (303.99249,111.69353707) and (329.46714,1.83349707)   ..
            (362.13712,1.60669707)   .. controls
            (394.8071,1.37989707)    and (415.22326,51.87129707)  ..
            (415.22326,51.87129707);
        }%
        \newcommand{\MyWaveA}{%
            (2.2036044,23.21319707)  .. controls
            (2.2036044,23.21319707)  and (7.1896734,29.00179707)  ..
            (11.896862,37.80959707)  .. controls
            (14.463984,42.61309707)  and (16.430404,45.91309707)  ..
            (19.958399,53.79003707)  .. controls
            (23.178819,60.98023707)  and (34.001913,68.16115707)  ..
            (44.701755,69.73908707)  .. controls
            (55.401595,71.31701707)  and (62.430335,112.41311707) ..
            (84.285711,112.14277707) .. controls
            (106.14109,111.87243707) and (131.61574,2.01239707)   ..
            (164.28572,1.78559707)   .. controls
            (196.9557,1.55879707)    and (217.37186,52.05019707)  ..
            (217.37186,52.05019707);
        }%
  
    \centering
    \begin{tikzpicture}[ 
        scale=0.7,
        tangent/.style={
          decoration={
            markings,% switch on markings
            mark= at position #1 with
              {
                \coordinate (tangent point-\pgfkeysvalueof{/pgf/decoration/mark 
                             info/sequence number}) at (0pt,0pt);
                \coordinate (tangent unit vector-\pgfkeysvalueof{/pgf/decoration/mark 
                             info/sequence number}) at (1,0pt);
                \coordinate (tangent orthogonal unit vector-\pgfkeysvalueof{/pgf/decoration/mark  
                             info/sequence number}) at (0pt,1);
              }
          },
          postaction=decorate
        },
        use tangent/.style={
            shift=(tangent point-#1),
            x=(tangent unit vector-#1),
            y=(tangent orthogonal unit vector-#1)
        },
        use tangent/.default=1
    ]

        \def\xmin{-30}
        \def\xmax{450}
        \def\ymin{-80}
        \def\ymax{150}
        \def\nula{52.05019707}
        \def\phase{2.2}
        \def\myscale{0.7}

        \begin{scope}[draw=black,line join=round, miter limit=4.00,
                      line width=0.5pt, y=1pt,x=1pt, scale=\myscale]   
          \fill [fill=yellow!80] (18.309015,\nula) -- \MyPath -- (216.7025,\nula);
          \path[name path=fce1,draw=black,line join=round,even odd rule,line cap=butt,miter   
            limit=4.00, tangent=0.36695, tangent=0.7770] \MyWaveA;  
          \path[name path=fce2,draw=black,line join=round,even odd rule,line cap=butt,miter   
            limit=4.00] \MyWaveB;           
          \draw[name path=axeX,->] (\xmin,\nula) -- (\xmax,\nula)   
               node[right] {$t$} coordinate(x axis);
          \draw[name path=axeY,->] (\phase,\ymin) -- (\phase,\ymax) 
               node[left]  {$v(t)$} coordinate(y axis);     
          \draw[densely dashed, thin, use tangent=1] 
               (0,0) coordinate(vrch) -- (0,-60*\myscale) % firt part
               (0,0) -- (-70*\myscale,0)    node[left] {$V_m$};     % delka krivky: 0.3667 - 0.3668
          \draw[densely dashed, thin, use tangent=2] 
               (0,0) coordinate(dul) --  (0,50*\myscale)  % first part 
               (0,0) -- (-120*\myscale,0) node[left] {$V_{min}$}; %               0.7770 - 0.7775
          \path[name intersections={of=fce1 and axeX, name=point}]; % prusecik osy y s funkcí - posun
          \draw[dotted, name path=line1] (point-1) --+(0,-100) coordinate(Ta);  
          \draw[dotted, name path=line2] (point-2) --+(0,-100) coordinate(Tb);
          \draw[dotted, name path=line3] (point-3) --+(0,-100) coordinate(Tc);
          \draw[fill, color=black] ($ (Ta)!0.5!(Tb) $) node[above, black] {$\frac{T}{2}$};  
          \draw[fill, color=black] ($ (Tb)!0.5!(Tc) $) node[above, black] {$\frac{T}{2}$};            
          \draw[<->] (Ta) -- (Tb);  
          \draw[<->] (Tb) -- (Tc);  
          \path[name intersections={of=axeX and axeY, name=pocatek}]; 
          \node[below left] at (pocatek-1) {$0$}; 
          \draw[fill=white] (pocatek-1) circle(2pt);          
          \draw[fill=white] (vrch) circle(1pt); 
          \draw[fill=white] (dul)  circle(1pt);                  
        \end{scope} 
    \end{tikzpicture}
  \caption{Časový průběh \textbf{střídavé veličiny} $v=v(t)$, pro kterou platí, že obsahy ploch
           v jednom cyklu nad osou $t$ a pod osou $t$ jsou totožné}\label{TEO:fig_02}
\end{figure}
  
% \end{document}  
      %----------------------------------    
    \begin{itemize}
      \item Veličiny $v$, jež během svého cyklu \emph{změní znaménko} (obr.\ref{TEO:fig_01})
            nazýváme \textbf{kmitavé}. \emph{Speciálním případem} kmitavých veličin jsou
            \textbf{střídavé veličiny}, jež mají tu vlastnost, že po dobu $T/2$ jsou trvale kladné,
            po dobu $T/2$ naopak záporné a obsahy ploch omezených grafem funkce $v=v(t)$ v jednom
            cyklu nad osou $t$ a pod osou $t$ jsou \emph{totožné} (obr. \ref{TEO:fig_02}).
      \item Veličiny $v$, jež \emph{nemění své znaménko}, tj. jsou trvale kladné nebo trvale
            záporné (obr.\ref{TEO:fig_03}) nazýváme \textbf{pulzující}. \emph{Speciálním případem}
            jsou \textbf{stejnosměrné veličiny}, které nemění svou hodnotu, tj. $v=konst$
            (obr.\ref{TEO:fig_03} (b)).
    \end{itemize} 

      %----------------------------------
      % image: TEO_fce03.tex label: \label{TEO:fig_03}
        % \documentclass{article}
% \usepackage{tikz}
% \usetikzlibrary{decorations.markings}
% \usetikzlibrary{intersections}
% \usepackage{subfigure} 
% \usetikzlibrary{calc}

% \begin{document}
  \begin{figure}[htb]
     \newcommand{\MyPath}[1]{%
        (0 + #1,18) .. controls
        (0 + #1,18) and (7 + #1,50) ..
        (17 + #1,50) .. controls
        (27 + #1,50) and (25 + #1,33) ..
        (33 + #1,37) .. controls
        (41 + #1,41) and (48 + #1,18) ..
        (48 + #1,18);
    }
    
    \newcommand{\MyXY}{%
        \draw[name path=axeX,->] (\xmin,\nula) -- (\xmax,\nula)   node[right] {$t$} coordinate(x axis);
        \draw[name path=axeY,->] (\phase,\ymin) -- (\phase,\ymax) node[left]  {$v(t)$} coordinate(y axis);
        \path[name intersections={of=axeX and axeY, name=pocatek}];
        \node[below left] at (pocatek-1) {$0$};
        \draw[fill=white] (pocatek-1) circle(2pt);
    }
     
    \centering
        \def\xmin{-20}
        \def\xmax{160}
        \def\ymin{-15}
        \def\ymax{80}
        \def\nula{0}
        \def\phase{10}
        \def\period{48}
        \def\myscale{0.8}
           
    \subfloat[ ]   { 
      \begin{tikzpicture}      
        \begin{scope}[draw=black,line join=round, miter limit=4.00,line width=0.5pt, y=1pt,x=1pt, scale=\myscale]  
          \MyXY;         
          \draw[thick, name path=fce1, draw=black,line join=round,even odd rule,line cap=butt,miter   
            limit=4.00] \MyPath{0}; 
          \draw[] (\period + \phase,0) ++ (0,2) -- +(0,-4) node[below] {$T$};       
          \draw[thick, draw=black,line join=round,even odd rule,line cap=butt,miter   
            limit=4.00] \MyPath{\period}; 
          \draw[thick, draw=black,line join=round,even odd rule,line cap=butt,miter   
            limit=4.00] \MyPath{2*\period};             
        \end{scope}  
      \end{tikzpicture}
    }
    \subfloat[  ]   { 
      \begin{tikzpicture}  
        \begin{scope}[draw=black,line join=round, miter limit=4.00,line width=0.5pt, y=1pt,x=1pt, scale=\myscale]  
          \MyXY;     
          \draw[thick] (\xmin+10, \nula + 40) -- (\xmax-30, \nula + 40);  
          \draw[] (\period + \phase,0) ++ (0,2) -- +(0,-4) node[below] {$T$};   
        \end{scope}    
      \end{tikzpicture} 
    }   
    \caption{Časový průběh pulsující periodické veličiny a konstantní veličiny}\label{TEO:fig_03}
  \end{figure}
%\end{document}  
      %----------------------------------        
    Praktický význam mají zejména tyto hodnoty periodických veličin:
    \begin{itemize}
      \item \emph{Maximální hodnota} $V_m$ periodické veličiny $v$, tj. největší hodnota, které
            tato veličina dosahuje $v_m=\max v(t)$
      \item \emph{Minimální hodnota} $V_{min}$ periodické veličiny $v$, tj. nejmenší hodnota, které
            tato veličina dosahuje $v_m=\min v(t)$
    \end{itemize}
      
    Maximální a minimální hodnoty střídavé veličiny se nazývají též \emph{vrcholovými hodnotami}
    (kladnými nebo zápornými), obr. \ref{TEO:fig_01} a \ref{TEO:fig_02}. 
    
    \fbox{Střední hodnota} veličiny $v$ v intervalu $\langle t_i, t_j\rangle$ je 
    \begin{equation}\label{TEO:eq_harm03}
      V_s = \frac{1}{t_j-t_s}\int_{t_j}^{t_s}v(t)dt
    \end{equation}
    U periodické veličiny se spravidla počítá střední hodnota v jedno cyklu. U střídavé veličiny je
    v jednom cyklu $V_s = 0$,  a proto střední hodnotu vyjadřujeme v takovém intervalu v němž je
    $v\geq0$.
    
    \fbox{Efektivní hodnota} periodické veličiny v intervalu $\langle 0, T\rangle$ je 
    \begin{equation}\label{TEO:eq_harm04}
      V = \sqrt{\frac{1}{T}\int_{0}^{T}v^2(t)dt}
    \end{equation}   
    U periodických napětí a proudů má praktický význam především jejich efektivní hodnota.
    Efektivní hodnotu periodického proudu $i=i(t)$ procházejícího konstatním odporem $R$ lze
    interpretovat jako stejnosměrný proud $I$, při němž se za dobu $T$ vyvine v odporu $R$ stejná
    tepelná energie, jako průchodem proudu $i$. Podle \emph{Joulova-Lenzova} zákona je totiž
    \begin{equation}\label{TEO:eq_harm05}
      RI^2T = \sqrt{\frac{1}{T}\int_{0}^{T}Ri^2(t)dt}
    \end{equation}       
    z čehož lze určit $I$ v souladu s rovnicí \ref{TEO:eq_harm04}. Obdobně lze fyzikálně
    interpretovat efektivní hodnotu napětí.
    
    Střední hodnotu periodického proudu $i=i(t)$ lze fyzikálně interpretovat jako stejnosměrný
    proud $I_s$, jimž se za dobu $T$ přenese stejný náboj $Q$ jako proudem $i$:
    \begin{equation}\label{TEO:eq_harm06}
      Q = I_sT = \int_{0}^{T}i(t)dt
    \end{equation}       
    z čehož plyne $I_s$ v souladu s rovnicí \ref{TEO:eq_harm03}.  
    
    Efektivní hodnotu napětí (proudu) lze změřit např. feromagnetickým, elektrodynamickým nebo
    tepelným voltmetrem (ampérmetrem). Střední hodnotu napětí (proudu) magnetoelektrickým
    voltmetrem (ampérmetrem) a střední hodnotu výkonu elektrodynamickým wattmetrem.
 
      %----------------------------------
      % image: TEO_fce03.tex label: \label{TEO:fig_04}
        % \documentclass{article}
% \usepackage{tikz}
% \usetikzlibrary{decorations.markings}
% \usetikzlibrary{intersections}
% \usepackage{subfigure} 
% \usetikzlibrary{calc}

% \newcommand{\MyXYcross}{%
%   \draw[name path=axeX,->] (\xmin,\nula) -- (\xmax,\nula)   
%     node[right] {$\omega t$} coordinate(x axis);
%   \draw[name path=axeY,->] (\phase,\ymin) -- (\phase,\ymax) 
%     node[left]  {$v(t)$} coordinate(y axis);
%   \path[name intersections={of=axeX and axeY, name=pocatek}];
%   \node[below left] at (pocatek-1) {$0$};
%   \draw[fill=white] (pocatek-1) circle(2pt);
% }
%\begin{document}
 
  \begin{figure}[ht!]
    \centering
        \def\xmin{-15}
        \def\xmax{180}
        \def\ymin{-60}
        \def\ymax{+70}
        \def\nula{0}
        \def\phase{15}
        \def\period{48}
        \def\myscale{0.8}   
      \begin{tikzpicture}
        \begin{scope}[domain=-0.3*pi:2.5*pi,draw=black,line join=round, miter limit=4.00,line
                      width=0.5pt, y=1pt,x=1pt, scale=\myscale] \pgfmathsetmacro\bx{sin(pi*0.5 r)}        
          % osový kříž      
          \draw[name path=axeX,->] (\xmin,\nula) -- (\xmax,\nula)   
            node[right] {$\omega t$} coordinate(x axis);
          \draw[name path=axeY,->] (\phase,\ymin) -- (\phase,\ymax) 
            node[left]  {$v(t)$} coordinate(y axis);
          \path[name intersections={of=axeX and axeY, name=pocatek}];
          \node[below left] at (pocatek-1) {$0$};
          \draw[fill=white] (pocatek-1) circle(2pt);
          % funkce
          \draw[name path=sinx, color=blue, smooth, x=20pt, y=50pt]   
              plot[mark=triangle*] (\x,{sin(\x r)}); % r .. radian    
          \path[name intersections={of=axeX and sinx, name=point}]; % prusecik osy x s funkcí sin(x)
            \draw[dotted, name path=line1] (point-1) --+(0,-70) coordinate(Ta); 
            \draw[dotted, name path=line3] (point-3) --+(0,-70) coordinate(Tc);
            \draw[<->] (Ta) -- (Tc); 
            \draw[fill, color=black] ($ (Ta)!0.5!(Tc) $) node[above, black] {$T$}; 
            \draw[-]  (Ta) ++(-1,30)  -- +(\phase+1.5,0);
            \draw[->] (Ta) ++(-5, 30) -- +(+5,0); % left arrow for dimension \varphi
            \draw[->] (Ta)  +(\phase*0.5,30) 
              node[above] {$\varphi$} ++(\phase+5,30) -- +(-5,0); % left arrow for dimension \varphi
          % amplitude Vm
          \draw[dotted, x=20pt, y=50pt] (pi*0.5,\bx) -- ++(3,0) coordinate(Vm);
          \draw[dotted, x=20pt, y=50pt] (pi*0.5,\bx) -- ++(3,0) coordinate(Vm);
          % http://tex.stackexchange.com/questions/85079/tikz using functions for calculaton
          % x-y parts of the coordinate in the stream 
          % \draw[fill=white, x=20pt, y=50pt] ({pi*0.5},{sin(pi*0.5 r)}) circle(1pt);
          \draw[fill=white, x=20pt, y=50pt] (pi*0.5,\bx) circle(1pt);         
          \draw[<->, x=20pt, y=50pt] (Vm) -- +(0,-0.5*\bx) node[right] {$V_m$} -- ++(0,-\bx);         
        \end{scope}  
      \end{tikzpicture}
      \caption{Haromincká funkce $v = V_m\cos(\omega t + \varphi)$ resp. $v= V_m\cos(\omega t +
               \varphi')$ kde je $\varphi' = \varphi - \frac{T}{4}$}
      \label{TEO:fig_04}
  \end{figure}

  
%\end{document}  
      %---------------------------------- 
         
    Střídavou veličinu $v$ lze též do jisté míry charakterizovat \emph{činitelem tvaru} $\beta$,
    \emph{činitelem výkyvu} $\gamma$ a \emph{činitelem plnění} $\alpha$ definovanými vztahy
    \begin{equation}\label{TEO:eq_harm07}
      \beta = \frac{V}{V_s}, \quad \gamma = \frac{V_m}{V}, \quad \alpha = \frac{V_s}{V_m}
    \end{equation}    
    Je zřejmé, že platí $\alpha\beta\gamma = 1$.
    
    V elektrotechnice mají velkou důležitost periodická napětí a proudy, jejichž závislost je dána
    sinusovou nebo kosinusovou funkcí, tj.
    \begin{equation}\label{TEO:eq_harm08}
      v = V_m\sin(\omega t + \varphi),
    \end{equation}        
    nebo
    \begin{equation}\label{TEO:eq_harm09}
      v = V_m\cos(\omega t + \varphi),
    \end{equation}  
    kde $V_m$, $\omega$, $\varphi$ jsou konstanty (obr. \ref{TEO:fig_04})
    
    Jelikož, tato napětí, resp. proudy představují \emph{harmonické kmity}, nazýváme je
    \emph{harmonicky proměnné}, nebo krátce \emph{harmonická napětí} resp. \emph{harmonická
    proudy}. Konstanta $V_m$ je maximální hodnota, či-li \emph{amplituda}, $\omega t + \varphi$ je
    \emph{fáze}, $\omega = 2\pi f = \frac{2\pi}{T}$ je \emph{úhlový kmitočet} a $\varphi$ je
    \emph{počáteční fáze} harmonické funkce.
    
    Rozdíl fází dvou harmonických veličin (stejného kmitočtu) nazýváme \emph{fázový posun}.
    
    % --------example: Efektivní hodnota výpočet -----------
    % \label{TEO:exam007}
    % !TeX spellcheck = cs_CZ
\begin{example}\label{TEO:exam007}
  Pro harmonickou veličinu, určete efektivní hodnotu, střední hodnotu, činitele tvaru, činitele
  výkyvu a činitele plnění \newline
  \textbf{Řešení:} Efektivní hodnota je:
  \begin{align}
    V &= \sqrt{\frac{1}{T}\int_0^TV_m^2\cos^2{(\omega t + \varphi)}\,dt}    \nonumber  \\
      &= \sqrt{\frac{1}{T}\int_0^TV_m^2\sin^2{(\omega t + \varphi)}\,dt} = 
         \frac{1}{\sqrt{2}}V_m \doteq 0.707 V_m   
  \end{align}
  Podrobný výpočet tohoto integrálu pomocí substituce $\omega t + \varphi=\dfrac{\alpha}{2}$ je
  poněkud zdlouhavější:
  \begin{align*}
      \omega t + \varphi=\dfrac{\alpha}{2}   
    & \rightarrow  2(\omega t + \varphi) = \alpha      \\ 
      \omega dt = \frac{1}{2}d\alpha         
    & \rightarrow dt = \frac{1}{2\omega}d\alpha
  \end{align*}
  Nesmíme zapomenout přepočítat meze $\alpha_d|_{t=0}=2\varphi$ a $\alpha_h|_{t=T} = 
  4\pi+2\varphi$ nového integrálu.
  \begin{align*}
    V^2  &= \frac{V_m}{2T\omega}\int_{\alpha_d}^{\alpha_h}\cos^2\frac{\alpha}{2}\,d\alpha    \\
         &= \frac{V_m}{4\pi}\int_{\alpha_d}^{\alpha_h}\frac{1+\cos\alpha}{2}\,d\alpha  
          = \frac{V_m}{4\pi}\left(\frac{\alpha}{2}|_{\alpha_d}^{\alpha_h}
          + \frac{1}{2}\sin\alpha|_{\alpha_d}^{\alpha_h}\right)                              \\
         &= \frac{V_m}{4\pi}\left(2\pi+\varphi-\varphi 
          + \frac{1}{2}\sin(4\pi+2\varphi)
          - \frac{1}{2}\sin(2\varphi)\right) = \frac{V_m}{2}.  
  \end{align*}  
  Při zjednodušování integrálu je užito známého goniometrického vzorce \(\cos^2\dfrac{\alpha}{2} = 
  \dfrac{1+\cos\alpha}{2}\) a faktu \(\sin(x+2k\pi)=\sin x\)
  
  Střední hodnota kladné půlvlny je 
  \begin{align*}
    V_s &= \frac{2}{T}\int\limits_{-S\frac{T}{4}-
           \frac{\varphi}{\omega}}^{\frac{T}{4}-
           \frac{\varphi}{\omega}}{V_m\cos(\omega t +\varphi)}\,dt                            \\
        &= \frac{2}{T}\int\limits_{-\frac{\varphi}{\omega}}^{-\frac{T}{2}-
           \frac{\varphi}{\omega}}{V_m\sin(\omega t +\varphi)}\,dt                            \\
        &= \frac{2}{\pi}V_m \doteq 0,637V_m
  \end{align*}
  činitele tvaru, výkyvu a plnění jsou 
  \begin{align*}
    \beta  &=\frac{V}{V_s}   =\frac{\pi}{2\sqrt{2}}=1,111, \\ 
    \gamma &=\frac{V_m}{V}   =\sqrt{2}\doteq1.414,         \\
    \alpha &=\frac{V_s}{V_m} =\frac{2}{\pi}\doteq0,637 
  \end{align*}
\end{example}  
    %-------------------------------------------------------
      
 %--------------------------------------------------------------------------------------------------
  \section{Obvody s nastavitelnými parametry}
    V praxi se setkáváme s obvody, u nichž lze (spojitě nebo stupňovitě) nastavit odpor odporníku, 
    kapacitu kondenzátoru, vlastní nebo vzájemnou indukčnost cívek, amplitudu, fázi nebo kmitočet
    zdroje (napětí nebo proud). Nazveme je \emph{obvody s nastavitelnými parametry}.
 %--------------------------------------------------------------------------------------------------

} % tikzset
%---------------------------------------------------------------------------------------------------
\printbibliography[title={Seznam literatury}, heading=subbibliography]
\addcontentsline{toc}{section}{Seznam literatury}