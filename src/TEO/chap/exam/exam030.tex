% !TeX spellcheck = cs_CZ
\wikitextrule
\begin{example}\label{MAI:exam030}
  Je dána funkce \(f: f(x) = \frac{x + 1}{x}\). Sledujme její chování, když hodnoty argumentu \(x\) 
  budou vzrůstat nade všechny meze neboli, jak říkáme, \(x\) se bude blížit k \(+\infty\) (což 
  zapisujeme \(x \to + \infty\) (viz obr. \ref{mai_fig019}). Můžeme psát \(f(x) = 1 + 1/x\). 
  Vzrůstají-li neomezeně hodnoty proměnné \(x\), blíží se hodnoty výrazu \(1/x\) čím dál tím více 
  nule, takže funkční hodnoty \(f(x)\) jsou čím dál tím blíže číslu \(1\). V tomto případě píšeme 
  \(lim_{x\to+\infty} f(x) = 1\) nebo \(f(x) \to 1\) pro \(x\to +\infty\) a říkáme, že funkce \(f\) 
  má v bodě \(+\infty\) limitu rovnou \(1\). Přesně to znamená toto: Zvolíme-li libovolně malé 
  \(\varepsilon > 0\), můžeme nalézt \(p > 0\) tak, že pro \(x > p\) platí \(\abs{f(x) — l} < 
  \varepsilon\). (Viz obr. \ref{mai_fig019}.) Můžeme to říci i takto: Zvolíme-li libovolně okolí 
  bodu \(1\), existuje okolí bodu \(+\infty\) tak, že pro každé \(x\) (konečné) z tohoto okolí je 
  \(f(x)\) ve zvoleném okolí bodu \(1\).
  
  {\centering
   \captionsetup{type=figure}
   \begin{tikzpicture}[thick,scale=0.7, 
    every node/.style={transform shape},
    ]

\tikzset{->-/.style={decoration={
  markings,
  mark=at position #1 with {\arrow{stealth}}},postaction={decorate}}}
  
  \begin{axis}[
    xmin = -0.5, xmax = 5.5, ymin = 0, ymax = 4.5,  % osy
    domain =0.2:5,
    restrict y to domain=0:4,
    axis equal image,
    grid = major,   % both
    grid style={line width=.1pt, draw=gray!20},
    major grid style={dashed, line width=.2pt, draw=gray!40},
    clip = true,
    clip mode=individual,
    xtick={1,2,3,4,5}, % make steps of length 0.2
    ytick={0,1,2,3,4}, 
    axis x line = middle,
    axis y line = middle,
    xlabel={$x$}, ylabel={$y$},
    enlarge y limits={rel=0.07},
    enlarge x limits={rel=0.07},
    ]

    \addplot[color=Gold3, samples=100, smooth, ultra thick, unbounded coords=jump,
             no markers, domain = 0.1:5, name path global=func1] 
       gnuplot{1+1/x};

    \node [fill=white] at (rel axis cs: 0.4,0.75) {\(y=\dfrac{x+1}{x}\)};

    \path[name path=line] (0,1.7) -- (3,1.7); 
        % Intersections points
        \path [name intersections={of=func1 and line,by={P1}}] (P1) node [] {};
    
    \draw[black,fill=black] (P1) circle (.3ex);      
    \path (P1 |- 3,-0.1) node [below, fill=white] (X) {p} -- (P1) -- (P1 -| 0,3);

    \draw[thick,red, fill=white] ([shift=(90:2mm)]X) 
         arc (270:360:1mm) node(Y) {} arc (360:450:1mm);
    \draw[thin] (P1 |- 3,-0.1) -- (P1) -- (P1 -| 0,3);
    \draw[line width = 2pt,red] (Y |- 3,0)  -- ++(3.5,0);
    
    \path[name path=line] (2.4,0) -- ++(0,2.5); 
        % Intersections points
        \path [name intersections={of=func1 and line,by={P1}}] (P1) node [] {};
        \draw[->-=.4, dashed, gray] (P1 |- 3,-0.05) node[below] {\(x\)} -- (P1);
        \draw[->-=1,  dashed, gray] (P1) -- (P1 -| 0,3) 
          node[left] {\small\(f(x)\)};
 
    \draw[line width = 1pt, black, dashed] (0,1) -- ++(5,0);
    \draw[line width = 3pt, red, line cap=butt] (0,0.3) -- (0,1.7);
    \draw [thick] (-.2, 0.3) node[left] {\(1-\varepsilon\)} -- (0.1, 0.3);
    \draw [thick] (-.2, 1.7) node[left] {\(1+\varepsilon\)} -- (0.1, 1.7 );

    \draw[black,fill=white] (0,1) circle (.4ex);
  \end{axis}
\end{tikzpicture}
   \captionof{figure}{K příkladu \ref{MAI:exam030}
   \cite[s.~119]{Brabec1989}
   \label{mai_fig019}}
  \par}
\end{example}















