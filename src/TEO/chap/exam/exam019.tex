% !TeX spellcheck = cs_CZ
\wikitextrule
\begin{example}\label{MAI:exam019} 
  Funkce je dána vzorcem 
  \begin{equation*}
    f(x):y=\abs{x}.
  \end{equation*} 
  Přirozeným definičním oborem této funkce je množina $\realset$. Táž funkce může být dána i 
  vzorcem
  \begin{equation*}
    f(x):y=\sqrt{x^2},
  \end{equation*}    
  nebo dvěma rovnicemi
  \begin{equation*}
    f(x):y=
       \begin{cases}
           x & \text{je-li} x \geq 0. \\
          -x & \text{je-li} x < 0,
       \end{cases}                 
  \end{equation*}  
  což je zřejmé, uvědomíme-li si jak je definována absolutní hodnota. Graf funkce je na obr. 
  \ref{mai_fig007}.
  
  {\centering
   \captionsetup{type=figure}
   % xelatex --enable-write18 MAI002a.tex
\begin{tikzpicture}[thick,scale=0.7, every node/.style={transform shape}]
  \begin{axis}[
    xmin = -3, xmax = 3, ymin = 0, ymax = 3,  % osy
    domain = -3:3,
    restrict y to domain=0:3,
    grid = major,   % both
    grid style={line width=.1pt, draw=gray!20},
    major grid style={dashed, line width=.2pt, draw=gray!40},
    minor tick num=5,
    clip = true,
    clip mode=individual,
    axis x line = middle,
    axis y line = middle,
    xlabel={$x$},
  %  xlabel style={at=(current axis.right of origin), anchor=west},
    ylabel={$y$},
  %  ylabel style={at=(current axis.above origin), anchor=south},
    enlarge y limits={rel=0.13},
    enlarge x limits={rel=0.07},
  ]
  
   \addplot[color=Gold3, samples=200, smooth, ultra thick, unbounded coords=jump, no markers] 
      gnuplot{abs(x)};  
  \end{axis}
\end{tikzpicture} 
   \captionof{figure}{Graf funkce $y=\abs{x}$}
   \label{mai_fig007}
  \par}
\end{example}