% !TeX spellcheck = cs_CZ
%====================== Sbírka řešených příkladů ==================================================
  \begin{example}\label{mai:exam016}
    \begin{equation}\label{mai:exam016_001}
      \int{xe^x\dx}
    \end{equation}
    Užijeme metodu per partes 
      \(\left[\begin{array}{cc}
              u=x   & dv=e^x \\
              du=dx & v=e^x
        \end{array}\right]\) 
    \begin{equation*}
      \int{xe^xdx} = xe^x-\int{e^x\dx} = xe^x - e^x+ c
    \end{equation*}
%--------------------------------------------------------------------------------------------------
    \begin{equation}\label{mai:exam016_002}
      \int{\arctan x\dx}\qquad x\in R
    \end{equation}
    Užijeme metodu per partes 
       \(\left[\begin{array}{cc} 
                u =\arctan x                     &  dv= 1  \\ 
               du =\displaystyle\frac{1}{x^2+1}  &   v= x
             \end{array}
       \right]\)
       
    \begin{align*}
       \int{\arctan xdx}                      &= x\arctan x-\int\frac{x}{x^2+1}         \\
       x\arctan x-\int\frac{x}{x^2+1}         &= 
         \left[\begin{array}{c} 
                  x^2 + 1 = t  \Rightarrow 2xdx = dt        \\ 
                      xdx = \displaystyle{\frac{dt}{2}}
               \end{array} 
         \right] =                                                                    \\ 
       x\arctan x-\frac{1}{2}\int\frac{dt}{t} &= x\arctan x-\frac{1}{2}\ln|t|         \\
         &=   x\arctan x-\frac{1}{2}\ln|1+x^2|+ c                                     \\
    \end{align*}
%--------------------------------------------------------------------------------------------------
    \begin{equation}\label{mai:exam016_003}
      \int{\sqrt{x^2+a}\dx}, \text{kde} a\neq0, x^2+a>0
    \end{equation}  
    \begin{align*}
      \int{\sqrt{x^2+a}\dx}                           &=
        \left[
          \begin{array}{cc} 
             u =\sqrt{x^2+a}              & dv = 1 \\ 
            du =\displaystyle
                  \frac{x}{\sqrt{x^2+a}}  &  v = x
          \end{array}
        \right]                                                                                   \\
      x\sqrt{x^2+a}-\int{\frac{x^2}{\sqrt{x^2+a}}\dx} &= 
        \displaystyle{x\sqrt{x^2+a}-\int{\frac{x^2+a-a}{\sqrt{x^2+a}}\dx}}                        
    \end{align*}\vspace*{-1em}
    \begin{align*}
      \int{\sqrt{x^2+a}\dx}                           &= 
        \displaystyle{x\sqrt{x^2+a}-\int{\sqrt{x^2+a}\dx} + \int{\frac{a}{\sqrt{x^2+a}}\dx}}      \\
      \int{\sqrt{x^2+a}\dx}                           &= 
        \frac{1}{2}\left[x\sqrt{x^2+a}+a\int{\frac{1}{\sqrt{x^2+a}}}\dx\right]
    \end{align*}
    
    Integrál \(\int\frac{1}{\sqrt{x^2+a}}\dx\) na pravé straně vyjádříme podle příkladu 
    \ref{ma:ex_sub_metoda1} a výsledek do\-sta\-ne\-me ve tvaru
    \begin{equation*}
      \int\displaystyle{{\sqrt{x^2+a}\dx}
         =\frac{1}{2}\left[x\sqrt{x^2+a}+a\ln{|x + \sqrt{x^2+a}|}\right]}
    \end{equation*}
  \end{example}
  
  \small\begin{example}
    \begin{equation}\label{mai:int_ex_02}
      \int{\frac{2x^4-5x^2+14x+13}{x^2-x-2}\dx} \qquad x\in R - \{1,2\}
    \end{equation}
    Dělením čitatele integrandu jmenovatelem dostaneme rozklad integrandu na součet funkcí, jejich 
    integrály najdeme snadno:
    \tiny\begin{equation*}
      \polylongdiv[style=C,div=:]{2x^4-5x^2+14x+13}{x^2-x-2}
    \end{equation*}\small

    Zbytek po dělení představuje integrál, jež je počítán v příkladu \ref{MA:eq_ex1} a proto ho 
    vynecháme. 
    \begin{align*}
       &= 2\int x^2\dx + 2\int x\dx + \int\dx + \int\frac{19x+15}{x^2-x-2}\dx     \\
       &= \frac{2}{3}x^3 + x^2 + x + \frac{4}{3}\ln|x+1| - \frac{53}{3}\ln|x-2| + C 
    \end{align*}                
  \end{example}