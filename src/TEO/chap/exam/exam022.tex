% !TeX spellcheck = cs_CZ
\wikitextrule
\begin{example}\label{MAI:exam022}
  \begin{itemize}
  \item[]
  \item  Rovnicí $x+2y-3=0$ je implicitně definována funkce  
         $f:y=-\dfrac{1}{2}x+\dfrac{3}{2}$.
  \item  Rovnicí $x^2+y^2=1$ a podmínkou $y\geq0$ je definována implicitní funkce z příkladu 
         \ref{MAI:exam020}. Relace $\{(x,y)\in\realset^2;\ x^2+y^2=1\}$ není ovšem jednoznačná, 
         každé hodnotě $x\in(-1,1)$ odpovídají dvě hodnoty $y: y_1=\sqrt{1-x^2}$, $y: y_2 = 
         -\sqrt{1-x^2}$. Podmínkou $y\geq0$ druhou hodnotu vylučujeme. Místo podmínky $y\geq0$ 
         bychom mohli uvést i jiné podmínky, aby rovnice $x^2+y^2=1$ určovala implicitní funkci.   
  \end{itemize}
\end{example}